%%%%%%%%%%%%%%%%%%%%%%%%%%%%%%%%%%%%%%%%%%%%%%%%%%%%%%%%%%%%%%%%%%%%%%%%
\section{Evaluations and Downstream Tasks Related Work}
%%%%%%%%%%%%%%%%%%%%%%%%%%%%%%%%%%%%%%%%%%%%%%%%%%%%%%%%%%%%%%%%%%%%%%%%

%%%%%%%%%%%%%%%%%%%%%%%%%%%%%%%%%%%%%%%%%%%%%%%%%%%%%%%%%%%%%%%%%%%%%%%%
\section{FTB Related Work}
%%%%%%%%%%%%%%%%%%%%%%%%%%%%%%%%%%%%%%%%%%%%%%%%%%%%%%%%%%%%%%%%%%%%%%%%


\subsection{Brief state of the art of NER}
\label{subsec:sota}
%BS: this section is heavily inspired (understand: partly copy-pasted) from the \camembert paper. Maybe some rewriting could be a good idea.

As mentioned above, NER was first addressed using rule-based approaches, followed by statistical and now neural machine learning techniques. In addition, many systems use a lexicon of named entity mentions, usually called a ``gazetteer'' in this context.

Most of the advances in NER  have been achieved on English, in particular with the CoNLL 2003 \cite{tjong-kim-sang-de-meulder-2003-introduction} and  Ontonotes~v5 \cite{pradhan-etal-2012-conll,pradhan-etal-2013-towards} corpora. In recent years, NER was traditionally tackled using Conditional Random Fields (CRF) \cite{lafferty-etal-2001-conditional} which are quite suited for NER; CRFs were later used as decoding layers for Bi-LSTM architectures \cite{huang-etal-2015-bidirectional,lample-etal-2016-neural} showing considerable improvements over CRFs alone. These Bi-LSTM-CRF architectures were later enhanced with contextualized word embeddings which yet again brought major improvements to the task \cite{peters-etal-2018-deep,akbik-etal-2018-contextual}. Finally, large pre-trained architectures settled the current state of the art showing a small yet important improvement over previous NER-specific architectures \cite{devlin-etal-2019-bert,baevski-etal-2019-cloze}.

For French, rule-based system have been developed until relatively recently, due to the lack of proper training data \cite{sekine-nobata-2004-definition,rosset-etal-2005-interaction,stern-sagot-2010-resources,nouvel-etal-2014-pattern}. The limited availability of a few annotated corpora (cf.~Section~\ref{sec:intro}) made it possible to apply statistical machine learning techniques \cite{bechet-charton-2010-unsupervised,dupont-tellier-2014-named,dupont-2017-exploration} as well as hybrid techniques combining handcrafted grammars and machine learning \cite{bechet-etal-2011-cooperation}. To the best of our knowledge, the best results previously published on FTB NER are those obtained by \newcite{dupont-2017-exploration}, who trained both CRF and BiLSTM-CRF architectures and improved them using heuristics and pre-trained word embeddings. We use this system as our strong baseline.

Leaving aside French and English, the CoNLL 2002 shared task included NER corpora for Spanish and Dutch corpora \cite{tjong-kim-sang-2002-introduction} while the CoNLL 2003 shared task included a German corpus \cite{tjong-kim-sang-de-meulder-2003-introduction}. The recent efforts by \newcite{strakova-etal-2019-neural} settled the state of the art for Spanish and Dutch, while \newcite{akbik-etal-2018-contextual} did so for German.



\subsection{The original named entity FTB layer}
\label{subsec:originalannotations}


\newcite{sagot-etal-2012-annotation} annotated the FTB with the span, absolute type\footnote{
    Every mention of \emph{France} is annotated as a \texttt{Location} with subtype \texttt{Country}, as given in \aleda database, even if in context the mentioned entity is a political organization, the French people, a sports team, etc.}, sometimes subtype and \aleda unique identifier of each named entity mention.\footnote{Only proper nouns are considered as named entity mentions, thereby excluding other types of referential expressions.} Annotations are restricted to person, location, organization and company names, as well as a few product names.\footnote{More precisely, we used a tagset of 7 base NE types: \texttt{Person}, \texttt{Location}, \texttt{Organization}, \texttt{Company}, \texttt{Product}, \texttt{POI} (Point of Interest) and \texttt{FictionChar}.} There are no nested entities. Non capitalized entity mentions (e.g.~\emph{banque mondiale} `World Bank') are annotated only if they can be disambiguated independently of their context. Entity mentions that require the context to be disambiguated (e.g.~\emph{Banque centrale}) are only annotated if they are capitalized.
\footnote{So for instance, in \emph{université de Nantes} `Nantes university', only \emph{Nantes} is annotated, as a city, as \emph{université} is written in lowercase letters. However, \emph{Université de Nantes} `Nantes University' is wholly annotated as an organization. It is non-ambiguous because \emph{Université} is capitalized. \emph{Université de Montpellier} `Montpellier University' being ambiguous when the text of the FTB was written and when the named entity annotations were produced, only \emph{Montpellier} is annotated, as a city.}
For person names, grammatical or contextual words around the mention are not included in the mention (e.g.~in \emph{M.~Jacques Chirac} or \emph{le Président Jacques Chirac}, only \emph{Jacques Chirac} is included in the mention).


Tags used for the annotation have the following information:
\begin{itemize}
    \item the identifier of the NE in the \aleda database (\texttt{eid} attribute); when a named entity is not present in the database, the identifier is \texttt{null},\footnote{Specific conventions for entities that have merged, changed name, ceased to exist as such (e.g.~\emph{Tchequoslovaquie}) or evolved in other ways are described in \newcite{sagot-etal-2012-annotation}.}
    \item the normalized named of the named entity as given in \aleda; for locations it is their name as given in GeoNames and for the others, it is the title of the corresponding French Wikipedia article,
    \item the type and, when relevant, the subtype of the entity.
\end{itemize}
Here are two annotation examples:\\
\noindent{\small\texttt{<ENAMEX type="Organization" eid="1000000000016778"
        name="Confédération\\
        française démocratique du travail">CFDT</ENAMEX>\\
        <ENAMEX type="Location"
        sub\_type="Country"
        eid="2000000001861060"\\
        name="Japan">Japon</ENAMEX>}}

\newcite{sagot-etal-2012-annotation} annotated the 2007 version of the FTB treebank (with the exception of sentences that did not receive any functional annotation), i.e.~12,351 sentences comprising 350,931 tokens. The annotation process consisted in a manual correction and validation of the output of a rule- and heuristics-based named entity recognition and linking tool in an XML editor.
Only a single person annotated the corpus, despite the limitations of such a protocol, as acknowledged by \newcite{sagot-etal-2012-annotation}.

In total, 5,890 of the 12,351 sentences contain at least a named entity mention. 11,636 mentions were annotated, which are distributed as follows:
3,761 location names, 3,357 company names, 2,381 organization names, 2,025 person names, 67 product names, 29 fiction character names and 15 points of interest.


%%%%%%%%%%%%%%%%%%%%%%%%%%%%%%%%%%%%%%%%%%%%%%%%%%%%%%%%%%%%%%%%%%%%%%%%
\section{SinNER Related Work}
%%%%%%%%%%%%%%%%%%%%%%%%%%%%%%%%%%%%%%%%%%%%%%%%%%%%%%%%%%%%%%%%%%%%%%%%

\subsection{Related Work on Named Entity Recognition}
\label{sec:sota}

Named Entity Recognition came into light as a prerequisite for designing robust Information Extraction (IE) systems in the MUC conferences \cite{grishman-sundheim-1995-design}. This task soon began to be treated independently from IE since it can serve multiple purposes, like Information retrieval or Media Monitoring for instance \cite{yangarber-etal-2002-unsupervised}. As such, shared task specifically dedicated to NER started to rise like the CoNLL 2003 shared task \cite{tjong-kim-sang-de-meulder-2003-introduction}. Two main paths were followed by the community: (i) since NER was at first used for general purposes, domain extension start to gain interest \cite{evans-2003-a}; (ii) since the majority of NER systems were designed for English, the extension to novel languages (including low resource languages) became of importance \cite{rossler-2004-adapting}.

One can say that NER followed the different trends in NLP. The first approaches were based on gazeeters and handcrafted rules. Initially NER was considered to be solved by a patient process involving careful syntactic analysis \cite{hobbs-1993-generic}. Supervised learning approaches came to fashion with the increase of available data and the rise of shared tasks on NER. Decision trees and Markov models were soon outperformed by Condition Random Fields (CRF).
%By taking advantage of the sequentiality of textual data, CRF helped to set new state-of-the-art results in the domain \cite{finkel-etal-2005-incorporating}.
Thanks to its ability to model dependencies and to take advantage of the sequentiality of textual data, CRF helped to set new state-of-the-art results in the domain \cite{finkel-etal-2005-incorporating}.
Since supervised learning results were bound by the size of training data, lighter approaches were tested in the beginning of the 2000's, among them we can cite weakly supervision \cite{yangarber-2003-counter} and active learning \cite{shen-etal-2004-multi}.

During a time, most of promising approaches involved an addition to improve CRFs : word embeddings \cite{passos-etal-2014-lexicon}, (bi-)LSTMs \cite{lample-etal-2016-neural} % \cite{Ma-2016}
or contextual embeddings \cite{peters-etal-2018-deep}.
More recently, the improvements in contextual word embeddings made the CRFs disappear as standalone models for systems reaching state-of-the-art results, see \cite{stanislawek-etal-2019-named} for a review on the subject and a very interesting discussion on the limits attained by state-of-the-art systems, the \textit{Glass Ceiling}.

\subsubsection{Contextualized word embeddings}

\emph{Embeddings from Language Models} (ELMo) \cite{peters-etal-2018-deep} is a Language Model, i.e, a model that given a sequence of $N$ tokens, $(t_1, t_2, ..., t_N)$, computes the probability of the sequence
by modeling the probability of token $t_k$ given the history $(t_1, ..., t_{k-1})$:
\[
    p(t_1, t_2, \ldots, t_N) = \prod_{k=1}^N p({t_k} \mid t_1, t_2, \ldots, t_{k-1}).
\]
However, ELMo in particular uses a bidirectional language model (biLM) consisting of $L$ LSTM layers, that is, it combines both a forward and a backward language model jointly maximizing the log likelihood of the forward and backward directions:
\begin{align*}
     & \sum_{k=1}^N \left( \right. \log p({t_k} \mid t_1, \ldots, t_{k-1}; \Theta_x, \overrightarrow{\Theta}_{LSTM}, \Theta_s) \\
     & + \log p({t_k} \mid t_{k+1}, \ldots, t_{N}; \Theta_x, \overleftarrow{\Theta}_{LSTM}, \Theta_s)
    \left. \right).
\end{align*}
where at each position $k$, each LSTM layer $l$ outputs a context-dependent representation $\overrightarrow{\mathbf{h}}^{LM}_{k,l}$ with $l=1, \ldots, L$ for a forward LSTM, and $\overleftarrow{\mathbf{h}}^{LM}_{k,l}$ of $t_k$ given $(t_{k+1}, \ldots, t_N)$ for a backward LSTM.

ELMo also computes a context-independent token representation $\mathbf{x}^{LM}_{k}$ via token embeddings or via a CNN over characters. ELMo then ties the parameters for the token representation ($\Theta_x$) and Softmax layer ($\Theta_s$) in the forward and backward direction while maintaining separate parameters for the LSTMs in each direction.

ELMo is a task specific combination of the intermediate layer representations in the biLM, that is,
for each token $t_k$, a $L$-layer biLM computes a set of $2L + 1$ representations
\begin{align*}
    R_k & =  \{\mathbf{x}^{LM}_{k}, \overrightarrow{\mathbf{h}}^{LM}_{k,l}, \overleftarrow{\mathbf{h}}^{LM}_{k,l} \ |\  l =1, \ldots, L \} \\
        & =  \{\mathbf{h}^{LM}_{k,l}\ | \ l=0, \ldots, L\},
\end{align*}
where $\mathbf{h}^{LM}_{k,0}$ is the token layer and
\[
    \mathbf{h}^{LM}_{k,l} = [\overrightarrow{\mathbf{h}}^{LM}_{k,l}; \overleftarrow{\mathbf{h}}^{LM}_{k,l}],
\]
for each biLSTM layer.


When included in a downstream model, as it is the case in this paper, ELMo collapses all $L$ layers in $R$ into a single vector $\mathbf{ELMo}_k = E(R_k; \mathbf{\Theta}_e)$, generally computing a task specific weighting of all biLM layers:
\begin{align*}
    \mathbf{ELMo}^{task}_k & = E(R_k; \Theta^{task})                                       \\
                           & =\gamma^{task} \sum_{l=0}^L s^{task}_l \mathbf{h}^{LM}_{k,l}.
\end{align*}
applying layer normalization to each biLM layer before weighting.

Following \cite{peters-etal-2018-deep}, we use in this paper ELMo models where $L=2$, i.e., the ELMo architecture involves a character-level CNN layer followed by a 2-layer biLSTM.

