\newpage  % chktex 1
\thispagestyle{empty}
\phantomsection  % chktex 1
\addcontentsline{toc}{chapter}{Back}

\begin{otherlanguage}{french}
    \begin{center}
        Une approche basée sur les données pour le traitement automatique du langage naturel en français contemporain et historique
    \end{center}

    \begingroup
    \scriptsize
    Depuis plusieurs années, les approches neuronales ont régulièrement amélioré l'état de l'art du traitement automatique des langues (TAL) sur une grande variété de tâches. L'un des principaux facteurs ayant permis ces progrès continus est l'utilisation de techniques d'apprentissage par transfert. Ces méthodes consistent à partir d'un modèle pré-entraîné et à le réutiliser, avec peu ou pas d'entraînement supplémentaire, pour traiter d'autres tâches. Même si ces modèles présentent des avantages évidents, leur principal inconvénient est la quantité de données nécessaire pour les pré-entraîner. Ainsi, le manque de données disponibles à grande échelle a freiné le développement de tels modèles pour le français contemporain et a fortiori pour ses états de langue plus anciens.

    Cette thèse met l'accent sur le développement de corpus pour le pré-entraînement de telles architectures. Cette approche s'avère extrêmement efficace car nous sommes en mesure d'améliorer l'état de l'art pour un large éventail de tâches de TAL pour le français contemporain et historique, ainsi que pour six autres langues contemporaines. De plus, nous montrons que ces modèles sont extrêmement sensibles à la qualité, à l'hétérogénéité et à l'équilibre des données de pré-entraînement et montrons que ces trois caractéristiques sont de meilleurs prédicteurs de la performance des modèles que la taille des données de pré-entraînement. Nous montrons également que l'importance de la taille des données de pré-entraînement a été surestimée en démontrant à plusieurs reprises que l'on peut pré-entraîner de tels modèles avec des corpus de taille assez modeste.

    \textbf{Mots-clés :} modèle de langue, corpus de pré-entraînement, traitement automatique des langues, français contemporain, français historique, apprentissage par transfert.
    \endgroup
\end{otherlanguage}

\vspace{1cm}

\begin{center}
    A Data-driven Approach to Natural Language Processing for Contemporary and Historical French
\end{center}

\begingroup
\scriptsize
In recent years, neural methods for Natural Language Processing (NLP) have consistently and repeatedly improved the state of the art in a wide variety of NLP tasks. One of the main contributing reasons for this steady improvement is the increased use of transfer learning techniques. These methods consist in taking a pre-trained model and reusing it, with little to no further training, to solve other tasks. Even though these models have clear advantages, their main drawback is the amount of data that is needed to pre-train them. The lack of availability of large-scale data previously hindered the development of such models for contemporary French, and even more so for its historical states.

In this thesis, we focus on developing corpora for the pre-training of these transfer learning architectures. This approach proves to be extremely effective, as we are able to establish a new state of the art for a wide range of tasks in NLP for contemporary, medieval and early modern French as well as for six other contemporary languages. Furthermore, we are able to determine, not only that these models are extremely sensitive to pre-training data quality, heterogeneity and balance, but we also show that these three features are better predictors of the pre-trained models' performance in downstream tasks than the pre-training data size itself. In fact, we determine that the importance of the pre-training dataset size was largely overestimated, as we are able to repeatedly show that such models can be pre-trained with corpora of a modest size.

\textbf{Keywords:} language model, pre-training corpora, natural language processing, contemporary french, historical french, transfer learning.
\endgroup

\vfill

\begin{center}
    Sorbonne Université - École doctorale EDITE (ED130)\\
    Campus Pierre et Marie Curie, Tour 24-25 bureau 210\\
    4, place Jussieu, 75252 Paris Cedex 05, France
\end{center}