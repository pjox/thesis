\chapter{CaBeRnet: A Contemporary French Balanced Corpus}

The question of quality versus size of training corpora is increasingly gaining attention and interest in the context of the latest developments in neural language models' performance.
The longstanding issue of corpora "representativeness" is here addressed, in order to grasp to what extent a linguistically balanced cross-genre language sample is sufficient for a language model to gain in accuracy for contextualized word-embeddings on different NLP tasks.

Several increasingly larger corpora are nowadays compiled from the web, i.e. frWAC \citep{baroni-etal-2009-the}, CCNet \citep{wenzek-etal-2020-ccnet} and OSCAR-fr \citep{ortiz-suarez-etal-2019-asynchronous}. However, does large size necessarily go along with better performance for language model training? Their alleged lack of representativeness has called for inventive ways of building a French balanced corpus offering new insights into language variation and NLP.

Following Biber's definition, “representativeness refers to the extent to which a sample includes the full range of variability in a population” \citep{biber-1993-representativeness}. We adopt a balanced approach by sampling a wide spectrum of language use and its cross-genre variability, be it situational (e.g.\ format, author, addressee, purposes, settings or topics) or linguistic, e.g.\ linked to distributional parameters like frequencies of word classes and genres.
In this way, we developed two newly built corpora. The French Balanced Reference Corpus - \textit{CaBeRnet} - includes a wide-ranging and balanced coverage of cross-genre language use to be maximally representative of French language and therefore yield good generalizations from. The second corpus, the \textit{French Children Book Test} (CBT-fr), includes both narrative material and oral language use as present in youth literature, and will be used for domain-specific language model training. Both are inspired by existing American and English corpora, respectively  COCA, the balanced Corpus of Contemporary American English \citep{davies-2009-the, davies-2010-the}, and the Children Book Test \citep[CBT]{hill-etal-2016-the}.

%better performing word-embeddings
%evaluate the quality of word embedding generated by pre-training 
The second main contribution of this paper lies in the evaluation of the quality of the word-embeddings obtained by pre-training and fine-tuning on different corpora, that are made here publicly available.
Based on the underlying assumption that a linguistically representative corpus would possibly generate better word-embeddings. %being more representative of real language use, would tentatively preform better in downstream tasks. 
We provide an evaluation-based investigation of how a balanced cross-genre corpus can yield improvements in the performance of neural language models like ELMo \citep{peters-etal-2018-deep} on various downstream tasks.
The two corpora, \Cabernet and CBT-fr, and the ELMos will be distributed freely under Creative Commons License.

Specifically, we want to investigate the contribution of oral language use as present in different corpora. Through a series of comparisons, we contrast a more domain-specific and written corpus like Wikipedia-fr with the newly built domain-specific CBT-fr corpus which additionally features oral style dialogues, like the ones one can find in youth literature. To test for the effect of corpus size, we further compare a wide ranging corpora characterized by a variety of linguistic phenomena crawled from internet, like OSCAR \citep{ortiz-suarez-etal-2019-asynchronous}, with our newly built French Balanced Reference Corpus \Cabernet.
Our aim is assess the benefits that can be gained from a balanced, multi-domain corpus such as \Cabernet, despite its being 34 times smaller than the web-based OSCAR.

%Methodologically, our approach constitutes an original proof-of-concept that fine-tuning with resources that are up to 35 times smaller than pre-training corpora has a observable impact on classical NLP tasks scores. Secondly, we show that pre-training a language model on a very small sample like the French Children Book Test corpus yields unexpected positive results.
%\iffalse
%Resources associated to this paper encompass\footnote{The link to the corpus and FrElMos will be available upon acceptance of the paper. Following the link the reader will have access to a dedicated website \textit{cabernet-corpus.fr} where raw text version and metadata for each sub-part of the corpus are also be available.} five version of FrELMo trained on the four corpora presented in this paper and two newly brewed corpora - The French Balanced Reference Corpus CaBeRnet and the French Children Book Test.
%\fi

%In sum, this paper offers three main contributions: (1) two newly built corpora one French Balanced Reference Corpus and a second domain-specific corpus having both oral and written style, (2) five versions of FrELMo, and (3) a whole array of computational results that deepen our understanding on the effects of balance and register in NLP evaluation. 


%The data for corpus creation has been extracted from selected sections of open data including dumped data from the web and section of already published corpora that specialize on a given register of source.


%%%%%%%%%%%%%%%%%%%%%%%%%%%%%%%%%%%%%%%%%%%%%%%%%%%%%%%%%
%\subsection{The present Paper}\label{ssect:Paper_struct}

%The paper presents two substantially different training corpora, proposing a set of evaluations to enquire on the accuracy of Language Models

%We report a few experiments designed to better understand the computational impact of the quality and size of training corpora with the method and tasks described so far. All the experiments are performed on xxx.

%Crucially, manipulating the presence of oral transcriptions and oral based text will be interesting to understand the impact on accuracy of our language model and their impact on several language tasks after fine-tuning. 

%\paragraph{Structure of the paper}
The paper is organized as follows. Sections \ref{sec:DescribeCorpora} and \ref{sec:CompareCorpora} are dedicated to a descriptive overlook of the building of our two newly brewed corpora \Cabernet and CBT-fr, including quantitative measures like type-token ratio and morphological richness. %corpora building and data collection. The construction process of  is here presented.% through
%in this section that summarizes information details that can be found in corpora metadata. The achievement of linguistic balance in \Cabernet is detailed in section \ref{subsec:DescribeCaBeRnet}. 
%statistics on the distribution of lexical, syntactic and morphological features of the different sub-parts of the corpus are also presented. 
%Section \ref{sec:CompareCorpora} focuses on several quantitative measures characterizing the corpora under comparison : number of sentences, type-token ratio and morphological richness. %The characteristics of CBT-fr and \Cabernet are compared to the other corpora under analysis (OSCAR-fr, Wiki-fr). 
Section \ref{sect:EvalMethod} presents the evaluation methods for POS-tagging, NER and dependency Parsing tasks, while results are introduced in §\ref{sect:ResultsCorpora}
Finally, we conclude in §\ref{sec:Concl} on the computational relevance of word-embeddings obtained through a balanced and representative corpus, and broaden the discussion on the benefits of smaller and noiseless corpora in neural NLP.% and some future developments of \Cabernet.

%\notemumu{@ALL should we say here why each corpus ? and summarize the table here under \ref{Table_copus_feature} commented here under}

%\begin{table}[htbp]
%\centering
%\begin{tabular}{lccccc}\\\toprule
%Corpora     & noise     & Range     & O/W    & Ling. \\\midrule
%OSCAR-fr    &  -        &   +      &  -      &   -    \\%\midrule 
%\Cabernet   &  +        &   +      &  both   &   +     \\%\midrule 
%Wiki-fr     &  +        &   -      &  -      &   -    \\%\midrule 
%CBT-fr      &  +        &   -      &  both   &   -     \\\bottomrule  
%\end{tabular}
%\caption{\label{Table_copus_feature} Comparison of corpora. Full-fledged linguistic representativity.}
%\end{table}

\section{Corpora Building}%- Quantitative Description
\label{sec:DescribeCorpora}
%Two main criteria guided corpus building, the first was balance and representativeness, and the second was maximizing the usage of open resources to build our corpora.

\subsection{\Cabernet} \label{subsec:DescribeCaBeRnet}

\Cabernet corpus was inspired by the genre partition of the American balanced corpus COCA, %\footnote{\url{https://www.corpusdata.org}}
which currently contains over 618 million words of text (20 million words each year 1990-2019) and is equally divided among spoken, fiction, popular magazines, newspapers, and academic texts \citep{davies-2009-the, davies-2010-the}. A second reference, guiding our approach and sampling method, is one of the earliest precursors of balanced reference corpora: the BNC \citep{bnc-2007-the}, first covered a wide variety of genres, with the intention to be a representative sample of spoken and written language.

\Cabernet was obtained by compiling existing data-sets and web-text extracted from different sources as detailed in this section. As shown in Table \ref{Table_Morpho_CabernetSub}, genres sources are evenly divided ($\sim$120 million words each) into spoken, fiction, magazine, newspaper, academic to achieve genre-balanced between oral and written modality in newspapers or popular written style, technical reports and Wikipedia entries, fiction, literature or academic production).
%(cf. Metadata)
%Encompassing five different genres and registers : xxx

\paragraph{\Cabernet Oral} \label{subsec:DescribeCaBeRnetOral}
The oral sub-portion gathers both oral transcriptions (\textsc{ORFEO} and Rhapsodie\footnote{\textsc{ORFEO} corpus available at \url{www.cocoon.huma-num.fr/exist/crdo/} ; Rhapsodie corpus at \url{www.projet-rhapsodie.fr}.}) and Films subtitles (Open Subtitles.org), pruned from diacritics, interlocutors tagging and time stamps. To these transcriptions, the French European Parliament Proceedings (1996-2011), as presented in \citet{koehn-2005-europarl}, contributed a sample of more complex oral style with longer sentences and richer vocabulary.%\url{www.opensubtitles.org/fr}

\paragraph{\Cabernet Popular Press} \label{subsec:DescribeCaBeRnetPop}
The whole sub-portion of Popular Press is gathered from an open data-set from the \textit{Est  Républicain} (1999, 2002 and 2003), a regional press format\footnote{Corpus available at \url{www.cnrtl.fr/corpus/estrepublicain/}.}. %Ce corpus est constitué des données textuelles correspondant à deux années de toutes les éditions intégrales du quotidien régional.
It was selected to match popular style as it is characterized by easy-to-read press style and a wide range of every-day topics characterizing local regional press.

\paragraph{\Cabernet Fiction \& Literature} \label{subsec:DescribeCaBeRnetFic}
The Fiction \& Literature sub-portion was compiled from march 2019's Wiki Source and WikiBooks dump and extracted using WikiExtractor.py, a script that extracts and cleans text from a WikiMedia database dumps, by performing template expansion and preprocessing of template definitions.\footnote{Script available at \url{https://github.com/attardi/wikiextractor}.}

\paragraph{\Cabernet News} \label{subsec:DescribeCaBeRnetNews}
The News sub-portion builds upon web crawled elements, including Wikimedia's NewsComments and WikiNews reports from may 2019 WikiMedia dump, collected with a custom version of WikiExtractor.py.
Newspaper's content gathered by the Chambers-Rostand Corpus (i.e. Le Monde 2002-2003, La Dépèche 2002-2003, L'Humanité 2002-2003) and \textit{Le Monde diplomatique} open-source corpus were assembled to represent a higher register of written news style from different political and thematic horizons.
Several months of French Press Agency reports (AFP, 2007-2011-2012) competed with more simple and telegraphic style the newspaper written sample of the corpus.\footnote{At the time being, this part of \Cabernet corpus is still subject to Licence restrictions. This restricted amount of AFP news reports can reasonably fall in the public domain.}

\paragraph{\Cabernet Academic} \label{subsec:DescribeCaBeRnetAcad}
The academic genre was also built from different sources including technical and educational texts from WikiBooks and Wikipedia dump (prior to 2016) for their thematic variety of highly specialized written production. \textsc{ORFEO} Corpus offered a small sample of academic writings like PHD dissertations and scientific articles encompassing a wide choice of disciplinary topics, and TALN Corpus\footnote{TALN proceedings corpus (about 2 million) builds on a subset of 586 scientific articles (from 2007 to 2013), namely TALN and RECITAL. Available at \url{redac.univ-tlse2.fr/corpus/taln_en.html}.} was included to represent more concise written style characterizing scientific abstracts and proceedings.

\begin{table}[ht]
    \centering\small
        \begin{tabular}{lrrr}                                                                                      \\\toprule
            {\textsc{\Cabernet Sub-set}} & {\textsc{Tokens}} & {\textsc{Unique Forms}} & {\textsc{TTR}} \\\midrule
            Oral                         & 122 864 888       & 291 744                 & 0.0024         \\
            Popular                      & 131 444 017       & 458 521                 & 0.0035         \\
            News                         & 132 708 943       & 462 971                 & 0.0035         \\
            Fiction                      & 198 343 802       & 983 195                 & 0.0050         \\
            Academic                     & 126 431 211       & 1 433 663               & 0.0113         \\
            \textit{Total}               & 711 792 861       & 2 558 513               & 0.0036         \\ \bottomrule
        \end{tabular}
    \caption{\label{Table_Morpho_CabernetSub} Comparison of number of unique forms in the different genres represented by \Cabernet partition. TTR: Type-Token Ration. Lemmatization and tokenization was performed as described in §\ref{sec:CompareCorpora}.}
\end{table}
%%%check !! 
%\begin{table}[htbp]
%\centering
%\scalebox{0.83}{
%\begin{tabular}{lrrr}\\
%\toprule
%\multicolumn{2}{}{French Balanced Reference Corpus - \textbf{\Cabernet}}\\\midrule
%{\sc \textbf{\Cabernet}}   & nb{\sc Tokens}    & nb {\sc Unique Forms}  & Mo \\\midrule
%Oral        &  122,864,888      & 291,744  & 735,4 Mo   \\ 
%Popular     &  131,444,017      & 458,521  & 758,5 Mo   \\  
%News        &  132,708,943      & 462,971  & 797,2 Mo   \\ 
%Fiction     &  198,343,802      & 983,195  & 1 080 Mo    \\
%Academic    &  126,431,211      & 1433663  & 810,8 Mo   \\
%Tot.        &  711,792,861     & 2,558,513 & 4 190 Mo   \\ \bottomrule  
%\end{tabular}}
%\caption{\label{Table_TTR_CabernetSub} Comparison of number of unique forms in the different genres represent by \Cabernet partition into Oral, Popular, News, Fiction and Academic. Mo: Mega Octet. lemmatisation and tokenisation was achieved as described in section \ref{sec:CompareCorpora}}
%\end{table}
%voir si remplaçabl

For all sub-portions of \Cabernet, visual inspection was performed to remove section titles, redundant meta-information linked to publishing schemes of each of the six news editor includes. This was manually achieved by compiling a rich set of regular expressions specific of each textual source to obtain clean plain text as an outcome.


\subsection{French Children Book Test (CBT-fr)}
\label{subsec:DescribeCBT}

The French Children Book Test (CBT-fr) was built upon its original English version, the Children Book Test (CBT) \citet{hill-etal-2016-the}\footnote{This data-set can be found at \url{www.fb.ai/babi/}.}, which consists of books freely available on \url{www.gutenberg.org}{Project Gutenberg}.

Using youth literature and children books guarantees a clear narrative structure, and a large amount of dialogues, which enrich with oral register the literary style of this corpus.
The English version of this corpus was originally built as benchmark data-set to test how well language models capture meaning in context. It contains 108 books, and a vocabulary size of 53,628.

French version of CBT, named CBT-fr, was constructed to guarantee enough linguistic similarities between the collected books in the two languages. 104 freely available books were included. One third of the books were purposely chosen because they were classical translations of English literary classics. % (see Metadata). %www.cabernet-corpus.fr 
Chapter heads, titles, notes and all types of editorial information were removed to obtain a plain narrative text. The effort of keeping proportion, genre, domain, and time as equal as possible yields a multilingual set of comparable corpora with a similar balance and representativeness.

\begin{table}[ht]
    \centering\small
        \begin{tabular}{lr}                                                             \\\toprule
            {\textsc{Children Book Test - fr}}           & { \textsc{words}} \\\midrule
            number of different lemmas                   & 25 139            \\
            total number of forms                        & 95 058            \\
            mean number of forms per lemma               & 3.78              \\
            Number of lemmas having more than one form : & 14 128            \\
            Percentage of lemmas with multiple forms     & 56.20             \\
            \bottomrule
        \end{tabular}
    \caption{\label{Table_DescribeCBTfr} Lexical statistics of French CBT, performed as described in §\ref{sec:CompareCorpora}}
\end{table}

%%%%%%%%%%%%%%%%%%%%%%%%%%%%%%%%%%%%%%%%%%%%%%%%%%%%%%%%%%%%%%%%%%
%%%%%%%%%%%%%%%%%%%%%%      Section 3    %%%%%%%%%%%%%%%%%%%%%%%%%
%%%%%%%%%%%%%%%%%%%%%%%%%%%%%%%%%%%%%%%%%%%%%%%%%%%%%%%%%%%%%%%%%%
\section{Corpora Descriptive Comparison} \label{sec:CompareCorpora}

We used two different tokenizers: SEM, Segmenteur-Étiqueteur Markovien standalone \citet{dupont-2017-exploration} and TreeTagger. Both are based on cascades of regular expressions, and both perform tokenization and sentence splitting.
The first was used for descriptive purposes because it technically allowed to segment and tokenize all corpora including OSCAR (23 billion words). Hence, all corpora were entirely segmented into sentences and tokenized using SEM.

The second tokenization method was run only on 3 million words samples to automatically tag them with TreeTagger into part-of-speech and lemmatize them.\footnote{Based on the tag-set available at \url{https://www.cis.uni-muenchen.de/~schmid/tools/TreeTagger/data/french-tagset.html}.} All corpora were randomly shuffled by sentence to then select samples of 3 million words, to be able to compare them in terms of lexical composition (Type-Token Ratio, see Table \ref{Table_MorphoRich}).
%All corpora were POS-Tagged for descriptive reasons using MElt POS-tagging (\citep{denis2012coupling} and \citep{sagot2016external}). Vocabulary size was evaluated after Lemmatization of each Corpus with MElt tool.

\subsection{Corpora Size and Composition}
%\notemumu{@Eric est-ce que je raporte la moyenne ou la mediane ou les deux ? je me rappelle que dans tes scripts tu as les deux.}

Length of sentences is a simple measure to quantify both sentence syntactic complexity and genre. Hence, the number of sentences reported in Table \ref{Table_nb_Words} shows interesting patterns of distributions across genres, consider the comparison between \Cabernet an Wiki-fr.
In our effort to evaluate the impact of corpora pre-training on ELMo-based contextualized word-embedding, we introduce here our two terms of comparison, namely the crawled corpus OSCAR-fr and the Wikipedia-fr one.
%an average length per sentence of 
%xxx for OSCAR-fr, xxx for frWac, xxx for Wiki-fr, xxx for \Cabernet and xx for CBT-fr.
%TODO

\subsubsection{OSCAR fr}
As it has been shown that pre-trained language models can be significantly improved by using more data \citep{liu-etal-2019-roberta,raffel-etal-2020-exploring}, we decided to include in our comparison a corpus of French text extracted from Common Crawl\footnote{More information available  at \url{https://commoncrawl.org/about/}.}. We leverage on a recently published corpus, OSCAR \citep{ortiz-suarez-etal-2019-asynchronous}, which offers a pre-classified and pre-filtered version of the November 2018 Common Craw snapshot.

OSCAR gathers a set of monolingual text extracted from Common Crawl - in plain text \emph{WET} format - where all HTML tags are removed and all text encodings are converted to UTF-8. It follows a similar approach to \citep{grave-etal-2018-learning} by using a language classification model based on the fastText linear classifier \citep{joulin-etal-2016-fasttext,joulin-etal-2017-bag} pre-trained on Wikipedia, Tatoeba and SETimes, supporting 176 different languages.

After language classification, a deduplication step is performed without introducing a specialized filtering scheme: paragraphs containing 100 or more UTF-8 encoded characters are kept. This makes OSCAR an example of unfiltered data that is nearly as noisy as to the original Crawled data.%\footnote{We did not use CCNet because of its difficult availability and download.}

%\subsubsection{frWac} %on l'enlève !
%The frWaC corpus is a French text corpus collected from the .fr domain with using medium-frequency words from the Le Monde Diplomatique corpus and basic French vocabulary lists as seeds. The corpus consists of French websites with total size 1.3 billion words (xxx).%add to biblio

%%%%%%%%%%%%%%%%%%%%%%%%%%%%%%%%%%%%%%%%%%%%%%%%%
\subsubsection{FrWIKI}
This corpus collects a selection of pages from Wikipedia-fr from a dump executed in April 2019, where HTML tags and tables were removed, together with template expansion using Attardi's tool (WikiExtractor, §\ref{subsec:DescribeCaBeRnetFic}). As reported on Table \ref{Table_nb_Words}, in this data-set (660 million words) sentences are relatively longer compared to other corpora. It has the advantage of having a comparable size to \Cabernet, but its homogeneity in terms of written genre is set to Wikipedia entries descriptive style.

\begin{table}[ht]
    \centering
        \begin{tabular}{lrrr}                                                                                 \\\toprule
            {\textsc{corpus}} & { \textsc{wordforms}} & { \textsc{tokens}} & { \textsc{sentences}} \\\midrule
            OSCAR-fr          & 23 212 459 287        & 27 439 082 933     & 1 003 261 066         \\
            Wiki-fr           & 665 599 545           & 802 283 130        & 21 775 351            \\
            \Cabernet         & 697 119 013           & 830 894 133        & 54 216 010            \\
            CBT-fr            & 5 697 584             & 6 910 201          & 317 239               \\\bottomrule
            %frWac       &  1,357,598,417  & 1,622,619,337  &  57,236,199  \\  
        \end{tabular}
    \caption{\label{Table_nb_Words} Comparing the corpora under study.}
\end{table}
%%%%%%%%%%%%%%%%%%%%%%%%%%%%%%%%%%%%%%%%%%%%%%%%%%%%%%%%%%%%%%%%
%%%%%%%%%%%%%%%%%%%%%%%%%%%%%%%%%%%%%%%%%%%%%%%%%%%%%%%%%%%%%%%%
\subsection{Corpora Lexical Variety}

Focusing on a useful measure of complexity that documents lexical richness or variety in vocabulary, we present the type-token ration (TTR) of the corpora under analysis. Generally used to asses language use aspects like the variety of different words used to communicate by learners or children, it represents the total number of unique words (types/forms) divided by the total number of tokens in a given sample of language production. Hence, the closer the TTR ratio is to 1, the greater the lexical richness of the corpus. Table \ref{Table_Morpho_CabernetSub} summarizes the lexical variety of the five sub-portions of \Cabernet, respectively taken as representative of Oral, Popular, Fiction, News, and Academic genres. Domain diversity of texts can be observed in the lexical statistics showing a gradual increase in the number of distinct lexical forms (cf. TTR). This pattern  reflects a generally acknowledged distributional pattern of vocabulary-size across genres. Oral style shows a poorer lexical variety compared to newspapers/magazines’ textual typology. The lexically rich fictional/classic literature is outreached by academic writing-style with its wide-ranging specialized vocabulary. All in all, Table \ref{Table_Morpho_CabernetSub} quantitatively demonstrates that the selected textual and oral materials are indeed representative of the five types of genres of CaBeRnet.


%\textbf{DEFINITION :} The Type Token Ratio (TTR). The TTR is the number of Types divided by the number of Tokens. The closer the TTR is to 1 the more lexical variety there is. Enter Henry's TTR for his written sample in Table 1 below.


%%%%%%%%%%%%%%%%%%%%%%%%%%%%%%%%%%%%%%%%%%%%%%%%%%%%%%%%%%%%%%%%%%
%%%%%%%%%%%%%%%%%%%%%%%%%%%%%%%%%%%%%%%%%%%%%%%%%%%%%%%%%%%%%%%%%%
\subsection{Corpora Morphological richness}

To select a measure that would help quantifying the different corpora morphological richness, we follow \citep{bonami-etal-2015-implicative}. Hence, the proportion of lemmas with multiple forms in a given vocabulary size was evaluated on randomly selected samples of 3-million-words from each corpus under analysis (see Table \ref{Table_MorphoRich}).
%Distribution of the lemmas in function of the percentage of their full flexion . 

%\notemumu{@Benoit : je suis pas sûre de bien ex-primer ici le contenu de nos discussions sur le sujet}

\begin{table}[ht]
    \centering
        \begin{tabular}{lrrrr}
            \toprule
            \textsc{3 M samples}  & \textsc{CBT-fr} & \textsc{\Cabernet} & \textsc{Fr-Wiki} & \textsc{OSCAR} \\
            \midrule
            nb of diff. lemmas    & 25 139          & 30 488             & 31 385           & 31 204         \\
            tot. nb forms         & 95 058          & 180 089            & 238 121          & 190 078        \\
            mean nb forms/lemma   & 3.78            & 6.19               & 7.85             & 6.40           \\
            nb lemmas $>$ 1 form  & 14 128          & 15 927             & 15 182           & 16 480         \\
            \% lemmas  $>$ 1 form & 56.20           & 52.24              & 48.37            & 52.81          \\
            \bottomrule
        \end{tabular}
    \caption{Lexical statistics on morphological richness over randomly selected samples of 3 million words from each corpus. nb : number}
    \label{Table_MorphoRich}
\end{table}

Table 4 reports some more in-depth lexical and morphological statistics across corpora. Although OSCAR is 34 times bigger than CaBeRnet, their total number of forms and the proportion of lemmas having more than one form in a 3-million-word sample are comparable. FrWiki shows a radically different lexical distribution with numerous hapaxes but a lower morphological richness. Although its total number of forms is more than one third higher than in OSCAR and CaBeRnet samples, the proportion of lemmas having more than one distinct form is around four points below CaBeRnet and OSCAR. Comparatively, youth literature in CBT-fr shows the greatest morphological richness, around 56\% of lemmas have more than one form.

%\begin{table}[ht]
%\centering\small
%\scalebox{0.95}{
%\begin{tabular}{lr}\\\toprule

%{\sc CBT-fr 3 m sample}   &   \\\midrule

%%%number of different lemmas    & 25.139   \\ 
%%total number of forms       &  95.058  \\  
%%mean number of forms per lemma     &  3,78   \\ 
%%Number of lemmas having more than one form :   &  14.128  \\
%%Percentage of lemmas with multiple forms  &  56,20 %\\\midrule

% CBT-fr
%number of forms: 70230
%number of lemmas: 22123
%mean number of forms per lemma: 74819 3.381955430999412
%proportion of lemmas with multiple forms: 11808 0.5337431632237942

%%{\sc CaBeRnetFRanc 3 m sample}   &   \\\midrule
%%number of different lemmas          & 30 488   \\ 
%%total number of forms               &  180.089  \\  
%%mean number of forms per lemma      &  6,19   \\ %188.675
%%Number of lemmas having more than one form :   &  15.927  \\
%%Percentage of lemmas with multiple forms  & 52,24 %\\\midrule

%CaBeRnetFRanc.txt
%number of forms: 
%number of lemmas: 30488
%mean number of forms per lemma: 188675 6.188500393597481
%proportion of lemmas with multiple forms: 15927 0.5224 022566255576

%{\sc frWAC 3 m sample}   &   \\\midrule
%number of different lemmas    & 30.892   \\ 
%total number of forms       &  194.562  \\  
%mean number of forms per lemma     &  6,62 \\ 
%Number of lemmas having more than one form :   &  16.197  \\
%Proportion of lemmas with multiple forms  &  52,43 \\\midrule
% frwac
%number of forms: 194562
%number of lemmas: 30892
%mean number of forms per lemma: 204400 6.61659976692995
%proportion of lemmas with multiple forms: 16197 0.5243 105011006086

%%{\sc frwiki 3 m sample}   &   \\\midrule
%%number of different lemmas    & 31.385   \\ 
%%total number of forms       &  238.121  \\  
%%mean number of forms per lemma     &  7,85   \\ 
%Number of lemmas having more than one form :   &  15.182  \\
%%Percentage of lemmas with multiple forms  &  48,37 %\\\midrule
% frwiki
%number of forms: 238121
%number of lemmas: 31385
%mean number of forms per lemma: 246418 7.851457702724232
%proportion of lemmas with multiple forms: 15182 0.4837 3426796240243

%{\sc OSCAR 3 m sample}   &   \\\midrule
%number of different lemmas    & 31.204   \\ 
%%total number of forms       &  190.078  \\  
%%mean number of forms per lemma     &  6,40   \\ 
%Number of lemmas having more than one form :   &  16.480  \\
%%Percentage of lemmas with multiple forms  &  52,81 %\\
% OSCAR
%number of forms: 190078
%number of lemmas: 31204
%mean number of forms per lemma: 199589 6.396263299576978
%proportion of lemmas with multiple forms: 16480 0.5281 374182797077
%\bottomrule  
%\end{tabular}}
%\caption{\label{Table_MorphoRich} Lexical Statistics comparing morphological richness of the corpora under study.}
%\end{table}

