\documentclass{mimosis}

\usepackage{metalogo}

%%%%%%%%%%%%%%%%%%%%%%%%%%%%%%%%%%%%%%%%%%%%%%%%%%%%%%%%%%%%%%%%%%%%%%%%
% Some of my favourite personal adjustments
%%%%%%%%%%%%%%%%%%%%%%%%%%%%%%%%%%%%%%%%%%%%%%%%%%%%%%%%%%%%%%%%%%%%%%%%
%
% These are the adjustments that I consider necessary for typesetting
% a nice thesis. However, they are *not* included in the template, as
% I do not want to force you to use them.

% This ensures that I am able to typeset bold font in table while still aligning the numbers
% correctly.
\usepackage{etoolbox}

\usepackage{siunitx}
\DeclareSIUnit\px{px}

\sisetup{%
  detect-all           = true,
  detect-family        = true,
  detect-mode          = true,
  detect-shape         = true,
  detect-weight        = true,
}

%%%%%%%%%%%%%%%%%%%%%%%%%%%%%%%%%%%%%%%%%%%%%%%%%%%%%%%%%%%%%%%%%%%%%%%%
% Hyperlinks & bookmarks
%%%%%%%%%%%%%%%%%%%%%%%%%%%%%%%%%%%%%%%%%%%%%%%%%%%%%%%%%%%%%%%%%%%%%%%%

\usepackage[%
  unicode,
  colorlinks = true,
  citecolor  = RoyalBlue,
  linkcolor  = RoyalBlue,
  urlcolor   = RoyalBlue,
  unicode,
  ]{hyperref}
\hypersetup{
  pdftitle=A Data-driven Approach to Natural Language Processing for Contemporary and Historical French,
  pdfauthor=Pedro Ortiz Suarez
}

\usepackage{bookmark}



%%%%%%%%%%%%%%%%%%%%%%%%%%%%%%%%%%%%%%%%%%%%%%%%%%%%%%%%%%%%%%%%%%%%%%%%
% Fonts
%%%%%%%%%%%%%%%%%%%%%%%%%%%%%%%%%%%%%%%%%%%%%%%%%%%%%%%%%%%%%%%%%%%%%%%%

\ifxetexorluatex
  \setmainfont[Ligatures=TeX]{TeX Gyre Pagella} % Palatino clone
  \linespread{1.05} % a bit more for Palatino
  \RequirePackage[warnings-off={mathtools-colon,mathtools-overbracket}]{unicode-math}
  \setmathfont{TeX Gyre Pagella Math}
\else
  \usepackage[lf]{ebgaramond}
  \usepackage[oldstyle,scale=0.7]{sourcecodepro}
  \singlespacing
\fi

\usepackage{microtype}

\renewcommand{\th}{\textsuperscript{\textup{th}}\xspace}

% \newacronym[description={Principal component analysis}]{PCA}{PCA}{principal component analysis}
% \newacronym                                            {SNF}{SNF}{Smith normal form}
% \newacronym[description={Topological data analysis}]   {TDA}{TDA}{topological data analysis}

% \newglossaryentry{LaTeX}{%
%   name        = {\LaTeX},
%   description = {A document preparation system},
%   sort        = {LaTeX},
% }

% \newglossaryentry{Real numbers}{%
%   name        = {$\real$},
%   description = {The set of real numbers},
%   sort        = {Real numbers},
% }

%\makeindex
%\makeglossaries

%%%%%%%%%%%%%%%%%%%%%%%%%%%%%%%%%%%%%%%%%%%%%%%%%%%%%%%%%%%%%%%%%%%%%%%%
% Custom Commands and Environments
%%%%%%%%%%%%%%%%%%%%%%%%%%%%%%%%%%%%%%%%%%%%%%%%%%%%%%%%%%%%%%%%%%%%%%%%

%%%% License %%%%
\usepackage[type={CC}, modifier={by}, version={4.0}]{doclicense}

\usepackage{arydshln}
\usepackage{fontawesome5}
\usepackage{longtable}
\usepackage{tikz}
\usepackage{tabu}
\usepackage{tabularx}
\usepackage{rotating}
\newcolumntype{x}[1]{>{\centering\arraybackslash\hspace{0pt}}p{#1}}
\usepackage{colortbl}
\definecolor{Gray}{gray}{0.925}
\usetikzlibrary{positioning}

\usepackage{siunitx}
    \sisetup{
        detect-all,
        group-separator=\text{\,},
    }
	\DeclareSIUnit{\quantity}{\relax}
	\DeclareSIUnit{\words}{words}
	\DeclareSIUnit{\sentences}{sentences}

  \usepackage[keeplayout]{covington}  % Gloses interlinéaires
  \setglossoptions{
      %fsi=\small,
      fsii=\small,
      fsiii=\small,
  }

\usepackage{cleveref}

\usepackage{tikz-dependency}

\usepackage{pgfplots}
    \pgfplotsset{compat=1.17}
    \usetikzlibrary{backgrounds}


\usepackage{stackengine}
\def\ucr{\scalebox{1}{\stackinset{c}{}{c}{-.1pt}{%
  \textcolor{white}{\sffamily\bfseries\small ?}}{%
  \rotatebox{45}{$\blacksquare$}}}}

\let\tablefontsize\normalsize

\newcommand{\camembert}{CamemBERT\xspace}
\newcommand{\camembertoscar}{CamemBERT\textsubscript{OSCAR}\xspace}
\newcommand{\camembertccnet}{CamemBERT\textsubscript{CCNet}\xspace}
\newcommand{\roberta}{RoBERTa\xspace}
\newcommand{\bert}{BERT\xspace}
\newcommand{\mbert}{mBERT\xspace}
\newcommand{\ccnet}{CCNet\xspace}
\newcommand{\Cabernet}{CaBeRnet\xspace}


\newcolumntype{P}[1]{>{\RaggedRight\hspace{0pt}}p{#1}}

\newcommand{\ELMooscar}{ELMo\textsubscript{OSCAR}\xspace}
\newcommand{\ELMowiki}{ELMo\textsubscript{Wikipedia}\xspace}
\newcommand{\ELMococa}{ELMo\textsubscript{\Cabernet}\xspace}
\newcommand{\ELMocaber}{ELMo\textsubscript{\Cabernet}\xspace}


\newcommand{\ELMocbt}{ELMo\textsubscript{CBT}\xspace}
\newcommand{\ELMocoscar}{ELMo\textsubscript{OSCAR+\Cabernet}\xspace}
\newcommand{\ELMocabercar}{ELMo\textsubscript{OSCAR+\Cabernet}\xspace}

%%%%%%%%%%%%%%%%
\newcommand{\camemberts}{CamemBERTs\xspace}
\newcommand{\camembertwiki}{CamemBERT\textsuperscript{4GB}\hspace{-1.2em}\textsubscript{Wikipedia}\xspace}
\newcommand{\camembertccnetmini}{CamemBERT\textsuperscript{4GB}\hspace{-1.2em}\textsubscript{CCNet}\xspace}
\newcommand{\camembertoscarmini}{CamemBERT\textsuperscript{4GB}\xspace}
%\newcommand{\camembertbase}{CamemBERT\textsubscript{BASE}\xspace}
\newcommand{\camembertccnetlarge}{CamemBERT\textsubscript{LARGE}\xspace}
\newcommand{\camembertlarge}{CamemBERT\textsubscript{LARGE}\xspace}
\newcommand{\camembertccnetlong}{CamemBERT\textsubscript{500k / CCNet}\xspace}
\newcommand{\camembertoscarlong}{CamemBERT\textsuperscript{500k}\hspace{-1.4em}\textsubscript{OSCAR}\xspace}
\newcommand{\camembertoscarswm}{CamemBERT\textsubscript{\it subword}\xspace}
\newcommand{\camembertccnetswm}{CamemBERT\textsubscript{CCNet-{\it subword}}\xspace}
%XLM-R\textsubscript{BASE} \cite{conneau2019xlmr} & 80.1 & 270M \\
\newcommand{\bertbase}{BERT\textsubscript{BASE}\xspace}
\newcommand{\bertlarge}{BERT\textsubscript{LARGE}\xspace}

\newcommand{\lm}[1]{{\color{red}\textbf{LM}: #1}}
\newcommand{\bm}[1]{{\color{blue}\textbf{BM}: #1}}
\newcommand{\po}[1]{{\color{orange}\textbf{PO}: #1}}
\newcommand{\bs}[1]{{\color{PineGreen}\textbf{BS}: #1}}
\newcommand{\ds}[1]{{\color{purple}\textbf{DS}: #1}}

\newcommand{\oscar}{OSCAR\xspace}
\newcommand{\xlmEnFr}{XLM\textsubscript{EN-FR}\xspace}
\newcommand{\xlmmultimlm}{XLM\textsubscript{17-MLM-1280}\xspace}
\newcommand{\xlmmlmtlm}{XLM\textsubscript{MLM-TLM}\xspace}


\newcommand{\secref}[1]{\StrSubstitute{\getrefnumber{#1}}{.}{ }}
\newcommand{\elmooscar}{ELMo\textsubscript{OSCAR}\xspace}
\newcommand{\elmooscars}{ELMo\textsubscript{OSCAR}'s\xspace}
\newcommand{\elmowiki}{ELMo\textsubscript{Wikipedia}\xspace}
\newcommand{\elmowikis}{ELMo\textsubscript{Wikipedia}'s\xspace}
\newcommand{\elmooscarone}{ELMo\textsubscript{OSCAR(1)}\xspace}
\newcommand{\elmowikione}{ELMo\textsubscript{Wikipedia(1)}\xspace}
\newcommand{\elmooscarthree}{ELMo\textsubscript{OSCAR(3)}\xspace}
\newcommand{\elmowikithree}{ELMo\textsubscript{Wikipedia(3)}\xspace}
\newcommand{\elmooscarfive}{ELMo\textsubscript{OSCAR(5)}\xspace}
\newcommand{\elmowikifive}{ELMo\textsubscript{Wikipedia(5)}\xspace}
\newcommand{\elmooscarten}{ELMo\textsubscript{OSCAR(10)}\xspace}
\newcommand{\elmowikiten}{ELMo\textsubscript{Wikipedia(10)}\xspace}

\def\aleda{\textsf{Aleda}\xspace}

\newcommand{\dalembert}{D'AlemBERT\xspace}
\newcommand{\freemmax}{\textsc{FreEM}\textsubscript{\emph{max}}\xspace}
\newcommand{\freemlpm}{\textsc{FreEM}\textsubscript{\emph{LPM}}\xspace}
\newcommand{\freemner}{\textsc{FreEM}\textsubscript{\emph{NER}}\xspace}
\newcommand{\pieextended}{Pie Extended\xspace}

\newcommand{\goclassy}{\texttt{goclassy}\xspace}

\newcommand{\teal}[1]{\textcolor{teal}{#1}}
\newcommand{\purp}[1]{\textcolor{purple}{#1}}
\newcommand{\ora}[1]{\textcolor{orange}{#1}}

%%%%%%%%%%%%%%%%%%%%%%%%%%%%%%%%%%%%%%%%%%%%%%%%%%%%%%%%%%%%%%%%%%%%%%%%
% Incipit
%%%%%%%%%%%%%%%%%%%%%%%%%%%%%%%%%%%%%%%%%%%%%%%%%%%%%%%%%%%%%%%%%%%%%%%%

\title{A Data-driven Approach to Natural Language Processing for Contemporary and Historical French}
%\subtitle{A minimal, modern \LaTeX{} package for typesetting your thesis}
\author{Pedro Ortiz Suarez}

\begin{document}

\frontmatter
\selectlanguage{french}

\begin{titlepage}

	\begin{center}
		\begin{minipage}[t]{0.4\textwidth}
			\includegraphics[height=2cm]{static/media/sorbonne}
		\end{minipage}%
		\hfill
		\begin{minipage}[t]{0.4\textwidth}
			\hfill
			\includegraphics[height=2cm]{static/media/inria}
		\end{minipage}

		\vspace{0.2cm}
		\LARGE \textsc{Sorbonne Université}\\
		\vspace{0.2cm}
		\normalsize \textsc{Ecole Doctorale Informatique, Télécommunications et Electronique} - ED130\\
		\vspace{0.2cm}
		\textsc{Inria de Paris / Équipe ALMAnaCH}\\

		\vspace{0.4cm}
		\Large \textsc{Thèse de doctorat}\\
		\normalsize Discipline : Informatique\\
		\vspace{0.4cm}
		\normalsize Présentée par \\
		\LARGE \textbf{Pedro \textsc{Ortiz Suarez}}\\
		\vspace{0.4cm}
		\normalsize Dirigée par \\
		\Large \textbf{Laurent \textsc{Romary}} et \textbf{Benoît \textsc{Sagot}}\\
		\vspace{0.4cm}
		\normalsize Pour obtenir le grade universitaire de\\
		\Large \textsc{Docteur} de \textsc{Sorbonne Université}

		\hrulefill\\[0.2cm]

		{\Large  \textbf{On Language Modeling and its Applications for Contemporary and Historical French}}\\[0.1cm]

		\hrulefill\\

		\vspace{0.cm}
		\normalsize Présentée et soutenue publiquement le 30 avril 2022 devant le jury composé de :\\
		\vspace{0.4cm}
		\begin{tabular*}{\linewidth}{@{\extracolsep{\fill}}l c r}
			Francis \textsc{Bach} & Inria - SIERRA & Examinateur \\
			Alexander \textsc{Geyken} & Berlin Academy & Examinateur \\
			Sebastian \textsc{Padó} & University of Stuttgart & Examinateur \\
			Barbara \textsc{Plank} & IT University of Copenhagen & Rapporteur \\
			Laurent \textsc{Romary} & Inria - ALMAnaCH & Directeur\\
			Benoît \textsc{Sagot} & Inria - ALMAnaCH & Directeur\\
			Holger \textsc{Schwenk} & Facebook AI Research & Rapporteur \\
			Achim \textsc{Stein} & University of Stuttgart & Examinateur \\
		\end{tabular*}
	\end{center}

\end{titlepage}

\selectlanguage{english}

\newpage
\null
\thispagestyle{empty}
\newpage

\begin{otherlanguage}{french}
    \begin{center}
        {\huge Une approche basée sur les données pour le traitement automatique du langage naturel en français contemporain et historique}
    \end{center}

    \section*{Résumé}
    \addcontentsline{toc}{section}{Résumé}

    Depuis plusieurs années, les approches neuronales ont régulièrement amélioré l'état de l'art du traitement automatique des langues (TAL) sur une grande variété de tâches. L'un des principaux facteurs ayant permis ces progrès continus est l'utilisation de techniques d'apprentissage par transfert. Ces méthodes consistent à partir d'un modèle pré-entraîné et à le réutiliser, avec peu ou pas d'entraînement supplémentaire, pour traiter d'autres tâches. Même si ces modèles présentent des avantages évidents, leur principal inconvénient est la quantité de données nécessaire pour les pré-entraîner. Ainsi, le manque de données disponibles à grande échelle a freiné le développement de tels modèles pour le français contemporain et a fortiori pour ses états de langue plus anciens.

    Cette thèse met l'accent sur le développement de corpus pour le pré-entraînement de telles architectures. Cette approche s'avère extrêmement efficace car nous sommes en mesure d'améliorer l'état de l'art pour un large éventail de tâches de TAL pour le français contemporain et historique, ainsi que pour six autres langues contemporaines. De plus, nous montrons que ces modèles sont extrêmement sensibles à la qualité, à l'hétérogénéité et à l'équilibre des données de pré-entraînement et montrons que ces trois caractéristiques sont de meilleurs prédicteurs de la performance des modèles que la taille des données de pré-entraînement. Nous montrons également que l'importance de la taille des données de pré-entraînement a été surestimée en démontrant à plusieurs reprises que l'on peut pré-entraîner de tels modèles avec des corpus de taille assez modeste.

    \vspace{1cm}
    \textbf{Mots-clés :} modèle de langue, corpus de pré-entraînement, traitement automatique des langues, français contemporain, français historique, apprentissage par transfert.
\end{otherlanguage}

\pagebreak

\begin{center}
    {\huge A Data-driven Approach to Natural Language Processing for Contemporary and Historical French}
\end{center}

\section*{Abstract}
\addcontentsline{toc}{section}{Abstract}

In recent years, neural methods for Natural Language Processing (NLP) have consistently and repeatedly improved the state of the art in a wide variety of NLP tasks. One of the main contributing reasons for this steady improvement is the increased use of transfer learning techniques. These methods consist in taking a pre-trained model and reusing it, with little to no further training, to solve other tasks. Even though these models have clear advantages, their main drawback is the amount of data that is needed to pre-train them. The lack of availability of large-scale data previously hindered the development of such models for contemporary French, and even more so for its historical states.

In this thesis, we focus on developing corpora for the pre-training of these transfer learning architectures. This approach proves to be extremely effective, as we are able to establish a new state of the art for a wide range of tasks in NLP for contemporary, medieval and early modern French as well as for six other contemporary languages. Furthermore, we are able to determine, not only that these models are extremely sensitive to pre-training data quality, heterogeneity and balance, but we also show that these three features are better predictors of the pre-trained models' performance in downstream tasks than the pre-training data size itself. In fact, we determine that the importance of the pre-training dataset size was largely overestimated, as we are able to repeatedly show that such models can be pre-trained with corpora of a modest size.

\vspace{1cm}
\textbf{Keywords:} language model, pre-training corpora, natural language processing, contemporary french, historical french, transfer learning.

\pagebreak

\section*{Acknowledgments}
\addcontentsline{toc}{section}{Acknowledgments}
First and foremost I am extremely grateful to my supervisors Laurent Romary and Benoît Sagot, who at the beginning of my Ph.D. accepted to supervise me even though I was coming from a different background and had absolutely no experience in Natural Language Processing (NLP) or Digital Humanities (DH). You both had the patience to teach me and accompany me throughout this Ph.D. journey, supporting me not only academically but also personally; you're both my mentors and for that I am extremely grateful. To Benoît, thank you so much for always being there during all the difficult deadlines ready to correct and advice. To Laurent, thank you so much for all your invaluable advise which undeniably helped go through the difficult situations that I encountered during this Ph.D.

I would like to thank all the members of the committee for agreeing to evaluate this thesis and its defense, for their comments, advice and questions which have greatly contributed to the improvement and finalization of this work. Special thanks to Anna Korhonen and Holger Schwenk, for accepting to read my manuscript in such a short span of time and for the helpful comments in your pre-reports. I would also like to thank the members of my \emph{comité de suivi de thèse}, for their support and their advice for the end of the thesis.

I would also like to thank the ANR BASNUM project (ANR-18-CE38-0003) for founding my Ph.D. thesis, as well as its members Geoffrey Williams, Ioana Galleron, Mohamed Khemakhem and Clarissa Stincone, with whom I had the pleasure of collaborating at the beginning of this Ph.D. thesis.

This thesis is not only the product of my work, but the fruit of numerous collaborations without which it would have never been possible. For my collaborators and coauthors, I would like to thank all the members of the \emph{CamemBERT} project, specially to Louis and Benjamin. We managed to publish a valuable and influential resource for French NLP and even for NLP in languages other than English. I will always be grateful to have participated in this project.

I also want to mention Rachel, Philippe and Alexandre from the \emph{Early Modern French} project; Tian Tian and Gaël Lejeune from the \emph{CLEF-HIPE2020} shared task; and Benoît Crabbé from the \emph{BERTrade} project. You taught me many things during our time collaborating together, and I cannot say enough how lucky I was to have the opportunity to work with you and how grateful I am to all of you.

To Simon Gabay, with whom I collaborated during the last part of my Ph.D. thesis, I'd like to thank for providing me with so much high quality data. Most of the experiments at the end of this manuscript would have never been possible without your work, so for this you have my whole gratitude.

To all the people at Masakhane and all the coauthors of the \emph{Quality at Glance} paper, specially to Julia and Isaac; I truly have no words to express how lucky I was to work with such an amazing group of people, you taught me about the importance of looking at my own data, and you also helped me restart the OSCAR project. Thank you so much for working with me and expressing interest in my corpus.

As the OSCAR project became larger, I no longer had the means to continue working on it alone, this is why I would like to express my deepest gratitude to Julien. Thank you so much for your patience, for taking the time to go through my messy code and making it readable, for managing our users while I was away and for taking care of OSCAR like it was your own. I only have gratitude and respect for you, you're truly one of the greatest engineers that I know, and even though we have sadly never met in person, know that you became a valuable friend and colleague to me during this last year.

To all my other colleagues at the ALMAnaCH team that I have worked with such as Djamé, Eric, Syrielle, Kim and Meriem; and to the newcomers such as Hugo, Rua and Roman. Thank you for making of Inria such a nice place to work at.

At the beginning of this Ph.D. thesis I was just a mathematician that did not have any clue about NLP or Machine Learning. However, I encountered a group of people that believed in me and supported me in working at Inria and starting this Ph.D. thesis in NLP. For this I would like to thank in particular Patrice, Alba and Professor Isis Truck.

While this Ph.D. thesis was an academic exercise, I cannot overstate how valuable was the personal support that the friends I made here in Paris gave me during the last four years. You made me feel at home in France and you actually helped me become one of your own; you accompanied me during very turbulent personal times and even through the pandemic. Some of you even became coauthors and collaborators. Thank you so much Yoann and Ariane, Alix and Jen, Clementine, Loïc, Murielle, Manon, Emilia, Ganesh, Tanti and Arij. I do not have words to truly express to you how thankful I am.

To all my Colombian friends that have always supported me, and specially to Zubieta, Juan Rafael, Pedro Pablo, Isabel, Susana, Henao, Pablo, Andrés, Maritza Carlos and Min. Thank you so much for sending me all your support from afar.

To my family, specially to my father, my grandmother, to Elena and Gloria and to my little Family in the US, Fercho and Beatriz. Thank you so much for all your support and your love, in spite of the distance and the pandemic keeping us apart for years, you were always there when I needed it the most. Thank you so much for always believing in me.

Finally, I'd like to thank Mathilde and her family, Paul, Anne, Thierry and Anne-Marie. Since the beginning you always welcomed me with open arms, and you have supported and accompanied me during the last two years. I have no words to express how grateful I am for everything that you have done for me and for the hospitality that you have always offered me. To Mathilde, you have all my gratitude for your constant support and specially for putting up with me during the last days of this Ph.D. thesis, I know it was difficult, but I do believe good times are ahead of us.


\tableofcontents

\cleardoublepage
\begin{flushright}
    \thispagestyle{empty}
	\vspace*{\stretch{1}}
	{\itshape
		To both my grandmother and my mother.
	}
    \vspace*{\stretch{2}}
\end{flushright}

\cleardoublepage

\mainmatter

\part{Introduction and Related Work}
%%%%%%%%%%%%%%%%%%%%%%%%%%%%%%%%%%%%%%%%%%%%%%%%%%%%%%%%%%%%%%%%%%%%%%%%
\chapter{Introduction}
%%%%%%%%%%%%%%%%%%%%%%%%%%%%%%%%%%%%%%%%%%%%%%%%%%%%%%%%%%%%%%%%%%%%%%%%

\begin{center}
  \begin{minipage}{0.5\textwidth}
    \begin{small}
      In which the reasons for doing this Ph.D. are laid bare for the whole world to see and we encounter some answers to questions in which, frankly, only an extremely small number of people were interested in the first place.
    \end{small}
  \end{minipage}
  \vspace{0.5cm}
\end{center}

\noindent This package contains a minimal, modern template for writing your
thesis. While originally meant to be used for a Ph.\,D.\ thesis, you can
equally well use it for your honour thesis, bachelor thesis, and so
on---some adjustments may be necessary, though.

\section{The BASNUM Project}

Cette thèse s'inscrit dans le cadre du projet ANR BASNUM (ANR-18-CE38-0003), qui avait comme objectif principal de numériser le Dictionnaire Universel (DU) d'Antoine Furetière, dans sa version de 1701 revue et corrigée par Basnage de Beauval \citep{furetiere-1701-dictionnaire}, et de l'analyser avec des outils numériques, afin de faire apparaître l'importance de cet ouvrage pour l'évolution des sciences et des mentalités au XVIIIe siècle. Le projet visait également à contribuer au mouvement actuel de conception de méthodes innovantes de numérisation, encodage et analyse des textes.

D'un point vue purement informatique le projet BASNUM comptait réaliser deux types de tâches :
\begin{enumerate}
  \item une première tâche de structuration où la macrostructure du dictionnaire serait annoté;
  \item une deuxième tâche d'enrichissement qui consistait à réaliser un large éventail de tâches d'extraction d'informations, d'annotation de la microstructure du dictionnaire et même de normalisation et modernisation du texte.
\end{enumerate}

La première tâche de structuration automatique des dictionnaires était déjà partiellement couverte par les travaux de \citet{khemakhem-etal-2017-automatic,khemakhem-etal-2018-enhancing} qui développent \emph{GROBID-dictionaries}, un sous-module de \emph{GROBID}\footnote{Bibliothèque d'apprentissage automatique pour extraire, analyser et restructurer des documents bruts tels que PDF, en documents structurés et encodés en TEI.} \citep{lopez-etal-2018-grobid} implémentant une bibliothèque d'apprentissage automatique Java pour la structuration de ressources lexicales numérisées au format \emph{TEI} \citep{tei-2018-guidelines}, afin de permettre l'analyse, l'extraction et la structuration d'informations textuelles dans de telles ressources. \emph{GROBID-dictionaries} avait déjà obtenu des résultats et des performances prometteurs \citep{khemakhem-2020-standard}, à tel point qu'il a été utilisé pour faire une première annotation de la macrostructure du Dictionnaire Universel.

Étant donnés les travaux réalisés par \citet{khemakhem-2020-standard}, nous avons décidé de nous concentrer sur la deuxième tâche d'enrichissement des dictionnaires, qui à l'époque restait assez générale et abstraite, notamment par contraste avec la première tâche de structuration. Pour aborder cette tâche, nous avions deux options: soit nous développions de multiples modèles et systèmes d'annotation dédiés à chacune des sous-tâches impliquées dans cet \emph{enrichissement} et ciblant très spécifiquement le Dictionnaire Universel, soit nous développions un seul modèle d'annotation générique capable d'aborder toutes les sous-tâches d'enrichissement et capable de traiter non seulement le Dictionnaire Universel, mais aussi d'autres textes et ressources de l'époque moderne\footnote{Entre le 16e et le 18e siècle.}.

Étant donnée la nature de la tâche d'enrichissement et le fait que de nouveaux modèles de langue neuronaux capables transférer des connaissances entre différentes tâches en traitement automatique du langage naturel (TALN) venaient d'être publiés au début de cette thèse \cite{peters-etal-2018-deep,devlin-etal-2019-bert}, nous avons décidé de nous concentrer sur la deuxième option et, en conséquence, de développer un seul modèle général que nous espérions capable d'être utilisé pour toutes ces tâches d'enrichissement et pour n'importe quel type de document en français moderne ou contemporain.

En choisissant cette approche, nous voulions aussi aborder d'une façon indirecte la première tâche de structuration automatique, parce que nous croyons qu'il était possible d'améliorer les premiers résultats de GROBID-dictionaries en utilisant des nouveaux modèles neuronaux. En effet GROBID-dictionaries reposait sur des modèles CRF (Champ aléatoire conditionnel) \cite{lafferty-etal-2001-conditional} qui étaient largement utilisés pour l'étiquetage et la classification des tokens, mais qui ont été remplacés par ces modèles neuronaux au cours des dernières années \citep{lample-etal-2016-neural,devlin-etal-2019-bert}. En outre, nous savions que les développeurs de GROBID avaient commencé à travailler avec certains de ces modèles neuronaux en écrivant \emph{DeLFT}, une bibliothèque pour le traitement de texte, couvrant l'étiquetage et la classification de tokens. Cette bibliothèque réimplémente les derniers modèles d'apprentissage automatique en TALN \citep{lopez-etal-2018-delft} et vise à améliorer les pipelines de \emph{GROBID}. Ce sont des outils et des idées comme ceux contenus dans \emph{DeLFT} qui pouvaient être appliqués à \emph{GROBID}-Dictionaries pour améliorer et étendre considérablement ses capacités au profit du projet \emph{BASNUM}, surtout en plus des ressources que nous avions décidé de développer.

Ayant choisi de développer ces nouveaux modèles pour le français, tels que \emph{ELMo} \citep{peters-etal-2018-deep} ou \emph{BERT} \cite{devlin-etal-2019-bert}, nous devions commencer par construire et rassembler notre propre corpus pour le pre-entraînement de ces architectures, puisque les corpus en français contemporain librement disponibles à l'époque, tels que Wikipedia ou frWAC \citep{baroni-etal-2009-the}, n'étaient pas considérés comme étant suffisamment grands pour cela \citep{liu-etal-2019-roberta}.

Notre plan était alors de développer un corpus de pre-entraînement en français contemporain, puis de pre-entraîner un modèle de langue pour le français contemporain et enfin d'utiliser les capacités de transfert de connaissances de ces architectures pour l'adapter au français moderne, au cas où nous ne serions pas en mesure de trouver assez de ressources textuelles pour pre-entrainer directement un tel modèle de langue pour le français moderne. Au cours de cette thèse, nous voulions également étudier la question de la quantité minimale de ressources requise pour pre-entraîner avec succès de telles modèles, quantité qui, à l'époque, était considérée comme plus élevée que ce qui est disponible pour les langues historiques \citep{peters-etal-2018-deep,liu-etal-2019-roberta}.

%%%%%%%%%%%%%%%%%%%%%%%%%%%%%%%%%%%%%%%%%%%%%%%%%%%%%%%%%%%%%%%%%%%%%%%%
\section{Goclassy}
%%%%%%%%%%%%%%%%%%%%%%%%%%%%%%%%%%%%%%%%%%%%%%%%%%%%%%%%%%%%%%%%%%%%%%%%

In recent years neural methods for Natural Language Processing (NLP) have consistently and repeatedly improved the state-of-the-art in a wide variety of NLP tasks such as parsing, PoS-tagging, named entity recognition, machine translation, text classification and reading comprehension among others. Probably the main contributing factor in this steady improvement for NLP models is the raise in usage of \emph{transfer learning} techniques in the field. These methods normally consist of taking a pre-trained model and reusing it, with little to no retraining, to solve a different task from the original one it was intended to solve; in other words, one \emph{transfers} the \emph{knowledge} from one task to another.

Most of the transfer learning done in NLP nowadays is done in an unsupervised manner, that is, it normally consists of a  \emph{language model} that is fed unannotated plain text in a particular language; so that it \emph{extracts} or \emph{learns} the basic \emph{features} and patterns of the given language, the model is subsequently used on top of an specialised architecture designed to tackle a particular NLP task. Probably the best known example of this type of model are \emph{word embeddings} which consist of real-valued vector representations that are trained for each word on a given corpus. Some notorious examples of word embeddings are word2vec \citep{mikolov-etal-2013-distributed}, GloVe \citep{pennington-etal-2014-glove} and \mbox{fastText} \citep{mikolov-etal-2018-advances}. All these models are \emph{context-free}, meaning that a given word has one single vector representation that is independent of context, thus for a polysemous word like Washington, one would have one single representation that is reused for the city, the state and the US president.

In order to overcome the problem of polysemy, \emph{contextual} models have recently appeared. Most notably ELMo \citep{peters-etal-2018-deep} which produces deep contextualised word representations out of the internal states of a deep bidirectional language model in order to model word use and how the usage varies across linguistic contexts. ELMo still needs to be used alongside a specialised architecture for each given downstream task, but newer architectures that can be fine-tuned have also appear. For these, the model is first fed unannotated data, and is then fine-tuned with annotated data to a particular downstream task without relying on any other architecture. The most remarkable examples of this type of model are GPT-1, GPT-2 \citep{radford-etal-2018-improving,radford-etal-2019-language}, BERT \citep{devlin-etal-2019-bert} and XLNet \citep{yang-etal-2019-xlnet}; the latter being the current state-of-the-art for multiple downstream tasks. All of these models are different arrangements of the Transformer architecture \citep{vaswani-etal-2017-attention} trained with different datasets, except for XLNet which is an instance of the Transformer-XL \citep{dai-etal-2019-transformer}.

Even though these models have clear advantages, their main drawback is the amount of data that is needed to train them in order to obtain a functional and efficient model. For the first English version of word2vec, \citet{mikolov-etal-2013-distributed} used a one billion word dataset consisting of various news articles. Later \citet{al-rfou-etal-2013-polyglot} and then \citet{bojanowski-etal-2017-enriching} used the plain text from Wikipedia to train distributions of word2vec and fastText respectively, for languages other than English. Now, the problem of obtaining large quantities of data aggravates even more for contextual models, as they normally need multiple instances of a given word in order to capture all its different uses and in order to avoid overfitting due to the large quantity of hyperparameters that these models have. \citet{peters-etal-2018-deep} for example use a 5.5 billion token\footnote{Punctuation marks are counted as tokens.} dataset comprised of crawled news articles plus the English Wikipedia in order to train ELMo, \citet{devlin-etal-2019-bert} use a 3.3 billion word\footnote{Space sparated tokens.} corpus made by merging the English Wikipedia with the BooksCorpus \citep{zhu-etal-2015-aligning}, and \citet{radford-etal-2019-language} use a 40GB English corpus created by scraping outbound links from Reddit.\footnote{\url{https://www.reddit.com/}}

While Wikipedia is freely available, and multiple pipelines exist\footnote{\url{https://github.com/attardi/wikiextractor}}\textsuperscript{,}\footnote{\url{https://github.com/hghodrati/wikifil}} to extract plain text from it, some of the bigger corpora mentioned above are not made available by the authors either due to copyright issues or probably because of the infrastructure needed to serve and distribute such big corpora. Moreover the vast majority of both these models and the corpora they are trained with are in English, meaning that the availability of high quality NLP for other languages, specially for low-resource languages, is rather limited.



%%%%%%%%%%%%%%%%%%%%%%%%%%%%%%%%%%%%%%%%%%%%%%%%%%%%%%%%%%%%%%%%%%%%%%%%
\section{Quality at Glance}
%%%%%%%%%%%%%%%%%%%%%%%%%%%%%%%%%%%%%%%%%%%%%%%%%%%%%%%%%%%%%%%%%%%%%%%%

Access to multilingual datasets for NLP research has vastly improved over the past years. A variety of web-derived collections for hundreds of languages is available for anyone to download, such as ParaCrawl~\citep{espla-etal-2019-paracrawl, banon-etal-2020-paracrawl}, WikiMatrix~\citep{schwenk-etal-2021-wikimatrix} CCAligned~\citep{el-kishky-etal-2020-ccaligned}, \mbox{OSCAR}~\citep{ortiz-suarez-etal-2019-asynchronous, ortiz-suarez-etal-2020-monolingual}, and several others.
These have in turn enabled a variety of highly multilingual models, like mT5 \citep{xue-etal-2021-mt5}, M2M\nobreakdash-100 \citep{fan-etal-2020-beyond}, M4 \citep{arivazhagan-etal-2019-massively}.


Curating such datasets relies on the websites giving clues about the language of their contents (e.g. a language identifier in the URL) and on automatic language classification (LangID).

It is commonly known that these automatically crawled and filtered datasets tend to have overall lower quality than hand-curated collections~\citep{koehn-etal-2020-findings}, but their quality is rarely measured directly, and is rather judged through the improvements they bring to downstream applications~\citep{schwenk-etal-2021-wikimatrix}.

Building NLP technologies with automatically crawled datasets is promising. This is especially true for low-resource languages, because data scarcity is one of the major bottlenecks for deep learning approaches.
However, there is a problem: There exists very little research on evaluating both data collections and automatic crawling and filtering tools for low-resource languages.
As a result, although many low-resource languages are covered by the latest multilingual crawl data releases, their quality and thus usability is unknown.

%%%%%%%%%%%%%%%%%%%%%%%%%%%%%%%%%%%%%%%%%%%%%%%%%%%%%%%%%%%%%%%%%%%%%%%%
\section{CamemBERT}
%%%%%%%%%%%%%%%%%%%%%%%%%%%%%%%%%%%%%%%%%%%%%%%%%%%%%%%%%%%%%%%%%%%%%%%%

Pretrained word representations have a long history in Natural Language Processing (NLP), from non-contextual \citep{brown-etal-1992-class,ando-zhang-2005-framework,mikolov-etal-2013-distributed,pennington-etal-2014-glove} to contextual word embeddings \citep{peters-etal-2018-deep,akbik-etal-2018-contextual}. Word representations are usually obtained by training language model architectures on large amounts of textual data and then fed as an input to more complex task-specific architectures. More recently, these specialized architectures have been replaced altogether by large-scale pretrained language models which are \emph{fine-tuned} for each application considered. This shift has resulted in large improvements in performance over a wide range of tasks \cite{devlin-etal-2019-bert,radford-etal-2019-language,liu-etal-2019-roberta,raffel-etal-2020-exploring}.

These transfer learning methods exhibit clear advantages over more traditional task-specific approaches. In particular, they can be trained in an \emph{unsupervized} manner, thereby taking advantage of the information contained in large amounts of raw text.
Yet they come with implementation challenges, namely the amount of data and computational resources needed for pre-training, which can reach hundreds of gigabytes of text and require hundreds of GPUs \citep{yang-etal-2019-xlnet,liu-etal-2019-roberta}. This has limited the availability of these state-of-the-art models to the English language, at least in the monolingual setting. This is particularly inconvenient as it hinders their practical use in NLP systems. It also prevents us from investigating their language modelling capacity, for instance in the case of morphologically rich languages.

%%%%%%%%%%%%%%%%%%%%%%%%%%%%%%%%%%%%%%%%%%%%%%%%%%%%%%%%%%%%%%%%%%%%%%%%
\section{FrELMo}
%%%%%%%%%%%%%%%%%%%%%%%%%%%%%%%%%%%%%%%%%%%%%%%%%%%%%%%%%%%%%%%%%%%%%%%%
\label{sec:intro}

Named entity recognition (NER) is the widely studied task consisting in identifying text spans that denote \emph{named entities} such as person, location and organization names, to name the most important types. Such text spans are called named entity \emph{mentions}. In NER, mentions are generally not only identified, but also classified according to a more or less fine-grained ontology, thereby allowing for instance to distinguish between the telecommunication company \emph{Orange} and the town \emph{Orange} in southern France (amongst others). Importantly, it has long been recognised that the type of named entities can be defined in two ways, which underlies the notion of metonymy: the intrinsic type (\emph{France} is always a location) and the contextual type (in \emph{la France a signé un traité} `France signed a treaty', \emph{France} denotes an organization).


NER has been an important task in natural language processing for quite some time. It was already the focus of the MUC conferences and associated shared tasks
\cite{marsh-perzanowski-1998-muc}, and later that of the CoNLL~2003 and ACE shared tasks \cite{tjong-kim-sang-de-meulder-2003-introduction,doddington-etal-2004-automatic}. Traditionally, as for instance was the case for the MUC shared tasks, only person names, location names, organization names, and sometimes ``other proper names'' are considered. However, the notion of named entity mention is sometimes extended to cover any text span that does not follow the general grammar of the language at hand, but a type- and often culture-specific grammar, thereby including entities ranging from product and brand names to dates and from URLs to monetary amounts and other types of numbers.

As for many other tasks, NER was first addressed using rule-based approaches, followed by statistical and now neural machine learning techniques (see Section~\ref{subsec:sota} for a brief discussion on NER approaches). Of course, evaluating NER systems as well as training machine-learning-based NER systems, statistical or neural, require named-entity-annotated corpora.
Unfortunately, most named entity annotated French corpora are oral transcripts, and they are not always freely available. The ESTER and ESTER2 corpora (60 plus 150 hours of NER-annotated broadcast transcripts)
\cite{galliano-etal-2005-the,galliano-etal-2009-the}, as well as the Quaero
\cite{grouin-etal-2011-proposal} corpus are based on oral transcripts (radio broadcasts). Interestingly, the Quaero corpus relies on an original, very rich and structured  definition of the notion of named entity \cite{rosset-etal-2011-entites}. It contains both the intrinsic and the contextual types of each mention, whereas the ESTER and ESTER2 corpora only provide the contextual type.


%%%%%%%%%%%%%%%%%%%%%%%%%%%%%%%%%%%%%%%%%%%%%%%%%%%%%%%%%%%%%%%%%%%%%%%%
\section{BERTrade}
%%%%%%%%%%%%%%%%%%%%%%%%%%%%%%%%%%%%%%%%%%%%%%%%%%%%%%%%%%%%%%%%%%%%%%%%

% Context
There is a growing interest in digital humanities for automatic processing and annotation of historical texts. In this work, we study how to take advantage of current NLP models of the BERT family to advance the state of the art in processing historical languages, taking Old French (9th-13th century French) as a use case.

Old French is one of the historical languages for which we have the largest amount of syntactically annotated data, and we expect that our results on these language states may be generalised and used as a source of inspiration for researchers currently developing annotated resources for other historical languages.

Using contextual word embeddings as input representations has brought clear gains in performances for most of the NLP tasks for which they have been used.
However, this has mostly been attested in languages where sufficient (raw) linguistic data is available.
For less-resourced languages, the most common approach has been to leverage multilingual models such as mBERT \citep{devlin-etal-2019-bert}

Historical languages are typical cases where available linguistic data is limited, with no chance of acquiring new texts. They are also not normalized by spelling and institutional conventions and tend to be more heterogeneous than contemporary lesser-resourced languages.

\subsection{Medeival French}

\begin{figure}[thb]
  \centering
  \includegraphics[scale=0.29]{static/media/mod_eval/bertrade/map-dialects2.png}
  \caption{Oïl languages}
  \label{fig:map-dialects}
\end{figure}

Medieval French covers both Old French (9th-13th c.) and Middle French (14th-15th c.). These stages are linguistically close and both precede the adoption of spelling norms. Middle French is more regular than Old French in some respects such as word order \citep{marchello-Nizia-etal-2020-grande} and less in others such as NP structure and pronouns system \citep{marchello-nizia-etal-1979-histoire}. Medieval French covers a set of \textit{Oïl} Romance languages spoken in the kingdom of France between the 9th and the 15th century (\cref{fig:map-dialects}).
There are around twenty such languages.

Older texts are close to Late Latin, and verse is prevalent until the end of the 13th century. Old French has a relatively free word order.
Until the mid-11th century, the prevalent order is \textit{Subject-Object-Verb} (SOV), which is then gradually supplanted by SVO, which is the default order in contemporary French. Unlike most languages with free word order, the functions of verbal arguments are not always given away by morphological clues, the already simplistic case system of Old French disappears progressively through the covered period.

There are also many cases of syntactic ambiguity. For example, in the following quote from \emph{Lancelot},\footnote{In the edition from Pierre Kunstmann, from the online \textit{Base de français médiéval}: \url{http://catalog.bfm-corpus.org/CharretteKu}.} (verse ~5436),
both \enquote{la dame} and \enquote{Lancelot} could be the subject or the object of \enquote{Vit} and only the context enables the reader to understand that \enquote{la dame} is the subject.

\digloss{Dolant et pansif Lancelot Vit la dame}
{Mournful and meditative Lancelot saw the lady}
{The lady saw that Lancelot was mournful and meditative.}

Word order is also relatively free within constituents. For example, a noun modifier can be on the left or on the right of its governor, and it is not necessarily preceded by a preposition. In contemporary French, it can only appear on the right, and it is found without a preposition only in some cases like named entities. Because of the general free word order and the absence of punctuation in our treebank, this adds up to the ambiguity of the analysis.

In each of the following examples from the SRCMF corpus, the noun following \emph{roi} (\enquote{king}) has a different analysis: head of \emph{roi}, modifier, argument of the same verb or a different one, with no explicit marking:

\begin{center}
    % beroul. modifieur à gauche.
    \begin{dependency}[theme=simple]
        \begin{deptext}[row 2/.style={font=\small}]
            \textit{Fus} \& \textit{tu} \& \textit{donc} \& \textit{pus} \& \textit{a} \& \textit{la} \& \textbf{\textit{roi}} \& \textit{cort} \\
            %VERB \& PRON \& ADV \& ADV \& ADP \& DET \& NOUN \& NOUN \\
            Were \& you \& then \& no more \& at \& the \& king \& court \\
        \end{deptext}
        \depedge{8}{7}{nmod}
        \depedge[edge start x offset=0.5em]{8}{6}{det}
        \depedge[edge start x offset=1em]{8}{5}{case}
    \end{dependency}

    \raggedright
    \enquote{Then were you not at the king's court anymore?} (\emph{Beroul Tristan})
\end{center}

\begin{center}
    % Graal. modifieur à droite.
    \begin{dependency}[theme=simple]
        \begin{deptext}[row 2/.style={font=\small}]
            \textit{la} \& \textit{fille} \& \textit{au} \& \textit{riche} \& \textbf{\textit{roi}} \& \textit{pescheor} \\
            the \& daughter \& of the \& rich \& king \& fisher \\
        \end{deptext}
        \depedge{5}{6}{flat}
    \end{dependency}

    \raggedright
    \enquote{the daughter of the rich Fisher King} (\emph{Queste del Saint Graal})
\end{center}

\begin{center}
    % Roland. arguments du même verbe.
    \begin{dependency}[theme=simple]
        \begin{deptext}[row 2/.style={font=\small}]
            \textit{De} \& \textit{Guenelun} \& \textit{atent} \& \textit{li} \& \textbf{\textit{reis}} \& \textit{nuveles} \\
            From \& Ganelon \& waits \& the \& king \& news \\
        \end{deptext}
        \depedge{3}{5}{nsubj}
        \depedge[edge start x offset=-0.5em]{3}{6}{obj}
    \end{dependency}

    \raggedright
    \enquote{The king waits for news from Ganelon.} (\emph{Chanson de Roland})
\end{center}

\begin{center}
    % Graal. arguments de verbes différents, et absence de ponctuation.
    \begin{dependency}[theme=simple]
        \begin{deptext}[row 2/.style={font=\small}]
            \textit{Biax} \& \textit{sire} \& \textit{fet} \& \textit{li} \& \textbf{\textit{rois}} \& \textit{escu} \& \textit{vos} \& \textit{envoiera} \& \textit{Diex} \\
            Dear \& Sir \& says \& the \& king \& shield \& you \& send-FUT \& God \\
        \end{deptext}
        \depedge{3}{5}{nsubj}
        \depedge{8}{6}{obj}
    \end{dependency}

    \raggedright
    \enquote{Dear Sir, says the king, God will send you a shield.} (\emph{Queste del Saint Graal})
\end{center}

Furthermore, overt subjects are not mandatory, and are often dropped in texts written in verse until the 12th century, after which the presence of subjects increases through time.
These phenomena are particularly prevalent in verse, where metric and rhyming constraints often lead to more contrived syntactic forms than in prose.

Another source of ambiguity is the variety of spellings, due to the lack of spelling standard. For example, the word \textit{moult} (transl. \textit{a lot (of), very}), emblematic of this period, is initially an adjective, and it is progressively grammaticalized, becoming an adverb. Several forms appear at the same time, some with a declension, some without, and the radical does not have a fixed spelling: \textit{molt(e)(s), molz, mult(e)(s), mul(t)z, mou(l)t}…

%%%%%%%%%%%%%%%%%%%%%%%%%%%%%%%%%%%%%%%%%%%%%%%%%%%%%%%%%%%%%%%%%%%%%%%%
\section{D'AlemBERT}
%%%%%%%%%%%%%%%%%%%%%%%%%%%%%%%%%%%%%%%%%%%%%%%%%%%%%%%%%%%%%%%%%%%%%%%%

With the rise of digital humanities, it is becoming increasingly important to develop high quality tools to automatically process old states of languages. Libraries, archives and museums, among others, are digitising large numbers of historical sources, from which high quality data must be extracted for further study by specialists of human sciences following new approaches such as ``distant reading'' \cite{moretti-2013-distant}. Many (sub)tasks such as automatic OCR post-correction \cite{rijhwani-etal-2021-lexically} and linguistic annotation \cite{camps-etal-2021-corpus} benefit from pretrained language models to improve their accuracy, and this is what motivated us to develop a BERT-like \cite{devlin-etal-2019-bert} contextualised language model for Early Modern French.

Languages evolve over time on many different levels: from one century to another, we see variations in spelling, syntax, the lexicon etc. However this variation is not uniform: it tends, at least for ``literate scriptors'' (literature, journalism, law, etc.), to converge towards a single norm over time, and this has especially been the case for French because of the prominent role of the \emph{Académie française} and the \emph{remarqueurs} \cite{ayres-bennett-etal-2011-remarques}. The result of this convergence is, for instance, that spelling and word order within sentences have become more strict, where they were less so in the past. From a computational perspective, historical states of language are therefore not only different from the contemporary state, but, from a computational perspective, are also more complex because they do not follow a strict and explicit norm. In French, this explicit norm  appeared in the 17\textsuperscript{th}~c. and was slowly integrated throughout the 18\textsuperscript{th}~c.

On top of this first linguistic problem, a second issue appears: because the production of textual sources has continued to grow exponentially, it is easier to collect a corpus for contemporary French  than for the 19\textsuperscript{th}\,c. French, which is itself easier than for the 18\textsuperscript{th}\,c. French, etc. The further we go back in time, the more scarce resources are, which creates the following paradox: we have more data when the language is homogeneous and simple for the computer to process, and less when it is heterogeneous and harder to process.

\subsection{Early Modern French}\label{def:early}

\begin{table*}[!htp]
    \centering\small
    \begin{tabular}{@{}p{0.3\linewidth}p{0.3\linewidth}p{0.3\linewidth}@{}}
        \toprule
        Source                                                                                                                                                                                                                                                                                               & Normalised & Translation \\
        \midrule
        Surquoy, SIRE, s’il plaiſt à voſtre Maieſté de ſe ſouuenir des miſeres de ſon Eſtat, dõt au moins ell’a tiré cét aduantage, qu’en vne grande ieuneſse ell’a acquis vne grande experi\~ece, elle verra que tous les mal-heurs de sõ bas âge ont pris leur commencement en ſemblables occaſions;       &
        \emph{Sur quoi, SIRE, s’il plaît à votre Majesté de se souvenir des misères de son état dont au moins elle a tiré cet avantage, qu’en une grande jeunesse elle a acquis une grande expérience, elle verra que tous les malheurs de son bas âge ont pris leur commencement en semblables occasions~;} &
        \textcolor{gray}{``Whereupon, SIR, if it pleases your Majesty to remember the miseries of her state, from which at least she has derived this advantage, that in great youth she has acquired great experience, she will see that all the misfortunes of her early life took their beginning on similar occasions;''}           \\
        \bottomrule
    \end{tabular}
    \caption{\label{tab:norm_examples}Example of normalisation taken from the \emph{Lettres} of \protect\newcite{balzac-1624-lettres}.}
\end{table*}

We loosely define Early Modern French as a state of language following Middle French in 1500---following here the \emph{terminus ad quem} used by the \emph{Dictionnaire de Moyen Français} \cite{martin-2020-dictionnaire}---and ending with the French Revolution in 1789. It therefore encompasses three centuries (16\textsuperscript{th}, 17\textsuperscript{th} and 18\textsuperscript{th}\,c.), or two linguistic periods: the \emph{français préclassique} or ``preclassical French'', 1500--1630 and the \emph{français classique} or ``classical French'', 1630--1689; both periodisations are currently used in French linguistics (\emph{e.g.}~by \newcite{vachon-2010-changement} and \newcite{amatuzzi-etal-2019-ameliorer}).

A typical example of Early Modern French, taken from ~\newcite{balzac-1624-lettres}, is given in Table~\ref{tab:norm_examples}. We note here the presence of several phenomena that have now disappeared in contemporary French, such as the presence of abbreviations (\emph{dõt}$\to$\emph{dont}), the long \emph{s} (\emph{ſ}, see\,\emph{miſeres}), the use of \emph{v} instead of \emph{u} (\emph{vne} for \emph{une}), the conservation of etymological letters (\emph{voſtre}$<$Latin~\emph{vŏster} rather than \emph{votre}) and calligraphic letters (\emph{-y} in \emph{Surquoy}), the absence of welding  (\emph{\mbox{mal-heurs}} and not \emph{malheurs}) and the opposite (\emph{Surquoy} and not \emph{Sur quoi}).

For NLP tasks, which process raw sequences, such differences with respect to contemporary French are not trivial, and they prevent the processing of historical texts with tools trained on recent sources.

%%%%%%%%%%%%%%%%%%%%%%%%%%%%%%%%%%%%%%%%%%%%%%%%%%%%%%%%%%%%%%%%%%%%%%%%
\chapter{On Raw Corpora for Language Modeling}
%%%%%%%%%%%%%%%%%%%%%%%%%%%%%%%%%%%%%%%%%%%%%%%%%%%%%%%%%%%%%%%%%%%%%%%%

The frWaC corpus is a French text corpus collected from the .fr domain with using medium-frequency words from the Le Monde Diplomatique corpus and basic French vocabulary lists as seeds. The corpus consists of French websites with total size 1.3 billion words.



\section{Common Crawl}

Common Crawl is a non-profit foundation which produces and maintains an open repository of web crawled data that is both accessible and analyzable.\footnote{\url{http://commoncrawl.org/about/}} Common Crawl's complete web archive consists of petabytes of data collected over 8 years of web crawling. The repository contains raw web page HTML data (WARC files), metadata extracts (WAT files) and plain text extracts (WET files). The organization's crawlers has always respected \texttt{nofollow}\footnote{\url{http://microformats.org/wiki/rel-nofollow}} and \texttt{robots.txt}\footnote{\url{https://www.robotstxt.org/}} policies.

Each monthly Common Crawl snapshot is in itself a massive multilingual corpus, where every single file contains data coming from multiple web pages written in a large variety of languages and covering all possible types of topics. Thus, in order to effectively use this corpus for Natural Language Processing and Machine Learning applications, one has first to extract, filter, clean and classify the data in the snapshot by language.

Throughout this thesis, we will use the WET files which contain the extracted plain texts from the websites mostly converted to UTF-8, as well as headers containing the metadata of each crawled document. Each WET file comes compressed in gzip format\footnote{\url{https://www.gnu.org/software/gzip/}} and is stored on Amazon Web Services.

Common Crawl has already been successfully used to train language models, even multilingual ones. The most notable example is probably fastText which was first trained for English using Common Crawl \citep{mikolov-etal-2018-advances} and then for other 157 different languages \citep{grave-etal-2018-learning}. In fact \citet{grave-etal-2018-learning} proposed a pipeline to filter, clean and classify Common Crawl, which we shall call the ``fastText pre-processing pipeline.'' They used the fastText linear classifier \citep{joulin-etal-2016-fasttext, joulin-etal-2017-bag} to classify each line of Common Crawl by language, and downloaded the initial corpus and schedule the I/O using some simple Bash scripts. Their solution, however, proved to be a synchronous blocking pipeline that works well on infrastructures having the necessary hardware to assure high I/O speeds even when storing tens of terabytes of data at a time. But that downscales poorly to medium-low resource infrastructures that rely on more traditional cost-effective electromechanical mediums in order to store this amount of data.

\subsection{fastText's Pipeline}

The \emph{``fastText pre-processing pipeline''} used by \citet{grave-etal-2018-learning} launches multiple process, preferably as many as available cores. Each of these processes first downloads one Common Crawl WET file which then proceeds to decompress after the download is over. After decompressing, an instance of the fastText linear classifier \citep{joulin-etal-2016-fasttext, joulin-etal-2017-bag} is launched, the classifier processes each WET file line by line, generating a language tag for each line. The tags are then stored in a tag file which holds a one-to-one correspondence between lines of the WET file and its corresponding language tag. The WET file and the tag files are read sequentially and each on the WET file line holding the condition of being longer than 100 bytes is appended to a language file containing only plain text (tags are discarded). Finally, the tag file and the WET files are deleted.

Only when one of these processes finishes another can be launched. This means that one can at most process and download as many files as cores the machine has. That is, if for example a machine has 24 cores, only 24 WET files can be downloaded and processed simultaneously, moreover, the 25\textsuperscript{th} file won't be downloaded until one of the previous 24 files is completely processed.

When all the WET files are classified, one would normally get around 160 language files, each file holding just plain text written in its corresponding language. These files still need to be filtered in order to get rid of all files containing invalid UTF-8 characters, so again a number of processes are launched, this time depending on the amount of memory of the machine. Each process reads a language file, first filters for invalid UTF-8 characters and then performs deduplication. A simple non-collision resistant hashing algorithm is used to deduplicate the files.

The fastText linear classifier works by representing sentences for classification as Bags of Words (BoW) and training a linear classifier. A weight matrix $A$ is used as a look-up table over the words and the word representations are then averaged into a text representation which is fed to the linear classifier. The architecture is in general similar to the CBoW model of \citet{mikolov-etal-2013-distributed}, but the middle word is replaced by a label. They uses a softmax function $f$ to compute the probability distribution over the classes. For a set of $N$ documents, the model is trained to minimize the negative log-likelihood over the classes:
\[
    -\frac{1}{N}\sum_{n=1}^{N} y_n\log\left(f(BAx_n)\right),
\]
where $x_n$ is the normalized bag of features of the $n$-th document, $y_n$ is the $n$-th label, and $A,B$ are the weight matrices. The pre-trained fastText model for language recognition \citep{grave-etal-2018-learning} is capable of recognising around 176 different languages and was trained using 400 million tokens from Wikipedia as well as sentences from the Tatoeba website\footnote{\url{https://tatoeba.org/}}.

\subsection{CCNet}

We note that the original Common-Crawl-based corpus created by \citet{grave-etal-2018-learning} to train fastText is not freely available. Since running the experiments described in this paper, a new architecture for creating a Common-Crawl-based corpus named CCNet \citep{wenzek-etal-2020-ccnet} has been published, although it includes specialized filtering based on the KenLM library \cite{heafield-2011-kenlm} and trained on Wikipedia, which might result in a cleaner corpus, the resulting CCNet corpus itself was never published in its entirety.


%%%%%%%%%%%%%%%%%%%%%%%%%%%%%%%%%%%%%%%%%%%%%%%%%%%%%%%%%%%%%%%%%%%%%%%%
\section{Quality at Glance Related Work}
%%%%%%%%%%%%%%%%%%%%%%%%%%%%%%%%%%%%%%%%%%%%%%%%%%%%%%%%%%%%%%%%%%%%%%%%

Corpora collected by web crawlers are known to be noisy~\citep{junczys-dowmunt-2019-microsoft,luccioni-viviano-2021-whats}. In highly multilingual settings, past work found that web-crawls of lower-resource languages have serious issues, especially with segment-level LangID~\citep{caswell-etal-2020-language}.

Cleaning and filtering web-crawls can boost general language modeling~\citep{gao-etal-2020-the,brown-etal-2020-language,raffel-etal-2020-exploring} and downstream task performance~\citep{moore-lewis-2010-intelligent,rarrick-etal-2011-mt,xu-koehn-2017-zipporah,khayrallah-koehn-2018-impact,brown-etal-2020-language}.

As the scale of ML research grows, it becomes increasingly difficult to validate automatically collected and curated datasets \citep{biderman-etal-2020-pitfalls,birhane-etal-2021-large,bender-etal-2021-on}.

Several works have focused on advancing methodologies and best practices to address these challenges. \citet{bender-friedman-2018-data} introduced data statements, a documentary framework for NLP datasets that seeks to provide a universal minimum bar for dataset description. Similar work has been done on systematizing documentation in other areas in data science and machine learning, including work focusing on
online news \citep{kevin-etal-2018-information}, data ethics \citep{sun-etal-2019-mithralabel}, and data exploration \citep{holland-etal-2018-the}, as well as generalist work such as \citep{gebru-etal-2018-datasheets}. 
Data quality is also implicitly documented by successes of filtering methods. There is a large literature on filtering data for various NLP tasks, e.g. \citet{axelrod-etal-2011-domain,moore-lewis-2010-intelligent,rarrick-etal-2011-mt,wang-etal-2018-denoising,kamholz-etal-2014-panlex,junczys-dowmunt-2018-dual,caswell-etal-2020-language}.

Closest to our work is the analysis of a highly multilingual (non-publicly available) web-crawl and LangID related quality issues by \citet{caswell-etal-2020-language}. They perform a brief analysis of the quality of OSCAR with the focus only on the presence of in-language content. \citet{dodge-etal-2021-documenting} automatically documented and analyzed the contents and sources of C4~\citep{raffel-etal-2020-exploring}, the English counterpart of mC4, which surfaced the presence of machine-translated contents and NLP benchmark data.

\subsection{Multilingual Corpora}\label{sec:crawls}

\begin{table*}[th!]
    \centering
    \resizebox{\textwidth}{!}{%
        \begin{tabular}{lccccc}
            \toprule
                            & \multicolumn{3}{c}{\textbf{Parallel}} & \multicolumn{2}{c}{\textbf{Monolingual}}                                                       \\
            \cmidrule(lr){2-4} \cmidrule(lr){5-6}
                            & \textbf{CCAligned}                    & \textbf{ParaCrawl v7.1}                  & \textbf{WikiMatrix} & \textbf{OSCAR} & \textbf{mC4} \\
            \midrule
            \#languages     & 137                                   & 41                                       & 85                  & 166            & 101          \\
            Source          & CC 2013--2020                         & selected websites                        & Wikipedia           & CC 11/2018     & CC all       \\
            Filtering level & document                              & sentence                                 & sentence            & document       & document     \\
            Langid          & FastText                              & CLD2                                     & FastText            & FastText       & CLD3         \\
            Alignment       & LASER                                 & Vec/Hun/BLEU-Align                       & LASER               & -              & -            \\
            Evaluation      & TED-6                                 & WMT-5                                    & TED-45              & POS/DEP-5      & XTREME       \\
            \bottomrule
        \end{tabular}%
    }
    \caption{Comparison of parallel and monolingual corpora extracted from web documents, including their downstream evaluation tasks. All parallel corpora are evaluated for machine translation (BLEU). TED-6: \texttt{da}, \texttt{cr}, \texttt{sl}, \texttt{sk}, \texttt{lt}, \texttt{et}; TED-45: 45-language subset of ~\citep{qi-etal-2018-pre}; WMT-5: \texttt{cs}, \texttt{de}, \texttt{fi}, \texttt{lv}, \texttt{ro}. POS/DEP-5: part-of-speech labeling and dependency parsing for \texttt{bg}, \texttt{ca}, \texttt{da}, \texttt{fi}, \texttt{id}.}
    \label{tab:corpora}
\end{table*}


Table \ref{tab:corpora} provides an overview of the corpora of interest in this work. We selected the corpora for their multilinguality and the inclusion of understudied languages in NLP. With the exception of WikiMatrix and ParaCrawl, all corpora are derived from CommonCrawl (CC).\footnote{\url{http://commoncrawl.org/}}

\paragraph{CCAligned~\citep{el-kishky-etal-2020-ccaligned}}is a parallel dataset built off 68 CC snapshots. Documents are aligned if they are in the same language according to FastText LangID~\citep{joulin-etal-2016-fasttext,joulin-etal-2017-bag}, and have the same URL but for a differing language code. These alignments are refined with cross-lingual LASER embeddings \citep{artetxe-schwenk-2019-massively}. For sentence-level data, they split on newlines and align with LASER, but perform no further filtering.
Human annotators evaluated the quality of document alignments for six languages (\texttt{de}, \texttt{zh}, \texttt{ar}, \texttt{ro}, \texttt{et}, \texttt{my}) selected for their different scripts and amount of retrieved documents, reporting precision of over 90\%. The quality of the extracted parallel sentences was evaluated in a machine translation (MT) task on six European (\texttt{da}, \texttt{cr}, \texttt{sl}, \texttt{sk}, \texttt{lt}, \texttt{et}) languages of the TED corpus~\citep{qi-etal-2018-pre}, where it compared favorably to systems built on crawled sentences from WikiMatrix and ParaCrawl v6.

\paragraph{Multilingual C4 (mC4)~\citep{xue-etal-2021-mt5}} is a document-level dataset used for training the mT5 language model. It consists of monolingual text in 101 languages and is generated from 71 CC snapshots. It filters out pages that contain less than three lines of at least 200 characters and pages that contain bad words.\footnote{\url{https://github.com/LDNOOBW/}} Since this is a document-level dataset, we split it by sentence and deduplicate it before rating. For language identification, it uses CLD3~\citep{botha-etal-2017-natural},\footnote{\url{https://github.com/google/cld3/}} a small feed-forward neural network that was trained to detect 107 languages. The mT5 model pre-trained on mC4 is evaluated on 6 tasks of the XTREME benchmark~\citep{hu-etal-2020-xtreme} covering a variety of languages and outperforms other multilingual pre-trained language models such as mBERT~\citep{devlin-etal-2019-bert} and XLM\nobreakdash-R~\citep{conneau-etal-2020-unsupervised}.

\paragraph{OSCAR~\citep{ortiz-suarez-etal-2019-asynchronous, ortiz-suarez-etal-2020-monolingual}}is a set of monolingual corpora extracted from CC snapshots, specifically from the plain text \emph{WET} format distributed by CC which removes all the HTML tags and converts the text to UTF-8. It is deduplicated and follows the approach by~\citep{grave-etal-2018-learning} of using FastText LangID~\citep{joulin-etal-2016-fasttext, joulin-etal-2017-bag} on a line-level.\footnote{\url{https://fasttext.cc/docs/en/language-identification.html} } No other filtering was applied. For five languages (\texttt{bg}, \texttt{ca}, \texttt{da}, \texttt{fi}, \texttt{id}) OSCAR was used by its original authors to train language models which were then evaluated on parsing and POS tagging \citep{ortiz-suarez-etal-2020-monolingual}. OSCAR has also been used in independent studies to train monolingual or multilingual language models (\texttt{ar}, \texttt{as}, \texttt{bn}, \texttt{de}, \texttt{el}, \texttt{fr}, \texttt{gu}, \texttt{he}, \texttt{hi}, \texttt{kn}, \texttt{ml}, \texttt{mr}, \texttt{nl}, \texttt{or}, \texttt{pa}, \texttt{ro}, \texttt{ta}, \texttt{te}) and subsequently evaluate them on various downstream tasks \citep{antoun-etal-2021-araelectra, kakwani-etal-2020-indicnlpsuite, wilie-etal-2020-indonlu, chan-etal-2020-germans, koutsikakis-etal-2020-greek, martin-etal-2020-camembert, chriqui-etal-2021-hebert, seker-etal-2021-alephbert, delobelle-etal-2020-robbert, dumitrescu-etal-2020-birth, masala-etal-2020-robert}.


\paragraph{ParaCrawl v7.1} is a parallel dataset with 41 language pairs primarily aligned with English (39 out of 41) and mined using the parallel-data-crawling tool Bitextor \citep{espla-etal-2019-paracrawl,banon-etal-2020-paracrawl} which includes downloading documents, preprocessing and normalization, aligning documents and segments, and filtering noisy data via Bicleaner.\footnote{\url{https://github.com/bitextor/bicleaner}}
ParaCrawl focuses on European languages, but also includes 9 lower-resource, non-European language pairs in v7.1. Sentence alignment and sentence pair filtering choices were optimized for five languages (\texttt{mt}, \texttt{et}, \texttt{hu}, \texttt{cs}, \texttt{de}) by training and evaluating MT models on the resulting parallel sentences. An earlier version (v5) was shown to improve translation quality on WMT benchmarks for~\texttt{cs}, \texttt{de}, \texttt{fi}, \texttt{lv}, \texttt{ro}.


\paragraph{WikiMatrix~\citep{schwenk-etal-2021-wikimatrix}} is a public dataset containing 135M parallel sentences in 1620 language pairs (85 languages) mined from Wikipedia. Out of the 135M parallel sentences, 34M are aligned with English. The text is extracted from Wikipedia pages, split into sentences, and duplicate sentences are removed. FastText LangID is used before identifying bitext with LASER's distance-based mining approach. The margin threshold is optimized by training and evaluating downstream MT models on four WMT benchmarks (\texttt{de-en}, \texttt{de-fr}, \texttt{cs-de}, \texttt{cs-fr}). The final dataset is used to train translation models that are then evaluated by automatically measuring the quality of their translations against human translations of TED talks in 45 languages, with highest quality for translations between English and e.g. \texttt{pt}, \texttt{es}, \texttt{da}, and lowest for \texttt{sr}, \texttt{ja}, \texttt{mr}, \texttt{zh\_TW}. In the audit we focus on language pairs with English on one side.

%%%%%%%%%%%%%%%%%%%%%%%%%%%%%%%%%%%%%%%%%%%%%%%%%%%%%%%%%%%%%%%%%%%%%%%%
\section{Monolingual Related Work}
%%%%%%%%%%%%%%%%%%%%%%%%%%%%%%%%%%%%%%%%%%%%%%%%%%%%%%%%%%%%%%%%%%%%%%%%

Since the introduction of \emph{word2vec} \citep{mikolov-etal-2013-distributed}, many attempts have been made to create multilingual language representations; for fixed word embeddings the most remarkable works are those of \citep{al-rfou-etal-2013-polyglot} and \citep{bojanowski-etal-2017-enriching} who created word embeddings for a large quantity of languages using Wikipedia, and later \citep{grave-etal-2018-learning} who trained the fastText word embeddings for 157 languages using Common Crawl and who in fact showed that using crawled data significantly increased the performance of the embeddings especially for mid- to low-resource languages.

Regarding contextualized models, the most notable non-English contribution has been that of the mBERT \citep{devlin-etal-2019-bert}, which is distributed as (i)~a single multilingual model for 100 different languages trained on Wikipedia data, and as (ii)~a single multilingual model for both Simplified and Traditional Chinese. Four monolingual fully trained ELMo models have been distributed for Japanese, Portuguese, German and Basque\footnote{\url{https://allennlp.org/elmo}}; 44 monolingual ELMo models\footnote{\url{https://github.com/HIT-SCIR/ELMoForManyLangs}} where also released by the \emph{HIT-SCIR} team \citep{che-etal-2018-towards} during the \emph{CoNLL 2018 Shared Task} \citep{zeman-etal-2018-conll}, but their training sets where capped at 20 million words.

For dependency parsing and POS tagging the most notable non-English specific contribution is that of the \emph{CoNLL 2018 Shared Task} \citep{zeman-etal-2018-conll}, where the 1\textsuperscript{st} place (LAS Ranking) was awarded to the \emph{HIT-SCIR} team \citep{che-etal-2018-towards} who used \citet{dozat-manning-2017-deep}'s \emph{Deep Bi-affine parser} and its extension described in \citep{dozat-etal-2017-stanfords}, coupled with deep contextualized ELMo embeddings \citep{peters-etal-2018-deep} (capping the training set at 20 million words). The 1\textsuperscript{st} place in universal POS tagging was awarded to \citet{smith-etal-2018-82} who used two separate instances of \citet{bohnet-etal-2018-morphosyntactic}'s tagger.

More recent developments in POS tagging and parsing include those of \citet{straka-strakova-2019-evaluating} which couples another CoNLL 2018 shared task participant, UDPipe 2.0 \citep{straka-2018-udpipe}, with mBERT greatly improving the scores of the original model, and UDify \citep{kondratyuk-straka-2019-75}, which adds an extra attention layer on top of mBERT plus a Deep Bi-affine attention layer for dependency parsing and a Softmax layer for POS tagging. UDify is actually trained by concatenating the training sets of 124 different UD treebanks, creating a single POS tagging and dependency parsing model that works across 75 different languages.

Concerning contextual models, \citet{baevski-etal-2019-cloze} trained a BERT-like bi-directional Transformer for English using Common Crawl. They followed the ``fastText pre-processing pipeline'' but they removed all copies of Wikipedia inside Common Crawl. They also trained their model using News Crawl \citep{bojar-etal-2018-findings} and using Wikipedia + BooksCorpus, they compared three models and showed that Common Crawl gives the best performance out of the three corpora.

The XLNet model was trained for English by joining the BookCorpus, English Wikipedia, Giga5 \citep{parker-etal-2011-english}, ClueWeb 2012-B \citep{callan-etal-2009-clueweb09} and Common Crawl. Particularly for Common Crawl, \citet{yang-etal-2019-xlnet} say they use ``heuristics to aggressively filter out short or low-quality articles'' from Common Crawl, however they don't give any detail about these ``heuristics'' nor about the pipeline they use to classify and extract the English part of Common Crawl.

It is important to note that none of these projects distributed their classified, filtered and cleaned versions of Common Crawl, making it difficult in general to faithfully reproduce their results.



%%%%%%%%%%%%%%%%%%%%%%%%%%%%%%%%%%%%%%%%%%%%%%%%%%%%%%%%%%%%%%%%%%%%%%%%
\chapter{On Language Models and Downstream Tasks}
%%%%%%%%%%%%%%%%%%%%%%%%%%%%%%%%%%%%%%%%%%%%%%%%%%%%%%%%%%%%%%%%%%%%%%%%


%%%%%%%%%%%%%%%%%%%%%%%%%%%%%%%%%%%%%%%%%%%%%%%%%%%%%%%%%%%%%%%%%%%%%%%%
\section{Monolingual approach}
%%%%%%%%%%%%%%%%%%%%%%%%%%%%%%%%%%%%%%%%%%%%%%%%%%%%%%%%%%%%%%%%%%%%%%%%

One of the key elements that has pushed the state of the art considerably in neural NLP in recent years has been the introduction and spread of transfer learning methods to the field. These methods can normally be classified in two categories according to how they are used:
\begin{itemize}
    \item \emph{Feature-based} methods, which involve pre-training real-valued vectors (``embeddings'') at the word, sentence, or paragraph level; and using them in conjunction with a specific architecture for each individual downstream task.
    \item \emph{Fine-tuning} methods, which introduce a minimal number of task-specific parameters, and instead copy the weights from a pre-trained network and then tune them to a particular downstream task.
\end{itemize}
Embeddings or language models can be divided into \emph{fixed}, meaning that they generate a single representation for each word in the vocabulary; and \emph{contextualized}, meaning that a representation is generated based on both the word and its surrounding context, so that a single word can have multiple representations, each one depending on how it is used.

In practice, most fixed embeddings are used as feature-based models. The most notable examples are \emph{word2vec} \citep{mikolov-etal-2013-distributed}, \emph{GloVe} \citep{pennington-etal-2014-glove} and \emph{fastText} \citep{mikolov-etal-2018-advances}. All of them are extensively used in a variety of applications nowadays. On the other hand, contextualized word representations and language models have been developed using both feature-based architectures, the most notable examples being ELMo and Flair \citep{peters-etal-2018-deep,akbik-etal-2018-contextual}, and transformer based architectures, that are commonly used in a fine-tune setting, as is the case of GPT-1, GPT-2 \citep{radford-etal-2018-improving,radford-etal-2019-language}, BERT and its derivatives \citep{devlin-etal-2019-bert,liu-etal-2019-roberta,lan-etal-2020-albert} and more recently T5 \citep{raffel-etal-2020-exploring}. All of them have repeatedly improved the state-of-the art in many downstream NLP tasks over the last year.

In general, the main advantage of using language models is that they are mostly built in an \emph{unsupervised} manner and they can be trained with raw, unannotated plain text. Their main drawback is that enormous quantities of data seem to be required to properly train them especially in the case of contextualized models, for which larger corpora are thought to be needed to properly address polysemy and cover the wide range of uses that commonly exist within languages.

For the first English version of word2vec, \citet{mikolov-etal-2013-distributed} used a one billion word dataset consisting of various news articles. Later \citet{al-rfou-etal-2013-polyglot} and then \citet{bojanowski-etal-2017-enriching} used the plain text from Wikipedia to train distributions of word2vec and fastText respectively, for languages other than English. Now, the problem of obtaining large quantities of data aggravates even more for contextual models, as they normally need multiple instances of a given word in order to capture all its different uses and in order to avoid overfitting due to the large quantity of hyperparameters that these models have. \citet{peters-etal-2018-deep} for example use a 5.5 billion token\footnote{Punctuation marks are counted as tokens.} dataset comprised of crawled news articles plus the English Wikipedia in order to train ELMo, \citet{devlin-etal-2019-bert} use a 3.3 billion word\footnote{Space sparated tokens.} corpus made by merging the English Wikipedia with the BooksCorpus \citep{zhu-etal-2015-aligning}, and \citet{radford-etal-2019-language} use a 40GB English corpus created by scraping outbound links from Reddit.\footnote{\url{https://www.reddit.com/}}


For gathering data in a wide range of languages, Wikipedia is a commonly used option. It has been used to train fixed embeddings \citep{al-rfou-etal-2013-polyglot,bojanowski-etal-2017-enriching} and more recently the multilingual BERT \citep{devlin-etal-2019-bert}, hereafter mBERT. However, for some languages, Wikipedia might not be large enough to train good quality contextualized word embeddings. Moreover, Wikipedia data all belong to the same specific genre and style. To address this problem, one can resort to crawled text from the internet; the largest and most widespread dataset of crawled text being Common Crawl.\footnote{\url{https://commoncrawl.org}} Such an approach generally solves the quantity and genre/style coverage problems but might introduce noise in the data, an issue which has earned the corpus some criticism, most notably by \citet{trinh-le-2018-a} and \citet{radford-etal-2019-language}. Using Common Crawl also leads to data management challenges as the corpus is distributed in the form of a large set of plain text each containing a large quantity of unclassified multilingual documents from different websites.

%%%%%%%%%%%%%%%%%%%%%%%%%%%%%%%%%%%%%%%%%%%%%%%%%%%%%%%%%%%%%%%%%%%%%%%%
\section{Monolingual Related Work}
%%%%%%%%%%%%%%%%%%%%%%%%%%%%%%%%%%%%%%%%%%%%%%%%%%%%%%%%%%%%%%%%%%%%%%%%

Since the introduction of \emph{word2vec} \citep{mikolov-etal-2013-distributed}, many attempts have been made to create multilingual language representations; for fixed word embeddings the most remarkable works are those of \citep{al-rfou-etal-2013-polyglot} and \citep{bojanowski-etal-2017-enriching} who created word embeddings for a large quantity of languages using Wikipedia, and later \citep{grave-etal-2018-learning} who trained the fastText word embeddings for 157 languages using Common Crawl and who in fact showed that using crawled data significantly increased the performance of the embeddings especially for mid- to low-resource languages.

Regarding contextualized models, the most notable non-English contribution has been that of the mBERT \citep{devlin-etal-2019-bert}, which is distributed as (i)~a single multilingual model for 100 different languages trained on Wikipedia data, and as (ii)~a single multilingual model for both Simplified and Traditional Chinese. Four monolingual fully trained ELMo models have been distributed for Japanese, Portuguese, German and Basque\footnote{\url{https://allennlp.org/elmo}}; 44 monolingual ELMo models\footnote{\url{https://github.com/HIT-SCIR/ELMoForManyLangs}} where also released by the \emph{HIT-SCIR} team \citep{che-etal-2018-towards} during the \emph{CoNLL 2018 Shared Task} \citep{zeman-etal-2018-conll}, but their training sets where capped at 20 million words.

For dependency parsing and POS tagging the most notable non-English specific contribution is that of the \emph{CoNLL 2018 Shared Task} \citep{zeman-etal-2018-conll}, where the 1\textsuperscript{st} place (LAS Ranking) was awarded to the \emph{HIT-SCIR} team \citep{che-etal-2018-towards} who used \citet{dozat-manning-2017-deep}'s \emph{Deep Bi-affine parser} and its extension described in \citep{dozat-etal-2017-stanfords}, coupled with deep contextualized ELMo embeddings \citep{peters-etal-2018-deep} (capping the training set at 20 million words). The 1\textsuperscript{st} place in universal POS tagging was awarded to \citet{smith-etal-2018-82} who used two separate instances of \citet{bohnet-etal-2018-morphosyntactic}'s tagger.

More recent developments in POS tagging and parsing include those of \citet{straka-strakova-2019-evaluating} which couples another CoNLL 2018 shared task participant, UDPipe 2.0 \citep{straka-2018-udpipe}, with mBERT greatly improving the scores of the original model, and UDify \citep{kondratyuk-straka-2019-75}, which adds an extra attention layer on top of mBERT plus a Deep Bi-affine attention layer for dependency parsing and a Softmax layer for POS tagging. UDify is actually trained by concatenating the training sets of 124 different UD treebanks, creating a single POS tagging and dependency parsing model that works across 75 different languages.

Concerning contextual models, \citet{baevski-etal-2019-cloze} trained a BERT-like bi-directional Transformer for English using Common Crawl. They followed the ``fastText pre-processing pipeline'' but they removed all copies of Wikipedia inside Common Crawl. They also trained their model using News Crawl \citep{bojar-etal-2018-findings} and using Wikipedia + BooksCorpus, they compared three models and showed that Common Crawl gives the best performance out of the three corpora.

The XLNet model was trained for English by joining the BookCorpus, English Wikipedia, Giga5 \citep{parker-etal-2011-english}, ClueWeb 2012-B \citep{callan-etal-2009-clueweb09} and Common Crawl. Particularly for Common Crawl, \citet{yang-etal-2019-xlnet} say they use ``heuristics to aggressively filter out short or low-quality articles'' from Common Crawl, however they don't give any detail about these ``heuristics'' nor about the pipeline they use to classify and extract the English part of Common Crawl.

It is important to note that none of these projects distributed their classified, filtered and cleaned versions of Common Crawl, making it difficult in general to faithfully reproduce their results.

%%%%%%%%%%%%%%%%%%%%%%%%%%%%%%%%%%%%%%%%%%%%%%%%%%%%%%%%%%%%%%%%%%%%%%%%
\section{ELMo: Contextualized word embeddings}
%%%%%%%%%%%%%%%%%%%%%%%%%%%%%%%%%%%%%%%%%%%%%%%%%%%%%%%%%%%%%%%%%%%%%%%%

\emph{Embeddings from Language Models} (ELMo) \cite{peters-etal-2018-deep} is a Language Model, i.e, a model that given a sequence of $N$ tokens, $(t_1, t_2, ..., t_N)$, computes the probability of the sequence
by modeling the probability of token $t_k$ given the history $(t_1, ..., t_{k-1})$:
\[
    p(t_1, t_2, \ldots, t_N) = \prod_{k=1}^N p({t_k} \mid t_1, t_2, \ldots, t_{k-1}).
\]
However, ELMo in particular uses a bidirectional language model (biLM) consisting of $L$ LSTM layers, that is, it combines both a forward and a backward language model jointly maximizing the log likelihood of the forward and backward directions:
\begin{align*}
     & \sum_{k=1}^N \left( \right. \log p({t_k} \mid t_1, \ldots, t_{k-1}; \Theta_x, \overrightarrow{\Theta}_{LSTM}, \Theta_s) \\
     & + \log p({t_k} \mid t_{k+1}, \ldots, t_{N}; \Theta_x, \overleftarrow{\Theta}_{LSTM}, \Theta_s)
    \left. \right).
\end{align*}
where at each position $k$, each LSTM layer $l$ outputs a context-dependent representation $\overrightarrow{\mathbf{h}}^{LM}_{k,l}$ with $l=1, \ldots, L$ for a forward LSTM, and $\overleftarrow{\mathbf{h}}^{LM}_{k,l}$ of $t_k$ given $(t_{k+1}, \ldots, t_N)$ for a backward LSTM.

ELMo also computes a context-independent token representation $\mathbf{x}^{LM}_{k}$ via token embeddings or via a CNN over characters. ELMo then ties the parameters for the token representation ($\Theta_x$) and Softmax layer ($\Theta_s$) in the forward and backward direction while maintaining separate parameters for the LSTMs in each direction.

ELMo is a task specific combination of the intermediate layer representations in the biLM, that is,
for each token $t_k$, a $L$-layer biLM computes a set of $2L + 1$ representations
\begin{align*}
    R_k & =  \{\mathbf{x}^{LM}_{k}, \overrightarrow{\mathbf{h}}^{LM}_{k,l}, \overleftarrow{\mathbf{h}}^{LM}_{k,l} \ |\  l =1, \ldots, L \} \\
        & =  \{\mathbf{h}^{LM}_{k,l}\ | \ l=0, \ldots, L\},
\end{align*}
where $\mathbf{h}^{LM}_{k,0}$ is the token layer and
\[
    \mathbf{h}^{LM}_{k,l} = [\overrightarrow{\mathbf{h}}^{LM}_{k,l}; \overleftarrow{\mathbf{h}}^{LM}_{k,l}],
\]
for each biLSTM layer.


When included in a downstream model, as it is the case in this paper, ELMo collapses all $L$ layers in $R$ into a single vector $\mathbf{ELMo}_k = E(R_k; \mathbf{\Theta}_e)$, generally computing a task specific weighting of all biLM layers:
\begin{align*}
    \mathbf{ELMo}^{task}_k & = E(R_k; \Theta^{task})                                       \\
                           & =\gamma^{task} \sum_{l=0}^L s^{task}_l \mathbf{h}^{LM}_{k,l}.
\end{align*}
applying layer normalization to each biLM layer before weighting.

Following \cite{peters-etal-2018-deep}, we use in this paper ELMo models where $L=2$, i.e., the ELMo architecture involves a character-level CNN layer followed by a 2-layer biLSTM.


%%%%%%%%%%%%%%%%%%%%%%%%%%%%%%%%%%%%%%%%%%%%%%%%%%%%%%%%%%%%%%%%%%%%%%%%
\section{CamemBERT Related Work}
%%%%%%%%%%%%%%%%%%%%%%%%%%%%%%%%%%%%%%%%%%%%%%%%%%%%%%%%%%%%%%%%%%%%%%%%

\subsection{Previous work}
\label{relatedwork}
\subsubsection{Contextual Language Models}
\paragraph{From non-contextual to contextual word embeddings}
The first neural word vector representations were non-contextualized word embeddings, most notably
word2vec \citep{mikolov-etal-2013-distributed}, GloVe \cite{pennington-etal-2014-glove} and fastText \cite{mikolov-etal-2018-advances}, which were designed to be used as input to task-specific neural architectures.
Contextualized word representations such as ELMo \cite{peters-etal-2018-deep} and flair \cite{akbik-etal-2018-contextual}, improved the representational power of word embeddings by taking context into account. Among other reasons, they improved the performance of models on many tasks by handling words polysemy.
This paved the way for larger contextualized models that replaced downstream architectures altogether in most tasks. Trained with language modeling objectives, these approaches range from LSTM-based architectures such as \cite{dai-le-2015-semi}, to the successful transformer-based architectures such
as GPT2 \cite{radford-etal-2019-language}, \bert \cite{devlin-etal-2019-bert}, \roberta \cite{liu-etal-2019-roberta} and more recently ALBERT \cite{lan-etal-2020-albert} and T5 \cite{raffel-etal-2020-exploring}.


\paragraph{Non-English contextualized models}
\label{contextualmodelsforotherlanguages}
Following the success of large pretrained language models, they were extended to the multilingual setting with multilingual \bert (hereafter \mbert) \cite{devlin-etal-2019-bert}, a single multilingual model for 104 different languages trained on Wikipedia data, and later XLM \cite{conneau-lample-2019-cross}, which significantly improved unsupervized machine translation.
More recently XLM-R \cite{conneau-etal-2020-unsupervised}, extended XLM by training on 2.5TB of data and outperformed previous scores on multilingual benchmarks. They show that multilingual models can obtain results competitive with monolingual models by leveraging higher quality data from other languages on specific downstream tasks.

A few non-English monolingual models have been released: ELMo models for Japanese, Portuguese, German and Basque\footnote{\url{https://allennlp.org/elmo}} and BERT for Simplified and Traditional Chinese \cite{devlin-etal-2019-bert} and German \cite{chan-etal-2019-german}.

However, to the best of our knowledge, no particular effort has been made toward training models for languages other than English at a scale similar to the latest English models (e.g.~\roberta trained on more than 100GB of data).

\paragraph{BERT and RoBERTa}
Our approach is based on \roberta \cite{liu-etal-2019-roberta} which itself is based on \bert \cite{devlin-etal-2019-bert}. \bert is a multi-layer bidirectional Transformer encoder trained with a masked language modeling (MLM) objective, inspired by the Cloze task \cite{taylor-1953-cloze}. It comes in two sizes: the \bertbase architecture and the \bertlarge architecture. The \bertbase architecture is 3 times smaller and therefore faster and easier to use while \bertlarge achieves increased performance on downstream tasks.
\roberta improves the original implementation of \bert by identifying key design choices for better performance, using dynamic masking, removing the next sentence prediction task, training with larger batches, on more data, and for longer.

\paragraph{Baselines}
In dependency parsing and POS-tagging we compare our model with:

\begin{itemize}
    \item \emph{\mbert}: The multilingual cased version of \bert (see Section~\ref{contextualmodelsforotherlanguages}). We fine-tune \mbert on each of the treebanks with an additional layer for POS-tagging and dependency parsing, in the same conditions as our \camembert model.
    \item \emph{\xlmmlmtlm}: A multilingual pretrained language model from \citet{conneau-lample-2019-cross}, which showed better performance than \mbert on NLI. We use the version available in the Hugging's Face transformer library \cite{wolf-etal-2019-huggingface}; like \mbert, we fine-tune it in the same conditions as our model.
    \item \emph{UDify} \cite{kondratyuk-straka-2019-75}: A multitask and multilingual model based on \mbert, UDify is trained simultaneously on 124 different UD treebanks, creating a single POS tagging and dependency parsing model that works across 75 different languages. We report the scores from \citet{kondratyuk-straka-2019-75} paper.
    \item \emph{UDPipe Future} \citep{straka-2018-udpipe}: An LSTM-based model ranked 3\textsuperscript{rd} in dependency parsing and 6\textsuperscript{th} in POS tagging at the CoNLL~2018 shared task \citep{seker-etal-2018-universal}. We report the scores from \citet{kondratyuk-straka-2019-75} paper.
    \item \emph{UDPipe Future + \mbert + Flair} \citep{straka-strakova-2019-evaluating}: The original UDPipe Future implementation using \mbert and Flair as feature-based contextualized word embeddings. We report the scores from \citet{straka-strakova-2019-evaluating} paper.
\end{itemize}

In French, no extensive work has been done on NER due to the limited availability of annotated corpora. Thus we compare our model with the only recent available baselines set by \citet{dupont-2017-exploration}, who trained both CRF \citep{lafferty-etal-2001-conditional} and BiLSTM-CRF \citep{lample-etal-2016-neural} architectures on the FTB and enhanced them using heuristics and pretrained word embeddings. Additionally, as for POS and dependency parsing, we compare our model to a fine-tuned version of \mbert for the NER task.

For XNLI, we provide the scores of \mbert which has been reported for French by \citet{wu-dredze-2019-beto}. We report scores from \xlmmlmtlm (described above), the best model from \citet{conneau-lample-2019-cross}. We also report the results of \mbox{XLM-R} \cite{conneau-etal-2020-unsupervised}.


% \subsection{Named Entity Recognition}\label{ner-section}
% Named Entity Recognition (NER) is a sequence labeling task that consists in predicting which words refer to real-world objects, such as people, locations, artifacts and organisations. We use the French Treebank\footnote{This dataset has only been stored and used on Inria's servers after signing the research-only agreement.} (FTB)  \cite{abeille-etal-2003-building} in its 2008 version introduced by \newcite{candito-crabbe-2009-improving} and with NER annotations by \newcite{sagot-etal-2012-annotation}.
% The NER-annotated FTB contains more than 12k sentences and more than 350k tokens extracted from articles of the newspaper \textit{Le Monde} published between 1989 and 1995. In total, it contains 11,636 entity mentions distributed among 7 different types of entities, namely: 2025 mentions of ``Person'', 3761 of ``Location'', 2382 of ``Organisation'', 3357 of ``Company'', 67 of ``Product'', 15 of ``POI'' (Point of Interest) and 29 of ``Fictional Character''. 

% A large proportion of the entity mentions in the treebank are multi-word entities. We therefore report the 3 metrics that are commonly used to evaluate models: precision, recall, and F1 score. Here precision measures the percentage of entities found by the system that are correctly tagged, recall measures the percentage of named entities present in the corpus that are found and the F1 score combines both precision and recall measures giving a general idea of a model's performance.

% \subsection{Natural Language Inference}
% We evaluate our model on Natural Language Inference (NLI), using the French part of the XNLI dataset \cite{conneau-etal-2018-xnli}.
% NLI consists in predicting whether a hypothesis sentence is entailed, neutral or contradicts a premise sentence.

% The XNLI dataset is the extension of the Multi-Genre NLI (MultiNLI) corpus \cite{williams-etal-2018-broad} to 15 languages by translating the validation and test sets manually into each of those languages.
% The English training set is machine translated for all languages.
% The dataset is composed of 122k train, 2490 valid and 5010 test examples.
% As usual, NLI performance is evaluated using accuracy.


%%%%%%%%%%%%%%%%%%%%%%%%%%%%%%%%%%%%%%%%%%%%%%%%%%%%%%%%%%%%%%%%%%%%%%%%
\section{Evaluations and Downstream Tasks Related Work}
%%%%%%%%%%%%%%%%%%%%%%%%%%%%%%%%%%%%%%%%%%%%%%%%%%%%%%%%%%%%%%%%%%%%%%%%

\section{The original named entity FTB layer}
\label{subsec:originalannotations}


\newcite{sagot-etal-2012-annotation} annotated the FTB with the span, absolute type\footnote{
    Every mention of \emph{France} is annotated as a \texttt{Location} with subtype \texttt{Country}, as given in \aleda database, even if in context the mentioned entity is a political organization, the French people, a sports team, etc.}, sometimes subtype and \aleda unique identifier of each named entity mention.\footnote{Only proper nouns are considered as named entity mentions, thereby excluding other types of referential expressions.} Annotations are restricted to person, location, organization and company names, as well as a few product names.\footnote{More precisely, we used a tagset of 7 base NE types: \texttt{Person}, \texttt{Location}, \texttt{Organization}, \texttt{Company}, \texttt{Product}, \texttt{POI} (Point of Interest) and \texttt{FictionChar}.} There are no nested entities. Non capitalized entity mentions (e.g.~\emph{banque mondiale} `World Bank') are annotated only if they can be disambiguated independently of their context. Entity mentions that require the context to be disambiguated (e.g.~\emph{Banque centrale}) are only annotated if they are capitalized.
\footnote{So for instance, in \emph{université de Nantes} `Nantes university', only \emph{Nantes} is annotated, as a city, as \emph{université} is written in lowercase letters. However, \emph{Université de Nantes} `Nantes University' is wholly annotated as an organization. It is non-ambiguous because \emph{Université} is capitalized. \emph{Université de Montpellier} `Montpellier University' being ambiguous when the text of the FTB was written and when the named entity annotations were produced, only \emph{Montpellier} is annotated, as a city.}
For person names, grammatical or contextual words around the mention are not included in the mention (e.g.~in \emph{M.~Jacques Chirac} or \emph{le Président Jacques Chirac}, only \emph{Jacques Chirac} is included in the mention).


Tags used for the annotation have the following information:
\begin{itemize}
    \item the identifier of the NE in the \aleda database (\texttt{eid} attribute); when a named entity is not present in the database, the identifier is \texttt{null},\footnote{Specific conventions for entities that have merged, changed name, ceased to exist as such (e.g.~\emph{Tchequoslovaquie}) or evolved in other ways are described in \newcite{sagot-etal-2012-annotation}.}
    \item the normalized named of the named entity as given in \aleda; for locations it is their name as given in GeoNames and for the others, it is the title of the corresponding French Wikipedia article,
    \item the type and, when relevant, the subtype of the entity.
\end{itemize}
Here are two annotation examples:\\
\noindent{\small\texttt{<ENAMEX type="Organization" eid="1000000000016778"
        name="Confédération\\
        française démocratique du travail">CFDT</ENAMEX>\\
        <ENAMEX type="Location"
        sub\_type="Country"
        eid="2000000001861060"\\
        name="Japan">Japon</ENAMEX>}}

\newcite{sagot-etal-2012-annotation} annotated the 2007 version of the FTB treebank (with the exception of sentences that did not receive any functional annotation), i.e.~12,351 sentences comprising 350,931 tokens. The annotation process consisted in a manual correction and validation of the output of a rule- and heuristics-based named entity recognition and linking tool in an XML editor.
Only a single person annotated the corpus, despite the limitations of such a protocol, as acknowledged by \newcite{sagot-etal-2012-annotation}.

In total, 5,890 of the 12,351 sentences contain at least a named entity mention. 11,636 mentions were annotated, which are distributed as follows:
3,761 location names, 3,357 company names, 2,381 organization names, 2,025 person names, 67 product names, 29 fiction character names and 15 points of interest.

\subsection{Evaluation Tasks}\label{MethodEVAL}


\paragraph{Syntactic tasks}
The evaluation tasks were selected to probe to what extent corpus "representativeness" and balance is impacting syntactic representations, in both (1) low-level syntactic relations in POS-tagging tasks, and (2) higher level syntactic relations at constituent- and sentence-level thanks to dependency-parsing evaluation task. Namely, POS-tagging is a low-level syntactic task, which consists in assigning to each word its corresponding grammatical category. Dependency-parsing consists of higher order syntactic task like predicting the labeled syntactic tree capturing the syntactic relations between words.
We evaluate the performance of our models using the standard UPOS accuracy for POS-tagging, and Unlabeled Attachment Score (UAS) and Labeled Attachment Score (LAS) for dependency parsing. We assume gold tokenisation and gold word segmentation as provided in the UD treebanks.
%Additionally, we include a contrast for the two corpora that are comparable in size on Language model perplexities, namely FrWiki and \Cabernet.

\paragraph{Lexical tasks}
To test for word-level representation obtained through the different pre-training corpora and fine-tunings, Named Entity Recognition task (NER) was retained (\ref{ner-section}). As it involves a sequence labeling task that consists in predicting which words refer to real-world objects, such as people, locations, artifacts and organizations, it directly probes the quality and specificity of semantic representations issued by the more or less balanced corpora under comparison.

%\notemumu{@All : est-ce qu eje peux dire ça ? Cette interprétation est-elle correcte ?}


%%%%%%%%%%%%%%%%%%%%%%%%%%%%%%%%%%%%%%%%%%%%%%%
%%%%%%%%%%%%%%%%%%%%%%%%%%%%%%%%%%%%%%%%%%%%%%%
\subsubsection{POS-tagging and dependency parsing}

%To build a state-of-the at baseline, we fist evaluate \camembert 



Different terms of comparisons were considered on the two downstream tasks of part-of-speech (POS) tagging and dependency parsing.
%%%%%%%%%%%%%%%%%%%%%%%%%%%%%%%%%%%%%%%%%%%%%%%
\paragraph{Treebanks test data-set}
We perform our work on the four freely available French UD treebanks in UD~v2.2: GSD, Sequoia, Spoken, and ParTUT, presented in Table \ref{treebanks-tab}.

\textbf{GSD} treebank \citep{mcdonald-etal-2013-universal} is the second-largest tree-bank available for French after the FTB (described in subsection \ref{ner-section}), it contains data from blogs, news, reviews, and Wikipedia.

\textbf{Sequoia} tree-bank \citep{candito-etal-2014-deep} comprises more than 3000 sentences, from the French Europarl, the regional newspaper \emph{L’Est Républicain}, the French Wikipedia and documents from the European Medicines Agency.

\textbf{Spoken} was automatically converted from the Rhapsodie tree-bank \citep{lacheret-etal-2014-rhapsodie} with manual corrections. It consists of 57 sound samples of spoken French with phonetic transcription aligned with sound (word boundaries, syllables, and phonemes), syntactic and prosodic annotations.

Finally, \textbf{ParTUT} is a conversion of a multilingual parallel treebank developed at the University of Turin, and consisting of a variety of text genres, including talks, legal texts, and Wikipedia articles, among others; ParTUT data is derived from the already-existing parallel treebank, Par(allel)TUT \citep{sanguinetti-Bosco-2015-parttut}. Table~\ref{treebanks-tab} contains a summary comparing the sizes of the treebanks.




\subsection{Downstream evaluation tasks}

In this section, we present the four downstream tasks that we use to evaluate \camembert, namely: Part-Of-Speech (POS) tagging, dependency parsing, Named Entity Recognition (NER) and Natural Language Inference (NLI). We also present the baselines that we will use for comparison.

%\lm{Merge all tasks together}
%\subsection{Part-of-speech tagging and dependency parsing}\label{subsection:pos_and_dp}

\paragraph{Tasks} POS tagging is a low-level syntactic task, which consists in assigning to each word its corresponding grammatical category. Dependency parsing consists in predicting the labeled syntactic tree in order to capture the syntactic relations between words.

For both of these tasks we run our experiments using the Universal Dependencies (UD)\footnote{\url{https://universaldependencies.org}} framework and its corresponding UD POS tag set \citep{petrov-etal-2012-universal} and UD treebank collection \citep{nivre-etal-2018-universal}, which was used for the CoNLL 2018 shared task \citep{seker-etal-2018-universal}. We perform our evaluations on the four freely available French UD treebanks in UD~v2.2: GSD \citep{mcdonald-etal-2013-universal}, Sequoia\footnote{\url{https://deep-sequoia.inria.fr}} \citep{candito-seddah-2012-le,candito-etal-2014-deep}, Spoken \citep{lacheret-etal-2014-rhapsodie,bawden-etal-2014-correcting}\footnote{Speech transcript uncased that includes annotated disfluencies without punctuation}, and ParTUT \cite{sanguinetti-Bosco-2015-parttut}. A brief overview of the size and content of each treebank can be found in Table \ref{treebanks-tab}.

\begin{table}[ht]
    \centering\small
        \begin{tabular}{lccl}
            \toprule
            Treebank                         & \#Tokens                         & \#Sentences                     & \multicolumn{1}{l}{Genres} \\
            \midrule
                                             &                                  &                                 & Blogs, News                \\
            \multirow{-2}{*}[1.5pt]{GSD}     & \multirow{-2}{*}[1.5pt]{389,363} & \multirow{-2}{*}[1.5pt]{16,342} & Reviews, Wiki              \\ \tabucline[\hbox {$\scriptstyle \cdot$}]{-}
                                             &                                  &                                 & Medical, News              \\
            \multirow{-2}{*}[0.7pt]{Sequoia} & \multirow{-2}{*}[0.7pt]{68,615}  & \multirow{-2}{*}[0.7pt]{3,099}  & Non-fiction, Wiki          \\ \tabucline[\hbox {$\scriptstyle \cdot$}]{-}
            Spoken                           & 34,972                           & 2,786                           & Spoken                     \\ \tabucline[\hbox {$\scriptstyle \cdot$}]{-}
            ParTUT                           & 27,658                           & 1,020                           & Legal, News, Wikis         \\ \tabucline[\hbox {$\scriptstyle \cdot$}]{-}
            FTB                              & 350,930                          & 27,658                          & News                       \\
            \bottomrule
        \end{tabular}
    \caption{Statistics on the treebanks used in POS tagging, dependency parsing, and NER (FTB).}\label{treebanks-tab}
\end{table}

We also evaluate our model in NER, which is a sequence labeling task predicting which words refer to real-world objects, such as people, locations, artifacts and organisations. We use the French Treebank\footnote{This dataset has only been stored and used on Inria's servers after signing the research-only agreement.} (FTB) \citep{abeille-etal-2003-building} in its 2008 version introduced by \citet{candito-crabbe-2009-improving} and with NER annotations by \citet{sagot-etal-2012-annotation}. The FTB contains more than 11 thousand entity mentions distributed among 7 different entity types. A brief overview of the FTB can also be found in Table \ref{treebanks-tab}.

Finally, we evaluate our model on NLI, using the French part of the XNLI dataset \cite{conneau-etal-2018-xnli}. NLI consists in predicting whether a hypothesis sentence is entailed, neutral or contradicts a premise sentence. The XNLI dataset is the extension of the Multi-Genre NLI (MultiNLI) corpus \cite{williams-etal-2018-broad} to 15 languages by translating the validation and test sets manually into each of those languages.
The English training set is machine translated for all languages other than English.
The dataset is composed of 122k train, 2490 development and 5010 test examples for each language. As usual, NLI performance is evaluated using accuracy.

%%%%%%%%%%%%%%%%%%%%%%%%%%%%%%%%%%%%%%%%%%%%%%%%%%%%%%%%%%%%%%%%%%%%%%%%
\section{FTB Related Work}
%%%%%%%%%%%%%%%%%%%%%%%%%%%%%%%%%%%%%%%%%%%%%%%%%%%%%%%%%%%%%%%%%%%%%%%%


\subsection{Brief state of the art of NER}
\label{subsec:sota}

As mentioned above, NER was first addressed using rule-based approaches, followed by statistical and now neural machine learning techniques. In addition, many systems use a lexicon of named entity mentions, usually called a ``gazetteer'' in this context.

Most of the advances in NER  have been achieved on English, in particular with the CoNLL 2003 \cite{tjong-kim-sang-de-meulder-2003-introduction} and  Ontonotes~v5 \cite{pradhan-etal-2012-conll,pradhan-etal-2013-towards} corpora. In recent years, NER was traditionally tackled using Conditional Random Fields (CRF) \cite{lafferty-etal-2001-conditional} which are quite suited for NER; CRFs were later used as decoding layers for Bi-LSTM architectures \cite{huang-etal-2015-bidirectional,lample-etal-2016-neural} showing considerable improvements over CRFs alone. These Bi-LSTM-CRF architectures were later enhanced with contextualized word embeddings which yet again brought major improvements to the task \cite{peters-etal-2018-deep,akbik-etal-2018-contextual}. Finally, large pre-trained architectures settled the current state of the art showing a small yet important improvement over previous NER-specific architectures \cite{devlin-etal-2019-bert,baevski-etal-2019-cloze}.

For French, rule-based system have been developed until relatively recently, due to the lack of proper training data \cite{sekine-nobata-2004-definition,rosset-etal-2005-interaction,stern-sagot-2010-resources,nouvel-etal-2014-pattern}. The limited availability of a few annotated corpora (cf.~Section~\ref{sec:intro}) made it possible to apply statistical machine learning techniques \cite{bechet-charton-2010-unsupervised,dupont-tellier-2014-named,dupont-2017-exploration} as well as hybrid techniques combining handcrafted grammars and machine learning \cite{bechet-etal-2011-cooperation}. To the best of our knowledge, the best results previously published on FTB NER are those obtained by \newcite{dupont-2017-exploration}, who trained both CRF and BiLSTM-CRF architectures and improved them using heuristics and pre-trained word embeddings. We use this system as our strong baseline.

Leaving aside French and English, the CoNLL 2002 shared task included NER corpora for Spanish and Dutch corpora \cite{tjong-kim-sang-2002-introduction} while the CoNLL 2003 shared task included a German corpus \cite{tjong-kim-sang-de-meulder-2003-introduction}. The recent efforts by \newcite{strakova-etal-2019-neural} settled the state of the art for Spanish and Dutch, while \newcite{akbik-etal-2018-contextual} did so for German.


\subsubsection{Named Entity Recognition}\label{ner-section}
\label{evalner}

%%%%%%%%%%%%%%%%%%%%%%%%%%%%%%%%%%%%%%%%%%%%%%%
\paragraph{Treebanks test data-set}
The benchmark data set from the French Treebank (FTB)  \citep{abeille-etal-2003-building} was selected in its 2008 version, as introduced by \citet{candito-crabbe-2009-improving} and complemented with NER annotations by \citet{sagot-etal-2012-annotation}\footnote{The NER-annotated FTB contains approximately than 12k sentences, and more than 350k tokens were extracted from articles of \emph{Le Monde} newspaper (1989 - 1995). As a whole, it encompasses 11,636 entity mentions distributed among 7 different types : 2025 mentions of ``Person'', 3761 of ``Location'', 2382 of ``Organisation'', 3357 of ``Company'', 67 of ``Product'', 15 of ``POI'' (Point of Interest) and 29 of ``Fictional Character''.}.
The tree-bank, shows a large proportion of the entity mentions that are multi-word entities. We therefore report the three metrics that are commonly used to evaluate models: precision, recall, and F1 score.

%%%%%%%%%%%%%%%%%%%%%%%%%%%%%%%%%%%%%%%%%%%%%%%%%%%%%%%%%%%%%%%%%%%%%%%%
\section{SinNER Related Work}
%%%%%%%%%%%%%%%%%%%%%%%%%%%%%%%%%%%%%%%%%%%%%%%%%%%%%%%%%%%%%%%%%%%%%%%%

\subsection{Related Work on Named Entity Recognition}
\label{sec:sota}

Named Entity Recognition came into light as a prerequisite for designing robust Information Extraction (IE) systems in the MUC conferences \cite{grishman-sundheim-1995-design}. This task soon began to be treated independently from IE since it can serve multiple purposes, like Information retrieval or Media Monitoring for instance \cite{yangarber-etal-2002-unsupervised}. As such, shared task specifically dedicated to NER started to rise like the CoNLL 2003 shared task \cite{tjong-kim-sang-de-meulder-2003-introduction}. Two main paths were followed by the community: (i) since NER was at first used for general purposes, domain extension start to gain interest \cite{evans-2003-a}; (ii) since the majority of NER systems were designed for English, the extension to novel languages (including low resource languages) became of importance \cite{rossler-2004-adapting}.

One can say that NER followed the different trends in NLP. The first approaches were based on gazeeters and handcrafted rules. Initially NER was considered to be solved by a patient process involving careful syntactic analysis \cite{hobbs-1993-generic}. Supervised learning approaches came to fashion with the increase of available data and the rise of shared tasks on NER. Decision trees and Markov models were soon outperformed by Condition Random Fields (CRF).
%By taking advantage of the sequentiality of textual data, CRF helped to set new state-of-the-art results in the domain \cite{finkel-etal-2005-incorporating}.
Thanks to its ability to model dependencies and to take advantage of the sequentiality of textual data, CRF helped to set new state-of-the-art results in the domain \cite{finkel-etal-2005-incorporating}.
Since supervised learning results were bound by the size of training data, lighter approaches were tested in the beginning of the 2000's, among them we can cite weakly supervision \cite{yangarber-2003-counter} and active learning \cite{shen-etal-2004-multi}.

During a time, most of promising approaches involved an addition to improve CRFs : word embeddings \cite{passos-etal-2014-lexicon}, (bi-)LSTMs \cite{lample-etal-2016-neural} % \cite{Ma-2016}
or contextual embeddings \cite{peters-etal-2018-deep}.
More recently, the improvements in contextual word embeddings made the CRFs disappear as standalone models for systems reaching state-of-the-art results, see \cite{stanislawek-etal-2019-named} for a review on the subject and a very interesting discussion on the limits attained by state-of-the-art systems, the \textit{Glass Ceiling}.

%%%%%%%%%%%%%%%%%%%%%%%%%%%%%%%%%%%%%%%%%%%%%%%%%%%%%%%%%%%%%%%%%%%%%%%%
\section{D'AlemBERT Related Work}
%%%%%%%%%%%%%%%%%%%%%%%%%%%%%%%%%%%%%%%%%%%%%%%%%%%%%%%%%%%%%%%%%%%%%%%%

Concerning language modelling in French, two main models are available for contemporary French, \camembert \cite{martin-etal-2020-camembert} and FlauBERT \cite{le-etal-2020-flaubert}. \camembert was trained on a freely available, automatically web-crawled corpus called OSCAR \cite{ortiz-suarez-etal-2019-asynchronous,ortiz-suarez-etal-2020-monolingual} while FlauBERT was trained on a mix of web-crawled data and manually curated (partly non freely available) contemporary French corpora. Neither of these models was explicitly pre-trained for historical French.\footnote{Note however that texts in Old, Middle and Modern French do exist on the internet, and might have found their way to the training corpus of these two models. This is especially the case for Modern French texts, which automatic language classification tools can easily classify as Contemporary French.} However efficient language models have been trained for less-resourced or extinct Languages such as Latin \cite{bamman-burns-2020-latin}, following the approach of \newcite{martin-etal-2020-camembert} for training language models with less data than was previously thought. There have also been some recent projects that specifically target Early Modern French such as that of \pieextended \cite{clerice-2020-pie} that uses the hierarchical encoding architecture originally proposed by \newcite{manjavacas-etal-2019-improving} which itself is constructed by stacking multiple Bi-LSTM-CRFs. \newcite{clerice-2020-pie} distributes pre-trained models for POS tagging and lemmatisation.

%%%%%%%%%%%%%%%%%%%%%%%%%%%%%%%%%%%%%%%%%%%%%%%%%%%%%%%%%%%%%%%%%%%%%%%%
\section{BERTrade Related Work}
%%%%%%%%%%%%%%%%%%%%%%%%%%%%%%%%%%%%%%%%%%%%%%%%%%%%%%%%%%%%%%%%%%%%%%%%
\label{sec-related}
Since the introduction of contextualized word representations \citep{peters-etal-2018-deep,akbik-etal-2018-contextual,devlin-etal-2019-bert} and the many improvements proposed for them in the consumption of computational resources \citep{clark-etal-2020-electra}, in the amount of data required to fine-tune them \citep{raffel-etal-2020-exploring}, and more recently in the length of the contextual window \citep{xiong-etal-2021-nystromformer}; there have also been important advancements from a digital humanities point of view on \emph{unsupervised domain adaptation} \citep{ramponi-plank-2020-neural}. In this case, one specializes a language model to a particular domain with unlabeled data in order to improve performance in downstream tasks. This can be achieved by  pre-training the models from scratch with specialized data \citep{beltagy-etal-2019-scibert} or by continuing the training of a general model with a new corpus \citep{lee-etal-2019-BioBERT, peng-etal-2019-transfer}. This last method has already been successfully implemented in the context of historical languages, in particular \citet{han-eisenstein-2019-unsupervised} showed that one can successfully adapt the original BERT \citep{devlin-etal-2019-bert} to Early Modern English by continuing the pre-training on historical raw texts.

In a multilingual context, transformer-based models such as mBERT have been adapted to low-resource languages and evaluated in dependency parsing and POS-tagging showing promising results \citep{chau-etal-2020-parsing, muller-etal-2021-unseen, gururangan-etal-2020-dont, wang-etal-2020-extending}. However, this multilingual approach has also been criticized for favoring monolingual pre-training even when data is scarce \citep{virtanen-etal-2019-multilingual, ortiz-suarez-etal-2020-monolingual}. Indeed, even when only small pre-training corpora are available, BERT-like models have also been successfully pre-trained, resulting in well-performing models \citep{micheli-etal-2020-importance}. Furthermore, compact BERT-like models have also been studied \citep{turc-etal-2019-well} and might prove useful in data constrained conditions, such as monolingual pre-training of contextualized word representation for low-resource languages.

Regarding corpora for historical languages, very few of them have manually annotated syntactical resources for their medieval states. English has three such treebanks \citep{oxford-2001-the,kroch-etal-2000-the,traugott-pintzuk-2008-coding} for Old and Middle English. The TOROT treebank for Old Church Slavonic, Old East Slavonic and Middle Russian is another large resource \citep{berdicevskis-eckhoff-2020-diachronic}. There is a treebank for Medieval Latin as well, the \emph{Index Thomisticus Treebank} \citep{passarotti-2019-project}. To our knowledge, the last large treebank containing medieval texts is IcePaHC for Icelandic \citep{rognvaldsson-etal-2012-icelandic}. Some other corpora were annotated automatically in order to reduce the cost of annotation. For example, \citet{rocio-etal-2003-automated} adapted a parsing pipeline for contemporary Portuguese and \citet{lee-kong-2014-a} used a previously annotated treebank \citep{lee-kong-2012-dependency} to parse a larger medieval Chinese corpus. Concerning contemporary regional Romance languages, \citet{miletic-etal-2020-building} also used a smaller treebank to generate new annotations, and concluded that using similar languages to train a model does not improve parsing. Although there are many resources for Latin, and some for Ancient Greek, we do not include them here, because they do not face the same challenges as medieval states of language, in particular the high level of spelling variability.

Lastly, concerning dependency parsing and POS-tagging of Old French in particular, the works of \citet{guibon-etal-2014-parsing} and \citet{stein-2014-parsing, stein-2016-old} are noteworthy. However, they use very different approaches to the one used in this paper and evaluate on previous versions of SRCMF, with incompatible annotation choices and slightly different texts. For the UD version of SRCMF, the most notable work is that of the winner of the \emph{CoNLL 2018 Shared Task} \citep{zeman-etal-2018-conll}, UDPipe 2.0 \citep{straka-2018-udpipe}, which was later enhanced by including contextualized word embeddings \citep{straka-strakova-2019-evaluating}.


\part{OSCAR}\label{part:oscar}
%%%%%%%%%%%%%%%%%%%%%%%%%%%%%%%%%%%%%%%%%%%%%%%%%%%%%%%%%%%%%%%%%%%%%%%%
\chapter{Goclassy and OSCAR 2019}
%%%%%%%%%%%%%%%%%%%%%%%%%%%%%%%%%%%%%%%%%%%%%%%%%%%%%%%%%%%%%%%%%%%%%%%%

\begin{center}
    \begin{minipage}{0.5\textwidth}
        \begin{small}
            In which we present the work of \citet{ortiz-suarez-etal-2019-asynchronous}, introducing the first OSCAR corpus, now known as \emph{OSCAR 2019}, as well as asynchronous pipeline \goclassy that was used to produce OSCAR 2019 and that was specically conceived to be used in low resource infrastructures.
        \end{small}
    \end{minipage}
    \vspace{0.5cm}
\end{center}

%\noindent Common Crawl is a considerably large, heterogeneous multilingual corpus comprised of crawled documents from the internet, surpassing 20TB of data and distributed as a set of more than 50 thousand plain text files where each contains many documents written in a wide variety of languages. Even though each document has a metadata block associated to it, this data lacks any information about the language in which each document is written, making it extremely difficult to use Common Crawl for monolingual applications. We propose a general, highly parallel, multithreaded pipeline to clean and classify Common Crawl by language; we specifically design it so that it runs efficiently on medium to low resource infrastructures where I/O speeds are the main constraint. We develop the pipeline so that it can be easily reapplied to any kind of  heterogeneous corpus and so that it can be parameterised to a wide range of infrastructures. We also distribute a 6.3TB version of Common Crawl, filtered, classified by language, shuffled at line level in order to avoid copyright issues, and ready to be used for NLP applications.

As previously mentioned, back in the fall of 2018 when this Ph.D. started, there was no freely available contemporary French corpus of the size that was thought to be needed at that time in order to train a state-of-the art language model. The only available resources were the French Wikipedia and frWAC \citep{baroni-etal-2009-the}. At the time, the original fastText's language classification pipeline \citep{grave-etal-2018-learning} was recently published, but while \citet{grave-etal-2018-learning} published word embeddings for a wide range of languages using the produced corpus, the corpus itself was never published. We thus decided to reproduce and improve \citet{grave-etal-2018-learning} in order to get enough raw textual French data to train a language model. Given that our pipeline ended up being capable of classifying text in a wide range of languages, we decided to publish to publish a multilingual corpus instead of a monolingual French one. In this chapter we thus lay the details of the \goclassy as well as the first version of the OSCAR corpus. 


\section{An Asynchronous Pipeline}

\begin{figure}
    \centering
    \begin{tikzpicture}[auto,scale=0.75, every node/.style={transform shape},font=\sffamily]
        \tikzstyle{nod}=[minimum width=1.65cm,minimum height=6cm,rectangle,rounded corners=10pt,
        fill=red!20, align=center, text width=1.65cm,text centered]
        \tikzstyle{ft} = [minimum width=1.5cm,minimum height=1cm,rectangle,rounded corners=10pt,
        fill=blue!20, align=center, text width=1.5cm,text centered]
        \tikzstyle{fin}=[minimum width=4cm,minimum height=4cm,rectangle,rounded corners=10pt,
        fill=green!20, align=center, text width=4cm,text centered]
        \tikzstyle{arr}=[->,>=stealth,thick]
        \tikzstyle{arr1}=[->,>=stealth,thick, dashed]

        \node[nod] (CC) at (0,0) {Common Crawl};

        \node[minimum width=2cm, text width=2cm,text centered] (TEX1) at (2.5,3.45) {\small Compressed Files};
        \node (GZ1) at (2.5,2.3) {\Huge \faFileArchive[regular]};
        \node (GZ2) at (2.5,1) {\Huge \faFileArchive[regular]};
        \node (GZ3) at (2.5, -0.3) {\Huge \faFileArchive[regular]};
        \node (DGZ) at (2.5, -1.2) {\Huge $\vdots$};
        \node (GZ4) at (2.5,-2.3) {\Huge \faFileArchive[regular]};


        \node[minimum width=1cm, text width=1cm,text centered] (TEX1) at (4.5,3.45) {\small WET Files};
        \node (T1) at (4.5,2.3) {\Huge \faFile*};
        \node (T2) at (4.5,1) {\Huge \faFile*};
        \node (T3) at (4.5, -0.3) {\Huge \faFile*};
        \node (DT) at (4.5, -1.2) {\Huge $\vdots$};
        \node (T4) at (4.5,-2.3) {\Huge \faFile*};

        \node[ft] (F1) at (7,2.3) {fastText};
        \node[ft] (F2) at (7,1) {fastText};
        \node[ft] (F3) at (7,-0.3) {fastText};
        \node (DF) at (7, -1.2) {\Huge $\vdots$};
        \node[ft] (F4) at (7,-2.3) {fastText};

        \node[minimum width=2.3cm, text width=2.3cm,text centered] (TEX1) at (10.25,3.45) {\small Filtered Files Language Tags};
        \node (TA1) at (10.25,2.3) {\Huge \faFile*[regular] $\,$ \faTags};
        \node (TA2) at (10.25,1) {\Huge \faFile*[regular] $\,$ \faTags};
        \node (TA3) at (10.25, -0.3) {\Huge \faFile*[regular] $\,$ \faTags};
        \node (DTA) at (10.25, -1.2) {\Huge $\vdots$};
        \node (TA4) at (10.25,-2.3) {\Huge \faFile*[regular] $\,$ \faTags};


        \node[minimum width=2.3cm, text width=2.3cm,text centered] (TEX1) at (15,3.45) {\small Files Classified by Language};
        \node[fin] (FF) at (15,0) {};
        \node (TF) at (15,1.3) {\Huge \faLanguage $\,\cdots$\faLanguage};
        \node (TF3) at (15, 0) {\Huge \faLanguage $\,\cdots$\faLanguage};
        \node (TF3) at (15, -1.3) {\Huge \faLanguage $\,\cdots$\faLanguage};


        \draw[arr] (1,2.3)--(GZ1);
        \draw[arr] (1,1)--(GZ2);
        \draw[arr] (1,-0.3)--(GZ3);
        \draw[arr] (1,-2.3)--(GZ4);


        \draw[arr] (GZ1)--(T1);
        \draw[arr] (GZ2)--(T2);
        \draw[arr] (GZ3)--(T3);
        \draw[arr] (GZ4)--(T4);


        \draw[arr1] (5,2.5)--(6.1,2.5);
        \draw[arr1] (5,2.3)--(6.1,2.3);
        \draw[arr1] (5,2.1)--(6.1,2.1);

        \draw[arr1] (5,1.2)--(6.1,1.2);
        \draw[arr1] (5,1)--(6.1,1);
        \draw[arr1] (5,0.8)--(6.1,0.8);

        \draw[arr1] (5,-0.1)--(6.1,-0.1);
        \draw[arr1] (5,-0.3)--(6.1,-0.3);
        \draw[arr1] (5,-0.5)--(6.1,-0.5);

        \draw[arr1] (5,-2.1)--(6.1,-2.1);
        \draw[arr1] (5,-2.3)--(6.1,-2.3);
        \draw[arr1] (5,-2.5)--(6.1,-2.5);


        \draw[arr] (8,2.5)--(9.1,2.5);
        \draw[arr] (8,2.3)--(9.1,2.3);
        \draw[arr] (8,2.1)--(9.1,2.1);

        \draw[arr] (8,1.2)--(9.1,1.2);
        \draw[arr] (8,1)--(9.1,1);
        \draw[arr] (8,0.8)--(9.1,0.8);

        \draw[arr] (8,-0.1)--(9.1,-0.1);
        \draw[arr] (8,-0.3)--(9.1,-0.3);
        \draw[arr] (8,-0.5)--(9.1,-0.5);

        \draw[arr] (8,-2.1)--(9.1,-2.1);
        \draw[arr] (8,-2.3)--(9.1,-2.3);
        \draw[arr] (8,-2.5)--(9.1,-2.5);


        \draw[arr] (TA1.0)--(FF);
        \draw[arr] (TA2.0)--(FF);
        \draw[arr] (TA3.0)--(FF);
        \draw[arr] (TA4.0)--(FF);
    \end{tikzpicture}
    \caption{A scheme of the \goclassy pipeline. The red square represents the Compressed WET files stored on Amazon Web Services. The {\faFileArchive[regular]} icons represent the gzip files stored locally, the {\faFile*} represent one of the 50K WET files. The {\faFile*[regular]} represents the filtered file and the {\faTags} represents a file of language tags, one tag per line in \faFile*[regular]. The {\faLanguage} represents one of the 166 classified files. Each arrow represents an asynchronous non blocking worker and dotted arrows represent a line filtering process.}
    \label{fig:D1}
\end{figure}

We propose a new pipeline derived from the fastText one which we call \goclassy, we reuse the fastText linear classifier \citep{joulin-etal-2016-fasttext, joulin-etal-2017-bag} and the pre-trained fastText model for language recognition \citep{grave-etal-2018-learning}, but we completely rewrite and parallelise their pipeline in an asynchronous manner.

The order of operations is more or less the same as in the fastText pre-processing pipeline but instead of clustering multiple operations into a single blocking process, we launch a worker for each operation and we bound the number of possible parallel operations at a given time by the number of available threads instead of the number of CPUs. We implement \goclassy using the Go programming language\footnote{\url{https://golang.org/}} so we let the Go runtime\footnote{\url{https://golang.org/src/runtime/mprof.go}} handle the scheduling of the processes. Thus in our pipeline we don't have to wait for a whole WET file to download, decompress and classify in order to start downloading and processing the next one, a new file will start downloading and processing as soon as the scheduler is able to allocate a new process.

When using electromechanical mediums of storage, I/O blocking is one of the main problems one encounters. To overcome this, we introduced buffers in all our I/O operations, a feature that is not present in the fastText pre-processing pipeline. We also create, from the start, a file for each of the 176 languages that the pre-trained fastText language classifier is capable of recognising, and we always leave them open, as we find that getting a file descriptor to each time we want to write, if we wanted leave them open just when needed, introduces a big overhead.

We also do the filtering and cleaning processes at line level before feeding each line to the classifier, which makes us create a new filtered file so that we can have a correspondence with the tag file, which in turn will consume more space, but that will also reduce the amount of unnecessary classifications performed by fastText. The filtered and file tags are then read and lines are appended to its corresponding language file. The writing in the classification step is asynchronous, meaning that process writing a line to the filtered files does not wait for the classifier to write a tag on the tag file. Figure \ref{fig:D1} shows the pipeline up to this point.

After all WET files are processed, we then use Isaac Whitfield's deduplication tool runiq\footnote{\url{https://github.com/whitfin/runiq}} which is based on Yann Collet's xxhash64\footnote{\url{https://github.com/Cyan4973/xxHash}}, an extremely fast non-cryptographic hash algorithm that is resistant to collisions. We finally use the Mark Adler's pigz\footnote{\url{https://zlib.net/pigz/}} for data compression, as opposed to the canonical UNIX tools proposed in the original fastText pipeline. We add both tools to our concurrent pipeline, executing multiple instances of them in parallel, in order to ensure we use the most of our available resources at a given time.

Beyond improving the computational time required to classify this corpus, we propose a simple improvement on the cleaning scheme in the fastText pre-processing pipeline. This improvement allows our pipeline to better take into account the multilingual nature of Common Crawl; that is, we count UTF-8 characters instead of bytes for setting the lower admissible bound for the length of a line to be fed into the classifier. This straightforward modification on the fastText pre-processing pipeline assures we take into account the multiple languages present in Common Crawl that use non-ASCII encoded characters.

Given that our implementation is written in Go, we release binary distributions \footnote{\url{https://github.com/oscar-corpus/goclassy}} of \goclassy for all major operating systems. Both pigz and runiq are also available for all major operating systems.

\section{Benchmarks}

\begin{table*}[ht!]
    \centering\small
    \resizebox{\linewidth}{!}{
        \begin{tabular}{lrrrcrrrcrrr}\toprule
                     & \multicolumn{3}{c}{10 files} & \phantom{a}             & \multicolumn{3}{c}{100 files} & \phantom{a} & \multicolumn{3}{c}{200 files}                                                                                                                                        \\
            \cmidrule{2-4} \cmidrule{6-8} \cmidrule{10-12}
                     & \multicolumn{1}{c}{Min}      & \multicolumn{1}{c}{Max} & \multicolumn{1}{c}{Mean}      &             & \multicolumn{1}{c}{Min}       & \multicolumn{1}{c}{Max} & \multicolumn{1}{c}{Mean} &  & \multicolumn{1}{c}{Min} & \multicolumn{1}{c}{Max} & \multicolumn{1}{c}{Mean} \\ \midrule
            \emph{real}                                                                                                                                                                                                                                                                            \\
            fastText & 2m50s                        & 6m45s                   & 3m31s                         &             & 13m46s                        & 38m38s                  & 17m39s                   &  & 26m20s                  & 47m48s                  & 31m4s                    \\
            \goclassy & 1m23s                        & 3m12s                   & 1m42s                         &             & 7m42s                         & 12m43s                  & 9m8s                     &  & 15m3s                   & 15m47s                  & 15m16s                   \\
            \emph{user}                                                                                                                                                                                                                                                                            \\
            fastText & 26m45s                       & 27m2s                   & 26m53s                        &             & 4h21m                         & 4h24m                   & 4h23m                    &  & 8h42m                   & 8h48m                   & 8h45m                    \\
            \goclassy & 10m26s                       & 12m53s                  & 11m0s                         &             & 1h46m                         & 1h54m                   & 1h49m                    &  & 3h37m                   & 3h40m                   & 3h38m                    \\
            \emph{sys}                                                                                                                                                                                                                                                                             \\
            fastText & 40.14s                       & 40.85s                  & 40.56s                        &             & 6m14s                         & 6m17s                   & 6m15s                    &  & 12m26s                  & 12m45s                  & 12m31s                   \\
            \goclassy & 37.34s                       & 45.98s                  & 39.67s                        &             & 5m7s                          & 5m34s                   & 5m16s                    &  & 9m57s                   & 10m14s                  & 10m5s                    \\
            \bottomrule
        \end{tabular}
    }
    \caption{Benchmarks are done using the UNIX \texttt{time} tool, are repeated 10 times each and are done for random samples of 10, 100 and 200 WET files. Only the classifying and filtering part are benchmarked. The table shows the minimum, maximum and mean time for the user, real and sys time over the 10 runs. Here ``fastText'' is used as short for the pipeline.}
    \label{tab:Bench}
\end{table*}

We test both pipelines against one another in an infrastructure using traditional electromechanical storage mediums that are connected to the main processing machine via an Ethernet interface, that is, a low I/O speed environment as compared to an infrastructure where one would have an array of SSDs connected directly to the main processing machine via a high speed interface. We use a machine with an Intel\textsuperscript{\textregistered} Xeon\textsuperscript{\textregistered} Processor E5-2650 2.00 GHz, 20M Cache, and 203.1 GiB of RAM. We make sure that no other processes apart from the benchmark and the Linux system processes are run. We do not include downloading, decompression or deduplication in our benchmarks as downloading takes far too much time, and deduplication and compression were performed with third party tools that don't make part of our main contribution. We are mainly interested in seeing how the way the data is fed to the classifier impacts the overall processing time.

Benchmarks in table \ref{tab:Bench} of our \goclassy pipeline show a drastic reduction in processing time compared to the original fastText prepossessing pipeline. We show that in our particular infrastructure, we are capable of reducing the \emph{real} time as measured by the \texttt{time} UNIX tool almost always by half. The \emph{user} time which represents the amount of CPU time spent in user-mode code (outside the kernel) within the process is almost three times lower for our \goclassy pipeline, this particular benchmark strongly suggest a substantial reduction in energy consumption of \goclassy with respect to the fastText pipeline.

As we understand that even an infrastructure with more than 20TB of free space in traditional electromechanical storage is not available to everyone and we propose a simple parametrization in our pipeline that actively deletes already processed data and that only downloads and decompresses files when needed, thus ensuring that no more than 10TB of storage are used at a given time. We nevertheless note that delaying decompression increases the amount of computation time, which is a trade-off that some users might make as it might be more suitable for their available infrastructure.

\section{OSCAR 2019}

We are aware that some users might not even have access to a big enough infrastructure to run our pipelines or just to store all the Common Crawl data. Moreover, even if previously used and cited in NLP and Machine Learning research, we note that at the time of OSCAR's 2019 publication there was no public distribution of Common Crawl that was filtered, classified by language and ready to use for Machine Learning or NLP applications. We thus decide to publish a pre-processed version of the November 2018 dump of Common Crawl which is comprised of usable data in 166 different languages, we publish\footnote{\url{https://oscar-corpus.com/post/oscar-2019/} and \url{https://huggingface.co/datasets/oscar}} our version under the name OSCAR 2019 which is short for \emph{Open Super-large Crawled Aggregated coRpus} 2019.

After processing all the data with \goclassy, the size of the whole Common Crawl corpus is reduced to 6.3TB, but in spite of this considerable reduction, OSCAR 2019 still dwarved all previously freely available corpora having more 800 billion ``words'' or spaced separated tokens and noting that this in fact is an understatement of how big OSCAR 2019 really is, as some of the largest languages within OSCAR 2019 such as Chinese and Japanese do not use spaces. The sizes in bytes for both the original and the deduplicated versions of OSCAR 2019 can be found in table \ref{tab:langs-goclassy}. OSCAR 2019 is published in both in \emph{unshuffled} and \emph{shuffled} distributions:

\begin{itemize}
    \item The \emph{unshuffled} distribution loosy respects the original documents, this is because by design \goclassy considers that a \emph{document} is a set of contigous lines (i.e. coming from the same url record) that share a language classification. Thus if a url record contains texts in mutiple languages, \goclassy will split this record in mutiple documents. The \emph{documents} here are separated by newlines. This \emph{unshuffled} OSCAR 2019 is distributed from France under a research-ony licese, or from the USA through the \emph{Hugging Face}'s datasets library under the \emph{Creative Commons CC0 license (``no rights reserved'')}\footnote{\url{http://creativecommons.org/publicdomain/zero/1.0/}}. This is in part due to the difference in copyright laws between the US and the EU.
    \item The \emph{shuffled} distribution is takes each language subcorpus of the \emph{unshuffled} distribution of OSCAR 2019 and shuffles it at ine level. There is no concept of document in this distribution of OSCAR 2019. As the original content is not reconstructible, we distribute the shuffled OSCAR 2019 from France under the \emph{Creative Commons CC0 license (``no rights reserved'')}\footnote{\url{http://creativecommons.org/publicdomain/zero/1.0/}}
\end{itemize}

\section{Conclusions}

We are sure that our work will greatly benefit researchers working on an either constrain infrastructure or a low budget setting. We are also confident, that by publishing a classified version of Common Crawl, we will substantially increase the amount of available public data for medium to low resource languages, thus improving and facilitating NLP research for them. Furthermore, as our pipeline speeds-up and simplifies the treatment of Common Crawl, we believe that our contribution can be further parallelised and adapted to treat multiple snapshots of Common Crawl opening the door to what would be otherwise costly diachronic studies of the use of a given language throughout the internet.

Finally, we note that both our proposed pipeline is data independent, which means that they can be reused to process, clean and classify any sort of big multilingual corpus that is available in plain text form and that is UTF-8 encoded; meaning that the impact of our work goes way beyond a single corpus.
\chapter{A Monolingual Approach to Contextualized Word Embeddings}

\begin{center}
    \begin{minipage}{0.66\textwidth}
        \begin{small}
            In which we present the work of \citet{ortiz-suarez-etal-2020-monolingual}, who propose the first evaluation of OSCAR 2019 as a pretraining corpus for language modeling. This evaluation was done by selecting OSCAR subcorpora for 5 morphologically and tipologically different mid-ressource languages and pre-training monolingual ELMo models \citep{peters-etal-2018-deep} for each of them. These ELMo models are then attached to the UDPipe 2.0 architecture \citep{straka-2018-udpipe,straka-strakova-2019-evaluating} and evaluated in dependency parsing and POS tagging.
        \end{small}
    \end{minipage}
    \vspace{0.5cm}
\end{center}

Having released OSCAR 2019, the first thing that we wanted to do with it was to evaluate how good it actually was for what it was mainly intended, that is, the pre-training of contextualized word embeddings that had just become available at the time we started working on OSCAR 2019. Such models included ULMFiT \citep{howard-ruder-2018-universal}, ELMo \citep{peters-etal-2018-deep} and BERT \citep{devlin-etal-2019-bert} at that time. For this first evaluation we decided to

\section{Corpora}
We train ELMo contextualized word embeddings for 5 languages: Bulgarian, Catalan, Danish, Finnish and Indonesian. We train one set of embeddings using only Wikipedia data, and another set using only  Common-Crawl-based OSCAR data. We chose these languages primarily because they are morphologically and typologically different from one another, but also because all of the OSCAR datasets for these languages were of a sufficiently manageable size such that the ELMo pre-training was doable in less than one month. Contrary to \emph{HIT-SCIR} team \citep{che-etal-2018-towards}, we do not impose any cap on the amount of data, and instead use the entirety of Wikipedia or OSCAR for each of our 5 chosen languages.

\subsection{Wikipedia}

\begin{table}[t!]
    \centering\small
    \scalebox{0.91}{
        \begin{tabular}{lrrrr}\toprule
            Language   & \multicolumn{1}{l}{Size} & \multicolumn{1}{l}{\#Ktokens} & \multicolumn{1}{l}{\#Kwords} & \multicolumn{1}{l}{\#Ksentences} \\ \midrule
            Bulgarian  & 609M                     & 64,190                        & 54,748                       & 3,685                            \\
            Catalan    & 1.1G                     & 211,627                       & 179,108                      & 8,293                            \\
            Danish     & 338M                     & 60,644                        & 52,538                       & 3,226                            \\
            Finnish    & 669M                     & 89,580                        & 76,035                       & 6,847                            \\
            Indonesian & 488M                     & 80,809                        & 68,955                       & 4,298                            \\
            \bottomrule
        \end{tabular}
    }
    \caption{Size of Wikipedia corpora, measured in bytes, thousands of tokens, words and sentences.}
    \label{tab:Wikipedia}
\end{table}

Wikipedia is the biggest online multilingual open encyclopedia, comprising more than 40 million articles in 301 different languages. Because articles are curated by language and written in an open collaboration model, its text tends to be of very high-quality in comparison to other free online resources. This is why Wikipedia has been extensively used in various NLP applications \citep{wu-weld-2010-open,mihalcea-2007-using,al-rfou-etal-2013-polyglot,bojanowski-etal-2017-enriching}. We downloaded the XML Wikipedia dumps\footnote{XML dumps from April 4, 2019.} and extracted the plain-text from them using the \texttt{wikiextractor.py} script\footnote{Available \href{https://github.com/attardi/wikiextractor}{here}.} from Giuseppe Attardi. We present the number of words and tokens available for each of our 5 languages in Table \ref{tab:Wikipedia}. We decided against deduplicating the Wikipedia data as the corpora are already quite small. We tokenize the 5 corpora using \emph{UDPipe} \citep{straka-strakova-2017-tokenizing}.

\subsection{OSCAR}

Common Crawl is a non-profit organization that produces and maintains an open, freely available repository of crawled data from the web. Common Crawl's complete archive consists of petabytes of monthly snapshots collected since 2011. \iffalse The snapshots are distributed as raw HTML documents, or as \emph{WET} files which contain the extracted plain text converted to UTF-8, as well as a header containing the metadata of each extracted document.\fi{} Common Crawl snapshots are not classified by language, and contain a certain level of noise (e.g.~one-word ``sentences'' such as ``OK'' and ``Cancel'' are unsurprisingly very frequent).

This is what motivated the creation of the freely available multilingual OSCAR corpus \citep{ortiz-suarez-etal-2019-asynchronous}, extracted from the November 2018 snapshot, which amounts to more than 20 terabytes of plain-text. In order to create OSCAR from this Common Crawl snapshot, \citet{ortiz-suarez-etal-2019-asynchronous}  reproduced the pipeline proposed by \citep{grave-etal-2018-learning} to process, filter and classify Common Crawl. More precisely,  language classification was performed using the \emph{fastText} linear classifier \citep{joulin-etal-2016-fasttext,joulin-etal-2017-bag}, which was trained by \citet{grave-etal-2018-learning} to recognize 176 languages and was shown to have an extremely good accuracy to processing time trade-off. The filtering step as performed by \citet{grave-etal-2018-learning} consisted in only keeping the lines exceeding 100 bytes in length.\footnote{Script available \href{https://github.com/facebookresearch/fastText/blob/master/crawl/process_wet_file.sh}{here}.} However, considering that Common Crawl is a mutilingual UTF-8 encoded corpus, this 100-byte threshold creates a huge disparity between ASCII and non-ASCII encoded languages. The filtering step used to create OSCAR therefore consisted in only keeping the lines containing at least 100 UTF-8-encoded characters. Finally, as in  \citep{grave-etal-2018-learning}, the OSCAR corpus is deduplicated, i.e.~for each language, only one occurrence of a given line is included.

As we did for Wikipedia, we tokenize OSCAR corpora for the 5 languages we chose for our study using UDPipe. Table \ref{tab:CC} provides quantitative information about the 5 resulting tokenized corpora.

We note that the original Common-Crawl-based corpus created by \citet{grave-etal-2018-learning} to train fastText is not freely available. Since running the experiments described in this paper, a new architecture for creating a Common-Crawl-based corpus named CCNet \citep{wenzek-etal-2020-ccnet} has been published, although it includes specialized filtering which might result in a cleaner corpus compared to OSCAR, the resulting CCNet corpus itself was not published. Thus we chose to keep OSCAR as it remains the only very large scale, Common-Crawl-based corpus currently available and easily downloadable.

\begin{table}[t]
    \centering\small
    \scalebox{0.91}{
        \begin{tabular}{lrrrr}\toprule
            Language   & \multicolumn{1}{l}{Size} & \multicolumn{1}{l}{\#Ktokens} & \multicolumn{1}{l}{\#Kwords} & \multicolumn{1}{l}{\#Ksentences} \\ \midrule
            Bulgarian  & 14G                      & 1,466,051                     & 1,268,115                    & 82,532                           \\
            Catalan    & 4.3G                     & 831,039                       & 729,333                      & 31,732                           \\
            Danish     & 9.7G                     & 1,828,881                     & 1,620,091                    & 99,766                           \\
            Finnish    & 14G                      & 1,854,440                     & 1,597,856                    & 142,215                          \\
            Indonesian & 16G                      & 2,701,627                     & 2,394,958                    & 140,138                          \\
            \bottomrule
        \end{tabular}
    }
    \caption{Size of OSCAR subcorpora, measured in bytes, thousands of tokens, words and sentences.}
    \label{tab:CC}
\end{table}

\subsection{Noisiness}

We wanted to address \citep{trinh-le-2018-a} and \citep{radford-etal-2019-language}'s criticisms of Common Crawl, so we devised a simple method to measure how noisy the OSCAR corpora were for our 5 languages. We randomly extract a number of lines from each corpus, such that the resulting random sample contains one million words.\footnote{We remove tokens that are capitalized or contain less than 4 UTF-8 encoded characters, allowing us to remove bias against Wikipedia, which traditionally contains a large quantity of proper nouns and acronyms.} We test if the words are in the corresponding \emph{GNU Aspell}\footnote{\url{http://aspell.net/}} dictionary. We repeat this task for each of the 5 languages, for both the OSCAR and the Wikipedia corpora. We compile in Table \ref{tab:OOV} the number of out-of-vocabulary tokens for each corpora.

As expected, this simple metric shows that in general the OSCAR samples contain more out-of-vocabulary words than the Wikipedia ones. However the difference in magnitude between the two is strikingly lower that one would have expected in view of the criticisms by \citet{trinh-le-2018-a} and \citet{radford-etal-2019-language}, thereby validating the usability of Common Crawl data when it is properly filtered, as was achieved by the OSCAR creators. We even observe that, for Danish, the number of out-of-vocabulary words in OSCAR is lower than that in Wikipedia.

\begin{table}[t]
    \centering\small
    \begin{tabular}{lrr}\toprule
        Language   & \multicolumn{1}{l}{OOV Wikipedia} & \multicolumn{1}{l}{OOV OSCAR} \\ \midrule
        Bulgarian  & 60,879                            & 66,558                        \\
        Catalan    & 34,919                            & 79,678                        \\
        Danish     & 134,677                           & 123,299                       \\
        Finnish    & 266,450                           & 267,525                       \\
        Indonesian & 116,714                           & 124,607                       \\
        \bottomrule
    \end{tabular}
    \caption{Number of out-of-vocabulary words in random samples of 1M words for OSCAR and Wikipedia.}
    \label{tab:OOV}
\end{table}

\section{Experimental Setting}

The main goal of this paper is to show the impact of training data on contextualized word representations when applied in particular downstream tasks. To this end, we train different versions of the \emph{Embeddings from Language Models} (ELMo) \citep{peters-etal-2018-deep} for both the Wikipedia and OSCAR corpora, for each of our selected 5 languages. We save the models' weights at different number of epochs for each language, in order to test how corpus size affect the embeddings and to see whether and when overfitting happens when training elmo on smaller corpora.

We take each of the trained ELMo models and use them in conjunction with the UDPipe 2.0 \citep{straka-2018-udpipe,straka-strakova-2019-evaluating} architecture for dependency parsing and POS-tagging to test our models. We train UDPipe 2.0 using gold tokenization and segmentation for each of our ELMo models, the only thing that changes from training to training is the ELMo model as hyperparameters always remain at the default values (except for number of training tokens) \citep{peters-etal-2018-deep}.

\subsection{Contextualized word embeddings}

\emph{Embeddings from Language Models} (ELMo) \citep{peters-etal-2018-deep} is an LSTM-based language model. More precisely, it uses a bidirectional language model, which combines a forward and a backward LSTM-based language model. ELMo also computes a context-independent token representation via a CNN over characters.

We train ELMo models for Bulgarian, Catalan, Danish, Finnish and Indonesian using the OSCAR corpora on the one hand and the Wikipedia corpora on the other. We train each model for 10 epochs, as was done for the original English ELMo \citep{peters-etal-2018-deep}. We save checkpoints at 1\textsuperscript{st}, 3\textsuperscript{rd} and 5\textsuperscript{th} epoch in order to investigate some concerns about possible overfitting for smaller corpora (Wikipedia in this case) raised by the original ELMo authors.\footnote{\url{https://github.com/allenai/bilm-tf/issues/135}}

\subsection{UDPipe 2.0} \label{udpipe-future}

For our POS tagging and dependency parsing evaluation, we use UDPipe 2.0, which has a freely available and ready to use implementation.\footnote{\url{https://github.com/CoNLL-UD-2018/UDPipe-Future}} This architecture was submitted as a participant to the \emph{2018 CoNLL Shared Task} \citep{zeman-etal-2018-conll}, obtaining the 3\textsuperscript{rd} place in LAS ranking. UDPipe 2.0 is a multi-task model that predicts POS tags, lemmas and dependency trees jointly.

The original UDPipe 2.0 implementation calculates 3 different embeddings, namely:

\begin{itemize}
    \item \emph{Pre-trained word embeddings}: In the original implementation, the Wikipedia version of fastText embeddings is used \citep{bojanowski-etal-2017-enriching}; we replace them in favor of the newer Common-Crawl-based fastText embeddings trained by \citet{grave-etal-2018-learning}.
    \item \emph{Trained word embeddings}: Randomly initialized word representations that are trained with the rest of the network.
    \item \emph{Character-level word embeddings}: Computed using bi-directional GRUs of dimension 256. They represent every UTF-8 encoded character with two 256 dimensional vectors, one for the forward and one for the backward layer. This two vector representations are concatenated and are trained along the whole network.
\end{itemize}

After the CoNLL 2018 Shared Task, the UDPipe 2.0 authors added the option to concatenate contextualized representations to the embedding section of the network \citep{straka-strakova-2019-evaluating}, we use this new implementation and we concatenate our pretrained deep contextualized ELMo embeddings to the three embeddings mentioned above.

Once the embedding step is completed, the concatenation of all vector representations for a word are fed to two shared bidirectional LSTM \citep{hochreiter-schmidhuber-1997-long} layers. The output of these two BiLSTMS is then fed to two separate specific LSTMs:
\begin{itemize}
    \item The tagger- and lemmatizer-specific bidirectional LSTMs, with Softmax classifiers on top, which process its output and generate UPOS, XPOS, UFeats and Lemmas. The lemma classifier also takes the character-level word embeddings as input.

    \item The parser-specific bidirectional LSTM layer, whose output is then passed to a bi-affine attention layer \citep{dozat-manning-2017-deep} producing labeled dependency trees.
\end{itemize}

\subsection{Treebanks}

\begin{table}[t!]
    \centering\small
    \begin{tabular}{lrr}\toprule
        Treebank       & \multicolumn{1}{l}{\#Ktokens} & \multicolumn{1}{l}{\#Ksentences} \\ \midrule
        Bulgarian-BTB  & 156                           & 11                               \\
        Catalan-AnCora & 530                           & 17                               \\
        Danish-DDT     & 100                           & 6                                \\
        Finnish-FTB    & 159                           & 19                               \\
        Finnish-TDT    & 202                           & 15                               \\
        Indonesian-GSD & 121                           & 6                                \\
        \bottomrule
    \end{tabular}
    \caption{Size of treebanks, measured in thousands of tokens and sentences.}
    \label{tab:treeb}
\end{table}

To train the selected parser and tagger (cf.~Section \ref{udpipe-future}) and evaluate the pre-trained language models in our 5 languages, we run our experiments using the Universal Dependencies (UD)\footnote{\url{https://universaldependencies.org}} paradigm and its corresponding UD POS tag set \citep{petrov-etal-2012-universal}. We use all the treebanks available for our five languages in the UD treebank collection version 2.2 \citep{nivre-etal-2018-universal}, which was used for the CoNLL 2018 shared task, thus we perform our evaluation tasks in 6 different treebanks (see Table~\ref{tab:treeb} for treebank size information).
\begin{itemize}
    \item \emph{Bulgarian BTB}: Created at the Institute of Information and Communication Technologies, Bulgarian Academy of Sciences, it consists of legal documents, news articles and fiction pieces.
    \item \emph{Catalan-AnCora}: Built on top of the Spanish-Catalan \emph{AnCora corpus} \citep{taule-etal-2008-ancora}, it contains mainly news articles.
    \item \emph{Danish-DDT}: Converted from the \emph{Danish Dependency Treebank} \citep{buch-kromann-2003-the}. It includes news articles, fiction and non fiction texts and oral transcriptions.
    \item \emph{Finnish-FTB}: Consists of manually annotated grammatical examples from VISK\footnote{\url{http://scripta.kotus.fi/visk}} (The Web Version of the Large Grammar of Finnish).
    \item \emph{Finnish-TDT}: Based on the Turku Dependency Treebank (TDT). Contains texts from Wikipedia, Wikinews, news articles, blog entries, magazine articles, grammar examples, Europarl speeches, legal texts and fiction.
    \item \emph{Indonesian-GSD}: Includes mainly blog entries and news articles.
\end{itemize}


\section{Results \& Discussion}


\begin{table}[ht!]
    \centering\small
    \resizebox{0.45\linewidth}{!}{
        \begin{tabular}{@{}llccc@{}}\toprule
            Treebank       & Model       & UPOS              & UAS               & LAS               \\
            \midrule
                           & UDify       & 98.89             & 95.54             & 92.40             \\
                           & UDPipe 2.0  & 98.98             & 93.38             & 90.35             \\
            Bulgarian BTB  & +mBERT      & \underline{99.20} & \underline{95.34} & \underline{92.62} \\
                           & +\elmowiki  & 99.17             & 94.93             & 92.05             \\
                           & +\elmooscar & \textbf{99.40}    & \textbf{96.01}    & \textbf{93.56}    \\
            \midrule
                           & UDify       & 98.89             & \underline{94.25} & 92.33             \\
                           & UDPipe 2.0  & 98.88             & 93.22             & 91.06             \\
            Catalan-AnCora & +mBERT      & \textbf{99.06}    & \textbf{94.49}    & \underline{92.74} \\
                           & +\elmowiki  & \underline{99.05} & 93.99             & 92.24             \\
                           & +\elmooscar & \textbf{99.06}    & \textbf{94.49}    & \textbf{92.88}    \\
            \midrule
                           & UDify       & 97.50             & 87.76             & 84.50             \\
                           & UDPipe 2.0  & 97.78             & 86.88             & 84.31             \\
            Danish-DDT     & +mBERT      & 98.21             & \underline{89.32} & \underline{87.24} \\
                           & +\elmowiki  & \underline{98.45} & 89.05             & 86.92             \\
                           & +\elmooscar & \textbf{98.62}    & \textbf{89.84}    & \textbf{87.95}    \\
            \bottomrule
        \end{tabular}
    }
    ~
    \resizebox{0.45\linewidth}{!}{
        \begin{tabular}{@{}llccc@{}}\toprule
            Treebank       & Model       & UPOS              & UAS               & LAS               \\
            \midrule
                           & UDify       & 93.80             & 86.37             & 81.40             \\
                           & UDPipe 2.0  & 96.65             & 90.68             & 87.89             \\
            Finnish-FTB    & +mBERT      & 96.97             & 91.68             & 89.02             \\
                           & +\elmowiki  & \underline{97.27} & \underline{92.05} & \underline{89.62} \\
                           & +\elmooscar & \textbf{98.13}    & \textbf{93.81}    & \textbf{92.02}    \\
            \midrule
                           & UDify       & 94.43             & 86.42             & 82.03             \\
                           & UDPipe 2.0  & 97.45             & 89.88             & 87.46             \\
            Finnish-TDT    & +mBERT      & 97.57             & \underline{91.66} & \underline{89.49} \\
                           & +\elmowiki  & \underline{97.65} & 91.60             & 89.34             \\
                           & +\elmooscar & \textbf{98.36}    & \textbf{93.54}    & \textbf{91.77}    \\
            \midrule
                           & UDify       & 93.36             & 86.45             & 80.10             \\
                           & UDPipe 2.0  & 93.69             & 85.31             & 78.99             \\
            Indonesian-GSD & +mBERT      & \underline{94.09} & \underline{86.47} & \underline{80.40} \\
                           & +\elmowiki  & 93.94             & 86.16             & 80.10             \\
                           & +\elmooscar & \textbf{94.12}    & \textbf{86.49}    & \textbf{80.59}    \\
            \bottomrule
        \end{tabular}
    }
    \caption{Scores from \mbox{UDPipe~2.0} \protect\citep[from][]{kondratyuk-straka-2019-75}, the previous state-of-the-art models UDPipe 2.0+mBERT \protect\citep{straka-strakova-2019-evaluating} and UDify \protect\citep{kondratyuk-straka-2019-75}, and our ELMo-enhanced UDPipe 2.0 models. Test scores are given for UPOS, UAS and LAS in all five languages. Best scores are shown in bold, second best scores are underlined.}
    \label{tab:parse}
\end{table}

\subsection{Parsing and POS tagging results}
We use UDPipe 2.0 without contextualized embeddings as our baseline for POS tagging and dependency parsing. However, we did not train the model without contextualized word embedding ourselves. We instead take the scores as they are reported in \citep{kondratyuk-straka-2019-75}. We also compare our UDPipe 2.0 + ELMo models against the state-of-the-art results (assuming gold tokenization) for these languages, which are either UDify \citep{kondratyuk-straka-2019-75} or UDPipe 2.0 + mBERT \citep{straka-strakova-2019-evaluating}.

Results for UPOS, UAS and LAS are shown in Table \ref{tab:parse}. We obtain the state of the art for the three metrics in each of the languages with the UDPipe 2.0 + \elmooscar models. We also see that in every single case the UDPipe 2.0 + \elmooscar result surpasses the UDPipe 2.0 + \elmowiki one, suggesting that the size of the pre-training data plays an important role in downstream task results. This is also supports our hypothesis that the OSCAR corpora, being multi-domain, exhibits a better coverage of the different styles, genres and uses present at least in these 5 languages.

Taking a closer look at the results for Danish, we see that \elmowiki, which was trained with a mere 300MB corpus, does not show any sign of overfitting, as the UDPipe 2.0 + \elmowiki results considerably improve the UDPipe 2.0 baseline. This is the case for all of our \elmowiki models as we never see any evidence of a negative impact when we add them to the baseline model. In fact, the results of UDPipe 2.0 + \elmowiki give better than previous state-of-the-art results in all metrics for the Finnish-FTB and in UPOS for the Finnish-TDT. The results for Finnish are actually quite interesting, as mBERT was pre-trained on Wikipedia and here we see that the multilingual setting in which UDify was fine-tuned exhibits sub-baseline results for all metrics, and that the UDPipe + mBERT scores are often lower than those of our UDPipe 2.0 + \elmowiki. This actually suggests that even though the multilingual approach of mBERT (in pre-training) or UDify (in pre-training and fine-tuning) leads to better performance for  high-resource languages or languages that are closely related to high-resource languages, it might also significantly degrade the representations for more isolated or even simply more morphologically rich languages like Finnish. In contrast, our monolingual approach with UDPipe 2.0 + \elmooscar improves the previous SOTA considerably, by more than 2 points for some metrics. Note however that Indonesian, which might also be seen as a relatively isolated language, does not behave in the same way as Finnish.

\subsection{Impact of the number of training epochs}

An important topic we wanted to address with our experiments was that of \emph{overfitting} and the number of epochs one should train the contextualized embeddings for. The ELMo authors have expressed that increasing the number of training epochs is generally better, as they argue that training the ELMo model for longer reduces held-out perplexity and further improves downstream task performance.\footnote{Their comments on the matter can be found \href{https://github.com/allenai/bilm-tf/issues/135}{here}.} This is why we intentionally fully pre-trained the \elmowiki to the 10 epochs of the original ELMo paper, as its authors also expressed concern over the possibility of overfitting for smaller corpora. We thus save checkpoints for each of our ELMo model at the 1, 3, 5 and 10 epoch marks so that we can properly probe for overfitting. The scores of all checkpoints are reported in Table \ref{tab:ablation-monolingual}. Here again we do not train the UDPipe 2.0 baselines without embedding, we just report the scores published in \citet{kondratyuk-straka-2019-75}.

The first striking finding is that even though all our Wikipedia data sets are smaller than 1GB in size (except for Catalan), none of the \elmowiki models show any sign of overfitting, as the results continue to improve for all metrics the more we train the ELMo models, with the best results consistently being those of the fully trained 10 epoch ELMos. For all of our Wikipedia models, but those of Catalan and Indonesian, we see sub-baseline results at 1 epoch; training the model for longer is better, even if the corpora are small in size.

\begin{table}[ht!]
    \centering\small
    \resizebox{0.45\linewidth}{!}{
        \begin{tabular}{@{}llccc@{}}\toprule
            Treebank       & Model            & UPOS              & UAS               & LAS               \\ \midrule
                           & UDPipe 2.0       & 98.98             & 93.38             & 90.35             \\
                           & +\elmowikione    & 98.81             & 93.60             & 90.21             \\
                           & +\elmowikithree  & 99.01             & 94.32             & 91.36             \\
                           & +\elmowikifive   & 99.03             & 94.32             & 91.38             \\
            Bulgarian BTB  & +\elmowikiten    & \underline{99.17} & \underline{94.93} & \underline{92.05} \\
                           & +\elmooscarone   & 99.28             & 95.45             & 92.98             \\
                           & +\elmooscarthree & 99.34             & 95.58             & 93.12             \\
                           & +\elmooscarfive  & 99.34             & 95.63             & 93.25             \\
                           & +\elmooscarten   & \textbf{99.40}    & \textbf{96.01}    & \textbf{93.56}    \\
            \midrule
                           & UDPipe 2.0       & 98.88             & 93.22             & 91.06             \\
                           & +\elmowikione    & 98.93             & 93.24             & 91.21             \\
                           & +\elmowikithree  & 99.02             & 93.75             & 91.93             \\
                           & +\elmowikifive   & 99.04             & 93.86             & 92.05             \\
            Catalan-AnCora & +\elmowikiten    & \underline{99.05} & \underline{93.99} & \underline{92.24} \\
                           & +\elmooscarone   & 99.07             & 93.92             & 92.29             \\
                           & +\elmooscarthree & \textbf{99.10}    & 94.29             & 92.69             \\
                           & +\elmooscarfive  & 99.07             & 94.38             & 92.75             \\
                           & +\elmooscarten   & 99.06             & \textbf{94.49}    & \textbf{92.88}    \\
            \midrule
                           & UDPipe 2.0       & 97.78             & 86.88             & 84.31             \\
                           & +\elmowikione    & 97.47             & 86.98             & 84.15             \\
                           & +\elmowikithree  & 98.03             & 88.16             & 85.81             \\
                           & +\elmowikifive   & 98.15             & 88.24             & 85.96             \\
            Danish-DDT     & +\elmowikiten    & \underline{98.45} & \underline{89.05} & \underline{86.92} \\
                           & +\elmooscarone   & 98.50             & 89.47             & 87.43             \\
                           & +\elmooscarthree & 98.59             & 89.68             & 87.77             \\
                           & +\elmooscarfive  & 98.59             & 89.46             & 87.64             \\
                           & +\elmooscarten   & \textbf{98.62}    & \textbf{89.84}    & \textbf{87.95}    \\
            %\iffalse
            \bottomrule
        \end{tabular}
    }
    ~
    \resizebox{0.45\linewidth}{!}{
        \begin{tabular}{@{}llccc@{}}\toprule
            Treebank       & Model            & UPOS              & UAS               & LAS               \\
            %\fi{}
            \midrule
                           & UDPipe 2.0       & 96.65             & 90.68             & 87.89             \\
                           & +\elmowikione    & 95.86             & 89.63             & 86.39             \\
                           & +\elmowikithree  & 96.76             & 91.02             & 88.27             \\
                           & +\elmowikifive   & 96.97             & 91.66             & 89.04             \\
            Finnish-FTB    & +\elmowikiten    & \underline{97.27} & \underline{92.05} & \underline{89.62} \\
                           & +\elmooscarone   & 97.91             & 93.41             & 91.43             \\
                           & +\elmooscarthree & 98.00             & \textbf{93.99}    & 91.98             \\
                           & +\elmooscarfive  & \textbf{98.15}    & 93.98             & \textbf{92.24}    \\
                           & +\elmooscarten   & 98.13             & 93.81             & 92.02             \\
            \midrule
                           & UDPipe 2.0       & 97.45             & 89.88             & 87.46             \\
                           & +\elmowikione    & 96.73             & 89.11             & 86.33             \\
                           & +\elmowikithree  & 97.55             & 90.84             & 88.50             \\
                           & +\elmowikifive   & 97.55             & 91.11             & 88.88             \\
            Finnish-TDT    & +\elmowikiten    & \underline{97.65} & \underline{91.60} & \underline{89.34} \\
                           & +\elmooscarone   & 98.27             & 93.03             & 91.29             \\
                           & +\elmooscarthree & 98.38             & \textbf{93.60}    & \textbf{91.83}    \\
                           & +\elmooscarfive  & \textbf{98.39}    & 93.57             & 91.80             \\
                           & +\elmooscarten   & 98.36             & 93.54             & 91.77             \\
            \midrule
                           & UDPipe 2.0       & 93.69             & 85.31             & 78.99             \\
                           & +\elmowikione    & 93.70             & 85.81             & 79.46             \\
                           & +\elmowikithree  & 93.90             & 86.04             & 79.72             \\
                           & +\elmowikifive   & 94.04             & 85.93             & 79.97             \\
            Indonesian-GSD & +\elmowikiten    & \underline{93.94} & \underline{86.16} & \underline{80.10} \\
                           & +\elmooscarone   & 93.95             & 86.25             & 80.23             \\
                           & +\elmooscarthree & 94.00             & 86.21             & 80.14             \\
                           & +\elmooscarfive  & \textbf{94.23}    & 86.37             & 80.40             \\
                           & +\elmooscarten   & 94.12             & \textbf{86.49}    & \textbf{80.59}    \\
            \bottomrule
        \end{tabular}
    }
    \caption{UPOS, UAS and LAS scores for the UDPipe 2.0 baseline reported by \protect\citep{kondratyuk-straka-2019-75}, plus the scores for checkpoints at 1, 3, 5 and 10 epochs for all the \elmooscar and \elmowiki. All scores are test scores. Best \elmooscar scores are shown in bold while best \elmowiki scores are underlined.}
    \label{tab:ablation-monolingual}
\end{table}

\elmooscar models exhibit exactly the same behavior as \elmowiki models where the scores continue to improve the longer they are pre-trained, except for the case of Finnish. Here we actually see an unexpected behavior where the model performance caps around the 3\textsuperscript{rd} to 5\textsuperscript{th} epoch. This is surprising because the Finnish OSCAR corpus is more than 20 times bigger than our smallest Wikipedia corpus, the Danish Wikipedia, that did not exhibit this behavior. As previously mentioned Finnish is morphologically richer than the other languages in which we trained ELMo, we hypothesize that the representation space given by the ELMo embeddings might not be sufficiently big to extract more features from the Finnish OSCAR corpus beyond the 5\textsuperscript{th} epoch mark, however in order to test this we would need to train a larger language model like BERT which is sadly beyond our computing infrastructure limits (cf. Subsection \ref{cost}). However we do note that pre-training our current language model architectures in a morphologically rich language like Finnish might actually better expose the limits of our existing approaches to language modeling.

One last thing that it is important to note with respect to the number of training epochs is that even though we fully pre-trained our \elmowikis and \elmooscars to the recommended 10 epoch mark, and then compared them against one another, the number of training steps between both pre-trained models differs drastically due to the big difference in corpus size (for Indonesian, for instance, 10 epochs correspond to 78K steps for \elmowiki and to 2.6M steps for OSCAR; the complete picture is provided in the Appendix, in Table~\ref{tab:steps}). In fact, we can see in Table \ref{tab:ablation-monolingual} that all the UDPipe 2.0 + \elmooscarone perform better than the UDPipe 2.0 + \elmowikione models across all metrics. Thus we believe that talking in terms of training steps as opposed to training epochs might be a more transparent way of comparing two pre-trained models.

\subsection{Computational cost and carbon footprint}\label{cost}

Considering the discussion above, we believe an interesting follow-up to our experiments would be training the ELMo models for more of the languages included in the OSCAR corpus. However training ELMo is computationally costly, and one way to estimate this cost, as pointed out by \citet{strubell-etal-2019-energy}, is by using the training times of each model to compute both power consumption and CO\textsubscript{2} emissions.

\begin{table}[t]
    \centering\small
    \scalebox{0.93}{
        \begin{tabular}{@{}lrrrrr@{}}\toprule
            Language                                        & Power & Hours  & Days  & KWh$\cdotp$PUE & CO\textsubscript{2}e \\
            \midrule
            \multicolumn{6}{l}{\hspace*{6mm}\em OSCAR-Based ELMos}                                                           \\[0.5mm]
            Bulgarian                                       & 1183  & 515.00 & 21.45 & 962.61         & 49.09                \\
            Catalan                                         & 1118  & 199.98 & 8.33  & 353.25         & 18.02                \\
            Danish                                          & 1183  & 200.89 & 8.58  & 375.49         & 19.15                \\
            Finnish                                         & 1118  & 591.25 & 24.63 & 1044.40        & 53.26                \\
            Indonesian                                      & 1183  & 694.26 & 28.93 & 1297.67        & 66.18                \\
            \midrule\multicolumn{6}{l}{\hspace*{6mm}\em Wikipedia-Based ELMos}                                               \\[0.5mm]
            Bulgarian                                       & 1118  & 15.45  & 0.64  & 27.29          & 1.39                 \\
            Catalan                                         & 1118  & 51.08  & 2.13  & 90.22          & 4.60                 \\
            Danish                                          & 1118  & 14.56  & 0.61  & 25,72          & 1.31                 \\
            Finnish                                         & 1118  & 21.79  & 0.91  & 38.49          & 1.96                 \\
            Indonesian                                      & 1118  & 20.28  & 0.84  & 35.82          & 1.82                 \\
            \midrule
            \multicolumn{2}{@{}l}{\textsc{Total emissions}} &       &        &       & 216.78                                \\
            \bottomrule
        \end{tabular}
    }
    \caption{Average power draw (Watts), training times (in both hours and days), mean power consumption (KWh) and CO\textsubscript{2} emissions (kg) for each ELMo model trained.}
    \label{tab:carbon}
\end{table}

In our set-up we used two different machines, each one having 4 NVIDIA GeForce GTX 1080 Ti graphic cards and 128GB of RAM, the difference between the machines being that one uses a single Intel Xeon Gold 5118 processor, while the other uses two Intel Xeon E5-2630 v4 processors. One GeForce GTX 1080 Ti card is rated at around 250 W,\footnote{\url{https://www.geforce.com/hardware/desktop-gpus/geforce-gtx-1080-ti/specifications}} the Xeon Gold 5118 processor is rated at 105 W,\footnote{\url{https://ark.intel.com/content/www/us/en/ark/products/120473/intel-xeon-gold-5118-processor-16-5m-cache-2-30-ghz.html}} while one Xeon E5-2630 v4 is rated at 85 W.\footnote{\url{https://ark.intel.com/content/www/us/en/ark/products/92981/intel-xeon-processor-e5-2630-v4-25m-cache-2-20-ghz.html}} For the DRAM we can use the work of \citet{desrochers-etal-2016-a} to estimate the total power draw of 128GB of RAM at around 13W. Having this information, we can now use the formula proposed by \citet{strubell-etal-2019-energy} in order to compute the total power required to train one ELMo model:
\[
    p_t = \frac{1.58t(cp_{c} + p_r + gp_g)}{1000}
\]
Where $c$ and $g$ are the number of CPUs and GPUs respectively, $p_c$ is the average power draw (in Watts) from all CPU sockets, $p_r$ the average power draw from all DRAM sockets, and $p_g$ the average power draw of a single GPU. We estimate the total power consumption by adding GPU, CPU and DRAM consumptions, and then multiplying by the \emph{Power Usage Effectiveness} (PUE), which accounts for the additional energy required to support the compute infrastructure. We use a PUE coefficient of 1.58, the 2018 global average for data centers \citep{strubell-etal-2019-energy}. In table \ref{tab:carbon} we report the training times in both hours and days, as well as the total power draw (in Watts) of the system used to train each individual ELMo model. We use this information to compute the total power consumption of each ELMo, also reported in table \ref{tab:carbon}.

We can further estimate the CO\textsubscript{2} emissions in kilograms of each single model by multiplying the total power consumption by the average CO\textsubscript{2} emissions per kWh in France (where the models were trained). According to the RTE (Réseau de transport d'électricité / Electricity Transmission Network) the average emission per kWh were around 51g/kWh in November 2019,\footnote{\url{https://www.rte-france.com/fr/eco2mix/eco2mix-co2}} when the models were trained. Thus the total CO\textsubscript{2} emissions in kg for one single model can be computed as:
\[
    \text{CO}_{2}\text{e} = 0.051 p_t
\]
All emissions for the ELMo models are also reported in table \ref{tab:carbon}.

We do not report the power consumption or the carbon footprint of training the UDPipe 2.0 architecture, as each model took less than 4 hours to train on a machine using a single NVIDIA Tesla V100 card. Also, this machine was shared during training time, so it would be extremely difficult to accurately estimate the power consumption of these models.

Even though it would have been interesting to replicate all our experiments and computational cost estimations with state-of-the-art fine-tuning models such as BERT, XLNet, RoBERTa or ALBERT, we recall that these transformer-based architectures are extremely costly to train, as noted by the BERT authors on the official BERT GitHub repository,\footnote{\url{https://github.com/google-research/bert}} and are currently beyond the scope of our computational infrastructure. However we believe that ELMo contextualized word embeddings remain a useful model that still provide an extremely good trade-off between performance to training cost, even setting new state-of-the-art scores in parsing and POS tagging for our five chosen languages, performing even better than the multilingual mBERT model.


\section{Conclusions}

In this paper, we have explored the use of the Common-Crawl-based OSCAR corpora to train ELMo contextualized embeddings for five typologically diverse mid-resource languages. We have compared them with Wikipedia-based ELMo embeddings on two classical NLP tasks, POS tagging and parsing, using state-of-the-art neural architectures. Our goal was to explore whether the noisiness level of Common Crawl data, often invoked to criticize the use of such data, could be compensated by its larger size; for some languages, the OSCAR corpus is several orders of magnitude larger than the corresponding Wikipedia. Firstly, we found that when properly filtered, Common Crawl data is not massively noisier than Wikipedia. Secondly, we show that embeddings trained using OSCAR data consistently outperform Wikipedia-based embeddings, to the extent that they allow us to improve the state of the art in POS tagging and dependency parsing for all the 6 chosen treebanks. Thirdly, we observe that more training epochs generally results in better embeddings even when the training data is relatively small, as is the case for Wikipedia.

Our experiments show that Common-Crawl-based data such as the OSCAR corpus can be used to train high-quality contextualized embeddings, even for languages for which more standard textual resources lack volume or genre variety. This could result in better performances in a number of NLP tasks for many non highly resourced languages.

\subsection*{Acknowledgments}

We want to thank Ganesh Jawahar for his insightful comments and suggestions during the early stages of this project. This work was partly funded by the French national ANR grant BASNUM (\mbox{ANR-18-CE38-0003}), as well as by the last author's chair in the PRAIRIE institute,\footnote{\url{http://prairie-institute.fr/}} funded by the French national ANR as part of the ``Investissements d’avenir'' programme under the reference \mbox{ANR-19-P3IA-0001}. The authors are grateful to Inria Sophia Antipolis - Méditerranée ``Nef''\footnote{\url{https://wiki.inria.fr/wikis/ClustersSophia}} computation cluster for providing resources and support.

\section{Appendix}
\label{sec:appendix}
\subsection{Number of training steps for each checkpoint and each corpus}

\begin{table}[ht!]
    \centering\small
    \scalebox{0.96}{
        \begin{tabular}{@{}lrrrr@{}}\toprule
            Language   & 1 Epoch & 3 Epochs & 5 Epochs  & 10 Epochs        \\
            \midrule
            \multicolumn{5}{l}{\hspace*{6mm}\em Wikipedia-Based ELMos}     \\[0.5mm]
            Bulgarian  & 6,268   & 18,804   & 31,340    & 62,680           \\
            Catalan    & 20,666  & 61,998   & 103,330   & 206,660          \\
            Danish     & 5,922   & 17,766   & 29,610    & 59,220           \\
            Finnish    & 8,763   & 26,289   & 43,815    & 87,630           \\
            Indonesian & 7,891   & 23,673   & 39,455    & 78,910           \\
            \midrule\multicolumn{5}{l}{\hspace*{6mm}\em OSCAR-Based ELMos} \\[0.5mm]
            Bulgarian  & 143,169 & 429,507  & 715,845   & 1,431,690        \\
            Catalan    & 81,156  & 243,468  & 405,780   & 811,560          \\
            Danish     & 81,156  & 243,468  & 405,780   & 811,560          \\
            Finnish    & 181,230 & 543,690  & 906,150   & 1,812,300        \\
            Indonesian & 263,830 & 791,490  & 1,319,150 & 2,638,300        \\
            \bottomrule
        \end{tabular}
    }
    \caption{Number of training steps for each checkpoint, for the \elmowiki and \elmooscar of each language.}
    \label{tab:steps}
\end{table}


\chapter{Quality at Glance}

\begin{center}
    \begin{minipage}{0.66\textwidth}
        \begin{small}
            In which we present the work of \citet{kreutzer-etal-2021-quality}, who propose the first evaluation of OSCAR 2019 as a pretraining corpus for language modeling. This evaluation was done by selecting OSCAR subcorpora for 5 morphologically and tipologically different mid-ressource languages and pre-training monolingual ELMo models \citep{peters-etal-2018-deep} for each of them. These ELMo models are then attached to the UDPipe 2.0 architecture \citep{straka-2018-udpipe,straka-strakova-2019-evaluating} and evaluated in dependency parsing and POS tagging.
        \end{small}
    \end{minipage}
    \vspace{0.5cm}
\end{center}

\section{Auditing Data Quality}\label{sec:audit}
None of the above datasets has been evaluated for quality on the sentence level (exception: several languages in ParaCrawl v3), and downstream evaluations are centered around a small fraction of higher-resource languages. This is insufficient for drawing conclusions about the quality of individual or aligned sentences, and about the entirety of languages. In addition, there might be a publication bias preventing negative results with any of the above corpora with lower quality being published.

To close this gap, we conduct a human data quality audit focused on the lowest-resource and most under-evaluated languages, but also covering mid- and high-resource languages for comparison.


\begin{table*}[th]
    \small
    \centering
    \resizebox{\textwidth}{!}{
    \begin{tabular}{ll}
        \toprule
        \multicolumn{2}{c}{\textbf{Correct Codes}}                                                                                  \\
        \midrule
        \textbf{\texttt{C}}:  \textit{Correct translation, any} & Combined label for \texttt{CC}, \texttt{CB}, \texttt{CS}          \\
        \midrule
        \multicolumn{2}{l}{\textbf{\texttt{CC}:} \textit{Correct translation, natural sentence}}                                    \\
        %	\midrule
        \texttt{en} The Constitution of South Africa            & \texttt{nso} Molaotheo wa Rephabliki ya Afrika Borwa              \\
        \texttt{en} Transforming your swimming pool into a pond & \texttt{de} Umbau Ihres Swimmingpools zum Teich                   \\
        \midrule
        \multicolumn{2}{l}{\textbf{\texttt{CB}:} \textit{Correct translation, Boilerplate or low quality}}                          \\
        %	\midrule
        \texttt{en} Reference number: 13634                     & \texttt{ln} Motango ya référence: 13634                           \\
        \texttt{en} Latest Smell Stop Articles                  & \texttt{fil} Pinakabagong mga Artikulo Smell Stop                 \\
        %  \texttt{en} Weight (Male): 8,6 - 13,5 kg & \texttt{ts} Ntiko (Xinuna): 8, 6 - 13, 5 kg \\
        \midrule
        \multicolumn{2}{l}{\textbf{\texttt{CS}:} \textit{Correct translation, Short}}                                               \\
        % 	\midrule
        \texttt{en} movies, dad                                 & \texttt{it} cinema, pap\`{a}                                      \\
        \texttt{en} Halloween - without me                      & \texttt{ay} Hallowen – janiw nayampejj                            \\
        \midrule
        \midrule
        \multicolumn{2}{c}{\textbf{Error Codes}}                                                                                    \\
        \midrule
        \multicolumn{2}{l}{\textbf{X:} \textit{Incorrect translation, but both correct languages}}                                  \\
        %	\midrule
        \texttt{en} A map of the arrondissements of Paris       & \texttt{kg} Paris kele mbanza ya kimfumu ya Fwalansa.             \\
        \texttt{en} Ask a question                              & \texttt{tr} Soru sor Kullanıma g{\"o}re se\c{c}im                 \\
        \midrule
        \multicolumn{2}{l}{\textbf{\texttt{WL}:} \textit{Source OR target wrong language, but both still linguistic content}}       \\
        % 	\midrule
        \texttt{en} The ISO3 language code is zho               & \texttt{zza} T{\' a}im eadra bracach mar bhionns na frogannaidhe. \\
        %  \texttt{en} sicilianu: Johannesburg & \texttt{ve} Tshivenda: Johannesburg \\
        \texttt{en} Der Werwolf — sprach der gute Mann,         &
        \texttt{de} des Weswolfs, Genitiv sodann,                                                                                   \\
        \midrule
        \multicolumn{2}{l}{\textbf{NL:} \textit{Not a language: at least one of source and target are not linguistic content}}      \\
        %	\midrule
        \texttt{en} EntryScan 4 \_                              & \texttt{tn} TSA PM704 \_                                          \\
        \texttt{en} organic peanut butter                       & \texttt{ckb} \ucr \ucr \ucr \ucr \ucr \ucr \ucr                   \\
        \bottomrule
    \end{tabular}
    }
    \caption{Annotation codes for parallel data with sentence pair examples. The language code before each sentence indicates the language it is supposed to be in.}
    \label{tab:examples}
\end{table*}

\subsection{Auditing Process}

\paragraph{Participants} We recruited 51 volunteers from the NLP community, covering about 70 languages with proficient language skills.\footnote{This surprisingly high number comes in part because there are many closely related languages, e.g. one person may be proficient enough to rate many different Slavic or Turkic languages even if only one is their native language.} Each sentence is annotated by one rater.
To verify our hypothesis that those annotations can largely done by non-native speakers, we repeat a set of language expert annotations by a non-expert, and measure the accuracy of the non-expert.

\paragraph{Sample selection} For each language in each dataset, we took a random sample of 100 lines, which may be anywhere from single words to short paragraphs depending on segmentation.
We manually annotated them according to the error taxonomy described below. For WikiMatrix and CCAligned, we selected those languages that are paired with English, and for ParaCrawl, we also included those paired with Spanish (``total'' counts in Table~\ref{tab:results}).
We did not annotate all languages, but focused on the ones with the least number of sentences in each dataset (at least the smallest 10) and languages for which we found proficient speakers.
Since we annotate the same maximum number of sentences\footnote{Some languages had fewer than 100 sentences.} across all chosen languages regardless of their total number of sentences, the annotated samples are not an unbiased sample from the whole dataset.

\paragraph{Non-expert labeling strategies}
Although many of the volunteers were familiar with the languages in question or spoke related languages, in cases where no speaker of a relevant language could be found, volunteers used dictionaries and internet search to form educated guesses. We discuss this deeper in Appendix~\ref{app:strategies} to highlight how much of this low-resource focused evaluation can actually be done by non-proficient speakers with relatively low effort.
In general, we aim to find an upper bound on quality, so we encouraged annotators to be forgiving of translation mistakes when the overall meaning of the sentence or large parts thereof are conveyed, or when most of the sentence is in the correct language.

\paragraph{Effort} The individual effort was dependent on the quality and complexity of the data, and on the annotator's knowledge of the language(s), e.g., it took from less than two minutes for an English native speaker to pass through 100 well-formed English sentences (or similarly to annotate languages with 0\% in-language sentences), to two hours of ``detective work'' for well-formed content in languages for an annotator without familiarity.

\paragraph{Taxonomy}
In order to quantify errors, we developed a simple error taxonomy. Sentences and sentence pairs were annotated according to a simple rubric with error classes of Incorrect Translation (\texttt{X}, excluded for monolingual data), Wrong Language (\texttt{WL}), and Non-Linguistic Content (\texttt{NL}). Of correct sentences (\texttt{C}), we further mark single words or phrases (\texttt{CS}) and boilerplate contents (\texttt{CB}).
In addition, we asked annotators to flag offensive or pornographic content.
Table \ref{tab:examples} provides examples for parallel data, and Appendix~\ref{app:taxonomy} contains detailed annotation instructions.

\begin{table*}[th]
    \centering \small
    \resizebox{\textwidth}{!}{%
    \begin{tabular}{llccccc}
        \toprule
                                                                  &                             & \multicolumn{3}{c}{\textbf{Parallel}} & \multicolumn{2}{c}{\textbf{Monolingual}}                                                       \\
        \cmidrule(lr){3-5} \cmidrule(lr){6-7}
                                                                  &                             & \textbf{CCAligned}                    & \textbf{ParaCrawl v7.1}                  & \textbf{WikiMatrix} & \textbf{OSCAR} & \textbf{mC4} \\
        \midrule
        \multicolumn{2}{l}{\#langs audited / total}               & 65 / 119                    & 21 / 38                               & 20 / 78                                  & 51 / 166            & 48 / 108                      \\
        \multicolumn{2}{l}{\%langs audited}                       & 54.62\%                     & 55.26\%                               & 25.64\%                                  & 30.72\%             & 44.44\%                       \\
        \multicolumn{2}{l}{\#sents audited / total}               & 8037 / 907M                 & 2214 / 521M                           & 1997 / 95M                               & 3517 / 8.4B         & 5314 / 8.5B                   \\
        \multicolumn{2}{l}{\%sents audited}                       & 0.00089\%                   & 0.00043\%                             & 0.00211\%                                & 0.00004\%           & 0.00006\%                     \\
        \midrule
        \multirow{6}{*}{\rotatebox[origin=c]{90}{\textbf{macro}}} &
        \texttt{C}                                                & 29.25\%                     & 76.14\%                               & 23.74\%                                  & 87.21\%             & 72.40\%                       \\
                                                                  & \texttt{X}                  & 29.46\%                               & 19.17\%                                  & 68.18\%             & -              & -            \\
                                                                  & \texttt{WL}                 & 9.44\%                                & 3.43\%                                   & 6.08\%              & 6.26\%         & 15.98\%      \\
                                                                  & \texttt{NL}                 & 31.42\%                               & 1.13\%                                   & 1.60\%              & 6.54\%         & 11.40\%      \\
                                                                  & offensive                   & 0.01\%                                & 0.00\%                                   & 0.00\%              & 0.14\%         & 0.06\%       \\
                                                                  & porn                        & 5.30\%                                & 0.63\%                                   & 0.00\%              & 0.48\%         & 0.36\%       \\
        \midrule
        \multirow{6}{*}{\rotatebox[origin=c]{90}{\textbf{micro}}} &
        \texttt{C}                                                & 53.52\%                     & 83.00\%                               & 50.58\%                                  & 98.72\%             & 92.66\%                       \\
                                                                  & \texttt{X}                  & 32.25\%                               & 15.27\%                                  & 47.10\%             & -              & -            \\
                                                                  & \texttt{WL}                 & 3.60\%                                & 1.04\%                                   & 1.35\%              & 0.52\%         & 2.33\%       \\
                                                                  & \texttt{NL}                 & 10.53\%                               & 0.69\%                                   & 0.94\%              & 0.75\%         & 5.01\%       \\
                                                                  & offensive                   & 0.00\%                                & 0.00\%                                   & 0.00\%              & 0.18\%         & 0.03\%       \\
                                                                  & porn                        & 2.86\%                                & 0.33\%                                   & 0.00\%              & 1.63\%         & 0.08\%       \\
        \midrule
                                                                  & \#langs =0\% \texttt{C}     & 7                                     & 0                                        & 1                   & 7              & 0            \\
        %& \#langs $<$5\% \texttt{C} & 14 & 0 & 2 & 7  & 0 \\
        %& \#langs $<$20\% \texttt{C} & 27 & 0 & 10 & 7 & 4 \\
                                                                  & \#langs $<$50\% \texttt{C}  & 44                                    & 4                                        & 19                  & 11             & 9            \\
                                                                  & \#langs $>$50\% \texttt{NL} & 13                                    & 0                                        & 0                   & 7              & 1            \\
                                                                  & \#langs $>$50\% \texttt{WL} & 1                                     & 0                                        & 0                   & 3              & 4            \\
        \bottomrule
    \end{tabular}%
    }
    \caption{Averages of sentence-level annotations across datasets and selected languages. Macro-avg: Each language is weighted equally in the aggregation, regardless of its size. Micro-avg: Each label is weighted by the fraction of sentences for that language in the overall annotated corpus, i.e., the annotations for higher-represented languages are upweighted, and annotations for lower-represented languages are downweighted. The bottom rows contain the number of languages that have 0\% labeled \texttt{C} etc. Note that these are not true expectations since the languages audited were not randomly sampled. }
    \label{tab:results}
\end{table*}





\subsection{Human Audit Results}\label{sec:audit-res}


\paragraph{Interpretation of Results}
For each language, we compute the percentage of each label within the 100 audited sentences.
Then, we either aggregate the labels across languages with equal weights (macro-average), or weight them according to their presence in the overall dataset (micro-average). Results are shown in Table~\ref{tab:results}. The statistics for the correct codes (\texttt{CC}, \texttt{CB}, \texttt{CS}) are combined as \texttt{C}.
The number of languages, the numbers of sentences per language and the choice of languages differ across datasets, both in the original release and in the selection for our audit, so the comparison of numbers across datasets has to be taken with a grain of salt. Since the numbers are based on a small sample of sentences that were partially annotated by non-experts, the error statistics are only rough estimates.
Our audit captures a decent ratio of languages (25--55\%, 2nd row in Table~\ref{tab:results}), but only a tiny fraction of the overall number of sentences (0.00004--0.002\%).
When we speak of ``low-'' and ``high''-resource languages, we mean languages with smaller or larger representation in the datasets at hand. When reporting language-specific results we use the original language identifiers of the datasets.



\paragraph{Which datasets have quality issues?}

The macro-averaged results show that the ratio of correct samples (\texttt{C}) ranges from 24\% to 87\%, with a large variance across the five audited datasets.
% Note that mC4 and OSCAR are monolingual datasets, so they do not require a correct alignment to another language to be labeled as correct.
Particularly severe problems were found in CCAligned and WikiMatrix, with 44 of the 65 languages that we audited for CCAligned containing under 50\% correct sentences, and 19 of the 20 in WikiMatrix. In total, 15 of the 205 language specific samples (7.3\%) contained not a single correct sentence.
For the parallel datasets we are also interested in the quantity of misaligned/mistranslated sentences (\texttt{X}). For WikiMatrix, two-thirds of the audited samples were on average misaligned. We noticed that sentences were often similar in structure, but described different facts (see Table~\ref{tab:not_actually_parallel}). This might originate from the nature of the underlying Wikipedia articles, since they are often comparable rather than parallel~\citep{schwenk-etal-2021-wikimatrix}.

%While Table~\ref{tab:results} gives means and numbers of corpora passing certain thresholds, 
Figure~\ref{fig:ratio_c} illustrates per-corpus correctness more completely, showing for each dataset what percent of audited corpora are under each possible threshold of correctness.

\begin{figure}[th!]
    \centering
    \includegraphics[width=\columnwidth]{static/media/oscar/quality/num_C_ratio.pdf}
    \caption{Fraction of languages in each dataset below a given quality threshold (percent correct).}% The larger the AUC, the better.}
    \label{fig:ratio_c}
\end{figure}

\begin{figure*}[th]
    \centering
    \begin{subfigure}{.5\textwidth}
        \centering
        \includegraphics[width=\linewidth]{static/media/oscar/quality/C_mono.pdf}
        \caption{Monolingual corpora}
        \label{fig:C_mono}
    \end{subfigure}%
    \begin{subfigure}{.5\textwidth}
        \centering
        \includegraphics[width=\linewidth]{static/media/oscar/quality/C_para.pdf}
        \caption{Parallel corpora}
        \label{fig:C_para}
    \end{subfigure}
    \caption{Percentage of sentences labeled as correct vs. log N sentences for all audited languages.}
    \label{fig:C}
\end{figure*}

\paragraph{Why haven't these problems been reported before?}
The findings above are averaged on a per-language basis (i.e. macro-average), and therefore give low and high-resource languages equal weight. If we instead estimate the quality on a per-sentence basis, i.e. down-weight lower-resource languages in the computation of the average, the numbers paint a more optimistic picture (``micro'' block in Table~\ref{tab:results}). This is especially relevant for the monolingual datasets because they contain audits for English, which makes up for 43\% of all sentences in OSCAR and 36\% in mC4. To illustrate the effect of this imbalance: A random sample from the entire mC4 dataset with over 63\% chance will be from one of the 8 largest languages (\texttt{en}, \texttt{ru}, \texttt{es}, \texttt{de}, \texttt{fr}, \texttt{it}, \texttt{pt}, \texttt{pl}, $>$100M sentences each), %\footnote{mC4 contains 22\% \texttt{und} sentences, i.e. sentences with undefined language.} 
of which all have near perfect quality. Analogously, evaluation and tuning of web mining pipelines and resulting corpora in downstream applications focused largely on higher-resource languages (Section~\ref{sec:crawls}), so the low quality of underrepresented languages might go unnoticed if there is no dedicated evaluation, or no proficient speakers are involved in the curation~\citep{nekoto-etal-2020-participatory}.


\paragraph{How much content is nonlinguistic or in the wrong language?}
Nonlinguistic content is a more common problem than wrong-language content. Among the parallel datasets, CCAligned contains the highest percentage of nonlinguistic content, at 31.42\% on average across all rated corpora, and also the highest percent of wrong-language content, at 9.44\%. Among the monolingual datasets, mC4 contains the highest ratio both of sentences in incorrect languages (15.98\% average) and nonlinguistic content (11.40\% average), with 4 of the 48 audited languages having more than 50\% contents in other languages. The low amount of wrong language in ParaCrawl shows the benefits of selecting domains by the amount in-language text, but the dataset also covers the smallest amount of languages. The low ratio of wrong language samples in OSCAR may reflect the success of line-level LangID filtering.
These numbers provide evidence that more research in LangID could improve the overall quality, especially with respect to nonlinguistic content.

\paragraph{Which languages got confused?} The languages that were confused were frequently related higher-resource languages. However, there were also a significant number of ``out-of-model cousin" cases, where languages not supported by the LangID model ended up in a similar-seeming language. For instance in mC4, much of the Shona (\texttt{sn}, Bantu language spoken in Zimbabwe and Mozambique) corpus is actually Kinyarwanda (\texttt{rw}, Bantu language spoken in mostly in Rwanda and Uganda)---and, peculiarly, much of the Hawaiian (\texttt{haw}, Polynesian language spoken in Hawaii) is actually Twi (\texttt{tw}/\texttt{ak}, Central Tano language spoken mostly in Ghana).




\paragraph{Do low-resource languages have lower quality?}
Low-resource datasets tend to have lower human-judged quality.
The Spearman rank correlation between quality (\%\texttt{C}) and size is positive in all cases. The trend is strongest for mC4 ($r=0.66$), %,p=6.8e-8$), 
and gradually declines for CCAligned ($r=0.53$), %,p=0.0001$),
WikiMatrix ($r=0.49$), %,p=0.03$), 
ParaCrawl ($r=0.43$), %,p=0.02$) 
and OSCAR ($r=0.37$). %,p=0.10$).
Figure~\ref{fig:C} compares the number of sentences for each language against the proportion of correct sentences: %that we found during the audit:
%The correlation between quality (\%\texttt{C}) and size is strongest for WikiMatrix (Pearson's $r=0.66$), while mC4, CCAligned, ParaCrawl have comparatively lower correlation (0.21, 0.25, 0.29), and OSCAR the lowest with $r=0.13$.
%In general, we observe that languages with low representation tend to contain fewer correct sentences, with an exception of a dozen of languages from OSCAR.
Not all higher-resource languages ($>10^6$ sentences) have high quality, in particular for CCAligned (e.g. Javanese (\texttt{en\nobreakdash-jv\_ID}) with 5\%\texttt{C}, or Tagalog (\texttt{en\nobreakdash-tl\_XX}) with 13\%\texttt{C}). For mid-resource languages ($10^4$\nobreakdash--$10^6$ sentences) the picture is inconclusive, with some languages having high quality, and others having extremely low quality, even within the same datasets, e.g. Urdu in CCAligned \texttt{en-ur\_PK} has 100\%\texttt{C} vs. its romanized counterpart \texttt{en\nobreakdash-ur\_PK\_rom} 0.5\% \texttt{C}.
%\footnote{\texttt{\_rom} corpora have been removed in the latest CCAligned release.}
For individual error codes trends are less clear (not depicted).

\begin{table*}[!htbp]
    \centering
    \resizebox{\textwidth}{!}{%

        \begin{tabular}{lcccccccccccc}
            \toprule
                           & \texttt{es\_XX} & \texttt{bm\_ML} & \texttt{yo\_NG} & \texttt{tr\_TR} & \texttt{ku\_TR} & \texttt{zh\_CN} & \texttt{af\_ZA} & \texttt{jv\_ID} & \texttt{zh\_TW} & \texttt{it\_IT} & \textbf{mean} \\
            \midrule
            \textbf{Acc-6} & 0.58            & 0.73            & 0.41            & 0.45            & 0.43            & 0.55            & 0.65            & 0.55            & 0.46            & 0.55            & 0.66          \\
            \textbf{Acc-4} & 0.77            & 0.73            & 0.60            & 0.55            & 0.56            & 0.72            & 0.72            & 0.57            & 0.58            & 0.66            & 0.72          \\
            \textbf{Acc-2} & 0.91            & 0.96            & 0.72            & 0.64            & 0.71            & 0.79            & 0.77            & 0.92            & 0.81            & 0.69            & 0.79          \\
            \bottomrule
        \end{tabular}%
    }
    \caption{Rater evaluation for a subset of audits from \textbf{CCAligned} (translated from English) measured by the accuracy (Acc-$n$) of annotations by non-proficient speaker against annotations by proficient speakers.
        %$n$ indicates the granularity of the classes.  For $n=6$ all classes of the taxonomy were distinguished, for $n=4$ the \texttt{C} subclasses were combined, and for $n=2$ it is binary decision between \texttt{C} and the rest of the error classes.
    }
    \label{tab:agreement_ccaligned}
\end{table*}

\begin{table}[!htbp]
    \centering\small

        \begin{tabular}{lccccccccc}
            \toprule
                           & \texttt{tyv} & \texttt{rm} & \texttt{bar} & \texttt{eml} & \texttt{zh} & \texttt{la} & \textbf{mean} \\
            \midrule
            \textbf{Acc-6} & 1.0          & 0.98        & 1.0          & 1.0          & 0.86        & 1.0         & 0.98          \\
            \textbf{Acc-4} & 1.0          & 1.0         & 1.0          & 1.0          & 0.87        & 1.0         & 0.98          \\
            \textbf{Acc-2} & 1.0          & 1.0         & 1.0          & 1.0          & 0.87        & 1.0         & 0.98          \\
            \bottomrule
        \end{tabular}%
    \caption{Rater evaluation for a subset of audits from \textbf{OSCAR} measured by the accuracy (Acc-$n$) of annotations by non-proficient speaker against annotations by proficient speakers.}
    \label{tab:agreement_oscar}
\end{table}

\paragraph{Which languages have the lowest quality?} Across datasets we observe that the quality is particularly poor for languages that are included in romanized script (\texttt{\_rom}/\texttt{\_latn}), but are more commonly written in other scripts, e.g., Urdu (\texttt{ur}), Japanese (\texttt{ja}), Arabic (\texttt{ar}).
%\footnote{These romanized versions have been removed from CCAligned in a later release.} 
These are not transliterations of other scripts, but mostly contain non-linguistic material or wrong languages (e.g. the romanized Japanese corpus in mC4 (\texttt{ja\_latn}) contains Spanish, French, English, Portuguese, amongst others). %, Chinese (\texttt{zh}), Telugu (\texttt{te}) and Bulgarian (\texttt{bg}).  
In terms of geography, the poorest quality is found for African languages (Bambara (\texttt{bm}), Fula (\texttt{ff}), Kikongo (\texttt{kg}), Luganda (\texttt{lg}), Lingala (\texttt{ln}), Norther Sotho (\texttt{nso}), Oromo (\texttt{om}), Shona (\texttt{sn}), Somali (\texttt{so}), Tswana (\texttt{tn}), Wolof (\texttt{wo})), minority languages in Europe and the Middle East that are closely related to higher-resource languages (Azerbaijani (\texttt{az-IR}), North Frisian (\texttt{frr}), Neapolitan (\texttt{nap}), Silesian (\texttt{szl}), Zaza (\texttt{zza})), lesser spoken Chinese languages sharing a script with Mandarin (Yue (\texttt{yue}), Wu (\texttt{wuu})), four major Austronesian (Central Bikol (\texttt{bcl}), Chavacano (\texttt{cbk}), Javanese (\texttt{jv}), Sundanese (\texttt{su})), and some South-Asian languages, in particular Sinhala (\texttt{si}).
Appendix~\ref{app:stats} contains the detailed per-language statistics for all corpora.
% Omitted from above: mt

\paragraph{What is the incidence of offensive and pornographic content?}
Overall, the sampled sentences did not contain a large amount of offensive contents. However, there were notable amounts of pornographic content ($>10\%$) found in CCAligned for 11 languages. % not fully annotated: tl_XX, lt_LV ?


\paragraph{Annotation quality}
For a subset of audited languages from CCAligned and OSCAR we measure the accuracy (Acc) of the labels assigned by non-proficient speakers against the labels assigned by proficient speakers for all audited sentences. This can be understood as a directed measure of annotator agreement for the special case where one rater is an expert and the other is not. Results for varying label granularity are reported in Tables~\ref{tab:agreement_ccaligned} and \ref{tab:agreement_oscar}. For $n=6$ all classes of the taxonomy were distinguished, for $n=4$ the \texttt{C} subclasses were combined, and for $n=2$ it is binary decision between \texttt{C} and the rest of the error classes. With the full 6-class taxonomy (Acc-6) we find a mean accuracy of 0.66
%($\sigma^2 =0.02$) 
for CCAligned audits, and 0.98
%($\sigma^2 =0.002$) 
for OSCAR audits. % (see appendix~\ref{app:agreement} for language-specific results).
With a binary taxonomy (Acc-2) distinguishing \texttt{C} from the rest, the accuracy further increases to 0.79
%($\sigma^2=0.01$) 
for CCAligned. This provides strong evidence that good quality annotations are not limited to those proficient in a language.

However, the significant drop of accuracy for finer-grained labels hints at that our taxonomy can be further improved, especially for parallel sentences.
The error taxonomy lacks at least one category of error, namely ``correct/in-language but unnatural".  Similarly, the definition of ``correct-short" and ``correct-boilerplate" were not understood equally by all annotators and the concept of ``correct-short" has potential issues for agglutinative languages like Turkish. Finally, it was unclear what to do with related dialects, e.g. when a sentence is ``almost correct but wrong dialect" or when it is unclear which dialect a sentence belongs to. We recommend including these categories for future audits

\subsection{Automatic Filtering}
Given the frequency of \texttt{WL} and \texttt{NL} annotations, it might be tempting to use open-source LangID models to post-filter data on a per-sentence(-pair) level, as OSCAR does. Unfortunately, this turns out to have its own issues.

\paragraph{Sentence-level n-gram LangID filtering}
We classify all sentence pairs of CCAligned with CLD3, an n-gram based LangID model. By comparing its predictions to the audit labels, we evaluate its quality on the subset of annotated samples: the classifier should detect both correct languages when the pair is annotated as \texttt{C} and \texttt{X}, and should detect incorrect languages in the pair when \texttt{WL} and \texttt{NL}. On this task, the CLD3 classifier
%\footnote{\texttt{filter=0.976 Prec, 0.962 Rec, 0.969 F1.}} 
achieves an average precision of only 40.6\%. %,
%n average accuracy of 56.4\% against our annotators across all audited sentences, 
%underlining the issues with LangID on web domain data~\citep{caswell-etal-2020-language}. %Its recall for detecting those pairs with wrong language(s) is 77.8\%, and its precision 35.9\%. 

\paragraph{Sentence-level Transformer LangID filtering}
N-gram LangID models like CLD3 have known problems. However, \citet{caswell-etal-2020-language} demonstrate that semi-supervised Transformer-based LangID models strongly out-perform them. We train a comparable Transformer-based LangID model and apply it to our annotated CCAligned data. We find that filtering noisy corpora ($<$ 50\% correct) on LangID for both source and target leads to gains in median precision, rising from 13.8\% pre-filter to 43.9\% post-filter. However, this comes at a steep cost of 77.5\% loss in recall.
The biggest winners were Lingala, whose precision climbs from 8\% to 80\%, and Oromo, which soars from 2\% to 33\% in-language. Both of these, however, come at the cost of losing 50\% of the correct in-language sentences, being reduced from ~22k sentences to 3k and 1k sentences respectively, which would likely be too small for building downstream models. The moral is that, at least at the current stage, there is no one-size-fits-all approach for sentence-level LangID filtering.


\section{Dataset Mis-labeling}
\label{sec:codes}
Standardized and unambiguous representations of language codes are important for practical data use and exchange. The standard used by most academic and industry applications is BCP-47~\citep{phillips-etal-2005-tags}, which builds off the two-letter ISO639-2 codes and three-letter ISO639\nobreakdash-3 codes, but also allows to add subtags for scripts (e.g. Hindi in Latin script: \texttt{hi-Latn}) or regional varieties (e.g. French spoken in Canada: \texttt{fr-CA}). It would enhance transparency and interoperability if adopted consistently, especially with growing language diversity in NLP. % since it allows to add subtags for scripts or regional varieties. 
%, which builds off the two-letter ISO639-2 codes and three-letter ISO639\nobreakdash-3 codes. Codes may additionally specify ISO15924 script subtags to indicate that a nonstandard script is used (e.g. \texttt{hi-Latn} for Hindi written in Latin script), ISO3166-1 country codes to indicate regional varieties (e.g. \texttt{fr-CA} for Canadian French), or extensions for private use (e.g. \texttt{ca-x-val} for Valencian Catalan). Some BCP-47 codes represent groups of languages---for instance, \texttt{kg} represents the Kongo language, and \texttt{kng}, \texttt{ldi}, \texttt{kwy}, and \texttt{yom} represent particular varieties of Kongo.

We find a variety of errors and inconsistencies in language code usage, ranging from serious mislabelings to small transgressions against standard conventions. For this analysis, we also include the JW300~\citep{agic-vulic-2019-jw300} dataset, a multilingual dataset crawled from \url{jw.org}. %, which was otherwise not audited in this paper. 
In summary, we find 8 nonstandard codes in CCAligned, 3 in OSCAR, 1 in mC4, 1 in WikiMatrix, and 70 in JW300, for 83 in total. This does not include the 59 codes affected by superset issues. %0 in ParaCrawl, 
Full details are given in Appendix~\ref{app:jw300}.

\paragraph{Inconsistent Language Codes} One common issue is simply using nonstandard or invented codes. For example, CCAligned uses only two-letter codes, so when the BCP-47 code for a language is three letters it is either shortened (e.g. \texttt{zza} $\rightarrow$ \texttt{zz})
%, \texttt{szl}  $\rightarrow$ \texttt{sz}, \texttt{nso}  $\rightarrow$ \texttt{ns}, \texttt{ckb}  $\rightarrow$ \texttt{cb}, \texttt{ber}  $\rightarrow$ \texttt{tz} \footnote{Tamazight (BCP-47 ber) goes by various codes, so this may have been a shortening of e.g. \texttt{tzm}}) 
or invented (\texttt{shn}  $\rightarrow$ \texttt{qa}).
%, \texttt{kac}  $\rightarrow$ \texttt{qd}, \texttt{ceb}  $\rightarrow$ \texttt{cx}), which can lead to 
%this can lead to confusion and limits the compatibility with other tools and resources.
Similarly, OSCAR contains data labeled as \texttt{als} (BCP-47 for Tosk Albanian) that is actually in \texttt{gsw} (Allemannic).\footnote{This is a result of the language code used by the \href{https://en.wikipedia.org/wiki/Alemannic\_Wikipedia}{Alemannic Wikipedia} and affects any corpus or tool that uses Wikipedia data without correcting for this, like FastText.}
% \citep{joulin-etal-2017-bag, joulin-etal-2016-fasttext}.
%\footnote{\url{https://en.wikipedia.org/wiki/Alemannic_Wikipedia}
% And in JW300~\citep{agic-vulic-2019-jw300}, a multilingual dataset otherwise not audited in this paper, there are five codes ({\tt cat}, {\tt daf}, {\tt que}, {\tt nya}, {\tt run}) which use ISO693-3 instead of BCP-47 codes. Puzzlingly, in most of these cases there exist separate datasets for the equivalent ISO693-2 and 3 codes (e.g. both {\tt ny} and {\tt nya}).
22 additional language codes in JW300 have similar issues,
%mostly from mis-parsed private-use extensions, 
including 12 codes that start with \texttt{jw\_} but are not %(as they may appear)
Javanese.

\paragraph{False Sign Languages}
12\% (48/417) of JW300
%has a much stranger problem than nonstandard codes. It has the peculiar issue that a full 12\% (48/417) of the languages it claims to cover 
carry language codes for sign languages. %While it is possible to transcribe sign languages using glosses, this is not what these corpora are. 
Instead of sign language transcripts they are texts in another high resource language, mostly English or Spanish---for example, the \texttt{en-zsl} (Zambian sign language) data is actually English-English parallel data (copies), details in Appendix~\ref{app:jw300}. This was likely caused by videos with sign language interpretation embedded on the crawled websites.\footnote{Kudos to Rebecca Knowles for this explanation.} %Details are in Appendix Table~\ref{tab:signlanguages}.


\paragraph{Mysterious supersets}
When datasets contain language codes that are supersets of other language codes, it is difficult to determine which particular language the text contains. WikiMatrix has Serbian (\texttt{sr}), Croatian (\texttt{hr}), Bosnian (\texttt{bs}), and Serbo-Croatian (\texttt{sh})---their superset.\footnote{\url{https://iso639-3.sil.org/code/hbs}}
%. And while there may be some debate whether \texttt{bs},  \texttt{hr},  \texttt{cnr},  and \texttt{sr} are different languages, \texttt{sh} (\texttt{hbs}) is by definition a superset of all of them.\footnote{https://iso639-3.sil.org/code/hbs} 
The issue of codes that are supersets of others is common enough to include a small table dedicated to it (Appendix Table~\ref{tab:supersets}).
In some cases this may not be an issue, as with Arabic, where \texttt{ar} conventionally refers to Modern Standard Arabic, even though the code technically encompasses all dialects.
%, or where \texttt{no} typically refers to Norwegian Bokm\r{a}l (\texttt{nb}), though it technically is the superset of \texttt{nb} and \texttt{nn}. 
In many cases, the nature of the data in the superset code remains a mystery.
% requiring detective work.


\paragraph{Deprecated codes} Finally, there are several deprecated codes that are used: \texttt{sh} in Wikimatrix, \texttt{iw} in mC4, \texttt{sh} and \texttt{eml} in Oscar, and \texttt{daf} in JW300.

\section{Risks of Low-Quality Data}\label{sec:risk}

\paragraph{Low quality in downstream applications}
Text corpora today are building blocks for many downstream NLP applications like question answering and text summarization---for instance, a common approach is to first train translation models on such data and then automatically translate training data for downstream models~\citep{conneau-etal-2018-xnli}. If the data used for the original systems is flawed, derived technology may fail for those languages far down the line without knowing the causes.
This risk of undesired downstream effects calls for future studies with a careful treatment of intertwined effects such as data size and domain, language-specific phenomena, evaluation data and metric biases.
%Furthermore, there are not many existing public models trained on these specific subsets of data that we can analyze. 
To give the reader a brief glimpse of the impact of data quality for the example of translation, we compare the \texttt{C}\% metric from our audit with the translation quality (sentencepiece-BLEU, spBLEU) of the multilingual translation model M2M124 for 124 languages~\citep{goyal-etal-2021-flores-101}. It was trained on WikiMatrix and CCAligned, and similar data collected with the same tools, which we expect to show similar biases. Translation quality is evaluated on the trusted, human-translated FloReS benchmark~\citep{goyal-etal-2021-flores-101}.
%For language pairs that were both covered in the WikiMatrix and the CCAligned audit, we compute an average of their \% \texttt{C} scores weighted by their size. 
For the 21 languages present in both the audit and the FloReS benchmark, we found a positive correlation (Spearman) between the data quality scores and spBLEU of $\rho=0.44$ $(p=0.041)$. This is not as large as the correlation with data size ($\rho=0.66$, $p=0.00078$), but it nonetheless helps to explain translation quality---the correlation between the product of \texttt{C}\% and data size (in other words, the expected total number of good sentences in the dataset), is the highest yet, with a value of $\rho=0.73$ $(p=0.00013)$.\footnote{For the translation from English, BLEU scores are less comparable but the trend holds nonetheless, with values of ($\rho=0.32$, $p=0.14$), ($\rho=0.74$, $p=0.000078$), and ($\rho=0.80$, $p=0.0000087$) respectively.}
% The human inspection and auditing of e.g. trained vector representations to detect possible risks and misrepresentations of a subset of languages is arguably harder than manually inspecting a few samples as we did in this work.
% As our analysis has shown, low-resource languages are disproportionately affected by such problems in automatic data curation pipelines.

\paragraph{Representation washing}
Since there are datasets which contain many low-resource languages, the community may feel a sense of progress and growing equity, despite the actual quality of the resources for these languages. %However, models often still perform poorly on NLP tasks for these languages
%Because there appear to be datasets for low-resource languages, the community may collectively feel as though progress is being made in these areas. 
Similarly, if low-quality datasets are used as benchmarks they may exaggerate model performance, making low-resource NLP appear more solved than it is---or conversely, if models perform poorly when trained with such data, it may be wrongly assumed that the task of learning models for these languages is harder than it actually is or infeasible given current resources. These effects could result in productive effort being redirected away from these tasks and languages.
%The result can be that productive effort will be directed away from these fields.

\begin{table}[t!]

        \centering\small
        \begin{tabular}{ll}
            \toprule

            \texttt{en}  & The prime minister of the \textbf{UK} is \textbf{Boris Johnson}.          \\
            \texttt{nl}  & De minister-president van \textbf{Nederland} is \textbf{Mark Rutte}.      \\
                         & \small{\texttt{en}: The prime minister of the Netherlands is Mark Rutte.} \\
            \midrule
            %\midrule
            % \texttt{en} &Sunglasses \\
            % \texttt{ig}	&ah\d{i}a Nyocha \\
            % \midrule
            \texttt{en}  & \textbf{24 March} 2018                                                    \\
            \texttt{pt}  & \textbf{14 Novembro} 2018                                                 \\
                         & \small{\texttt{en}: 14 November 2018 }                                    \\
            %\midrule
            \midrule
            % \texttt{en} &The current local time in \textbf{Sarasota} is \textbf{89} minutes ahead of apparent solar time. \\
            % \texttt{nn}	&Den lokale tiden i \textbf{Miami} er \textbf{86} minutt f\o{o}re sann soltid. \\
            \texttt{en}  & The current local time in \textbf{Sarasota} is \textbf{89} minutes.       \\
            \texttt{nn}  & Den lokale tiden i \textbf{Miami} er \textbf{86} minutt.                  \\
                         & \small{\texttt{en}: The local time in Miami is 86 minutes.}               \\
            %\midrule
            \midrule
            \texttt{en}  & In \textbf{1932} the highway was extended \textbf{north to LA}.           \\
            \texttt{bar} & \textbf{1938} is de Autobahn bei \textbf{Inglstod} fertig gstellt.        \\
                         & \small{\texttt{en}: The highway near Inglstod was completed in 1938.}     \\
            % \midrule
            % \textit{en:} He was engaged to the lawyer and actor João Lima Junior, with whom he dated from 2004 to 2006. \\
            % \textit{nds:} Ze woont in de tussentied samen met zanger en liedtiesschriever Johannes Oerding met wie ze sinds 2009 ook samen op de bühne stiet.\\
            \bottomrule
        \end{tabular}%
    \caption{Examples of ``parallel" data where the translation has a different meaning than the source, but the form looks the same. (We added translations of the non-English side.) Such data may encourage hallucinations of fake ``facts".}
    \label{tab:not_actually_parallel}
\end{table}

\paragraph{Trust in incorrect ``facts''} % and trust} %, algorithmic trust and automation bias}
We found many instances of parallel-looking sentences that are structurally and semantically similar, but not factually correct translations (Table~\ref{tab:not_actually_parallel}). They can cause models to produce plausible ``translations" that are factually wrong, but users may still trust them (\textit{algorithmic trust}) without verifying the information. %This is relevant for \textit{algorithmic trust}, when users increasingly trust the outputs of computers and ``algorithms" without verifying the information. 
Similarly, \textit{automation bias} \citep{skitka-etal-1999-does},
%from social psychology which refers to the bias of 
referring to humans favoring decisions made by automated systems over decisions made by humans, might amplify the issues of inaccurate translations caused by misalignments.
%One variant of this issue that occurs frequently in some datasets is pornographic content.
%, which in the majority of the cases we observed were parts of misaligned sentence pairs. 
%Another effect is that models trained on misaligned pornographic content may hallucinate such content, which may be disturbing to users.

\section{Future Work and Recommendations}\label{sec:recommendation}
Of the five multilingual corpora evaluated, we consistently found severe issues with quality, especially in the lower-resource languages. We rated samples of 205 languages, and found that 87 of them had under 50\% usable data, with a full 15 languages at 0\% in-language. We furthermore found consistent issues with mislabeled data and nonstandard language codes, particularly in the JW300 dataset, and identified 83 affected corpora, at least 48 of which were entirely spurious (Section~\ref{sec:codes}). While there might have been anecdotal evidence of insufficient quality for some of the datasets, the majority of these quality issues had not been reported, nor been investigated in depth. These issues might go unnoticed for languages that are not represented in the evaluation of the crawling methods, and cause harm in downstream applications~\citep{khayrallah-koehn-2018-impact}.

There are a variety of ways to improve both the ease and accuracy of human evaluation, as well a few classes of issues we ignored in this paper, like close dialects.
Ideally we would like to build a standard suite of automatic metrics for datasets, but more research is necessary to determine what the appropriate metrics would be. One important area missing from our analyses however is the estimated portion of a dataset which has been generated by MT~\citep{rarrick-etal-2011-mt}, LM systems, or bots/templates, as for example in the analysis of C4~\citep{dodge-etal-2021-documenting}. %A prominent example is the Lsjbot\footnote{\url{https://en.wikipedia.org/wiki/Lsjbot}} which is responsible for creating 80-90\% of content for Swedish, Cebuano and Waray Wikipedia. 
The information captured in machine-generated content might still be useful for modeling, but might falsely overrepresent typical generation patterns and introduce linguistic errors or unnatural artifacts.
% Malagasy wiktionary audit: https://meta.wikimedia.org/wiki/Requests_for_comment/Large-scale_errors_at_Malagasy_Wiktionary
%https://www.vice.com/en/article/4agamm/the-worlds-second-largest-wikipedia-is-written-almost-entirely-by-one-bot

% Finally, similar studies to this in future would do well ton work more on calibrating human raters, to ensure consistent use of error categories.

% An issue that arises with the progress in building technology for some of the languages is the retrieval of machine-generated output as in-language data. This is prominent for mid- to high-resource languages for which translation systems have reached sufficient quality for website translation, as we observed for example a significant amount of translations for Ukrainian. Non-native speakers might not have noticed them during annotations, so the problem might be even larger than we might estimate now. We leave a systematic investigation to future work. A more unexpected artifact is the retrieval of published BPE vocabularies for a range of low-resource languages, such as Sundanese.\footnote{\url{https://nlp.h-its.org/bpemb/su/su.wiki.bpe.vs100000.vocab}

We therefore strongly recommend looking at samples of any dataset before using it or releasing it to the public. As we have shown, one does not need to be proficient in a language to see when there are serious quality issues, and a quick scan of 100 sentences can be sufficient to detect major problems. Moreover, going through and annotating a small sample of data can bring actionable insights about new ways to filter or use it.

If data quality issues are found, a wide variety of techniques can be explored, like filtering on length-ratio, LangID, TF-IDF wordlists \cite{caswell-etal-2020-language} or dictionaries~\citep{kamholz-etal-2014-panlex}; to neural approaches like LM scoring \cite{axelrod-etal-2011-domain,moore-lewis-2010-intelligent,wang-etal-2018-denoising}. Unfortunately, none of these provides a quick and easy fix, especially for low-resource languages---data cleaning is no trivial task!

Noisy datasets are by no means useless, at least if they contain some desirable content. Therefore an alternative to filtering can be documentation~\citep{bender-etal-2021-on}. This can take the form of a per-language quality score and notes about known issues,
% ({ \it``language xx has high percentage non-linguistic content'' } etc.), 
a datasheet \citep{gebru-etal-2018-datasheets} or nutrition label \citep{holland-etal-2018-the}. However, we suggest researchers not release corpora with near-zero in-language content, as this may give the mistaken impression of usable resources.

Finally, we encourage the community to continue conducting evaluations and audits of public datasets---similar to system comparison papers.


\section*{Acknowledgements}
We would like to thank the TACL editors and reviewers, and AfricaNLP and Google reviewers who have helped us shape this paper. Furthermore, we are grateful for Ahmed El-Kishky's support and help with CCAligned and WikiMatrix size statistics.



\begin{table}[th!]
    \centering
    \begin{tabular}{lll}
        \toprule
        \textbf{Dataset } & \textbf{Supercode} & \textbf{Subcode(s)}                       \\
        \midrule
        JW300             & \texttt{kg}        & \texttt{kwy}                              \\
        JW300             & \texttt{mg}        & \texttt{tdx}                              \\
        JW300             & \texttt{qu}        & \texttt{que}, \texttt{qug}, \texttt{qus}, \\
                          &                    & \texttt{quw}, \texttt{quy}, \texttt{quz}, \\
                          &                    & \texttt{qvi}, \texttt{qvz}                \\
        JW300             & \texttt{sw}        & \texttt{swc}                              \\
        \midrule
        OSCAR             & \texttt{ar}        & \texttt{arz}                              \\
        OSCAR             & \texttt{az}        & \texttt{azb}                              \\
        OSCAR             & \texttt{sh}        & \texttt{bs}, \texttt{hr}, \texttt{sr}     \\
        OSCAR             & \texttt{ku}        & \texttt{ckb}                              \\
        OSCAR             & \texttt{ms}        & \texttt{id}, \texttt{min}                 \\
        OSCAR             & \texttt{no}        & \texttt{nn}                               \\
        OSCAR             & \texttt{sq}        & \texttt{als}$^{*}$                        \\
        OSCAR             & \texttt{zh}        & \texttt{yue}, \texttt{wuu}                \\
        %  \midrule
        % Tatoeba & ar & acm,afb,ajp,apc,arq,ary,arz,ayl \\
        % Tatoeba & ber & kab \\
        % Tatoeba & et & vro \\
        % Tatoeba & ku & ckb,kmr,sdh \\
        % Tatoeba & lv & ltg \\
        % Tatoeba & sq & aln \\
        \midrule
        WikiMatrix        & \texttt{ar}        & \texttt{arz}                              \\
        WikiMatrix        & \texttt{sh}        & \texttt{bs}, \texttt{hr}, \texttt{sr}     \\
        WikiMatrix        & \texttt{zh}        & \texttt{wuu}                              \\
        \bottomrule
    \end{tabular}
    \caption{Situations where two language codes are represented, but one is a superset of another by the ISO standard, leading to unclarity about the data in the supercode dataset. $^{*}$The \texttt{als} dataset is actually in \texttt{gsw}.}
    \label{tab:supersets}
\end{table}


\begin{table}[th!]
    \centering
    \small
    \begin{tabular}{ll}
        \toprule
        \textbf{ Actual language} & \textbf{Code in JW300}                                                \\
        \midrule
        \texttt{cs}               & \texttt{cse}                                                          \\
        \texttt{de}               & \texttt{gsg}                                                          \\
        \texttt{el}               & \texttt{gss}                                                          \\
        \texttt{en}               & \texttt{ase}, \texttt{asf}, \texttt{bfi}, \texttt{ins}, \texttt{psp}, \\
                                  & \texttt{sfs}, \texttt{zib}, \texttt{zsl}                              \\
        \texttt{es}               & \texttt{aed}, \texttt{bvl}, \texttt{csf}, \texttt{csg}, \texttt{csn}, \\
                                  & \texttt{csr}, \texttt{ecs}, \texttt{esn}, \texttt{gsm}, \texttt{hds}, \\
                                  & \texttt{lsp}, \texttt{mfs}, \texttt{ncs}, \texttt{prl}, \texttt{pys}, \\
                                  & \texttt{ssp}, \texttt{vsl}                                            \\
        \texttt{fi}               & \texttt{fse}                                                          \\
        \texttt{fr}               & \texttt{fcs},\texttt{fsl}                                             \\
        \texttt{hu}               & \texttt{hsh}                                                          \\
        \texttt{id}               & \texttt{inl}                                                          \\
        \texttt{it}               & \texttt{ise}                                                          \\
        \texttt{ja}               & \texttt{jsl}                                                          \\
        \texttt{ko}               & \texttt{kvk}                                                          \\
        \texttt{pl}               & \texttt{pso}                                                          \\
        \texttt{pt}               & \texttt{bzs}, \texttt{mzy}, \texttt{psr}, \texttt{sgn\_AO}            \\
        \texttt{ro}               & \texttt{rms}                                                          \\
        \texttt{ru}               & \texttt{rsl}                                                          \\
        \texttt{sk}               & \texttt{svk}                                                          \\
        \texttt{sq}               & \texttt{sql}                                                          \\
        \texttt{st}               & \texttt{jw\_ssa}                                                      \\
        \texttt{zh}               & \texttt{csl}, \texttt{tss}                                            \\
        \bottomrule
    \end{tabular}
    \caption{There are 48 languages in the JW300 corpus with language codes that correspond to sign languages, but in reality are unrelated high-resource languages (usually the most spoken language in the country of origin of the sign language). This table shows the actual language of the data corresponding to each sign language code.} %  For instance, the \texttt{ase-en} parallel data is actually \texttt{en-en} parallel data (copied source and target).
    \label{tab:signlanguages}
\end{table}

\section{Details on Language Code Issues}
\label{app:jw300}

Table \ref{tab:supersets} provides a complete lists of the corpora where one code is defined as a superset of the other by the ISO standard, and in Table \ref{tab:signlanguages} we provide a complete list of the language codes in JW300 which purport to be sign language but are actually unrelated high-resource languages.



\begin{table}[th!]
    \centering\small
        \begin{tabular}{lll}


            \toprule
            \textbf{Code in JW300} & \textbf{BCP-47 code} & \textbf{Actual Language Name} \\
            \multicolumn{3}{c}{}                                                          \\
            \multicolumn{3}{c}{\textbf{Incorrect private-use extensions}}                 \\
            \midrule
            hy\_arevmda            & hyw                  & Western Armenian              \\
            jw\_dgr                & os\_x\_dgr           & Digor Ossetian                \\
            jw\_dmr                & naq\_x\_dmr          & Damara Khoekhoe               \\
            jw\_ibi                & yom\_x\_ibi          & Ibinda Kongo                  \\
            jw\_paa                & pap\_x\_paa          & Papiamento (Aruba)            \\
            jw\_qcs                & qxl                  & Salasaca Highland Kichwa      \\
            jw\_rmg                & rmn\_x\_rmg          & Greek Romani (South)          \\
            jw\_rmv                & rmy\_x\_rmv          & Vlax Romani, Russia           \\
            jw\_spl                & nso\_x\_spl          & Sepulana                      \\
            jw\_ssa                & st\_ZA               & Sesotho (South Africa)        \\
            jw\_tpo                & pt\_PT               & Portuguese (Portugal)         \\
            jw\_vlc                & ca\_x\_vlc           & Catalan (Valencia)            \\
            jw\_vz                 & skg\_x\_vz           & Vezo Malagasy                 \\
            rmy\_AR                & rmy\_x\_?            & Kalderash                     \\

            \multicolumn{3}{c}{}                                                          \\
            \multicolumn{3}{c}{\textbf{Equivalent codes used in place of extensions}}     \\
            \midrule
            kmr\_latn              & kmr\_x\_rdu          & Kurmanji (Caucasus)           \\
            nya                    & ny\_x\_?             & Chinyanja (Zambia)            \\
            que                    & qu\_x\_?             & Quechua (Ancash)              \\

            \multicolumn{3}{c}{}                                                          \\
            \multicolumn{3}{c}{\textbf{Deprecated codes}}                                 \\
            \midrule
            daf                    & dnj/lda              & Dan                           \\
            % sgn\_AO & 	pt & 	Portuguese \\ 

            \multicolumn{3}{c}{}                                                          \\
            \multicolumn{3}{c}{\textbf{ISO-693-3 used in place of ISO-693-2}}             \\
            \midrule
            cat                    & ca                   & Catalan                       \\
            gug                    & gn                   & Guarani                       \\
            run                    & rn                   & Kirundi                       \\
            tso\_MZ                & ts\_MZ               & Changana (Mozambique)         \\
            \bottomrule
        \end{tabular}%
    \caption{Language code issues in the JW300 datasets for 22 language varieties not covered by Tables \ref{tab:supersets} and \ref{tab:signlanguages}.
        %Twelve languages have codes starting in \texttt{jw\_}, suggesting they are varieties of Javanese, but are instead mis-parsed private-use extensions. Three codes appear in addition to equivalent ISO codes, making it unclear which languages they are. One language uses a deprecated ISO code. Four languages use the ISO639-3 code instead of the ISO639-2 code, and therefore are not BCP-47. (Note: in this table, 
        Private use extensions are given as they appear in \url{jw.org}, and specified as `?' if they are absent from \url{jw.org}.}
    \label{tab:jw300nonbcp}
\end{table}

Special attention needs to be given to the JW300 dataset, which, in addition to the sign languages and superset code issues, has a variety of other peculiarities. These problems seem to originate in the codes used by \url{jw.org},\footnote{The \url{jw.org} website seems to use correct BCP-47 extensions now, however, and entering a code such as ``jw\_dmr" redirects to ``naq\_x\_dmr".} which were apparently not checked in the creation of the JW300 dataset. An overview is provided in Table \ref{tab:jw300nonbcp}, and the following paragraphs give specifics.

Twelve languages in JW300 have codes starting in \texttt{jw\_}, suggesting they are varieties of Javanese (ISO639-1 \texttt{jw}), but are instead attempts to represent language dialects for which there are no BCP-47 codes. These codes seem to have been updated in \url{jw.org} to appropriate BCP-47 private-use extensions in the form \texttt{<supercode>\_x\_<tag>}, which are provided in Table \ref{tab:jw300nonbcp}.
Twelve languages have codes starting in \texttt{jw\_}, suggesting they are varieties of Javanese, but are instead mis-parsed private-use extensions. Three codes appear in addition to equivalent ISO codes, making it unclear which languages they are. One language uses a deprecated ISO code. Four languages use the ISO639-3 code instead of the ISO639-2 code, and therefore are not BCP-47.

In addition to the \texttt{jw\_} tags, there are two other mis-used private subtags: \texttt{hy\_arevmda}, which in addition to lacking the mandatory \texttt{\_x\_} appears to represent standard Western Armenian (\texttt{hyw}); and \texttt{rmy\_AR}, which, rather than being Romany from Argentina, is Kalderash Romany.

There are also a few anomalies where private use extensions should have been used but other methods were found to convey the distinctions. Three codes appear in addition to equivalent ISO codes, making it unclear which languages they are. Two of these are equivalencies between  ISO639-2 and  ISO639-3 (\texttt{nya} and \texttt{ny} are both Chichewa, \texttt{qu} and \texttt{que} are both Quechua), and one is a script equivalency (\texttt{kmr} and \texttt{kmr\_latn} are both in Latin script). In these three cases the two codes do represent different languages---so a private use extension would have been appropriate.

Finally, there is the more minor issue that three languages use the ISO639-3 code instead of the ISO639-2 code, and therefore are not BCP-47.


In addition to the JW300-specific errors, Table \ref{tab:misc_codes} summarizes miscellaneous errors in CCAligned and OSCAR that were detailed in Section \ref{sec:codes}.

\begin{table}[!th]
    \small
    \centering
    \begin{tabular}{lll}
        \toprule
        \textbf{Dataset} & \textbf{Code in Corpus} & \textbf{Correct Code} \\
        \midrule
        CCAligned        & \texttt{zz}             & \texttt{zza}          \\
        CCAligned        & \texttt{sz}             & \texttt{szl}          \\
        CCAligned        & \texttt{ns}             & \texttt{nso}          \\
        CCAligned        & \texttt{cb}             & \texttt{ckb}          \\
        CCAligned        & \texttt{tz}             & \texttt{ber}          \\
        CCAligned        & \texttt{qa}             & \texttt{shn}          \\
        CCAligned        & \texttt{qd}             & \texttt{kac}          \\
        CCAligned        & \texttt{cx}             & \texttt{ceb}          \\
        \midrule
        mC4              & \texttt{iw}             & \texttt{he}           \\
        \midrule
        OSCAR            & \texttt{eml}            & \texttt{egl}          \\
        OSCAR            & \texttt{als}            & \texttt{gsw}          \\
        OSCAR            & \texttt{sh}             & \texttt{hbs}          \\
        \midrule
        WikiMatrix       & \texttt{sh}             & \texttt{hbs}          \\
        \bottomrule
    \end{tabular}
    \caption{Miscellaneous errors in language codes.}
    \label{tab:misc_codes}
\end{table}







\section{Complete Error Taxonomy and Instructions}~\label{app:taxonomy}
In addition to the examples given in Table \ref{tab:examples}, raters were provided with the following verbal notes on the error codes:
\begin{itemize}
    \item \textbf{\texttt{CC}: Correct translation, natural sentence:} It's OK if it's a sentence fragment instead of a whole sentence, as long as it is not too short (about 5 words or greater). The translation does not have to be perfect.
    \item \textbf{\texttt{\texttt{CS}}: Correct Translation, but single word or short phrase:} Also includes highly repeated short phrases, like ``the cat the cat the cat the cat the cat ..."
    \item \textbf{\texttt{CB}: Correct translation, but boilerplate: } This can be auto-generated or formulaic content, or content that one deems ``technically correct but generally not very useful to NLP models". Unfortunately, it's often not clear what should be counted as boilerplate...do your best.
    \item \textbf{\texttt{X}: Incorrect translation} [for parallel sentences] both source and target are in the correct language, but they are not adequate translations.
    \item \textbf{\texttt{WL}: Wrong language} For short sentences, especially with proper nouns, there is often a fine line between ``Wrong language" and ``Not language". Do your best.
    \item \textbf{\texttt{NL}: Not language} At least one of source and target are not linguistic content. Any sentence consisting only of a proper noun (e.g. ``Tyrone Ping") should be marked as \texttt{NL}.
    \item \textbf{\texttt{U}: Unknown} for sentences that need verification by a native speaker. This is an auxiliary label that is resolved in most cases.
\end{itemize}







\section{Methodological Notes}\label{app:strategies}

A surprising amount of work can be done without being an expert in the languages involved. The easiest approach is simply to search the internet for the sentence, which usually results in finding the exact page the sentence came from, which in turn frequently contains clues like language codes in the URL, or a headline like \textit{News in X language}, sometimes with references to a translated version of the same page. However, for the cases where this is insufficient, here are a few tips, tricks, and observations.

\paragraph{No Skills Required:}
Things that do not require knowledge of the language(s) in question.
\begin{enumerate}
    \item ``Not language'' can usually be identified by anyone who can read the script, though there are tricky cases with proper nouns.
    \item Frequently, ``parallel" sentences contain different numbers in the source and target (especially autogenerated content), and are easy to disqualify.
    \item Errors tend to repeat. If a word is mistranslated once, it will often be mistranslated many more times throughout a corpus, making it easy to spot.
\end{enumerate}

\paragraph{Basic Research Required:}
Things that do not require knowledge of the language(s) in question but can be done with basic research.
\begin{enumerate}
    \item If it's written in the wrong script it's considered wrong language. (Sometimes the writing system is indicated in the published corpus, e.g. \texttt{bg-Latn}, but usually the language has a ``default" script defined by ISO.)
    \item Some types of texts come with inherent labels or markers, such as enumerators or verse numbers.
          %For example, much of CCAligned's Odia text is Christian Bible verses, which are preceded by an identifier like ``Matt 12:37". 
    \item When all else fails, search the internet for the whole sentence or n-grams thereof! If the whole sentence can be found, frequently the language is betrayed by the web page (the language's autonym is useful in this case).
\end{enumerate}


\section{Complete Audit Results}\label{app:stats}
Tables \ref{tab:ccaligned-full}, \ref{tab:wikimatrix-full}, \ref{tab:paracrawl-full}, \ref{tab:mc4-full} and \ref{tab:oscar-full} give the complete annotation percentages for CCAligned, WikiMatrix, ParaCrawl, mC4 and OSCAR, respectively. For each annotation label, we report the ratio of the annotated sentences (of max 100 sentences) that were assigned that label by the primary annotator. Repeated annotations done for agreement measurement are not included. The \texttt{C} column aggregates all correct sub-codes (\texttt{CC}, \texttt{CS}, \texttt{CB}). We also report the total number of sentences that each dataset contains for each language and the average sentence length for the audited sentences to illustrate differences across languages. The original language codes as they are published with the datasets are maintained for the sake of consistency (but should be handled with care in future work, see Section~\ref{sec:codes}), and those with less than 20\% correct sentences are highlighted.

%%%%%%%%%%%%%%%%%%%%%%%%%%%%%%%%%%%%%%%%%%%%%%%%%%%%%%%%%%%%%%%%%%%%%%%%
\chapter{Ungoliant: The Second OSCAR pipeline}
%%%%%%%%%%%%%%%%%%%%%%%%%%%%%%%%%%%%%%%%%%%%%%%%%%%%%%%%%%%%%%%%%%%%%%%%

\begin{center}
    \begin{minipage}{0.66\textwidth}
        \begin{small}
            In which we present the work of \citet{abadji-etal-2021-ungoliant}, who after the evaluations discussed in the previous two chapters, completely rewrote the original OSCAR's \goclassy pipeline, added features to the corpus such as metadata extraction and published the second version of the OSCAR corpus now known as OSCAR 21.09.\footnotemark
        \end{small}
    \end{minipage}
    \vspace{0.5cm}
\end{center}

\footnotetext{Contributions: I designed most of the experiments and comparisons to be made between the two pipelines, I downloaded and prepared Common Crawl for extraction and I actively participated in the writing of the scientific article.}

As discussed in previous chapters, OSCAR 2019 was generated from the plain text data extracts (WET files) of the November 2018 Common Crawl dump, which was distributed in the form of 56,000 \emph{shards}, that were then filtered and classified by language \citep{ortiz-suarez-etal-2019-asynchronous,ortiz-suarez-etal-2020-monolingual}. OSCAR 2019 is now available for research through the Huma-Num servers \footnote{\url{https://oscar-corpus.com/post/oscar-2019/}} in Europe and for the public at large through Hugging Face's Datasets Hub \footnote{\url{https://huggingface.co/datasets/oscar}} where it now has more than 15 thousands downloads.

OSCAR 2019 came in four different versions, each one intended for different tasks. These versions were either \emph{unshuffled} or \emph{shuffled} (that is, for each language, lines have been shuffled, destroying record and thus document integrity), and \emph{non-deduplicated} or \emph{deduplicated} (since duplicate lines account for more than half of the total data\footnote{OSCAR-orig: 6.3TB, OSCAR-dedup: 3.2TB} generated by the pipeline). For the unshuffled versions, each language file contained paragraphs that came from the same record, and each paragraph is separated by a newline.

Simply put, OSCAR 2019 was composed of single language files that contained textual data (\texttt{ta.txt} for the Tamil language, for example). However, due to the often huge sizes of these files, and subsequently the impracticality of storage and distribution, OSCAR 2019 files were split and compressed in equally sized parts.

However, but OSCAR 2019 and its pipeline came with a number of limitations, which we will discuss in the following sections, and we will try to start fixing in this and the following chapter.

\section{Limitations of the OSCAR 2019 Corpus and its Generation Pipeline}

\subsection{OSCAR 2019}

OSCAR 2019 was inherently linked to its generation pipeline, and as such its quality partly depended on the pipeline's quality. While OSCAR 2019 was considered to be one of the cleanest multilingual corpora available \citep{caswell-etal-2020-language,kreutzer-etal-2021-quality}, several problems had been described, and the state of the publicly available code raised questions about maintenance and maintenability of the pipeline itself.

Apart from the fact that its content dated back to 2018, OSCAR 2019 corpus suffered from quality issues already discussed in chapter \ref{chap:quality} and of course more in depth in \citep{caswell-etal-2020-language,kreutzer-etal-2021-quality}, some of which include:

\begin{itemize}
    \item \textbf{Language label mismatches and inconsistencies}, which occurs earlier in the pipeline and would be fixable downstream,
    \item \textbf{Representation washing} as defined by \citet{kreutzer-etal-2021-quality}, whereby low resource languages, while present in the corpus, are of a significantly lower quality than higher resource languages without any quality metric available publicly.
\end{itemize}

Moreover, the more recent dumps of Common Crawl in 2021 contain more than 64,000 shards (almost 10,000 more than the dump used for OSCAR 2019). Furthermore, each of these shards is composed of numerous records, and each record holds textual content along with metadata. While Common Crawl shards hold document-level metadata that could be useful downstream, they were com discarded and do not appear in OSCAR 2019, whereas other corpora generated from Common Crawl do include them, e.g.~CCNet \citep{wenzek-etal-2020-ccnet}. This limits OSCAR 2019 users to the textual content only, whereas metadata could have been distributed along with the corpus itself.

\subsection{goclassy}

OSCAR 2019 was built using \goclassy, a high-performance asynchronous pipeline written in Go \citep{ortiz-suarez-etal-2019-asynchronous}. However, it suffered from several caveats that makes the re-generation and update of the corpus relatively complex in practice.

While \goclassy's source code was easily readable thanks to the choice of an uncluttered programming language and a pragmatic approach, the lack of structure in both the source and the project itself made \goclassy difficult to extend and maintain.

The pipeline was not functional out-of-the-box, as the user had to provide the compressed shards from CommonCrawl, manually install FastText \citep{joulin-etal-2016-fasttext,joulin-etal-2017-bag} and create specific directories by themselves, since only partial instructions are given in the supplied README file.

\goclassy also made heavy use of I/O, as data was saved and loaded repeatedly between steps; as an example, the identification step stored language identification data and individual sentences in two files, before generating the final files (one per language). Despite these limitations, \goclassy's performance remained acceptable mainly due to Go's emphasis on easy and efficient parallelization and inherent speed. The pipeline for instance used clever handling of file descriptors and employed extensive buffering, which limited I/O calls cost in some parts.

\section{Building a New Version of the OSCAR Corpus}

Having identified some shortcomings of both OSCAR 2019 and its pipeline, \goclassy, we decided to restart the OSCAR project by completely rewriting our pipeline. To that end, we introduce \emph{Ungoliant}, a new corpus generation pipeline that, like \goclassy, creates a large-scale multilingual text corpus from a Common Crawl dump. However, contrarily to \goclassy, Ungoliant is fully modular, better structured, and highly parametrizable; thereby allowing comparisons between several parallelization strategies. A specific effort was put in testing and documentation. Parts of Ungoliant are heavily inspired by \goclassy, although for its implementation we decided to use Rust rather than Go, which is often considered to be a faster more low level programming language.\footnote{\url{https://benchmarksgame-team.pages.debian.net/benchmarksgame/fastest/rust-go.html}}

We also use Ungoliant to generate a new version of the OSCAR corpus from a more recent Common Crawl dump. The new corpus includes metadata information while retaining backward compatibility with the OSCAR 2019 corpus.


\subsection{Ungoliant}

\subsubsection{Rationale and Scope}
\label{subsubsec:rationale}
While Ungoliant is heavily inspired by \goclassy, it provides a better set of tools to download, process, filter and aggregate textual and contextual data from Common Crawl. These operations can be sequential, parallel or both, depending on contexts and performance requirements.

We provide both batch and streaming processing, so that the whole pipeline could be run either online, with every step running on streams of data, or offline, with every step running on tangible files, or a mix of both, using already downloaded Common Crawl dumps but streaming the rest of the process. Moreover, we embed numerous filtering and deduplication utilities directly inside Ungoliant, making these features available for pipeline composition and post-processing.

Ungoliant features a loosely defined pipeline interface, on which we re-implement \goclassy's one, while improving performance by threading more aggressively and avoiding I/O where it is not necessary: While \goclassy uses intermediate files for tags and sentences, we try to keep everything in memory in order to avoid losing time loading or writing files. The Rust language provides constructs that helps us build complex abstractions and pipelines while limiting proactive file I/O or computing, since nearly all the reimplemented pipeline is built around lazy evaluation. File I/O is only used when loading shards, and when writing sentences in language files.

Through benchmarking we found that the best parallelization strategy is to use rayon\footnote{\url{https://github.com/rayon-rs/rayon}}, a work-stealing \cite{blumofe-etal-1999-scheduling} parallel and concurrent library enabling massive parallelization. We parallelize on \mbox{shard-,} record- and sentence-level processing.

To evaluate Ungoliant performance, we run both \goclassy and Ungoliant's implementation on 1, 10, 25 and 100 Common Crawl shards both on a middle-range laptop computer (i5-7200u, 8 GB RAM, NVMe SSD) and a HPC node (Xeon 5218 (64 Threads), 180 GB RAM). Results are shown in Table~\ref{tab:goclassy-bench}.

\begin{table}[t]
    \centering\small
    \scalebox{0.85}{
        \begin{tabular}{lrrrr}
            \toprule
            Platform                 & \#shards & goclassy & Ungoliant & Approx.~speedup \\
            \midrule
            \multirow{3}{*}{Desktop} & 1        & 30s      & 13s       & $\times$2.3     \\
                                     & 10       & 3m6s     & 2m12s     & $\times$1.3     \\
                                     & 25       & 9m10s    & 5m47s     & $\times$1.5     \\
            \midrule
            \multirow{3}{*}{HPC}     & 1        & 40s      & 6s        & $\times$6.6     \\
                                     & 25       & 2m40s    & 1m6s      & $\times$2.4     \\
                                     & 100      & 7m59s    & 4m14s     & $\times$1.8     \\
            \bottomrule
        \end{tabular}
    }
    \caption{Comparison of approximate generation times depending on platform and number of shards.}
    \label{tab:goclassy-bench}
\end{table}

Ungoliant performs better than \goclassy on all tasks, independently of the platform or number of shards processed. However, we can note that Ungoliant's speedup is higher on short tasks, which is explained by its aggressive multithreading strategy, while \goclassy uses a record-scope multithreading at its finest granularity.


\subsection{Iterating on the goclassy Pipeline}


Common Crawl dumps contain metadata that hold useful information such as related records, recognized language(s), or origin URLs. Since OSCAR's 2019 pipeline discarded metadata and sentences could be shuffled, we lost the ability to investigate the metadata itself, as well as working on potentially multilingual documents, since we separated text from metadata.

The new pipeline (and the resulting new corpus schema) aims to establish a first link between textual data and metadata from Common Crawl, while staying backward compatible with the existing OSCAR 2019 schema.

In other words, switching from the original OSCAR 2019 corpus and the newly generated one should be a drop-in operation.

% début ajout explication extraction metadata
\subsubsection{Metadata Extraction and Linking}
Our choice of keeping the corpus backward compatible with the original OSCAR 2019 introduces changes in the way the corpus is generated, namely regarding metadata: a record's body is composed of sentences that \textbf{aren't guaranteed to be of the same language}. Since OSCAR merges sentences from multiple records into a single file, special attention has to be paid to the metadata dispatch too.

Approaches to tackle this problem range from (1) storing all metadata in a single location to (2) having language-specific metadata files that contain the metadata for each line in the language file.

Both (1) and (2) have their strengths and weaknesses, namely:
\begin{enumerate}
    \item Having all metadata at the same place may facilitate wide queries about whole metadata, but at a cost of a very large size (which harms both accessibility and performance).
    \item Getting the metadata for a given line is fast since line numbers are synchronized, but there is repeated information and a potentially important increase in size.
\end{enumerate}

We thus choose a hybrid approach which keeps metadata local to each language, while trying to limit the information repetition by keeping an entry by group of \emph{chunks} rather than by line, where a \emph{chunk} is a series of contiguous sentences that share the same language from the same document.

An overview of the pipeline can be seen in Figure~\ref{fig:zoomed}, where we depict Ungoliant at a macro level in the first part of the figure, and where we also give a more precise view on record processing and metadata extraction in the second half of the figure.

Metadata is distributed via JSON-encoded files holding an ordered list of metadata entries, along with offsets ($o$) and paragraph lengths ($l$), enabling any user to get the content of a said metadata by querying for lines $(o, o+l]$ in the content file.

This approach still has drawbacks, in particular when looking for the corresponding metadata of a given sentence/paragraph, where one has to perform a search on the metadata file, or when working with multilingual documents. Another important drawback is the resulting cost of potentially merging back numerous language parts: Since metadata query is offset-based, merging back metadata files implies updating those offsets.

\begin{figure*}
    \centering
    \begin{tikzpicture}[auto,scale=0.75, every node/.style={transform shape},font=\sffamily]
        \tikzstyle{nod}=[minimum width=1.65cm,minimum height=6cm,rectangle,rounded corners=10pt,
        fill=red!20, align=center, text width=1.65cm,text centered]
        \tikzstyle{ft} = [minimum width=1.5cm,minimum height=1cm,rectangle,rounded corners=10pt,
        fill=blue!20, align=center, text width=1.5cm,text centered]
        \tikzstyle{fin}=[minimum width=4cm,minimum height=4cm,rectangle,rounded corners=10pt,
        fill=green!20, align=center, text width=4cm,text centered]
        \tikzstyle{arr}=[->,>=stealth,thick]
        \tikzstyle{arr1}=[->,>=stealth,thick]

        \node[nod] (CC) at (0,0) {Common Crawl};

        \node[minimum width=2cm, text width=2cm,text centered] (TEX1) at (2.5,3.45) {\small Compressed Files};
        \node (GZ1) at (2.5,2.3) {\Huge \faFileArchive};
        \node (GZ2) at (2.5,1) {\Huge \faFileArchive};
        \node (GZ3) at (2.5, -0.3) {\Huge \faFileArchive};
        \node (DGZ) at (2.5, -1.2) {\Huge $\vdots$};
        \node (GZ4) at (2.5,-2.3) {\Huge \faFileArchive};


        \node[minimum width=1cm, text width=1cm,text centered] (TEX1) at (4.5,3.45) {\small WET Files};
        \node (T1) at (4.5,2.3) {\Huge \faFile};
        \node (T2) at (4.5,1) {\Huge \faFile};
        \node (T3) at (4.5, -0.3) {\Huge \faFile};
        \node (DT) at (4.5, -1.2) {\Huge $\vdots$};
        \node (T4) at (4.5,-2.3) {\Huge \faFile};

        \node[ft] (F1) at (7,2.3) {Process Record};
        \node[ft] (F2) at (7,1) {Process Record};
        \node[ft] (F3) at (7,-0.3) {Process Record};
        \node (DF) at (7, -1.2) {\Huge $\vdots$};
        \node[ft] (F4) at (7,-2.3) {Process Record};

        \node[minimum width=2.3cm, text width=2.3cm,text centered] (TEX1) at (10.25,3.45) {\small Metadata, Paragraph, Language Tag};
        \node (TA1) at (10.25,2.3) {\faCode $\,$ \faParagraph $\,$ \faTags};
        \node (TA2) at (10.25,1) {\faCode $\,$ \faParagraph $\,$ \faTags};
        \node (TA3) at (10.25, -0.3) {\faCode $\,$ \faParagraph $\,$ \faTags};
        \node (DTA) at (10.25, -1.2) {\Huge $\vdots$};
        \node (TA4) at (10.25,-2.3) {\faCode $\,$ \faParagraph $\,$ \faTags};


        \node[minimum width=2.3cm, text width=2.3cm,text centered] (TEX1) at (15,3.45) {\small Files Classified by Language};
        \node[fin] (FF) at (15,0) {};
        \node (TF) at (15,1.3) {\Huge \faLanguage $\,\cdots$\faLanguage};
        \node (TF3) at (15, 0) {\Huge \faLanguage $\,\cdots$\faLanguage};
        \node (TF3) at (15, -1.3) {\Huge \faLanguage $\,\cdots$\faLanguage};


        \draw[arr] (1,2.3)--(GZ1);
        \draw[arr] (1,1)--(GZ2);
        \draw[arr] (1,-0.3)--(GZ3);
        \draw[arr] (1,-2.3)--(GZ4);


        \draw[arr] (GZ1)--(T1);
        \draw[arr] (GZ2)--(T2);
        \draw[arr] (GZ3)--(T3);
        \draw[arr] (GZ4)--(T4);


        \draw[arr1] (5,2.5)--(6.1,2.5);
        \draw[arr1] (5,2.3)--(6.1,2.3);
        \draw[arr1] (5,2.1)--(6.1,2.1);

        \draw[arr1] (5,1.2)--(6.1,1.2);
        \draw[arr1] (5,1)--(6.1,1);
        \draw[arr1] (5,0.8)--(6.1,0.8);

        \draw[arr1] (5,-0.1)--(6.1,-0.1);
        \draw[arr1] (5,-0.3)--(6.1,-0.3);
        \draw[arr1] (5,-0.5)--(6.1,-0.5);

        \draw[arr1] (5,-2.1)--(6.1,-2.1);
        \draw[arr1] (5,-2.3)--(6.1,-2.3);
        \draw[arr1] (5,-2.5)--(6.1,-2.5);


        \draw[arr] (8,2.5)--(9.1,2.5);
        \draw[arr] (8,2.3)--(9.1,2.3);
        \draw[arr] (8,2.1)--(9.1,2.1);

        \draw[arr] (8,1.2)--(9.1,1.2);
        \draw[arr] (8,1)--(9.1,1);
        \draw[arr] (8,0.8)--(9.1,0.8);

        \draw[arr] (8,-0.1)--(9.1,-0.1);
        \draw[arr] (8,-0.3)--(9.1,-0.3);
        \draw[arr] (8,-0.5)--(9.1,-0.5);

        \draw[arr] (8,-2.1)--(9.1,-2.1);
        \draw[arr] (8,-2.3)--(9.1,-2.3);
        \draw[arr] (8,-2.5)--(9.1,-2.5);


        \draw[arr] (TA1.0)--(FF);
        \draw[arr] (TA2.0)--(FF);
        \draw[arr] (TA3.0)--(FF);
        \draw[arr] (TA4.0)--(FF);
    \end{tikzpicture}


    \tikzset{every picture/.style={line width=0.75pt}} %set default line width to 0.75pt        

    \begin{tikzpicture}[x=0.75pt,y=0.75pt,yscale=-0.85,xscale=0.85, every node/.style={scale=0.85}, font=\sffamily]
        %uncomment if require: \path (0,489); %set diagram left start at 0, and has height of 489

        %Flowchart: Stored Data [id:dp4699321019986179] 
        \draw  [fill={rgb, 255:red, 80; green, 227; blue, 194 }  ,fill opacity=1 ] (238,30) -- (280,30) .. controls (275.58,30) and (272,38.95) .. (272,50) .. controls (272,61.05) and (275.58,70) .. (280,70) -- (238,70) .. controls (233.58,70) and (230,61.05) .. (230,50) .. controls (230,38.95) and (233.58,30) .. (238,30) -- cycle ;
        %Rounded Rect [id:dp4628690629420654] 
        \draw  [fill={rgb, 255:red, 245; green, 166; blue, 35 }  ,fill opacity=1 ] (20,24.6) .. controls (20,17.09) and (26.09,11) .. (33.6,11) -- (74.4,11) .. controls (81.91,11) and (88,17.09) .. (88,24.6) -- (88,276.4) .. controls (88,283.91) and (81.91,290) .. (74.4,290) -- (33.6,290) .. controls (26.09,290) and (20,283.91) .. (20,276.4) -- cycle ;
        %Straight Lines [id:da8313420982420904] 
        \draw    (90,40) -- (128,40) ;
        \draw [shift={(130,40)}, rotate = 180] [color={rgb, 255:red, 0; green, 0; blue, 0 }  ][line width=0.75]    (10.93,-3.29) .. controls (6.95,-1.4) and (3.31,-0.3) .. (0,0) .. controls (3.31,0.3) and (6.95,1.4) .. (10.93,3.29)   ;
        %Straight Lines [id:da4482967760350627] 
        \draw    (90,149) -- (128,149) ;
        \draw [shift={(130,149)}, rotate = 180] [color={rgb, 255:red, 0; green, 0; blue, 0 }  ][line width=0.75]    (10.93,-3.29) .. controls (6.95,-1.4) and (3.31,-0.3) .. (0,0) .. controls (3.31,0.3) and (6.95,1.4) .. (10.93,3.29)   ;
        %Rounded Rect [id:dp9849601809689044] 
        \draw  [fill={rgb, 255:red, 74; green, 144; blue, 226 }  ,fill opacity=1 ] (270,38) .. controls (270,33.58) and (273.58,30) .. (278,30) -- (332,30) .. controls (336.42,30) and (340,33.58) .. (340,38) -- (340,62) .. controls (340,66.42) and (336.42,70) .. (332,70) -- (278,70) .. controls (273.58,70) and (270,66.42) .. (270,62) -- cycle ;

        %Straight Lines [id:da8668621904345621] 
        \draw    (200,40) -- (228,40) ;
        \draw [shift={(230,40)}, rotate = 180] [color={rgb, 255:red, 0; green, 0; blue, 0 }  ][line width=0.75]    (10.93,-3.29) .. controls (6.95,-1.4) and (3.31,-0.3) .. (0,0) .. controls (3.31,0.3) and (6.95,1.4) .. (10.93,3.29)   ;
        %Straight Lines [id:da996242982187254] 
        \draw    (180,50) -- (180,60) ;
        %Straight Lines [id:da7644452497572334] 
        \draw    (190,50) -- (228,50) ;
        \draw [shift={(230,50)}, rotate = 180] [color={rgb, 255:red, 0; green, 0; blue, 0 }  ][line width=0.75]    (10.93,-3.29) .. controls (6.95,-1.4) and (3.31,-0.3) .. (0,0) .. controls (3.31,0.3) and (6.95,1.4) .. (10.93,3.29)   ;
        %Straight Lines [id:da8688952100791735] 
        \draw    (180,60) -- (228,60) ;
        \draw [shift={(230,60)}, rotate = 180] [color={rgb, 255:red, 0; green, 0; blue, 0 }  ][line width=0.75]    (10.93,-3.29) .. controls (6.95,-1.4) and (3.31,-0.3) .. (0,0) .. controls (3.31,0.3) and (6.95,1.4) .. (10.93,3.29)   ;
        %Straight Lines [id:da26148793342838117] 
        \draw    (340,40) -- (368,40) ;
        \draw [shift={(370,40)}, rotate = 180] [color={rgb, 255:red, 0; green, 0; blue, 0 }  ][line width=0.75]    (10.93,-3.29) .. controls (6.95,-1.4) and (3.31,-0.3) .. (0,0) .. controls (3.31,0.3) and (6.95,1.4) .. (10.93,3.29)   ;
        %Straight Lines [id:da4954783851514327] 
        \draw    (340,60) -- (368,60) ;
        \draw [shift={(370,60)}, rotate = 180] [color={rgb, 255:red, 0; green, 0; blue, 0 }  ][line width=0.75]    (10.93,-3.29) .. controls (6.95,-1.4) and (3.31,-0.3) .. (0,0) .. controls (3.31,0.3) and (6.95,1.4) .. (10.93,3.29)   ;
        %Straight Lines [id:da05936926008094867] 
        \draw    (340,50) -- (368,50) ;
        \draw [shift={(370,50)}, rotate = 180] [color={rgb, 255:red, 0; green, 0; blue, 0 }  ][line width=0.75]    (10.93,-3.29) .. controls (6.95,-1.4) and (3.31,-0.3) .. (0,0) .. controls (3.31,0.3) and (6.95,1.4) .. (10.93,3.29)   ;
        %Rounded Rect [id:dp49349855068480486] 
        \draw  [fill={rgb, 255:red, 126; green, 211; blue, 33 }  ,fill opacity=1 ] (460,86) .. controls (460,82.69) and (462.69,80) .. (466,80) -- (614,80) .. controls (617.31,80) and (620,82.69) .. (620,86) -- (620,104) .. controls (620,107.31) and (617.31,110) .. (614,110) -- (466,110) .. controls (462.69,110) and (460,107.31) .. (460,104) -- cycle ;

        %Straight Lines [id:da9110150675099965] 
        \draw    (445,40) -- (590,40) ;
        %Straight Lines [id:da9102373834436741] 
        \draw    (445,50) -- (540,50) ;
        %Straight Lines [id:da007117412031712345] 
        \draw    (445,60) -- (490,60) ;
        %Straight Lines [id:da6704245551358066] 
        \draw    (590,40) -- (590,78) ;
        \draw [shift={(590,80)}, rotate = 270] [color={rgb, 255:red, 0; green, 0; blue, 0 }  ][line width=0.75]    (10.93,-3.29) .. controls (6.95,-1.4) and (3.31,-0.3) .. (0,0) .. controls (3.31,0.3) and (6.95,1.4) .. (10.93,3.29)   ;
        %Straight Lines [id:da9774749934022594] 
        \draw    (540,50) -- (540,78) ;
        \draw [shift={(540,80)}, rotate = 270] [color={rgb, 255:red, 0; green, 0; blue, 0 }  ][line width=0.75]    (10.93,-3.29) .. controls (6.95,-1.4) and (3.31,-0.3) .. (0,0) .. controls (3.31,0.3) and (6.95,1.4) .. (10.93,3.29)   ;
        %Straight Lines [id:da9977400295475213] 
        \draw    (490,60) -- (490,78) ;
        \draw [shift={(490,80)}, rotate = 270] [color={rgb, 255:red, 0; green, 0; blue, 0 }  ][line width=0.75]    (10.93,-3.29) .. controls (6.95,-1.4) and (3.31,-0.3) .. (0,0) .. controls (3.31,0.3) and (6.95,1.4) .. (10.93,3.29)   ;
        %Shape: Rectangle [id:dp2031974452528671] 
        \draw  [fill={rgb, 255:red, 248; green, 231; blue, 28 }  ,fill opacity=1 ] (440,131) -- (500,131) -- (500,171) -- (440,171) -- cycle ;
        %Shape: Rectangle [id:dp3227217281027539] 
        \draw   (440,151) -- (470,151) -- (470,171) -- (440,171) -- cycle ;
        %Shape: Rectangle [id:dp8945259346838845] 
        \draw   (470,151) -- (500,151) -- (500,171) -- (470,171) -- cycle ;
        %Shape: Rectangle [id:dp8533666881534692] 
        \draw  [fill={rgb, 255:red, 248; green, 173; blue, 28 }  ,fill opacity=1 ] (580,131) -- (640,131) -- (640,171) -- (580,171) -- cycle ;
        %Shape: Rectangle [id:dp24559484415109678] 
        \draw   (580,151) -- (610,151) -- (610,171) -- (580,171) -- cycle ;
        %Shape: Rectangle [id:dp27135592725947755] 
        \draw   (610,151) -- (640,151) -- (640,171) -- (610,171) -- cycle ;
        %Shape: Rectangle [id:dp14887885636177012] 
        \draw  [fill={rgb, 255:red, 248; green, 255; blue, 28 }  ,fill opacity=1 ] (510,131) -- (570,131) -- (570,171) -- (510,171) -- cycle ;
        %Shape: Rectangle [id:dp6142444969447188] 
        \draw   (510,151) -- (540,151) -- (540,171) -- (510,171) -- cycle ;
        %Shape: Rectangle [id:dp3599126658761931] 
        \draw   (540,151) -- (570,151) -- (570,171) -- (540,171) -- cycle ;
        %Straight Lines [id:da6637979435067474] 
        \draw    (190,150) -- (420,150) ;
        %Straight Lines [id:da0639984733030543] 
        \draw    (420,150) -- (420,180) ;
        %Straight Lines [id:da668825270485392] 
        \draw    (420,180) -- (438,180) ;
        \draw [shift={(440,180)}, rotate = 180] [color={rgb, 255:red, 0; green, 0; blue, 0 }  ][line width=0.75]    (10.93,-3.29) .. controls (6.95,-1.4) and (3.31,-0.3) .. (0,0) .. controls (3.31,0.3) and (6.95,1.4) .. (10.93,3.29)   ;
        %Straight Lines [id:da3417800565981026] 
        \draw    (400,150) -- (400,160) ;
        %Straight Lines [id:da09888916391476721] 
        \draw    (400,160) -- (400,210) ;
        %Straight Lines [id:da6576198258261287] 
        \draw    (380,150) -- (380,230) ;
        %Straight Lines [id:da39423430620012667] 
        \draw    (400,210) -- (520,210) ;
        %Straight Lines [id:da36377145787111487] 
        \draw    (380,230) -- (590,230) ;
        %Straight Lines [id:da20949009357910353] 
        \draw    (520,210) -- (520,192) ;
        \draw [shift={(520,190)}, rotate = 450] [color={rgb, 255:red, 0; green, 0; blue, 0 }  ][line width=0.75]    (10.93,-3.29) .. controls (6.95,-1.4) and (3.31,-0.3) .. (0,0) .. controls (3.31,0.3) and (6.95,1.4) .. (10.93,3.29)   ;
        %Straight Lines [id:da9243090582667062] 
        \draw    (590,230) -- (590,192) ;
        \draw [shift={(590,190)}, rotate = 450] [color={rgb, 255:red, 0; green, 0; blue, 0 }  ][line width=0.75]    (10.93,-3.29) .. controls (6.95,-1.4) and (3.31,-0.3) .. (0,0) .. controls (3.31,0.3) and (6.95,1.4) .. (10.93,3.29)   ;
        %Snip Round Single Corner Rect [id:dp6248686818369311] 
        \draw   (500,187) .. controls (500,189.21) and (498.21,191) .. (496,191) -- (444,191) -- (440,187) -- (440,171) -- (500,171) -- cycle ;
        %Snip Round Single Corner Rect [id:dp4416079275569046] 
        \draw   (570,187) .. controls (570,189.21) and (568.21,191) .. (566,191) -- (514,191) -- (510,187) -- (510,171) -- (570,171) -- cycle ;
        %Snip Round Single Corner Rect [id:dp7248479469870895] 
        \draw   (640,187) .. controls (640,189.21) and (638.21,191) .. (636,191) -- (584,191) -- (580,187) -- (580,171) -- (640,171) -- cycle ;
        %Straight Lines [id:da02360656408875572] 
        \draw    (540,110) -- (540,128) ;
        \draw [shift={(540,130)}, rotate = 270] [color={rgb, 255:red, 0; green, 0; blue, 0 }  ][line width=0.75]    (10.93,-3.29) .. controls (6.95,-1.4) and (3.31,-0.3) .. (0,0) .. controls (3.31,0.3) and (6.95,1.4) .. (10.93,3.29)   ;
        %Straight Lines [id:da5151329005041937] 
        \draw    (610,110) -- (610,128) ;
        \draw [shift={(610,130)}, rotate = 270] [color={rgb, 255:red, 0; green, 0; blue, 0 }  ][line width=0.75]    (10.93,-3.29) .. controls (6.95,-1.4) and (3.31,-0.3) .. (0,0) .. controls (3.31,0.3) and (6.95,1.4) .. (10.93,3.29)   ;
        %Straight Lines [id:da5710017720882605] 
        \draw    (470,110) -- (470,128) ;
        \draw [shift={(470,130)}, rotate = 270] [color={rgb, 255:red, 0; green, 0; blue, 0 }  ][line width=0.75]    (10.93,-3.29) .. controls (6.95,-1.4) and (3.31,-0.3) .. (0,0) .. controls (3.31,0.3) and (6.95,1.4) .. (10.93,3.29)   ;
        %Rounded Rect [id:dp34777737838988] 
        \draw   (250,312) .. controls (250,288.8) and (268.8,270) .. (292,270) -- (598,270) .. controls (621.2,270) and (640,288.8) .. (640,312) -- (640,438) .. controls (640,461.2) and (621.2,480) .. (598,480) -- (292,480) .. controls (268.8,480) and (250,461.2) .. (250,438) -- cycle ;
        %Flowchart: Multidocument [id:dp38628004470840416] 
        \draw  [fill={rgb, 255:red, 255; green, 255; blue, 255 }  ,fill opacity=1 ] (562,290) -- (610,290) -- (610,316.51) .. controls (580,316.51) and (586,326.07) .. (562,319.88) -- cycle ; \draw  [fill={rgb, 255:red, 255; green, 255; blue, 255 }  ,fill opacity=1 ] (556,294.02) -- (604,294.02) -- (604,320.53) .. controls (574,320.53) and (580,330.09) .. (556,323.9) -- cycle ; \draw  [fill={rgb, 255:red, 255; green, 255; blue, 255 }  ,fill opacity=1 ] (550,298.03) -- (598,298.03) -- (598,324.54) .. controls (568,324.54) and (574,334.1) .. (550,327.92) -- cycle ;

        %Flowchart: Multidocument [id:dp29609362181778487] 
        \draw  [fill={rgb, 255:red, 255; green, 255; blue, 255 }  ,fill opacity=1 ] (562,350.25) -- (610,350.25) -- (610,376.59) .. controls (580,376.59) and (586,386.09) .. (562,379.95) -- cycle ; \draw  [fill={rgb, 255:red, 255; green, 255; blue, 255 }  ,fill opacity=1 ] (556,354.24) -- (604,354.24) -- (604,380.58) .. controls (574,380.58) and (580,390.09) .. (556,383.94) -- cycle ; \draw  [fill={rgb, 255:red, 255; green, 255; blue, 255 }  ,fill opacity=1 ] (550,358.23) -- (598,358.23) -- (598,384.58) .. controls (568,384.58) and (574,394.08) .. (550,387.93) -- cycle ;

        %Straight Lines [id:da0033612499475468294] 
        \draw    (470,190) -- (470,268) ;
        \draw [shift={(470,270)}, rotate = 270] [color={rgb, 255:red, 0; green, 0; blue, 0 }  ][line width=0.75]    (10.93,-3.29) .. controls (6.95,-1.4) and (3.31,-0.3) .. (0,0) .. controls (3.31,0.3) and (6.95,1.4) .. (10.93,3.29)   ;
        %Straight Lines [id:da08266362916000991] 
        \draw    (540,190) -- (540,268) ;
        \draw [shift={(540,270)}, rotate = 270] [color={rgb, 255:red, 0; green, 0; blue, 0 }  ][line width=0.75]    (10.93,-3.29) .. controls (6.95,-1.4) and (3.31,-0.3) .. (0,0) .. controls (3.31,0.3) and (6.95,1.4) .. (10.93,3.29)   ;
        %Straight Lines [id:da7915660415727729] 
        \draw    (612,190) -- (612,268) ;
        \draw [shift={(612,270)}, rotate = 270] [color={rgb, 255:red, 0; green, 0; blue, 0 }  ][line width=0.75]    (10.93,-3.29) .. controls (6.95,-1.4) and (3.31,-0.3) .. (0,0) .. controls (3.31,0.3) and (6.95,1.4) .. (10.93,3.29)   ;
        %Flowchart: Magnetic Disk [id:dp4806528931350428] 
        \draw   (450,428.75) -- (450,461.25) .. controls (450,466.08) and (425.38,470) .. (395,470) .. controls (364.62,470) and (340,466.08) .. (340,461.25) -- (340,428.75)(450,428.75) .. controls (450,433.58) and (425.38,437.5) .. (395,437.5) .. controls (364.62,437.5) and (340,433.58) .. (340,428.75) .. controls (340,423.92) and (364.62,420) .. (395,420) .. controls (425.38,420) and (450,423.92) .. (450,428.75) -- cycle ;

        %Straight Lines [id:da24403954969455177] 
        \draw    (340,300) -- (548,300) ;
        \draw [shift={(550,300)}, rotate = 180] [color={rgb, 255:red, 0; green, 0; blue, 0 }  ][line width=0.75]    (10.93,-3.29) .. controls (6.95,-1.4) and (3.31,-0.3) .. (0,0) .. controls (3.31,0.3) and (6.95,1.4) .. (10.93,3.29)   ;
        %Straight Lines [id:da4710037635075677] 
        \draw    (370,420) -- (370,410) ;
        \draw   (360,360) .. controls (360,354.48) and (364.48,350) .. (370,350) .. controls (375.52,350) and (380,354.48) .. (380,360) .. controls (380,365.52) and (375.52,370) .. (370,370) .. controls (364.48,370) and (360,365.52) .. (360,360) -- cycle ; \draw   (360,360) -- (380,360) ; \draw   (370,350) -- (370,370) ;
        %Straight Lines [id:da8378723284680856] 
        \draw    (330,360) -- (360,360) ;
        %Straight Lines [id:da16198507756193636] 
        \draw    (380,360) -- (548,360) ;
        \draw [shift={(550,360)}, rotate = 180] [color={rgb, 255:red, 0; green, 0; blue, 0 }  ][line width=0.75]    (10.93,-3.29) .. controls (6.95,-1.4) and (3.31,-0.3) .. (0,0) .. controls (3.31,0.3) and (6.95,1.4) .. (10.93,3.29)   ;
        %Straight Lines [id:da13573094047430956] 
        \draw    (370,390) -- (370,372) ;
        \draw [shift={(370,370)}, rotate = 450] [color={rgb, 255:red, 0; green, 0; blue, 0 }  ][line width=0.75]    (10.93,-3.29) .. controls (6.95,-1.4) and (3.31,-0.3) .. (0,0) .. controls (3.31,0.3) and (6.95,1.4) .. (10.93,3.29)   ;
        %Straight Lines [id:da8677494244961229] 
        \draw    (430,410) -- (430,418) ;
        \draw [shift={(430,420)}, rotate = 270] [color={rgb, 255:red, 0; green, 0; blue, 0 }  ][line width=0.75]    (10.93,-3.29) .. controls (6.95,-1.4) and (3.31,-0.3) .. (0,0) .. controls (3.31,0.3) and (6.95,1.4) .. (10.93,3.29)   ;
        %Straight Lines [id:da5717074508020997] 
        \draw    (430,390) -- (430,360) ;

        % Text Node
        \draw (29,142) node [anchor=north west][inner sep=0.75pt]   [align=left] {Record};
        % Text Node
        \draw (131,31) node [anchor=north west][inner sep=0.75pt]   [align=left] {sentences};
        % Text Node
        \draw (131,141) node [anchor=north west][inner sep=0.75pt]   [align=left] {headers};
        % Text Node
        \draw (281,42) node [anchor=north west][inner sep=0.75pt]   [align=left] {fasttext};
        % Text Node
        \draw (371,32) node [anchor=north west][inner sep=0.75pt]  [align=left] {{\scriptsize (sentence, tag)}};
        % Text Node
        \draw (371,42) node [anchor=north west][inner sep=0.75pt]  [align=left] {{\scriptsize (sentence, tag)}};
        % Text Node
        \draw (371,52) node [anchor=north west][inner sep=0.75pt]  [align=left] {{\scriptsize (sentence, tag)}};
        % Text Node
        \draw (235,42) node [anchor=north west][inner sep=0.75pt]  [align=left] {filter};
        % Text Node
        \draw (485,86.13) node [anchor=north west][inner sep=0.75pt]  [align=left] {\begin{minipage}[lt]{80pt}\setlength\topsep{0pt}
                \begin{center}
                    group languages
                \end{center}

            \end{minipage}};
        % Text Node
        \draw (442,174) node [anchor=north west][inner sep=0.75pt]   [align=left] {headers};
        % Text Node
        \draw (512,174) node [anchor=north west][inner sep=0.75pt]   [align=left] {headers};
        % Text Node
        \draw (582,174) node [anchor=north west][inner sep=0.75pt]   [align=left] {headers};
        % Text Node
        \draw (442,132) node [anchor=north west][inner sep=0.75pt]   [align=left] {{\footnotesize sentences}};
        % Text Node
        \draw (441,153) node [anchor=north west][inner sep=0.75pt]   [align=left] {{\scriptsize {lang}}};
        % Text Node
        \draw (462,153) node [anchor=north west][inner sep=0.75pt]   [align=left] {\begin{minipage}[lt]{19.86pt}\setlength\topsep{0pt}
                \begin{flushright}
                    {\scriptsize {len}}
                \end{flushright}

            \end{minipage}};
        % Text Node
        \draw (511,153) node [anchor=north west][inner sep=0.75pt]   [align=left] {{\scriptsize {lang}}};
        % Text Node
        \draw (532,153) node [anchor=north west][inner sep=0.75pt]   [align=left] {\begin{minipage}[lt]{19.86pt}\setlength\topsep{0pt}
                \begin{flushright}
                    {\scriptsize {len}}
                \end{flushright}

            \end{minipage}};
        % Text Node
        \draw (581,153) node [anchor=north west][inner sep=0.75pt]   [align=left] {{\scriptsize {lang}}};
        % Text Node
        \draw (602,153) node [anchor=north west][inner sep=0.75pt]   [align=left] {\begin{minipage}[lt]{19.86pt}\setlength\topsep{0pt}
                \begin{flushright}
                    {\scriptsize {len}}
                \end{flushright}

            \end{minipage}};
        % Text Node
        \draw (512,132) node [anchor=north west][inner sep=0.75pt]   [align=left] {{\footnotesize sentences}};
        % Text Node
        \draw (581,132) node [anchor=north west][inner sep=0.75pt]   [align=left] {{\footnotesize sentences}};
        % Text Node
        \draw (369.8,441) node [anchor=north west][inner sep=0.75pt]   [align=left] {offsets};
        % Text Node
        \draw (556.13,362.04) node [anchor=north west][inner sep=0.75pt]   [align=left] {meta};
        % Text Node
        \draw (562.5,301.25) node [anchor=north west][inner sep=0.75pt]   [align=left] {text};
        % Text Node
        \draw (271,291) node [anchor=north west][inner sep=0.75pt]   [align=left] {sentences};
        % Text Node
        \draw (271,351) node [anchor=north west][inner sep=0.75pt]   [align=left] {headers};
        % Text Node
        \draw (351,391) node [anchor=north west][inner sep=0.75pt]   [align=left] {{\small offset}};
        % Text Node
        \draw (401,391) node [anchor=north west][inner sep=0.75pt]   [align=left] {{\small offset + len}};
        % Text Node
        \draw (448,364) node [anchor=north west][inner sep=0.75pt]  [rotate=-90] [align=left] {{\tiny update}};


    \end{tikzpicture}

    \caption{Record processing with metadata extraction. Headers are kept aside while sentences are identified and grouped into same-language bins. Headers are then cloned for each bin, and are sequentially stamped with an offset that is recorded for the whole operation, and written to disk into text and metadata files by language.}
    \label{fig:zoomed}
\end{figure*}

Having paragraphs and metadata linked by offsets in a highly parallelized pipeline implies to take special care at the offset level. The solution is to use shard-scoped offsets (starting from $0$ for each language), and to keep global offsets protected by a mutex guard. This way, when a given shard is done processing and is ready to be written on disk, we convert shard-scoped offsets to global-scoped ones, update the global-scoped ones and then write text and metadata on disk.

\begin{table}[t]
    \centering\small
    \scalebox{0.85}{
        \begin{tabular}{lrrrr}
            \toprule
            Platform                 & \#shards & Without Metadata & With Metadata & Speedup     \\
            \midrule
            \multirow{3}{*}{Desktop} & 1        & 13s              & 12s           & $\times$1.1 \\
                                     & 10       & 2m12s            & 1m55s         & $\times$1.1 \\
                                     & 25       & 5m47s            & 4m50s         & $\times$1.2 \\
            \midrule
            \multirow{3}{*}{HPC}     & 1        & 6s               & 7s            & $\times$0.9 \\
                                     & 25       & 1m6s             & 1m12s         & $\times$0.9 \\
                                     & 100      & 4m14s            & 4m36s         & $\times$0.9 \\
            \bottomrule
        \end{tabular}
    }
    \caption{Comparison of approximate generation times with and without metadata generation.}
    \label{tab:pipelines-bench}
\end{table}

We compare running times for the reimplementation of the \goclassy pipeline, and our new pipeline adding metadata extraction, using both desktop and HPC contexts. The results are reported in Table ~\ref{tab:pipelines-bench}.

Metadata generation does not seem to influence generation time dramatically. However, we can notice a slight performance difference between HPC and Desktop contexts. These differences may lie in the storage medium differences, I/O layout, or algorithmic peculiarities benefiting desktop contexts because of other bottlenecks.


\subsection{Characteristics of the OSCAR 21.09 Corpus}

We evaluate the newly generated OSCAR 21.09 corpus (published on September 2021\footnote{\url{https://oscar-corpus.com/post/oscar-v21-09/}}), assessing its ability to reflect events that occurred after the publication of OSCAR 2019, that is, events that occurred after November 2018, and we detail the metadata format and potential use.

\subsubsection{Comparison with OSCAR 2019}

While it is expected that our new corpus has a larger file size than OSCAR 2019 since Common Crawl itself grew from 7.42 TB to 8.06 TB, metadata quickly adds up and accounts for nearly 15\% of the total uncompressed data in OSCAR 21.09.

\begin{table}[t]
    \centering\small
    \scalebox{0.82}{
        \begin{tabular}{lrrrr}
            \toprule
            OSCAR Version & Common Crawl & OSCAR (dedup) & Metadata & Total (increase) \\
            \midrule
            2019          & 7.42TB       & 6.3TB (3.2TB) & N/A      & 6.3TB            \\
            \midrule
            21.09         & 8.06TB       & 7.2TB (3.3TB) & 1.2TB    & 8.4TB ($+33\%$)  \\
            \bottomrule
        \end{tabular}
    }
    \caption{Comparison of the size of the Common Crawl dumps and their corresponding OSCAR sizes between the 2019 and the 21.09 versions. Compressed (Common Crawl) sources are from November 2018 and February 2021 dumps. Total is Textual + Metadata without deduplication.}
    \label{tab:oscar-size}
\end{table}

The size difference is not the same for each language, and while the corpus as a whole is bigger now, some languages are smaller than they were before.

\begin{figure}[ht]
    \centering
    \includegraphics[width=0.75\textwidth, angle=0]{static/media/oscar/ungoliant/size_evo}
    \caption{Comparison of language size (in bytes) between OSCAR 2018 and OSCAR 2021 (top/bottom 5 only). }
    \label{fig:lang-size}
\end{figure}

Results show that already largely represented languages gain more and more data (like the English language, which constituted more than a third of the original OSCAR 2019), except for the Russian language which loses approximately 100Gb of textual content. These results are summarized in Figure~\ref{fig:lang-size}.

However, in a context where the number of languages is very high (higher than 150) and of varying sizes, evolution can't be analyzed via a mere size evaluation. By computing, for each language, the relative size difference between the 2019 and 21.09 releases of OSCAR, less resourced languages do appear, hinting at a better representation of some of them. These results can be found in Figure~\ref{fig:lang-size-pctg}.

Note nonetheless that numerous languages have been omitted from Figure~\ref{fig:lang-size-pctg}, either:
\begin{itemize}
    \item because they were present in the original OSCAR 2019 and are now absent (\textit{Central Bikol} and \textit{Cantonese})
    \item or because they were absent in the original OSCAR 2019 and are now present (\textit{Manx}, \textit{Rusyn}, \textit{Scots} and \textit{West Flemish})
\end{itemize}

Precautions have to be taken when using these corpora and further work has to be done to correctly assess the quality of low-to-mid resource languages in order to better reflect the quality of each corpus to the OSCAR users. Some sub-corpora exhibited either a particularly low number of sentences or just very low quality data, and as such they are not really usable in practice. However, they still account for a language in the total language count of both the original OSCAR 2019 and the new OSCAR 21.09.

\begin{figure}[ht]
    \centering
    \includegraphics[width=0.75\textwidth, angle=0]{static/media/oscar/ungoliant/size_evo_pctg}
    \caption{Comparison of language percentage between OSCAR 2018 and OSCAR 2021 (top/bottom 5 only).}
    \label{fig:lang-size-pctg}
\end{figure}

\subsubsection{Metadata}

Metadata provides new contextual data that is useful to evaluate the corpus and draw metrics.

The total size of metadata is 1.2 TB, ranging from 4Kb to 500Gb, depending on the number of lines. Relative size varies from 100\% to 20\%, diminishing with the textual data size, which is expected.

Our choice of keeping metadata aside from the main content adds some complexity when working with both textual and contextual data:

\begin{itemize}
    \item When trying to get the metadata of given sentence, one has to get the line number $k$, then sequentially (or use a search algorithm since offsets are sorted) look for the record (with offset $o$ and length $l$), where $k \in [o, o+l]$.
    \item Looking for lines corresponding to a particular metadata entry is easier: one has to read the textual file, skipping until the $o$-th line, then read $l$ lines.
\end{itemize}


\subsubsection{Presence of events}

Using a sample of five sub-corpora, we perform a simple search of terms in order to assess and compare the presence of pre- and post- 2018 events and persons in both corpora. Terms and frequency are grouped in Table \ref{tab:word-frequency}.

\begin{table}[t]
    \centering\small
    \begin{tabular}{lrrrr}
        \toprule
        Language                  & Term                  & 2018  & 2021  \\
        \midrule
        \multirow{1}{*}{Arabic}   & Beirut port explosion & 0     & 31    \\
        \multirow{1}{*}{Burmese*} & Min Aung Hlaing       & 387   & 3439  \\
        \multirow{1}{*}{English}  & Obama                 & 30039 & 27639 \\
        \multirow{1}{*}{English}  & Biden                 & 990   & 19299 \\
        \multirow{1}{*}{French}   & Yellow Vests          & 2     & 96    \\
        \bottomrule
    \end{tabular}
    \caption{Comparison of occurrences of news-related terms between OSCAR and our corpus in a sample of 100 Common Crawl shards. For the Burmese language, we use the whole 2018 and 2021 corpus since it is a low resource language. Terms are translated to the target language.}
    \label{tab:word-frequency}
\end{table}

Our corpus keeps around the same number of occurrences for pre-2018 events or public figures such as Barack Obama, while increasing the occurrence of people linked to more recent events (Joe Biden).

We include search terms linked to post-2018 events in French and Arabic which are smaller corpora (resp. 200 and 80 GB), and in Burmese, a mid-resource language (approximately 2 GB). We observe a term occurrences evolution that reflects the linked events' timing and importance.

\subsection{License}

This new OSCAR 21.09 corpus is released under a research-only license that is compliant with the EU's exceptions for research in text and data mining. Contrarily to the original OSCAR 2019, no shuffled version of the corpus is distributed, instead we put in place an authentication system that allows us to verify that requests for the corpus come from research institutions. A contact form is also provided for independent researchers so that we can study their particular cases and determine if the utilization of the corpus corresponds to a legitimate research use.

Moreover, the introduction of metadata makes our corpus far more queryable, thus simplifying and speeding up the handling of take-down GDPR requests. For this reason, we release the complete set of metadata under a CC0 public domain license, so that any individual can check if their personal or even copyrighted data is in our new OSCAR 21.09 corpus and make a request accordingly.

\section{Conclusion}

Although the work presented in this particular chapter does not directly address some of the previous concerns raised by \citet{caswell-etal-2020-language,kreutzer-etal-2021-quality}. We do believe that a more efficient, more modular and better documented pipeline is the first step in making the OSCAR project more approachable by other members of the NLP and Digital Humanities communities.

Moreover, we also believe that the addition of metadata to OSCAR is a big step towards improving the quality of its content as it will provide us and other researchers willing to use OSCAR with enough information to better explore, audit, annotate and filter the corpus.

In the next and final chapter of the OSCAR part we will explore the question of document integrity which might be useful for researchers interested in document level tasks and which until now is not respected for Common Crawl records containing multilingual data. We will also continue improving Ungoliant and start using the metadata that we extract from the Common Crawl records to produce the first ever OSCAR annotations.
%%%%%%%%%%%%%%%%%%%%%%%%%%%%%%%%%%%%%%%%%%%%%%%%%%%%%%%%%%%%%%%%%%%%%%%%
\chapter{Towards a Cleaner Document-Oriented Annotated OSCAR Corpus}
%%%%%%%%%%%%%%%%%%%%%%%%%%%%%%%%%%%%%%%%%%%%%%%%%%%%%%%%%%%%%%%%%%%%%%%%

\begin{center}
    \begin{minipage}{0.66\textwidth}
        \begin{small}
            In which we present the work of \citet{abadji-etal-2022-towards}, who continued improving over the second OSCAR pipeline \emph{Ungoliant} by adding mechanisms to ensure document integrity, specially for multilingual records of Common Crawl, and also by adding the first methods for simple annotations of the OSCAR corpus that would allow users to more easily filter the data and obtain a cleaner dataset specially for language modeling applications.
        \end{small}
    \end{minipage}
    \vspace{0.5cm}
\end{center}

In this final chapter about the OSCAR project we present the first methods for adding simple annotators to the Ungoliant pipeline that build upon the improvements presented in the previous chapter and that actually allow us to finally start addressing some problems exposed in chapter \ref{chap:quality} and in far more detail in \citep{caswell-etal-2020-language,kreutzer-etal-2021-quality}. Moreover, we also introduce a new method for document level language classification that:

\begin{enumerate}
    \item Is based on line-level language classification allowing us to hopefully preserve the classification quality that we saw in chapter \ref{chap:quality}.
    \item Allow us to respect document integrity such that we can establish a one to one correspondence between OSCAR documents and Common Crawl records.
    \item Allows us to get multilingual documents that might one day serve as the basis of a parallel OSCAR corpus.
\end{enumerate}

\section{Filtering}

Previous OSCAR pipelines were line-oriented (where a line is defined as a string separated by \texttt{\textbackslash n}), which meant that the highest filtering granularity were lines.
Having a document-oriented corpus implies that:
\begin{itemize}
    \item We must try to keep the document integrity, by altering it in a way that does not completely destroy its coherence.
    \item Operations on the document (filtering, identification, annotation) must take into account the document as a whole.
\end{itemize}

We aim to produce a corpus that is similar in size and quality to OSCAR 21.09, looking for a set of filters that limits the inclusion of short, noisy lines in documents, while keeping a sufficient quantity of data, especially for low- and mid-resource languages. Those filters either keep/discard a given document, or remove lines from the document body then keep it.

\subsection {Header and footer filter}

Similar to previous OSCAR pipelines, we use a length-based filter discarding short-lines. However, we restrict the removal on contiguous sequences of short lines that are located either at the head or at the tail of the document. In the following document, only the lines preceded by an exclamation point would be kept.

\begin{verbatim}
Home
Login
Sign Up
Welcome to my Website
! Lorem Ipsum Dolor Sit Amet ....
! Lorem Ipsum Dolor Sit Amet ....
! Lorem Ipsum Dolor Sit Amet ....
! Lorem Ipsum Dolor Sit Amet ....
Copyright Myself
Legal
Contact
\end{verbatim}

The solution still has numerous drawbacks, especially when dealing with documents crawled from the internet, a source known to be extremely noisy and full of edge cases: Adding a long line at the very head and tail of the previous document would completely negate the benefits of the filter.

\subsection{Short lines proportion filter}

In order to refine the filtering process, we use a count-based filter that separates the data in two bins: One for short lines and one for long lines. The filter then checks which bin is bigger, and filters out documents where the short lines bin is bigger.

This filter may limit the impact of documents containing low-quality long lines at the head/tail, then a high number of short lines.


\section{Identification}

The backbone of the language identification process is similar to the one used in \goclassy \cite{ortiz-suarez-etal-2019-asynchronous} for the generation of OSCAR 2019 and Ungoliant \cite{abadji-etal-2021-ungoliant} for the generation of OSCAR 21.09. However, shifting to a document oriented corpus (with a single top-level identification per document) requires to infer the document identification, based on line identifications.


We define a document $\mathcal{D}$ as a pair $\mathcal{D}=(\mathcal{L}, \mathscr{L})$ where $\mathcal{L}=\{l_1,\ldots,l_n\}$ is the set of lines (strings separated by \texttt{\textbackslash n}) that constitute the document and $\mathscr{L} = \{g_1, \ldots, g_m\}$\footnote{Note that since FastText identifies one language by line, we have always have $m\le n$ for every document $\mathcal{D}$.} is the set of languages identified by FastText for the document $\mathcal{D}$. When FastText is not able to identify a language for a specific line, for instance because the confidence isn't higher than $0.8$, we tag said line with the \emph{No Identification Language} that we simply note by $g_0$. Furthermore, we define each line $l_i$ in a document $\mathcal{D}$ as a triplet $l_k=(g_i, p_i, s_i)$ where $g_i$ is the language identified by FastText with the highest confidence for the line $l_i$, $p_i$ is said confidence and $s_i$ is the size in bytes of the line $l_i$. We also note $|l_i|=s_i$, and we thus define the size $|\mathcal{D}|$ of a document $\mathcal{D}$ as
\[
    |\mathcal{D}| = \sum_{i=0}^{n} |l_i| = \sum_{i=0}^{n} s_i.
\]
Moreover, for each identified language $g_j \in \mathscr{L}$ in a document containing $n$ lines, we define its size $|g_j|$ as
\[
    |g_j| = \sum_{\mathclap{\{s_i \mid g_i = g_j\}}} s_i.
\]
Finally, for each language $g_j \in \mathscr{L}$ we can also compute its \emph{overall weighted confidence} $P_j$ throughout the document $\mathcal{D}$ as the following weighted mean:
\[
    P_j = |D|^{-1}\sum_{\mathclap{\{s_i|g_i=g_j\}}} s_jp_j.
\]

\subsection{Multilingual document identification}

A document can contain lines in multiple languages for several reasons:
\begin{enumerate}
    \item Identification mismatch, that can show up frequently, especially with languages that have significant vocabulary overlap (Czech and Slovak),
    \item Crawl from a website where the interface is written in a language, and the body is written in another one,
    \item Crawl from a translation page, where the same content is present in two (or more) different languages.
\end{enumerate}

In these examples, we should aim to limit the presence of 1. and 2., while maximizing the presence of 3.: documents having a balanced set of lines per language. Thus, we decide to take a cautious approach, restricting the multilingual document identification test to the documents that:
\begin{itemize}
    \item Have at least $5$ lines,
    \item Have at most $5$ different languages.
\end{itemize}
Next, we compute the \emph{proportion} for each language $g_j \in \mathscr{L}$ in the document $\mathcal{D}$ defined as follows
\[
    \mathrm{Pr}_g = \frac{|g|}{|\mathcal{D}|},
\]
including for the no identification language $g_0$.

A document $\mathcal{D}$ containing $n$ lines is identified as multilingual if and only if:
\[
    \begin{dcases}
        |g_j| \ge \frac{|\mathcal{D}|}{n+1} & \forall g_j \neq g_0, \text{and} \\
        |g_0| \le \frac{|\mathcal{D}|}{n+1}
    \end{dcases}
\]
As an example, a document holding $m=3$ languages is multilingual if each language makes up at least $\frac{1}{m+1} = \frac{1}{4}$ of the document, and that there is at most $\frac{1}{4}$ of the document that is of unknown identification.

\subsection{Monolingual identification}
We begin by identifying each line, keeping in memory the language identified, the confidence of the identification, and the size of the line. We keep track of lines that have not been identified with a special token, and a confidence of 1.

If the document does not pass the multilingual check, we then take the largest represented language and compute its overall confidence $P_j$ and use a minimum confidence threshold of $0.6$ that is way lower than the previous pipelines ($0.8$). This is motivated by the following reason: The document-based filtering removes documents containing lines that could have been kept by former pipelines, thus reducing the size of the generated data.

Using a lower threshold could help getting lower-quality documents that still hold high-confidence lines in themselves.

\section{Annotation}

While the filtering and identification steps are lenient by using lower thresholds than the previous pipelines, we introduce annotations, as non-destructive filters that enable more precise downstream filtering for the corpus users, as well as a useful resource to quickly assess the quality of a corpus. Annotations enable more aggressive filters to be run, since the non-destructive nature of annotations can in turn be used to refine annotation filters.

Numerous annotations are available, and each document can have several ones at the same time.

\subsection{Length-based annotations}

Some simple annotations are added when documents don't meet certain length requirements:

\begin{itemize}
    \item The document has a low ($\le 5$) number of lines (\emph{tiny})
    \item The document has a high number ($\ge 50\%$) of short lines (\emph{short\_sentences})
\end{itemize}

These annotations help to spot potentially tiny documents, where the line structure or the document size could negatively influence training tasks.

A third annotation checks the occurrence of short lines at the start of the document, and adds a \emph{header} annotation if it is the case, indicating that low-quality content could be present at the start of the document.

A fourth annotation named \emph{footer} works in the same way on the tail of the document.

\subsection{Noise detection}

Some documents make their way into the corpus while being extremely noisy or non-linguistic. As an example, source code can be found in English corpora because of the presence of English words in the source itself.

We use a filter that computes a ratio between letters and non-letters.

This filter is based on Unicode categories. We use categories \emph{Lu, Ll, Lt, Lm, Lo}\footnote{Lu: Uppercase letter, Ll: Lowercase letter, Lt: Titlecase, Lm: Modifier, Lo: Other} for letters, and we add categories \emph{Mn, Mc, Me}\footnote{Mn: Nonspacing mark, Ms: Spacing mark, Me: Enclosing mark} for accents and diacritics.

A \emph{noisy} annotation is added if the ratio passes a certain threshold, set to $0.5$.


\subsection{Adult documents}

We use the UT1 blocklist\footnote{\url{https://dsi.ut-capitole.fr/blacklists/}} as a base for adult content filtering.

The UT1 blocklist is a collection of thematic blocklists (adult, gambling, blogs, ...), usually utilized in internet access control for schools. The list is constituted and extended by both human and robots contributions (known indexes, search engines, exploration of already known addresses). The blocklist is updated twice to thrice a week by Fabrice Prigent.

Each folder contains URL and domain blocklists, enabling filtering of both websites that are centered around adult content, and websites hosting user-generated content that can be of adult nature (several social networks...).

The adult blocklist comprises roughly 3.7M records.


\section{Corpus}

We apply the aforementioned pipeline to the November/December 2021 crawl dump of Common Crawl. The result is a new corpus, OSCAR 22.01. While its structure is different from the previous OSCAR corpora (due to the choice of generating a document oriented corpus), we have attempted to compare the two corpora, especially in terms of size and news-related topic presence and recall. We also evaluate the occurrence and pertinence of the annotations.

\subsection{Comparison with OSCAR 21.09}
\subsubsection{Size distribution}

The data layout of OSCAR 22.01 may limit the relevance of raw size comparisons, since metadata are larger (annotations and line identifications were not present in previous OSCAR Corpora), and fused with textual data (metadata were distributed in separate files for OSCAR 21.09).

However, comparing the distribution of corpus sizes may help us ensure that the new corpus has a size distribution similar to the older one.

We compare the distribution of the sub-corpora sizes between OSCAR 21.09 and OSCAR 22.01 in figure \ref{fig.1}. We see that while the overall distribution is similar, the lower end of the distribution has more variance: The $[0\text{B}, 100\text{KB})$ range shows more corpora at its bounds than at its center. Furthermore, we also plot the empirical cumulative density function, that helps to assert the distribution similarity between OSCAR 21.09 and OSCAR 22.01.

\begin{figure*}[!ht]
    \begin{center}
        \includegraphics[width=\linewidth]{static/media/oscar/towards/size-comp}
        \caption{Corpus size distribution between OSCAR 21.09 and 22.01}
        \label{fig.1}
    \end{center}
\end{figure*}

We also select three low-resourced languages, three mid-resourced languages and three high-resources languages and compare their content (that is, textual data excluding metadata) between OSCAR 22.01 and OSCAR 21.09. Comparison is shown in figure \ref{fig.2}. While the overall sizes of these corpora  have slightly decreased, the sizes of the mid and high resource languages are similar enough.

\begin{figure}[!ht]
    \begin{center}
        \includegraphics[scale=0.50]{static/media/oscar/towards/size_comp_content}
        \caption{Content size comparison of selected languages in OSCAR 22.01 versus OSCAR 21.09}
        \label{fig.2}
    \end{center}
\end{figure}

\subsubsection{Size differences in low-resource languages}

The low-sized corpora exhibit important size changes. As an example, the Alemannic German corpus went from 7MB to 360KB between OSCAR 21.09 and OSCAR 22.01. This size decrease can be explained by the way the document identification works: by reasoning at a document level, documents containing a majority of German identified lines and a minority of Alemannic German identified lines will be identified as a German document, whereas previous OSCAR pipelines would have separated the lines and increase the size of the Alemannic German corpus.

By extracting the lines identified as Alemannic from the German corpus, we get around 30 MB of data, which could constitute an Alemannic corpus with a size comparable to the OSCAR 21.09 Alemannic corpus after confidence and length based filtering.


This situation can, in a way, help us investigate the cases of linguistic proximity, where languages have a lexical overlap: When a line identified as Alemannic German is found inside a document that has been identified as German:
\begin{enumerate}
    \item \label{one} Is the line in German, and it is an identification error?
    \item Is the line in Alemannic German, in a document that is in German? (ex: A German website related to the Alemannic German language)
    \item \label{three} Is the whole document in Alemannic German, and the identification classified the majority of Alemannic as German?
\end{enumerate}

Those three cases can arise and may help to enhance the detection of a said language, by finding (\ref{one}) identification mismatches, hoping that these cases would improve identification after training, or (\ref{three}), after verification by a speaker of the language, state that the whole document is in Alemannic. The new data collected could in turn be used to improve language detection.


\subsubsection{New themes}

As OSCAR 22.01 is based on a November/December 2021 dump (compared to OSCAR 21.09, based on a February 2021 dump), the corpus should include data related to events contemporary to February 2021. We conduct a simple word search similar to the one conducted for the generation of OSCAR 21.09 \cite{abadji-etal-2021-ungoliant}, using both old and new events, in order to give a rough idea of both the actuality and the memory of the corpus.

\begin{table}[t]
    \centering\small
    \begin{tabular}{lrrrr}
        \toprule
        Language                  & Term                  & 21.09 & 22.01 \\
        \midrule
        \multirow{1}{*}{Arabic}   & Beirut port explosion & 31    & 13    \\
        \multirow{1}{*}{Burmese*} & Min Aung Hlaing       & 3439  & 2736  \\
        \multirow{1}{*}{English}  & Obama                 & 27639 & 8697  \\
        \multirow{1}{*}{English}  & Biden                 & 19299 & 8232  \\
        \multirow{1}{*}{English}  & Omicron               & 131   & 417   \\
        \multirow{1}{*}{French}   & Yellow Vests          & 96    & 73    \\
        \multirow{1}{*}{Spanish}  & Aborto                & 1504  & 572   \\
        \bottomrule
    \end{tabular}
    \caption{Comparison of occurrences of news-related terms between OSCAR and our corpus in a sample of 100 Common Crawl shards. \\ *: For the Burmese language, we use the whole 21.09 and 22.01 corpus since it is a low resource language. Terms are translated in the corpus language.}
    \label{tab:word_frequency_towards}
\end{table}

We see that the events and terms related to events predating February 2021 are still present in the corpus, but have a lower count that nevertheless remains in the same order of magnitude.
We also count the occurrences of the term Omicron, related to the Omicron variant, and observe that the term has a higher count on the 21.01 sample.

\subsubsection{Absence of deduplication}

Contrary to OSCAR 21.09, we do not distribute a deduplicated version of the majority of OSCAR 22.01.

The line-level deduplication of documents would have destroyed the integrity of documents themselves, hampering human readability and even sequential sentence sense. We can imagine having forum discussions' sense destroyed because of identical responses, or song lyrics being altered.

Moreover, the similarity-based document-level deduplication procedure is very costly in terms of computing power and time \cite{gao-etal-2020-pile}.

We make the choice of distributing a non deduplicated version of OSCAR along with a deduplicated, line oriented version of the English corpus, while encouraging the use of deduplication in the context of training language models \cite{lee-etal-2021-deduplicating}.
A line-level deduplication tool will be available as part of the OSCAR toolkit\footnote{\url{https://github.com/oscar-corpus/oscar-tools}}. We will also distribute a deduplicated version of the English part of OSCAR 22.01, with a data layout similar to OSCAR 21.09 corpora.


\subsection{Annotations}

\subsubsection{Raw stats}

Annotations help us to infer the composition of the corpora: The \textit{tiny}, \textit{short\_sentences} and especially \textit{noisy} annotations may indicate documents of a varying poor quality, with \textit{noisy} being the worst.

Also, comparing corpora annotation distributions, especially related to their size, could highlight potentially very low quality corpora. This semi-automated quality checking process could be used to label corpora where data quality is bad.

We select 3 low-resource ($\simeq100KB$), 3 mid-resource ($\simeq100MB$) and 3 high-resource ($\simeq100GB$) languages and plot the number of documents per annotation, adding a \textit{total} legend for the total document count and a \textit{clean} legend for documents that do not have any annotation. We then plot the counts for each resource group using adapted scales.

We observe that the annotation distribution is similar for each resource group, but that the lower resourced languages have a higher proportion of documents annotated with \textit{short\_sentences} and \textit{tiny}.

\begin{figure*}[!ht]
    \begin{center}
        \includegraphics[scale=0.7]{static/media/oscar/towards/annot_count.pdf}
        \caption{Annotation count in selected low, mid and high resource languages (scales are adapted to corpus size)}
        \label{annot-count}
    \end{center}
\end{figure*}

In order to better compare the resource groups, we display the annotation distribution in a heat map (figure \ref{annot-heatmap}).
We notice important differences between low and mid/high resource groups.
A very large proportion of the low resource group is annotated as \textit{tiny} while simultaneously detaining few documents annotated \textit{short\_sentences}, indicating the presence of long sentences within documents with a low number of sentences.

\begin{figure}[!ht]
    \begin{center}
        \includegraphics[scale=0.6]{static/media/oscar/towards/annot_heatmap}
        \caption{Heat map of annotation distributions in selected low, mid and high resource languages.}
        \label{annot-heatmap}
    \end{center}
\end{figure}

\subsubsection{Multilinguality}

The OSCAR 22.01 Corpus also contains a multilingual corpus, composed of documents holding lines in multiple languages. Each document contains at least 2 languages, and at most 5.

We check the co-occurrence of languages, highlighting the coupling of language tuples. These tuples may highlight either linguistic similarity (Czech and Slovak, Russian and Uzbek) and subsequent poor classification, errors or languages commonly found together on documents. Due to the number of languages and the sparsity of the data, we show the language couples with a number of documents greater than 20 000 (Figure \ref{multi-confusion}).

We also note the presence of English in a high number of documents. This could be explained by boilerplate content in web pages, such as menu headers or footers.


\begin{figure}[!ht]
    \begin{center}
        \includegraphics[width=0.8\linewidth]{static/media/oscar/towards/multilingual_big}
        \caption{Count of $(l1, l2)$ language tuples in the multilingual corpus. Languages tuples with less than 20,000 occurrences are not shown.}
        \label{multi-confusion}
    \end{center}
\end{figure}

Using the clean annotation filter on the multilingual corpus may help to retrieve the highest quality multilingual documents.

\subsubsection{Clean documents}

We also look into documents that did not get annotated at all, and we find that these documents are usually of a high quality. However, their relative proportion in corpora may limit their usage.

We use a sample of the English corpus (183,497 documents, 1.3 GB) and compare the size of documents depending on the presence (or not) of annotations. The stacked counts are shown in figure \ref{clean_count}.

We observe that clean document mean length is slightly shorter than non-clean ones. Also, we note that while the length standard deviation of clean documents seems to be shorter, the computation yields larger numbers, caused by outliers in the high end (Annotations: $\mu=8606$ $\sigma=49874$, Clean: $\mu=6537$ $\sigma=14983$).
By removing the top and bottom 5\%, we get (Annotations: $\mu=3686$ $\sigma=4047$, Clean: $\mu=3582$ $\sigma=3202$).

These results are not sufficient to state on the intrinsic quality of the clean documents, but may ease the study of the filters and identify future filtering needs.

\begin{figure}[!ht]
    \begin{center}
        \includegraphics[width=\linewidth]{static/media/oscar/towards/num_doc_clean}
        \caption{Stacked distribution of annotated and non-annotated (clean) documents on a selection of the English corpus}
        \label{clean_count}
    \end{center}
\end{figure}

\subsubsection{Adult documents}

While very small in proportions, adult annotation documents highlight interesting facts.

The French sample contains 32,870 adult documents, out of 52,037,098.

We count if some documents coming from tetu.com are labeled as adult, in order to probe the possibility of finding LGBTQI+ content annotated as adult. We find 1063 documents, representing $\sim 3.2\%$ of the adult documents. This may imply that more LGBTQI+ content sites are present in the blocklist, thus increasing the ratio of LGBTQI+ content labeled as adult.

We take the first 100 adult documents of the French corpus and check whether they are properly classified.
\begin{itemize}
    \item \emph{true positives} documents that exhibit explicit sexual content geared towards pornography (pornographic websites, sexually explicit fictions)
    \item \emph{false positives} documents that do not meet these criteria,
\end{itemize}

We separately count websites that are simultaneously non-explicit and from LGBTQI+ websites.

We find:
\begin{itemize}
    \item $77$ true positives,
    \item $2$ false positives belonging to LGBTQI+ websites,
    \item $21$ false positives
\end{itemize}

While the majority of true positives are properly classified, numerous educational documents do appear: These type of documents exhibit an explicit language, but does feature a good document quality, and a better representation of sexuality that is less offensive compared to the usual associations between sexually explicit content and hate speech.  \cite{luccioni-viviano-2021-whats}.

The false positives are, for the majority, websites that do not belong in the blocklist in the first place. We suppose that the addresses were previously used as adult websites.

\subsubsection{Hard bounds problems}

Several pipeline steps (especially annotators), work using hard thresholds. As an example, any document that is less than 5 lines is considered to be \textit{tiny}. However, when exploring data, we can see that there is a number of documents whose number of lines is in the neighboring of the threshold, and quality is similar to the documents labeled as \textit{tiny}.

When plotting the distribution of clean and annotated corpus data, we can notice that a very high number of documents are of a tiny ($10^2 B$) size, which coincidentally happens to be the minimum size for a document to be accepted, since the first filter removes lines that are shorter than 100 characters $(\geq 10^2 B)$.


\section{Discussion}


\subsection{Corpus}
We provide a new, document-oriented corpus of the same size of OSCAR 21.09 that keeps document integrity and is easier to filter thanks to annotations.

While the mid and high resourced languages are of a similar size, several low resource languages have seen an important decrease of size.
We still have to check whether this size decrease comes with a quality increase, since previous low resource OSCAR corpora sometimes exhibited extremely poor quality: Many non-linguistic corpora that were published and deemed unusable weeks or months after release.

We also note that documents of similar languages could have been merged into larger corpora, and we show that the German corpus holds $\sim 30$MB of Alemannic that, with appropriate filtering, could be treated as an independent corpus. These cases of merging are also interesting to investigate, as they can explain identification mismatches and could, in turn, help to build better language identification models.
More work has to be done in order to properly map the connection between low-resource languages and mid and high resource languages potentially containing data in these languages.

\subsection{Annotations}

The selected annotations exhibit numerous caveats that have to be addressed in the future iterations of OSCAR generation pipelines.

The length-based annotations are widespread in the corpus, especially in mid to high resource languages ($\sim50\%$ in Czech) highlighting the potential low quality of a high number of documents as well as the need of better characterizing the nature of these line length discrepancies. Web crawls often contain boilerplate content extracted from headers, footers and sidebars, and these lines are present in the Common Crawl dumps.
Another solution would be to base the whole OSCAR generation pipeline on raw HTML files, potentially multiplying the computational cost and complexity of generating corpora.

The \textit{adult} annotation, based from an adult URL blocklist, is present on a very limited set of documents. However, studies have shown that adult content has been present in a previous version of OSCAR in a larger proportion than the one measured here \cite{kreutzer-etal-2021-quality}, hinting at a bad performance of the blocklist based adult content filtering approach. Moreover, we noticed that the blocklist contained websites representing LGBTQI+ related topics, which damages the representation of the LGBTQI+ (association with adult content, filtering out LGBTQI+ documents, which in turn could limit the representation in downstream tasks...).
Model-based approaches may help in improving the \textit{adult} annotation, and should be the next step towards a better annotation of adult content \cite{luccioni-viviano-2021-whats}.


\section{Conclusion}

With the improvements to the Ungoliant pipeline described in this chapter and the release of OSCAR 22.01, we believe we are moving the OSCAR project in a direction were we are capable of distributing high quality up-to-date textual data for a wide range of NLP and Digital Humanities applications.

While we're aware that not all the problems and concerns around the OSCAR corpus have been addressed, we hope we can continue working on this project as it has already had a significant impact on the NLP community, specially for studies in previously underrepresented languages.

We believe however that the next steps in improving our corpus will require a more close involvement and participation of the OSCAR users. We thus hope that in the coming months and years we will be able to build an active open source community around the OSCAR project where people will be able to collaborate and contribute directly to the development of future versions of OSCAR and its pipeline Ungoliant.

While this chapter marks the end of the multilingual discussion of this thesis, the French sub-corpus of OSCAR will be consequential to the development of our models and resources for both contemporary and historical French, as we will see in the coming chapters.


\part{French Corpora}
%%%%%%%%%%%%%%%%%%%%%%%%%%%%%%%%%%%%%%%%%%%%%%%%%%%%%%%%%%%%%%%%%%%%%%%%
\chapter{Contemporary French Corpora}
%%%%%%%%%%%%%%%%%%%%%%%%%%%%%%%%%%%%%%%%%%%%%%%%%%%%%%%%%%%%%%%%%%%%%%%%

\begin{center}
    \begin{minipage}{0.66\textwidth}
        \begin{small}
            In which we present a part of the work of \citet{popa-fabre-etal-2020-french} who construct a balanced corpus for contemporary French that could be used for language modeling; and of \citet{ortiz-suarez-etal-2020-establishing} who aligned both the Universal Dependencies and the TEI-annotated NER version of the French Treebank, correcting multiple annotation mistakes and discrepancies, and who then converted the NER annotations to a more machine-ready CoNLL-like format that is more often used for training neural models.
        \end{small}
    \end{minipage}
    \vspace{0.5cm}
\end{center}

Having constructed a multilingual corpus out of web data that was in theory big enough to train a state-of-the-art language model \citep{liu-etal-2019-roberta} for a wide range of languages, and having addressed some of the quality concerns that some researchers had expressed about this type of corpus. We wanted to focus a little more on constructing resources specifically for Contemporary and Historical French, as this was the originally intended task when we first started working on OSCAR, which was always intended to be a French corpus only, but that ended being multilingual due to the multilingual nature of Common Crawl.

In this chapter we will present \emph{CaBeRnet} \citep{popa-fabre-etal-2020-french} A Contemporary French Balanced Corpus that is orders of magnitude smaller than the French OSCAR sub-corpus, but that as opposed to OSCAR, it is manually curated and specifically designed to be a linguistically balanced cross-genre corpus for the French language. We will also briefly present the work of \citet{ortiz-suarez-etal-2020-establishing} who aligned both the Universal Dependencies and the TEI-annotated NER version of the French Treebank, giving us a more consistent a more user-friendly NER French corpus that will be used for evaluation in later chapters.

\section{CaBeRnet: A Contemporary French Balanced Corpus}
\label{sec:DescribeCorpora}

While working on OSCAR 2019,\footnote{All the work on \Cabernet was conducted prior to the existence of OSCAR 21.09 and OSCAR 22.01. As such, all the mentions of OSCAR in this chapter refer to OSCAR 2019.} the question of quality versus size of corpus caught our attention. We wanted to study in particular the issue of corpus ``representativeness'' in order to grasp to what extent a linguistically balanced cross-genre language sample would be sufficient to pre-train a language model. Here for ``representativeness'' we follow Biber's definition: \emph{“representativeness refers to the extent to which a sample includes the full range of variability in a population”} \citep{biber-1993-representativeness}.

To construct our corpora we adopt a balanced approach by sampling a wide spectrum of language use and its cross-genre variability, be it situational (e.g. format, author, addressee, purposes, settings or topics) or linguistic, e.g. linked to distributional parameters like frequencies of word classes and genres. In this fashion, we developed two corpora:
\begin{enumerate}
    \item The French Balanced Reference Corpus (\emph{CaBeRnet}), which includes a wide-ranging and balanced coverage of cross-genre language use to be maximally representative of the French language and therefore yield good generalizations from.
    \item The \emph{French Children Book Test} (CBT-fr), which includes both narrative material and oral language use as present in youth literature, and which could be used for domain-specific language model training.
\end{enumerate}

Both corpora are inspired by existing American and English corpora, respectively  COCA, the balanced Corpus of Contemporary American English \citep{davies-2009-the, davies-2010-the}, and the Children Book Test \citep[CBT]{hill-etal-2016-the}.

\subsection{\Cabernet} \label{subsec:DescribeCaBeRnet}

The \Cabernet corpus was inspired by the genre partition of the American balanced corpus COCA, \footnote{\url{https://www.english-corpora.org/coca/}} which at the end of 2019, when this study was conducted, contained over 618 million words of text (20 million words each year 1990-2019) and was equally divided among spoken, fiction, popular magazines, newspapers, and academic texts \citep{davies-2009-the, davies-2010-the}. A second reference, guiding our approach and sampling method, was one of the earliest precursors of balanced reference corpora: the BNC \citep{bnc-2007-the}, which covered a wide variety of genres, with the intention to be a representative sample of spoken and written language.

\begin{table}[ht]
    \centering\small
    \begin{tabular}{lrrr}                                                                                      \\\toprule
        {\textsc{\Cabernet Sub-set}} & {\textsc{Tokens}} & {\textsc{Unique Forms}} & {\textsc{TTR}} \\\midrule
        Oral                         & 122 864 888       & 291 744                 & 0.0024         \\
        Popular                      & 131 444 017       & 458 521                 & 0.0035         \\
        News                         & 132 708 943       & 462 971                 & 0.0035         \\
        Fiction                      & 198 343 802       & 983 195                 & 0.0050         \\
        Academic                     & 126 431 211       & 1 433 663               & 0.0113         \\
        \textit{Total}               & 711 792 861       & 2 558 513               & 0.0036         \\ \bottomrule
    \end{tabular}
    \caption{\label{Table_Morpho_CabernetSub} Comparison of number of unique forms in the different genres represented by \Cabernet partition. TTR: Type-Token Ration. Lemmatization and tokenization was performed as described in §\ref{sec:CompareCorpora}.}
\end{table}

\Cabernet was obtained by compiling existing data-sets and web-text extracted from different sources as detailed in this subsection. As shown in Table \ref{Table_Morpho_CabernetSub}, genres sources are evenly divided ($\sim$120 million words each) into spoken, fiction, magazine, newspaper, academic to achieve genre-balanced between oral and written modality in newspapers and popular written style, technical reports and Wikipedia entries, fiction, literature and academic production.

\paragraph{\Cabernet Oral} \label{subsec:DescribeCaBeRnetOral}
The oral sub-portion gathers both oral transcriptions (\textsc{ORFEO} and Rhapsodie\footnote{\textsc{ORFEO} corpus available at \url{www.cocoon.huma-num.fr/exist/crdo/} ; Rhapsodie corpus at \url{www.projet-rhapsodie.fr}.}) and Films subtitles (Open Subtitles.org), pruned from diacritics, interlocutors tagging and time stamps. To these transcriptions, we add the French European Parliament Proceedings (1996-2011), as presented in \citet{koehn-2005-europarl}, which contribute a sample of more complex oral style with longer sentences and richer vocabulary.

\paragraph{\Cabernet Popular Press} \label{subsec:DescribeCaBeRnetPop}
The whole sub-portion of Popular Press is gathered from an open data-set from the \textit{Est Républicain} (1999, 2002 and 2003), a regional press format\footnote{Corpus available at \url{www.cnrtl.fr/corpus/estrepublicain/}.}. It was selected to match popular style as it is characterized by easy-to-read press style and a wide range of every-day topics characterizing local regional french press.

\paragraph{\Cabernet Fiction \& Literature} \label{subsec:DescribeCaBeRnetFic}
The Fiction \& Literature sub-portion was compiled from March 2019's Wiki Source and WikiBooks dump and extracted using WikiExtractor.py, a script that extracts and cleans text from a WikiMedia database dumps, by performing template expansion and preprocessing of template definitions.\footnote{Script available at \url{https://github.com/attardi/wikiextractor}.}

\paragraph{\Cabernet News} \label{subsec:DescribeCaBeRnetNews}
The News sub-portion builds upon web crawled elements, including Wikimedia's NewsComments and WikiNews reports from the May 2019 WikiMedia dump, collected with a custom version of WikiExtractor.py. We also add newspaper's content gathered by the Chambers-Rostand Corpus (i.e. Le Monde 2002-2003, La Dépèche 2002-2003, L'Humanité 2002-2003) and \emph{Le Monde diplomatique}. This open-source corpora were assembled to represent a higher register of written news style from different political and thematic horizons. Several months of French Press Agency reports are also added (AFP, 2007-2011-2012), which contribute with a more simple and telegraphic style than the others newspaper written samples of the corpus.\footnote{This part of \Cabernet corpus is still subject to Licence restrictions. However, this restricted amount of AFP news reports can reasonably fall in the public domain.}

\paragraph{\Cabernet Academic} \label{subsec:DescribeCaBeRnetAcad}
The academic genre was also built from different sources including technical and educational texts from WikiBooks and Wikipedia dump (prior to 2016) for their thematic variety of highly specialized written production. The \textsc{ORFEO} Corpus offered a small sample of academic writings like PHD dissertations and scientific articles encompassing a wide choice of disciplinary topics, and the TALN Corpus\footnote{TALN proceedings corpus (about 2 million) builds on a subset of 586 scientific articles (from 2007 to 2013), namely TALN and RECITAL. Available at \url{redac.univ-tlse2.fr/corpus/taln_en.html}.} was included to represent more concise written style characterizing scientific abstracts and proceedings.

For all sub-portions of \Cabernet, visual inspection was performed to remove section titles, redundant meta-information linked to publishing schemes of each of the six news editor included. This was manually achieved by compiling a rich set of regular expressions specific of each textual source to obtain clean plain text as an output.

\subsection{French Children Book Test (CBT-fr)}
\label{subsec:DescribeCBT}

The French Children Book Test (CBT-fr) was built upon its original English version, the Children Book Test (CBT) \citep{hill-etal-2016-the}\footnote{This data-set can be found at \url{www.fb.ai/babi/}.}, which consists of books freely available from Project Gutenberg. \footnote{\url{{www.gutenberg.org}.}}

Using youth literature and children books guarantees a clear narrative structure, and a large amount of dialogues, which enriches with oral register the literary style of this corpus. The English version of this corpus was originally built as a benchmark data-set to test how well language models capture meaning in context. It contains 108 books, and a vocabulary size of 53,628 tokens.

The French version of CBT, named CBT-fr, was constructed to guarantee enough linguistic similarities between the collected books in the two languages. 104 freely available books were included. One third of the books were purposely chosen because they were classical translations of English literary classics. Chapter heads, titles, notes and all types of editorial information were removed to obtain a plain narrative text. The effort of keeping proportion, genre, domain, and time as equal as possible yields a multilingual set of comparable corpora with a similar balance and representativeness.

\begin{table}[ht]
    \centering\small
    \begin{tabular}{lr}                                                             \\\toprule
        {\textsc{Children Book Test - fr}}           & { \textsc{Words}} \\\midrule
        Number of different lemmas                   & 25 139            \\
        Total number of forms                        & 95 058            \\
        Mean number of forms per lemma               & 3.78              \\
        Number of lemmas having more than one form : & 14 128            \\
        Percentage of lemmas with multiple forms     & 56.20             \\
        \bottomrule
    \end{tabular}
    \caption{\label{Table_DescribeCBTfr} Lexical statistics of French CBT, performed as described in §\ref{sec:CompareCorpora}}
\end{table}

\subsection{Corpora Descriptive Comparison} \label{sec:CompareCorpora}

Having put together these two different balanced corpora, we wanted to perform a descriptive comparison between them, the French subcorpus of OSCAR 2019 and Wikipedia. In order to perform this comparison we start by tokenizing all corpora. For this we used two different tokenizers: A standalone version of SEM, (Segmenteur-Étiqueteur Markovien) \citep{dupont-2017-exploration} and TreeTagger \citep{schmid-1999-improvements}. Both are based on cascades of regular expressions, and both perform tokenization and sentence splitting. The first was used for descriptive purposes because it technically allowed to segment and tokenize all corpora including OSCAR (23 billion words). Hence, all corpora were entirely segmented into sentences and tokenized using SEM.

While the second tokenization method was only run on 3 million words samples to automatically tag them with TreeTagger into part-of-speech and lemmatize them.\footnote{Based on the tag-set available at \url{https://www.cis.uni-muenchen.de/~schmid/tools/TreeTagger/data/french-tagset.html}.} All corpora were randomly shuffled by sentence to then select samples of 3 million words, to be able to compare them in terms of lexical composition (Type-Token Ratio, see Table \ref{Table_MorphoRich}).

\subsubsection{Corpora Size and Composition}

Length of sentences is a simple measure to quantify both sentence syntactic complexity and genre. Hence, the number of sentences reported in Table \ref{Table_nb_Words} shows interesting patterns of distributions across genres, consider the comparison between \Cabernet an Wiki-fr. In our effort to evaluate the impact of corpora pre-training on ELMo-based contextualized word-embedding, we introduce here our two terms of comparison, namely the crawled corpus OSCAR-fr and the Wikipedia-fr one.

\paragraph{OSCAR fr}
As it has been shown that pre-trained language models can be significantly improved by using more data \citep{liu-etal-2019-roberta,raffel-etal-2020-exploring}, we decided to include in our comparison a corpus of French text extracted from Common Crawl\footnote{More information available  at \url{https://commoncrawl.org/about/}.}. We leverage on a recently published corpus, OSCAR \citep{ortiz-suarez-etal-2019-asynchronous}, which offers a pre-classified and pre-filtered version of the November 2018 Common Craw snapshot.

OSCAR gathers a set of monolingual text extracted from Common Crawl - in plain text \emph{WET} format - where all HTML tags are removed and all text encodings are converted to UTF-8. It follows a similar approach to \citep{grave-etal-2018-learning} by using a language classification model based on the fastText linear classifier \citep{joulin-etal-2016-fasttext,joulin-etal-2017-bag} pre-trained on Wikipedia, Tatoeba and SETimes, supporting 176 different languages.

After language classification, a deduplication step is performed without introducing a specialized filtering scheme: paragraphs containing 100 or more UTF-8 encoded characters are kept. This makes OSCAR an example of unfiltered data that is nearly as noisy as to the original Crawled data.

%%%%%%%%%%%%%%%%%%%%%%%%%%%%%%%%%%%%%%%%%%%%%%%%%
\paragraph{FrWIKI}
This corpus collects a selection of pages from Wikipedia-fr from a dump executed in April 2019, where HTML tags and tables were removed, together with template expansion using Attardi's tool (WikiExtractor, §\ref{subsec:DescribeCaBeRnetFic}). As reported on Table \ref{Table_nb_Words}, in this data-set (660 million words) sentences are relatively longer compared to other corpora. It has the advantage of having a comparable size to \Cabernet, but its homogeneity in terms of written genre is set to Wikipedia entries descriptive style.

\begin{table}[ht]
    \centering
    \begin{tabular}{lrrr}                                                                                 \\\toprule
        {\textsc{corpus}} & { \textsc{wordforms}} & { \textsc{tokens}} & { \textsc{sentences}} \\\midrule
        OSCAR-fr          & 23 212 459 287        & 27 439 082 933     & 1 003 261 066         \\
        Wiki-fr           & 665 599 545           & 802 283 130        & 21 775 351            \\
        \Cabernet         & 697 119 013           & 830 894 133        & 54 216 010            \\
        CBT-fr            & 5 697 584             & 6 910 201          & 317 239               \\\bottomrule
        %frWac       &  1,357,598,417  & 1,622,619,337  &  57,236,199  \\  
    \end{tabular}
    \caption{\label{Table_nb_Words} Comparing the corpora under study.}
\end{table}
%%%%%%%%%%%%%%%%%%%%%%%%%%%%%%%%%%%%%%%%%%%%%%%%%%%%%%%%%%%%%%%%
%%%%%%%%%%%%%%%%%%%%%%%%%%%%%%%%%%%%%%%%%%%%%%%%%%%%%%%%%%%%%%%%
\subsubsection{Corpora Lexical Variety}

Focusing on a useful measure of complexity that documents lexical richness or variety in vocabulary, we present the type-token ration (TTR) of the corpora under analysis. Generally used to assess language use aspects like the variety of different words used to communicate by learners or children, it represents the total number of unique words (types/forms) divided by the total number of tokens in a given sample of language production. Hence, the closer the TTR ratio is to 1, the greater the lexical richness of the corpus. Table \ref{Table_Morpho_CabernetSub} summarizes the lexical variety of the five sub-portions of \Cabernet, respectively taken as representative of Oral, Popular, Fiction, News, and Academic genres. Domain diversity of texts can be observed in the lexical statistics showing a gradual increase in the number of distinct lexical forms (cf. TTR). This pattern  reflects a generally acknowledged distributional pattern of vocabulary-size across genres. Oral style shows a poorer lexical variety compared to newspapers/magazines’ textual typology. The lexically rich fictional/classic literature is outreached by academic writing-style with its wide-ranging specialized vocabulary. All in all, Table \ref{Table_Morpho_CabernetSub} quantitatively demonstrates that the selected textual and oral materials are indeed representative of the five types of genres of CaBeRnet.

%%%%%%%%%%%%%%%%%%%%%%%%%%%%%%%%%%%%%%%%%%%%%%%%%%%%%%%%%%%%%%%%%%
%%%%%%%%%%%%%%%%%%%%%%%%%%%%%%%%%%%%%%%%%%%%%%%%%%%%%%%%%%%%%%%%%%
\subsubsection{Corpora Morphological richness}

To select a measure that would help quantifying the different corpora morphological richness, we follow \citep{bonami-etal-2015-implicative}. Hence, the proportion of lemmas with multiple forms in a given vocabulary size was evaluated on randomly selected samples of 3-million-words from each corpus under analysis (see Table \ref{Table_MorphoRich}).

\begin{table}[ht]
    \centering
    \begin{tabular}{lrrrr}
        \toprule
        \textsc{3 M samples}  & \textsc{CBT-fr} & \textsc{\Cabernet} & \textsc{Fr-Wiki} & \textsc{OSCAR} \\
        \midrule
        nb of diff. lemmas    & 25 139          & 30 488             & 31 385           & 31 204         \\
        tot. nb forms         & 95 058          & 180 089            & 238 121          & 190 078        \\
        mean nb forms/lemma   & 3.78            & 6.19               & 7.85             & 6.40           \\
        nb lemmas $>$ 1 form  & 14 128          & 15 927             & 15 182           & 16 480         \\
        \% lemmas  $>$ 1 form & 56.20           & 52.24              & 48.37            & 52.81          \\
        \bottomrule
    \end{tabular}
    \caption{Lexical statistics on morphological richness over randomly selected samples of 3 million words from each corpus. nb : number}
    \label{Table_MorphoRich}
\end{table}

Table 4 reports some more in-depth lexical and morphological statistics across corpora. Although OSCAR is 34 times bigger than CaBeRnet, their total number of forms and the proportion of lemmas having more than one form in a 3-million-word sample are comparable. FrWiki shows a radically different lexical distribution with numerous hapaxes but a lower morphological richness. Although its total number of forms is more than one third higher than in OSCAR and CaBeRnet samples, the proportion of lemmas having more than one distinct form is around four points below CaBeRnet and OSCAR. Comparatively, youth literature in CBT-fr shows the greatest morphological richness, around 56\% of lemmas have more than one form.



%%%%%%%%%%%%%%%%%%%%%%%%%%%%%%%%%%%%%%%%%%%%%%%%%%%%%%%%%%%%%%%%%%%%%%%%
\chapter{Historical French Corpora}\label{chap:historical}
%%%%%%%%%%%%%%%%%%%%%%%%%%%%%%%%%%%%%%%%%%%%%%%%%%%%%%%%%%%%%%%%%%%%%%%%

\begin{center}
    \begin{minipage}{0.66\textwidth}
        \begin{small}
            In which we present a part of the work of \citet{grobol-etal-2022-bertrade} who put together a raw corpus for Medieval French intended to pre-train language models. We also present part of the work of \citet{gabay-etal-2022-from} who construct a \freemmax,a corpus for Early Modern French for the pre-training of language models, as well as \freemner an evaluation corpus annotated in named entity recognition. We also briefly present \freemlpm an evaluation corpus annotated in part-of-speech tagging.
        \end{small}
    \end{minipage}
    \vspace{0.5cm}
\end{center}

Having extensively worked in Contemporary French corpora, we wanted to refocus and develop some historical French resources. As mentioned in the introduction of this thesis, this Ph.D. project was conceived and financed by ANR BASNUM (ANR-18-CE38-0003) project, whose main objective was to digitize and enrich Antoine Furetière's \emph{Dictionnaire Universel} (DU), in its 1701 version reviewed and corrected by Basnage de Beauval \citep{furetiere-1701-dictionnaire}; A text written in its entirety in Early Modern French. As such, in section \ref{data-freem} we develop resources for Early Modern French intended for both the pre-training and the evaluation of neural language models. This will allow us to develop in the next part, state-of-the-art models capable of conducting the dictionary enriching task originally planned by the BASNUM project.

Having said this, we also decided to participate in the curation of a small Medieval French corpus for the pre-training of a language model. Participating in this endeavor means that at the end of this chapter we would have developed resources for French covering a period going from the 9\textsuperscript{th} century to the present day, that is, we would have developed and curated textual resources for effectively all language states of French.

\section{Medieval French Corpus}
\label{sec-data}

This section describes the raw corpus of Medieval French we gathered in order to train unsupervised language models for Old French. To our knowledge, it is one of the largest such dataset gathered for Medieval French, although it remains quite small (\SI{55}{\mebi\byte} in total) relatively to the corpora usually used for pre-training contextual embeddings models.

We chose to include a few texts from the early Middle French period (14th-15th c.) in this raw corpus, which brings a valuable complement of the prose documents that are lacking for Old French, while staying close enough to late Old French, the boundary between the two epochs being somewhat fuzzy. These texts precede the adoption of norms established by editors after the invention of Gutenberg's printing press. Middle French is more regular than Old French in some respects such as word order \citep{marchello-Nizia-etal-2020-grande} and less in others such as NP structure and pronouns system \citep{marchello-nizia-etal-1979-histoire}, but they share most of their lexicon and for these relatively early texts, the syntax is not too different from that of late Old French texts.

%Corpora
\begin{table}[thb]
    \centering
    \tablefontsize
    \begin{tabular}{l S[table-format=2.1]}
        \toprule
        {\textbf{Corpus}}                              & {\textbf{Size / \si{\mebi\byte}}} \\ %& {\textbf{\# texts}}
        \midrule
        BFM \citep{guillot-etal-2018-base}             & 20.7                              \\ %139
        AND \citep{rothwell-etal-2005-anglo}           & 17.2                              \\ %73
        NCA \citep{kunstmann-stein-2007-le}            & 9.7                               \\ %271
        Chartes Douai \citep{glessen-2003-elaboration} & 3.1                               \\ %1
        OpenMedFr \citep{wrisley-2018-the}             & 1.7                               \\ %19
        Geste \citep{camps-etal-2019-geste}            & 1.5                               \\ %32
        MCVF \citep{martineau-2008-un}                 & 1.4                               \\ %17
        Chartes Aube \citep{reenen-etal-2007-chartes}  & 0.2                               \\ %75
        \midrule
        Total                                          & 55.3                              \\ %627
        \bottomrule
    \end{tabular}
    \caption{Data collection}
    \label{tab:texts_train}
\end{table}

\begin{figure}[thb]
    \centering
    \begin{tikzpicture}
        \begin{axis}[
                xbar,
                colormap name=viridis,
                cycle list={
                        {color of colormap=200, draw=., preaction={fill=.}, pattern=grid},
                        {color of colormap=800, draw=., preaction={fill=.}, pattern=dots},
                    },
                font=\footnotesize,
                width=0.5\linewidth,
                yticklabel style={
                        xshift=0.5cm,
                        align=right,
                    },
                y axis line style = { opacity = 0 },
                xlabel = Datasize (\unit{\mebi\byte}),
                %axis x line       = none,
                tickwidth         = 0pt,
                ytick             = data,
                % enlarge y limits  = 0.15,
                % enlarge x limits  = 0.2,
                symbolic y coords = {Legal, Historical, Didactic, Religious, Literature},
                nodes near coords,
                nodes near coords style={black},
                legend style={at={(0.95,0.6)}},
                reverse legend,
            ]
            \addplot
            coordinates {(13.3321,Literature) (4.35204,Religious)
                    (4.35204,Didactic) (0.687576,Historical) (0,Legal)};
            \addlegendentry{Verse}
            \addplot
            coordinates {(2.352817,Literature) (4.433186,Religious)
                    (3.041925,Didactic) (8.362256,Historical) (15.705591,Legal)};
            \addlegendentry{Prose}
            % 			\legend{Verse,Prose}
        \end{axis}
    \end{tikzpicture}
    \caption{Distribution of form and domain, gathered from documents metadata and manual annotation.}
    \label{fig:metadata}
\end{figure}

Medieval French has many factors of variation: language evolution, dialects, domains, forms of text (verse or prose) and lack of standard. Our dataset gives us a representation of Medieval French that is as accurate and diversified as possible, given the limited amount of material that survived to these days.
The detailed instructions to replicate this dataset are described in the Appendix. No particular processing is done on the original documents.

In order to get a sound evaluation of the contextual embeddings trained with this dataset, we filter out the documents that are also present in the SRCMF treebank used for evaluation purposes in section \ref{sec-experiments}\footnote{As noted by \citet{gururangan-etal-2020-dont}, pre-training on task specific data provides an additional boost, that would muddle our results, since our objective here is not so much task optimization as embeddings benchmarking.}. The resulting corpus is quite heterogeneous: legal texts and verse literature are in the majority, whereas other domains, such as historical and didactic texts, are under-represented, as can be seen in \cref{fig:metadata}.


\section{Early Modern French Corpora}\label{data-freem}

For the past few years, we have been involved in the development of linguistic resources for Early Modern French. The initiative, called \textsc{FreEM} (which stands for \emph{FREnch Early Modern}), aims to collect the corpora required for various NLP tasks such as lemmatization, POS tagging, linguistic normalization and named entity recognition. Two of these corpora are introduced here: \freemmax (see Section~\ref{freem_max}) and \freemlpm (see Section~\ref{freem_lpm}).

\subsection{\texorpdfstring{\freemmax}{FREEM max}}\label{freem_max}

Usable historical documents are difficult to find because, as previously mentioned, they are more rare than contemporary ones; editors tend to normalize the language (\emph{i.e.}~use the spelling conventions of contemporary French, see~\citep{gabay-2014-pourquoi}), transcriptions are not (always) distributed in a digital format. \freemmax \citep{gabay-etal-2022-FreEM} is an attempt to solve this problem, and the aim of this dataset is to group together the largest number of texts possible written in Early Modern French.

The texts we have curated have a variety of sources, which can be grouped into three main types:
\begin{itemize}
    \item Two institutional datasets have been used and are non open-sourced:
          \begin{itemize}
              \item \textsc{Frantext} \emph{intégral} \citep{atilf-1998-frantext}, the biggest database of French texts (only the texts between 1500 and 1800), a very small portion of which is open access: \textsc{Frantext} \emph{Démonstration} \citep{atilf-1998-frantext-d};
              \item \emph{Electronic Enlightenment} \citep{bodleian-2008-electronic}, an online collection of edited correspondences of the Early Modern period;
          \end{itemize}
    \item Several come from research projects distributing transcriptions online:
          \begin{itemize}
              \item The \emph{Antonomaz project},  French \emph{mazarinades} (\url{https://cahier.hypotheses.org/antonomaz});
              \item The II.B section (in French) of the \emph{Actis Pacis Westphalicae}, diplomatic letters for the Peace of Westphalia (\url{http://kaskade.dwds.de/dstar/apwcf/});
              \item The Bibliothèques virtuelles humanistes, 16\textsuperscript{th}\,c.~French literature (\url{http://www.bvh.univ-tours.fr});
              \item The \emph{Corpus électronique de la première modernité}, 17\textsuperscript{th}\,c.~French literature (\url{http://www.cepm.paris-sorbonne.fr})
              \item The \emph{Condé} project, \emph{coutumiers normands} (\url{https://conde.hypotheses.org})
              \item The Corpus Descartes, works of René Descartes (\url{https://www.unicaen.fr/puc/sources/prodescartes/});
              \item The \emph{Bibliothèque dramatique} of the CELLF, 17\textsuperscript{th}\,c.~French plays (\url{http://bibdramatique.huma-num.fr});
              \item The \emph{Fabula numerica} project, French fables (\url{https://obvil.sorbonne-universite.fr/projets/fabula-numerica});
              \item The \emph{ Fonds Boissy}, plays of Louis de Boissy (\url{https://www.licorn-research.fr/Boissy.html});
              \item The \emph{Mercure Galant} project, the famous French \emph{gazette} and literary magazine between 1672 and 1710 (\url{https://obvil.sorbonne-universite.fr/corpus/mercure-galant});
              \item The \emph{Rousseau online} project, works of Jean-Jacques Rousseau (\url{https://www.rousseauonline.ch});
              \item The \emph{Sermo} project, sermons of the 16\textsuperscript{th} and 17\textsuperscript{th}\,c. (\url{http://sermo.unine.ch});
              \item The \emph{Théâtre classique} project, 17\textsuperscript{th} and 18\textsuperscript{th}\,c.~French plays (\url{http://www.theatre-classique.fr});
          \end{itemize}
    \item Additional sources come from researchers who kindly accepted to offer their personal transcriptions or data scrapped by our team:
          \begin{itemize}
              \item Transcriptions of Anne-Élisabeth Spica (17\textsuperscript{th}\,c. French novels);
              \item Transcriptions found on \emph{Wikisource} (\url{https://fr.wikisource.org});
              \item Transcriptions (ePub files) found on \emph{Gallica} (\url{https://gallica.bnf.fr});
              \item Transcriptions found on various websites online.
          \end{itemize}
\end{itemize}

Additional data for later states of the language, up to the 1920's (mainly from FRANTEXT \emph{intégral}), are also provided for two main reasons: on the one hand, it is common to normalize Early Modern French into Contemporary French \citep{gabay-2014-pourquoi} because of the linguistic proximity between these the two states of the language, and on the other hand, it helps to collect (precious) additional data to avoid ending up with too small of a corpus for our needs.

\begin{table}[ht]
    \centering\small
    \resizebox{\linewidth}{!}{
        \begin{tabular}{lrrlr}
            \toprule
            Origin                      & \#Tokens  &  & Origin                                               & \#Tokens    \\
            \midrule
            Spica corpus                & 691,467   &  & \textsc{Frantext} \emph{intégral} ($>$1500, $<$1800) & 60,018,390  \\
            Antonomaz project           & 119,194   &  & \textsc{Frantext} \emph{intégral} ($>$1800)          & 71,504,440  \\
            Acta Pacis Westphlicae II B & 2,463,047 &  & \textsc{Frantext} \emph{Démonstration}               & 1,255,454   \\
            Bibliothèque Bleue          & 776,838   &  & Gallica                                              & 5,212,333   \\
            BVH                         & 2,434,657 &  & Boissy project                                       & 438,215     \\
            CEPM                        & 2,707,432 &  & Mercure galant                                       & 5,427,469   \\
            Condé project               & 3,173,845 &  & Rousseau Online project                              & 2,428,587   \\
            Descartes                   & 1,025,337 &  & Scrapping                                            & 1,936,835   \\
            CELLF                       & 1,873,772 &  & Sermo project                                        & 529,647     \\
            Electronic enlightenment    & 6,568,047 &  & Théâtre classique project                            & 13,916,169  \\
            Fabula project              & 145,978   &  & Wikisource                                           & 996,329     \\
            \midrule
            \textbf{TOTAL}              &           &  &                                                      & 185,643,482 \\
            \bottomrule
        \end{tabular}
    }
    \caption{Breakdown of the \freemmax corpus by text origin.}
    \label{tab:my_label}
\end{table}

The final result is far from being balanced or representative (see Figure~\ref{fig:FreEMmax_desc}). 16\textsuperscript{th}\,c. French documents are under-represented, as well as 18\textsuperscript{th}\,c.~literature. The 17\textsuperscript{th}\,c. is clearly over-represented, especially its second half---probably one of the most important of French literature, which could explain this situation (on top of our personal interest for this specific period).

\begin{figure}[ht]
    \centering
    \includegraphics[width=0.75\linewidth]{static/media/mod_eval/dalembert/desc_DalemBERT.png}
    \caption{Distribution of the documents in the \freemmax corpus per year}
    \label{fig:FreEMmax_desc}
\end{figure}

As some texts are still (partially) protected by restrictive licenses, the \freemmax corpus exists in both open and non-open versions, only the open one being distributed. In order to limit the impact of licenses forbidding the modification of files, we have designed a pipeline to distribute the data as it was found and recreate it (see Figure~\ref{fig:pipeline}).

Metadata is prepared manually in order to have the same categories for each document, whatever its origin. As well as the author, the title and the date (where relevant), we also provide the genre (``theater''), sometimes a subgenre (``tragedy''), the linguistic status (normalized or not) and the license attached to the transcription.

\begin{figure}[ht]
    \centering
    \includegraphics[width=0.75\linewidth]{static/media/mod_eval/dalembert/corpus_trans.png}
    \caption{\freemmax compilation pipeline. All files are kept in their original format. Metadata is manually prepared in separate files in order to automatically transform and clean (in blue) all the available documents into XML TEI files following the same encoding. It allows us to distribute open data (in green) but also data distributed with restrictions regarding the modification of the original format (in orange). Non-open texts (in red) are not distributed.}
    \label{fig:pipeline}
\end{figure}

\subsection{\texorpdfstring{\freemlpm}{FREEM LPM}}\label{freem_lpm}

The \freemlpm (``Lemma, POS tags, Morphology'') has already been presented \citep{gabay-etal-2020-standardizing}. The POS-annotated data, is a mixture of two different sources. On the one hand, there is the \emph{CornMol} corpus \citep{camps-etal-2021-corpus}, made up of normalized 17\textsuperscript{th}\,c.~French comedies. On the other hand, there is a gold subset of the \emph{Presto} corpus \citep{blumenthal-etal-2017-presto}, made up of texts of different genres written during the 16\textsuperscript{th}, 17\textsuperscript{th} and 18\textsuperscript{th}\,c., which have previously used to train annotation tools \citep{diwersy-etal-2017-ressources}, and was heavily corrected by us to match our annotation principles \citep{gabay-etal-2020-manuel}.

On top of traditional in-domain tests, an out-of-domain testing dataset was prepared to control the capacity of the model to generalize to other genres and periods. Centuries covered are the 16\textsuperscript{th}, 17\textsuperscript{th}, 18\textsuperscript{th}, 19\textsuperscript{th} and 20\textsuperscript{th}. There are two test sets for each century: one made up only of theater, the other of everything but theater. Each test set comprises 10 short samples (c.\,100 tokens), as representative as possible of the linguistic production of the century (female and male authors, decade of publication, genre, etc.).

All the data from \freemlpm (but almost none of the out-of-domain) can be found in \freemmax.

\subsection{\texorpdfstring{\freemner}{FREEM NER}}\label{freem_ner}

\begin{figure}[!htp]
    \centering
    \includegraphics[width=0.6\linewidth]{static/media/mod_eval/dalembert/distribution_tokens_corpus.png}
    \caption{Number of tokens per century.}
    \label{fig:description}
\end{figure}

Rather than designing a new corpus, we have decided to use a subpart of the ``core corpus'' of the \textit{Presto} project~\citep{blumenthal-etal-2017-presto}, namely the text written during the French \textit{Ancien Régime} (c.15\textsuperscript{th}-18\textsuperscript{th}\,c., \textit{i.e.} 34 texts)\footnote{A text has been withdrawn: the \textit{Histoire d'un voyage faict en la terre du Brésil} by Jean de Léry, the transcription being too faulty to be able to correctly annotate the document.}. This choice is driven by our will to limit the number of annotated corpora for historical French, the same set of documents having already been abundantly corrected to train a lemmatizer~\citep{gabay-etal-2020-standardizing}, but also to avoid a complex selection of works supposed to ensure a relative representativeness of literary documents from the \textit{Ancien Régime}, already perfectly done by our colleagues.

The number of genres covered is extremely large: poetry, drama, novel, correspondence, grammar, philosophy, short stories, encyclopedic literature, etc. and guarantees, here again, a reasonable representativeness of the range of possibilities of \textit{Belles-Lettres}\footnote{We do not offer a detailed description of the genres covered, these overlapping easily: poetry can be theological, political correspondence\dots}. The corpus is balanced regarding the distribution per century (c.\,10/century) but not regarding the length of the texts, which increases over time (cf.\,fig.~\ref{fig:description}), following a possible trend in literature.

\subsubsection{Annotation}

\begin{table}[!htp]
    \centering\small
    \begin{tabular}{llllll}
        %{@{}p{0.15\linewidth}p{0.15\linewidth}p{0.15\linewidth}p{0.15\linewidth}p{0.15\linewidth}p{0.15\linewidth}@{}}
        \toprule
        \rowcolor{lightgray}
        \multicolumn{3}{c}{Person}                                                                   & \multicolumn{3}{c}{Function}                                                              \\
        \multicolumn{3}{c}{\begin{tabular}{cc} \texttt{pers.ind} & \texttt{pers.coll} \end{tabular}} &
        \multicolumn{3}{c}{\begin{tabular}{cc} \texttt{func.ind} & \texttt{func.coll} \end{tabular}}                                                                                             \\
        \rowcolor{lightgray}
        \multicolumn{3}{c}{Location}                                                                 & \multicolumn{3}{c}{Production}                                                            \\
        \texttt{loc.adm.town}                                                                        & \texttt{loc.phys.geo}                      & \texttt{loc.fac}                           &
        \texttt{prod.art}                                                                            & \texttt{prod.rule}                         & \texttt{prod.object}                         \\
        \texttt{loc.adm.reg}                                                                         & \texttt{loc.phys.hydro}                    & \texttt{loc.oro}                           &
        \multicolumn{3}{c}{\begin{tabular}{cc}  &  \end{tabular}}                                                                                                                                \\
        \texttt{loc.adm.nat}                                                                         &                                            &                                            &
        \multicolumn{3}{c}{\begin{tabular}{cc}  &  \end{tabular}}                                                                                                                                \\
        \texttt{loc.adm.sup}                                                                         &                                            &                                            &
        \multicolumn{3}{c}{\begin{tabular}{cc}  &  \end{tabular}}                                                                                                                                \\
        \rowcolor{lightgray}
        \multicolumn{2}{c}{Organization}                                                             & \multicolumn{2}{c}{Time}                   &
        \multicolumn{1}{c}{Event}                                                                    &
        \multicolumn{1}{c}{Quantity}                                                                                                                                                             \\
        \texttt{org.adm}                                                                             & \texttt{org.ent}                           & \multicolumn{2}{c}{\texttt{time.date.abs}} &
        \multicolumn{1}{c}{\texttt{event}}                                                           &
        \multicolumn{1}{c}{\texttt{amount}}                                                                                                                                                      \\
        \multicolumn{2}{c}{}                                                                         & \multicolumn{2}{c}{\texttt{time.date.rel}} &
        \multicolumn{2}{c}{}                                                                                                                                                                     \\
    \end{tabular}
    \caption{Types (in gray) and subtypes taken from the \emph{Quaero} typology.}
    \label{tab:types}
\end{table}

Because two important historical corpora presented \textit{supra} (\textit{Quaero} and \textit{Impresso}) have chosen to follow the \textit{Quaero} annotation guide~\citep{rosset-etal-2011-entites}, it seemed logical to use this same typology. Because our texts and interests diverge from those of the aforementioned corpora, only some types and subtypes have been kept (cf.\,tab.~\ref{tab:types}) from the \textit{Quaero} annotation scheme. The details of our annotation choices can be found in a dedicated annotation manual~\citep{gabay-etal-2020-manuel}.

\begin{table}[!htp]
    \centering\scriptsize
    \begin{tabular}{llrrrrrr}
        \toprule
        Token     & Lemma    & POS & COARSE & FINE       & FINE-COMP    & NESTED        & Wikidata ID \\
        \midrule
        Les       & le       & Da  & O      & O          & O            & O             & \_          \\
        allemands & allemand & Nc  & O      & O          & O            & O             & \_          \\
        élurent   & élire    & Vvc & O      & O          & O            & O             & \_          \\
        pour      & pour     & S   & O      & O          & O            & O             & \_          \\
        empereur  & empereur & Nc  & B-pers & B-pers.ind & B-comp.title & O             & Q438435     \\
        Rodolphe  & Rodolphe & Np  & I-pers & I-pers.ind & B-comp.name  & O             & Q438435     \\
        duc       & duc      & Nc  & I-pers & I-pers.ind & B-comp.title & O             & Q438435     \\
        de        & de       & S   & I-pers & I-pers.ind & I-comp.title & O             & Q438435     \\
        Suabe     & Souabe   & Np  & I-pers & I-pers.ind & I-comp.title & B-loc.adm.reg & Q438435     \\
        \bottomrule
    \end{tabular}
    \caption{NERC Fine-Grained annotation avec EL}
    \label{tab:data}
\end{table}

The annotated texts are available in multi-columns \texttt{tsv} files (cf.\,tab.~\ref{tab:data}). Each token has a lemma (manually corrected) and a POS (produced by the \textit{Presto} project, non-systematically corrected but fairly reliable) using the MULTEXT tag set. We propose a coarse-grained annotation for high-level entity types and fine-grained annotation using subtypes using the following syntax:
\begin{quote}
    \texttt{\ora{BIO}}-\texttt{\purp{TYPE}}.\texttt{\teal{SUBTYPE}} \\
    \textit{For instance: } \texttt{\ora{B}}-\texttt{\purp{loc}}.\texttt{\teal{adm.town}}
\end{quote}

\noindent Subtypes are sometimes simple (\texttt{B-org.\teal{town}}) sometimes double (\texttt{B-loc.\teal{phys.geo}}), depending of the complexity of the entity to annotate. Nested entities (\textit{i.e.} an entity in an entity, such as a place name in a person name in \textit{Henri d'\textbf{Angleterre}}, ``Henry of England``) follow exactly the same syntax, and components a similar one, using six transverse elements:

\begin{itemize}
    \item \texttt{name} to annotate tokens that are names (\textit{Louis}, \textit{Philippe}\dots)
    \item \texttt{title}  to annotate tokens that are titles (\textit{sieur}, \textit{duc}, \textit{abbé}\dots)
    \item \texttt{qualifier} to annotate tokens that are adjectives (\textit{l'Inde \textbf{orientale}}, \textit{l'Arabie \textbf{heureuse}}, \textit{la mer \textbf{athlantique}}, \textit{l'\textbf{ancienne} Colchide})… but also the generation (\textit{Henri \textbf{IV}}) or a cardinal position
    \item \texttt{kind} to annotate tokens that are hyperonyms (\textit{l'\textbf{Empire} de Constantinople}, \textit{la \textbf{mer} du Japon}
    \item \texttt{unit} to annotate tokens that are units (meters, league, inches, pounds…)
    \item \texttt{val} to annotate tokens that are values (a number) that is linked to a unit to annotate an \texttt{amount}.
\end{itemize}

\begin{figure}[!htp]
    \centering
    \includegraphics[width=0.75\linewidth]{static/media/mod_eval/dalembert/corpus_desc_1.png}
    \caption{Number of entities (\textit{$\log_{10}$} scale) per category.}
    \label{fig:repartition}
\end{figure}

We have decided not to annotate metaphorical uses differently or in a separate column: everything is annotated in a literal sense. Thus, in \textit{\textbf{France} goes to war}, \textit{France} is labelled \texttt{loc.adm.nat} (\textit{i.e.} the country) and not \texttt{org.adm} (\textit{i.e.} the French government).

We have also started a first phase of semantic annotation, using Wikidata~\citep{vrandecic-krotzsch-2014-wikidata} identifiers, which remains imperfect. Due to the complexity of analyzing certain entities, in particular personal names (e.g. \textit{Pope John}), it was decided to annotate them only very marginally, only in the event of the absence of ambiguity (e.g. \textit{Pope John V}). The annotation of place names, on the other hand, is more advanced and almost functional.

A first layer of annotation was made using regular expressions, before moving on to a manual correction phase. Given the size of the corpus, it is obvious that each token has not been checked, and that the final result does not claim to be perfect. Occasional checks, however, concluded that the annotation was of high enough quality to move on to the training phase. All the annotation work was carried out by a single person, in order to ensure the consistency of the data. The structure of the file and the form of the tags was controlled by a specific parser, designed specifically for this corpus. Figure \ref{fig:repartition} shows the distribution of the coarse entity categories throughout \freemner on a logarithmic scale. For more detail please refer to figures \ref{fig:entities-by-text} and \ref{fig:entity-type-per-text} in the appendix.

\section{Conclusion}

In this chapter we have presented two raw textual corpora for historical French intended to be used in the pre-training of state-of-the-art language models, one for Medieval French and another one for Early Modern French. These corpora will be used in the next par of this thesis to produce two language models in order to tackle the textual enriching task proposed by the BASNUM project. As these two corpora are in fact quite general and diverse, we believe that the models they will produce will allow researcher in Digital Humanities to enrich and better study not only for the \emph{Dictionnaire Universel}, but also any other text in Early Modern or Medieval French.

We also present a NER annotated corpus in Early Modern French, that will allow not only to evaluate our language models for Early Modern French in the upcoming part of the thesis, but also to produce a general ready-to-use state-of-the-art model for NER in Early Modern French.


\part{Models and Evaluation}
%%%%%%%%%%%%%%%%%%%%%%%%%%%%%%%%%%%%%%%%%%%%%%%%%%%%%%%%%%%%%%%%%%%%%%%%
\chapter{CamemBERT}\label{chap:camembert}
%%%%%%%%%%%%%%%%%%%%%%%%%%%%%%%%%%%%%%%%%%%%%%%%%%%%%%%%%%%%%%%%%%%%%%%%

\begin{center}
    \begin{minipage}{0.66\textwidth}
        \begin{small}
            In which we present a part of the work of \citet{martin-etal-2020-camembert} who pre-trained the first transformer based language model for Contemporary French using the French subcorpus of OSCAR 2019 \citep{ortiz-suarez-etal-2019-asynchronous,ortiz-suarez-etal-2020-monolingual}. The model that we call \camembert is then evaluated in dependency parsing, part-of-speech tagging, named entity recognition and natural language inference. We also study the question of how corpus size and diversity affects the performance of an architecture like \roberta \citep{liu-etal-2019-roberta} in downstream tasks.\footnotemark
        \end{small}
    \end{minipage}
    \vspace{0.5cm}
\end{center}

\footnotetext{Contributions: I prepared OSCAR 2019 for the pre-training of \camembert and actually had to re-write the whole pipeline in order to produce the first unshuffled version of OSCAR. I did all the experiments where \camembert is used in embedding form. I also wrote the code to synchronize and extract fixed token embeddings from \camembert which was necessary at the time since this option did not exist in Hugging Face Transformer library at the time. Moreover, the whole section \ref{sec:origin_and_size} and one of the main scientific contributions of the article was originally devised by me as one of the experiments that we wanted to conduct for the OSCAR project and was supposed to be part of \citep{ortiz-suarez-etal-2020-monolingual} presented in chapter \ref{chap:monolingual}. However, due to time and space constraints we preferred to do these experiments as part of the \camembert project. Finally, I actively participated in writing \citep{martin-etal-2020-camembert}.}

Having extensively worked into creating and curating textual resources in previous chapters and parts of this thesis, we wanted to use these resources in order to train a monolingual contextual language model for Contemporary French.

When we started the experiments that will be discussed in this chapter, the availability of large monolingual transformer based models was limited to English-only models \citep{devlin-etal-2019-bert,radford-etal-2019-language,liu-etal-2019-roberta,yang-etal-2019-xlnet,raffel-etal-2020-exploring} and most of the work in other languages was being done through multilingual models like \mbert \citep{devlin-etal-2019-bert}. And even though multilingual models gave remarkable results at the time, they were often larger, and their results, as we will observe for French, could lag behind their monolingual counterparts for high-resource languages.

In order to reproduce and validate results that had so far only been obtained for English, we took advantage of the first version of OSCAR\footnote{Now OSCAR 2019.} \citep{ortiz-suarez-etal-2019-asynchronous} which had just been released at that time. We used the French subcorpus of OSCAR 2019 to train a monolingual language model for French, dubbed \camembert. We also trained alternative versions of \camembert on different smaller corpora with different levels of homogeneity in genre and style in order to assess the impact of these parameters on downstream task performance.
\camembert used the \roberta architecture \citep{liu-etal-2019-roberta}.

We then evaluated our model on four different downstream tasks for French: part-of-speech (POS) tagging, dependency parsing, named entity recognition (NER) and natural language inference (NLI). \camembert improved on the state of the art in all four tasks compared to previous monolingual and multilingual approaches including \mbert, XLM and XLM-R, which confirmed the effectiveness of pre-trained contextual language models for French.

\section{\camembert: A Contemporary French Language Model}\label{sec:Camembert}
In this section, we describe the pre-training data, architecture, training objective and optimization setup we use for \camembert.

\subsection{Training data}
Pre-trained language models benefits from being trained on large datasets \citep{devlin-etal-2019-bert,liu-etal-2019-roberta,raffel-etal-2020-exploring}. We therefore use the French subcorpus of OSCAR 2019 \citep{ortiz-suarez-etal-2019-asynchronous,ortiz-suarez-etal-2020-monolingual}. No other filtering is done. We use the deduplicated non-shuffled version of the French subcorpus, which amounts to 138GB of raw text and to around 32.7B tokens after subword tokenization.

\subsection{Pre-processing}
We segment the input text data into subword units using SentencePiece \citep{kudo-richardson-2018-sentencepiece}. SentencePiece is an extension of Byte-Pair encoding (BPE) \citep{sennrich-etal-2016-neural} and WordPiece \citep{kudo-2018-subword} that does not require pre-tokenization (at the word or token level), thus removing the need for language-specific tokenisers. We use a vocabulary size of 32k subword tokens. These subwords are learned on $10^7$ sentences sampled randomly from the pre-training dataset.
We do not use subword regularization (i.e.~sampling from multiple possible segmentations) for the sake of simplicity.


\subsection{Language Modeling}

\paragraph{Transformer}
Similar to \roberta and \bert, \camembert is a multi-layer bidirectional Transformer \citep{vaswani-etal-2017-attention}. \camembert uses the original architectures of \bertbase (12 layers, 768 hidden dimensions, 12 attention heads, 110M parameters) and \bertlarge (24 layers, 1024 hidden dimensions, 16 attention heads, 335M parameters). \camembert is very similar to \roberta, the main difference being the use of whole-word masking and the usage of SentencePiece tokenization \citep{kudo-richardson-2018-sentencepiece} instead of WordPiece \citep{schuster-nakajima-2012-japanese}.

\paragraph{Pretraining Objective}
We train our model on the Masked Language Modeling (MLM) task.
Given an input text sequence composed of $N$ tokens $x_1, ..., x_N$, we select 15\% of tokens for possible replacement. Among those selected tokens, 80\% are replaced with the special \texttt{<MASK>} token, 10\% are left unchanged and 10\% are replaced by a random token. The model is then trained to predict the initial masked tokens using cross-entropy loss.

Following the \roberta approach, we dynamically mask tokens instead of fixing them statically for the whole dataset during preprocessing. This improves variability and makes the model more robust when training for multiple epochs.

Since we use SentencePiece to tokenize our corpus, the input tokens to the model are a mix of whole words and subwords. An upgraded version of \bert\footnote{\url{https://github.com/google-research/bert/blob/master/README.md}} and \citet{joshi-etal-2020-spanbert} have shown that masking whole words instead of individual subwords leads to improved performance. Whole-word Masking (WWM) makes the training task more difficult because the model has to predict a whole word rather than predicting only part of the word given the rest. We train our models using WWM by using white spaces in the initial non-tokenized text as word delimiters.

WWM is implemented by first randomly sampling 15\% of the words in the sequence and then considering all subword tokens in each of this 15\% for candidate replacement. This amounts to a proportion of selected tokens that is close to the original 15\%. These tokens are then either replaced by \texttt{<MASK>} tokens (80\%), left unchanged (10\%) or replaced by a random token.

Subsequent work has shown that the next sentence prediction (NSP) task originally used in \bert does not improve downstream task performance \citep{conneau-lample-2019-cross,liu-etal-2019-roberta}, thus we also remove it.

\paragraph{Optimization}
Following \citep{liu-etal-2019-roberta}, we optimize the model using Adam \citep{kingma-ba-2015-adam} ($\beta_1 = 0.9$, $\beta_2 = 0.98$) for 100k steps with large batch sizes of 8192 sequences, each sequence containing at most 512 tokens. We enforce each sequence to only contain complete paragraphs (which correspond to lines in the pre-training dataset).

\paragraph{Pre-training}
We use the \roberta implementation in the fairseq library \citep{ott-etal-2019-fairseq}. Our learning rate is warmed up for 10k steps up to a peak value of $0.0007$ instead of the original $0.0001$ given our large batch size, and then fades to zero with polynomial decay. Unless otherwise specified, our models use the BASE architecture, and are pre-trained for 100k backpropagation steps on 256 Nvidia V100 GPUs (32 GB each) for a day. We do not train our models for longer due to practical considerations, even though the performance still seemed to continue increasing afterwards.

\subsection{Using \camembert for downstream tasks}
We use the pretrained \camembert in two ways. In the first one, which we refer to as \textit{fine-tuning}, we fine-tune the model on a specific task in an end-to-end manner. In the second one, referred to as \textit{feature-based embeddings} or simply \textit{embeddings}, we extract frozen contextual embedding vectors from \camembert.
These two complementary approaches shed light on the quality of the pretrained hidden representations captured by \camembert.


\paragraph{Fine-tuning}
For each task, we append the relevant predictive layer on top of \camembert's  architecture. Following the work done on the \bert paper \citep{devlin-etal-2019-bert}, for sequence tagging and sequence labeling we append a linear layer that respectively takes as input the last hidden representation of the \texttt{<s>} special token and the last hidden representation of the first subword token of each word. For dependency parsing, we plug a bi-affine graph predictor head as inspired by \citet{dozat-manning-2017-deep}. We fine-tune on XNLI by adding a classification head composed of one hidden layer with a non-linearity and one linear projection layer, with input dropout for both.

We fine-tune \camembert independently for each task and each dataset, optimizing the model using the Adam optimizer \citep{kingma-ba-2015-adam} with a fixed learning rate. Likewise, we run a grid search on a combination of learning rates and batch sizes. Furthermore, we select the best model on the validation set out of the 30 first epochs. For NLI we use the default hyper-parameters provided by the authors of RoBERTa on the MNLI task.\footnote{More details at \url{https://github.com/pytorch/fairseq/blob/master/examples/roberta/README.glue.md}.} Although this might have pushed the performances even further, we do not apply any regularization techniques such as weight decay, learning rate warm-up or discriminative fine-tuning, except for NLI. We show that fine-tuning \camembert in a straightforward manner leads to state-of-the-art results on all tasks and outperforms the existing multilingual \bert-based models in all cases. The POS tagging, dependency parsing, and NER experiments are run using Hugging Face's Transformer library extended to support \camembert and dependency parsing \citep{wolf-etal-2019-huggingface}. The NLI experiments use the fairseq library following the \roberta implementation.

\paragraph{Embeddings}

Following \citet{strakova-etal-2019-neural} and \citet{straka-strakova-2019-evaluating} for \mbert and the English BERT, we make use of \camembert in a feature-based embeddings setting. In order to obtain a representation for a given token, we first compute the average of each sub-word’s representations in the last four layers of the Transformer, and then average the resulting sub-word vectors.

We evaluate \camembert in the embeddings setting for POS tagging, dependency parsing and NER; using the open-source implementations of \citet{straka-strakova-2019-evaluating} and \citet{strakova-etal-2019-neural}.\footnote{UDPipe Future is available at \url{https://github.com/CoNLL-UD-2018/UDPipe-Future}, and the code for nested NER is available at \url{https://github.com/ufal/acl2019_nested_ner}.}


\paragraph{Dowstream Tasks}

For POS tagging and dependency parsing, we run our experiments using the Universal Dependencies (UD)\footnote{\url{https://universaldependencies.org}.} framework and its corresponding UD POS tag set \citep{petrov-etal-2012-universal} and UD treebank collection \citep{nivre-etal-2018-universal}, which was used for the CoNLL 2018 shared task \citep{seker-etal-2018-universal}. We perform our evaluations on the four freely available French UD treebanks in UD~v2.2: GSD \citep{mcdonald-etal-2013-universal}, Sequoia\footnote{\url{https://deep-sequoia.inria.fr}.} \citep{candito-seddah-2012-le,candito-etal-2014-deep}, Spoken \citep{lacheret-etal-2014-rhapsodie,bawden-etal-2014-correcting},\footnote{Speech transcript uncased that includes annotated disfluencies without punctuation.} and ParTUT \cite{sanguinetti-Bosco-2015-parttut}.

For NER, we use the French Treebank (FTB) \citep{abeille-etal-2003-building} in its 2008 version introduced by \citet{candito-crabbe-2009-improving} and with NER annotations by \citet{sagot-etal-2012-annotation}. More precisely, we used the corrected and synchronized version \citep{ortiz-suarez-etal-2020-establishing} presented in subsection \ref{subsec:alignment}.

Finally, we evaluate our model on NLI, using the French part of the XNLI dataset \cite{conneau-etal-2018-xnli}. The XNLI dataset is the extension of the Multi-Genre NLI (MultiNLI) corpus \cite{williams-etal-2018-broad} to 15 languages by translating the validation and test sets manually into each of those languages. The English training set is machine translated for all languages other than English.

\section{Evaluation of \camembert}

In this section, we measure the performance of our models by evaluating them on the four aforementioned tasks: POS tagging, dependency parsing, NER and NLI.

\paragraph{POS Tagging and Dependency Parsing}
For POS tagging and dependency parsing, we compare \camembert with other models in the two settings: \textit{fine-tuning} and as \textit{feature-based embeddings}. We report the results in Table~\ref{tab:pos_and_dp_results}.

\begin{table}[ht]
    \small\centering
    \resizebox{\linewidth}{!}{
        \begin{tabu}{ l  c  c @{\hspace{0.35cm}}  @{\hspace{0.35cm}} c  c @{\hspace{0.35cm}}  @{\hspace{0.35cm}} c  c  @{\hspace{0.35cm}}  @{\hspace{0.35cm}} c  c }
            \toprule
                                                                               & \multicolumn{2}{c @{\hspace{0.5cm}}}{\textsc{GSD}} & \multicolumn{2}{c @{\hspace{0.7cm}}}{\textsc{Sequoia}} & \multicolumn{2}{c @{\hspace{0.7cm}}}{\textsc{Spoken}} & \multicolumn{2}{c @{\hspace{0.35cm}}}{\textsc{ParTUT}}                                                                                 \\
            \cmidrule(l{2pt}r{0.4cm}){2-3}\cmidrule(l{-0.2cm}r{0.4cm}){4-5}\cmidrule(l{-0.2cm}r{0.4cm}){6-7}\cmidrule(l{-0.2cm}r{2pt}){8-9}
            \multirow{-2}{*}[1pt]{\textsc{Model}}                              & \textsc{UPOS}                                      & \textsc{LAS}                                           & \textsc{UPOS}                                         & \textsc{LAS}                                           & \textsc{UPOS}     & \textsc{LAS}      & \textsc{UPOS}     & \textsc{LAS}      \\
            \midrule
            \mbert  (fine-tuned)                                               & 97.48                                              & 89.73                                                  & 98.41                                                 & 91.24                                                  & 96.02             & 78.63             & 97.35             & 91.37             \\
            \xlmmlmtlm (fine-tuned)                                            & 98.13                                              & 90.03                                                  & 98.51                                                 & 91.62                                                  & 96.18             & 80.89             & 97.39             & 89.43             \\ % 10138744 new XLM %& 97.71             & -                 & 98.51          & 91.62             & 95.29             & 74.17             & 96.84             & 89.22             \\ % 10138744 new XLM  % partut sequoia parsing 10138787 
            UDify \citep{kondratyuk-straka-2019-75}                            & 97.83                                              & \underline{91.45}                                      & 97.89                                                 & 90.05                                                  & 96.23             & 80.01             & 96.12             & 88.06             \\
            %\xlmEnFr & 97.51 & 93.49 & 90.72 & 98.30 &  93.62 & 91.565.42 & 96.53 & 93.03 & 90.64 \\ 
            UDPipe Future \citep{straka-2018-udpipe}                           & 97.63                                              & 88.06                                                  & 98.79                                                 & 90.73                                                  & 95.91             & 77.53             & 96.93             & 89.63             \\
            \: + mBERT + Flair  (emb.) \citep{straka-strakova-2019-evaluating} & \underline{97.98}                                  & 90.31                                                  & \textbf{99.32}                                        & 93.81                                                  & \textbf{97.23}    & \underline{81.40} & \underline{97.64} & \underline{92.47} \\
            \tabucline[\hbox {$\scriptstyle \cdot$}]{-}
            \camembert (fine-tuned)                                            & \textbf{98.18}                                     & \textbf{92.57}                                         & \underline{99.29}                                     & \textbf{94.20}                                         & 96.99             & 81.37             & \textbf{97.65}    & \textbf{93.43}    \\ % 10125734 : POS best seed PARSING gsd 10126431 PARSING other : 10126429
            UDPipe Future \mbox{+ \camembert} (embeddings)                     & 97.96                                              & 90.57                                                  & 99.25                                                 & \underline{93.89}                                      & \underline{97.09} & \textbf{81.81}    & 97.50             & 92.32             \\
            \bottomrule
        \end{tabu}
    }
    \caption{\textbf{POS} and \textbf{dependency parsing} scores on 4 French treebanks, reported on test sets assuming gold tokenization and segmentation (best model selected on validation out of 4). Best scores in bold, second best underlined.}%\lm{uniformize the notations of XLM models}}%\comment{Report best seed and not average?}
    \label{tab:pos_and_dp_results}
\end{table}

\camembert reaches state-of-the-art scores on all treebanks and metrics in both scenarios. The two approaches achieve similar scores, with a slight advantage for the fine-tuned version of \camembert, thus questioning the need for complex task-specific architectures such as UDPipe Future.

Despite a much simpler optimization process and no task specific architecture, fine-tuning \camembert outperforms UDify on all treebanks and sometimes by a large margin (e.g. +4.15\% LAS on Sequoia and +5.37 LAS on ParTUT). \camembert also reaches better performance  than other multilingual pre-trained models such as \mbert and \xlmmlmtlm on all treebanks.

\camembert achieves overall slightly better results than the previous state-of-the-art and task-specific architecture UDPipe Future+mBERT+Flair, except for POS tagging on Sequoia and POS tagging on Spoken, where \camembert lags by 0.03\% and 0.14\% UPOS respectively.
UDPipe Future+mBERT+Flair uses the contextualized string embeddings Flair \citep{akbik-etal-2018-contextual}, which are in fact pre-trained contextualized character-level word embeddings specifically designed to handle misspelled words as well as subword structures such as prefixes and suffixes. This design choice might explain the difference in score for POS tagging with CamemBERT, especially for the Spoken treebank where words are not capitalized, a factor that might pose a problem for CamemBERT which was trained on capitalized data, but that might be properly handle by Flair on the UDPipe Future+mBERT+Flair model.

\paragraph{Named-Entity Recognition}
For NER, we similarly evaluate \camembert in the fine-tuning setting and as input embeddings to the task specific architecture LSTM+CRF. We report these scores in Table~\ref{table:ner_ablation}.

\begin{table}[ht]
    \centering \small
    \begin{tabu}{lc}
        \toprule
        Model                                     & F1                \\
        \midrule
        SEM (CRF) \citep{dupont-2017-exploration} & 85.02             \\
        LSTM-CRF \citep{dupont-2017-exploration}  & 85.57             \\
        \mbert (fine-tuned)                       & 87.35             \\
        \tabucline[\hbox {$\scriptstyle \cdot$}]{-}
        \camembert (fine-tuned)                   & \underline{89.08} \\% 10129644  %91.30 dev 10129153 (2 seeds only)
        LSTM+CRF+\camembert (embeddings)          & \textbf{89.55}    \\
        %            \midrule
        %            \multicolumn{2}{c}{\em Supplement: subword masking model}\\
        %            LSTM+CRF+\camembertoscarswm (embeddings)  & \textbf{90.25} \\
        \bottomrule
    \end{tabu}
    \caption{\textbf{NER} scores on the FTB (best model selected on validation out of 4). Best scores in bold, second best underlined.
        \label{table:ner_ablation}}
\end{table}

In both scenarios, \camembert achieves higher F1 scores than the traditional CRF-based architectures (both non-neural and neural), and than the fine-tuned multilingual BERT models.\footnote{\xlmmlmtlm is a lower-case model. Case is crucial for NER, therefore we do not report its low performance (84.37\%)}

Using \camembert as embeddings to the traditional LSTM+CRF architecture gives slightly higher scores than by fine-tuning the model (89.08 vs.~89.55).
This demonstrates that even though \camembert can be used successfully without any task-specific architecture, it can still produce high quality contextualized embeddings that might be useful in scenarios where powerful downstream architectures exist.

\paragraph{Natural Language Inference}
On the XNLI benchmark, we compare \camembert to previous state-of-the-art multilingual models in the fine-tuning setting. In addition to the standard \camembert model with a BASE architecture, we train another model with the LARGE architecture, referred to as \camembertccnetlarge, for a fair comparison with XLM-R\textsubscript{LARGE}. This model was trained with the \ccnet corpus, described in Sec.~\ref{sec:origin_and_size}, for 100k steps.\footnote{We train our LARGE model with the \ccnet corpus for practical reasons, mainly due to the fact that it was more readily available on the Facebook infrastructure we used to train \camembert. Given that BASE models reach similar performance when using \oscar or \ccnet as pretraining corpus (Appendix Table~\ref{tab:ablation}), we expect an \oscar LARGE model to reach comparable scores.} We expect that training the model for longer would yield even better performance.

\begin{table}[ht]
    \centering\small
    \begin{tabu}{lcc}
        \toprule
        Model                                                             & Acc.             & \#Params \\
        \midrule
        %BiLSTM-max \citep{conneau-etal-2018-xnli} & 68.3 & - \\
        \mbert \citep{devlin-etal-2019-bert}                              & 76.9             & 175M     \\
        \xlmmlmtlm \citep{conneau-lample-2019-cross}                      & \underline{80.2} & 250M     \\
        XLM-R\textsubscript{BASE} \citep{conneau-etal-2020-unsupervised}  & 80.1             & 270M     \\
        \tabucline[\hbox {$\scriptstyle \cdot$}]{-}
        \camembert (fine-tuned)                                           & \textbf{82.5}    & 110M     \\
        \midrule
        \multicolumn{3}{c}{\em Supplement: LARGE models}                                                \\
        XLM-R\textsubscript{LARGE} \citep{conneau-etal-2020-unsupervised} & \underline{85.2} & 550M     \\
        \tabucline[\hbox {$\scriptstyle \cdot$}]{-}
        \camembertccnetlarge (fine-tuned)                                 & \textbf{85.7}    & 335M     \\
        \bottomrule
    \end{tabu}
    \caption{\textbf{NLI} accuracy on the French XNLI test set (best model selected on validation out of 10). Best scores in bold, second best underlined.\label{table:xnli}}
\end{table}

\camembert reaches higher accuracy than its BASE counterparts reaching +5.6\% over \mbert, +2.3 over \xlmmlmtlm, and +2.4 over XLM-R\textsubscript{BASE}. \camembert also uses as few as half as many parameters (110M vs. 270M for XLM-R\textsubscript{BASE}).

\camembertccnetlarge achieves a state-of-the-art accuracy of 85.7\% on the XNLI benchmark, as opposed to 85.2, for the recent XLM-R\textsubscript{LARGE}.

\camembert uses fewer parameters than multilingual models, mostly because of its smaller vocabulary size (e.g. 32k vs. 250k for XLM-R). Two elements might explain the better performance of \camembert over XLM-R. Even though XLM-R was trained on an impressive amount of data (2.5TB), only 57GB of this data is in French, whereas we used 138GB of French data. Additionally, XLM-R also handles 100 languages, and the authors show that when reducing the number of languages to 7, they can reach 82.5\% accuracy for French XNLI with their BASE architecture.

\paragraph{Summary of \camembert's results}
\camembert improves the state of the art for the 4 downstream tasks considered, thereby confirming the usefulness of a monolingual Transformer-based models for contemporary French. We obtain these results when using \camembert as a fine-tuned model or when used as contextual embeddings with task-specific architectures. This questions the need for more complex downstream architectures, similar to what was shown for English \citep{devlin-etal-2019-bert}. Additionally, this suggests that \camembert is also able to produce high-quality representations out-of-the-box without further tuning.

\section{Impact of corpus origin and size}
\label{sec:origin_and_size}

In this section we investigate the influence of the homogeneity and size of the pre-training corpus on downstream task performance. With this aim, we train alternative version of \camembert by varying the pre-training datasets. For this experiment, we fix the number of pre-training steps to 100k, and allow the number of epochs to vary accordingly (more epochs for smaller dataset sizes). All models use the BASE architecture.

In order to investigate the need for homogeneous clean data versus more diverse and possibly noisier data, we use alternative sources of pre-training data in addition to \oscar 2019:
\begin{itemize}
    \item \textbf{Wikipedia}, which is homogeneous in terms of genre and style. We use the official 2019 French Wikipedia dumps.\footnote{ \url{https://dumps.wikimedia.org/backup-index.html}.} We remove HTML tags and tables using Giuseppe Attardi's  \emph{WikiExtractor}.\footnote{ \url{https://github.com/attardi/wikiextractor}.}
    \item \textbf{\ccnet} \citep{wenzek-etal-2020-ccnet}, a dataset extracted from Common Crawl with a different filtering process than for \oscar. It was built using a language model trained on Wikipedia, in order to filter out bad quality texts such as code or tables.\footnote{We use the \textsc{head} split, which corresponds to the top 33\% of documents in terms of filtering perplexity.} As this filtering step biases the noisy data from Common Crawl to more Wikipedia-like text, we expect \ccnet to act as a middle ground between the unfiltered ``noisy'' \oscar 2019 dataset, and the ``clean'' Wikipedia dataset. As a result of the different filtering processes, \ccnet contains longer documents on average compared to \oscar 2019 with smaller---and often noisier---documents weeded out.
\end{itemize}
Table~\ref{table:corpora_statistics} summarizes statistics of these different corpora.

\begin{table}[ht]
    \centering\small
    \begin{tabular}{lcccccc}
        \toprule
        Corpus     & Size   & \#tokens & \#docs & \multicolumn{3}{c}{Tokens/doc}                 \\
                   &        &          &        & \multicolumn{3}{c}{Percentiles:}               \\
                   &        &          &        & 5\%                              & 50\% & 95\% \\
        \midrule
        Wikipedia  & 4 GB   & 990M     & 1.4M   & 102                              & 363  & 2530 \\
        CCNet      & 135 GB & 31.9B    & 33.1M  & 128                              & 414  & 2869 \\
        OSCAR 2019 & 138 GB & 32.7B    & 59.4M  & 28                               & 201  & 1946 \\
        \bottomrule
    \end{tabular}
    \caption{Statistics on the pre-training datasets used.}
    \label{table:corpora_statistics}
\end{table}

In order to make a fair comparison between these three sources of pre-training data, we randomly sample 4 GB (the size of Wikipedia) of text (at the document level) from \oscar and \ccnet, thereby creating samples of both Common-Crawl-based corpora of the same size as the French Wikipedia. These smaller 4GB samples also provides us a way to investigate the impact of pre-training data size. Downstream task performance for our alternative versions of \camembert are provided in Table~\ref{tab:ablation_data_size}.
The upper section reports scores in the fine-tuning setting while the lower section reports scores for the embeddings.

\subsection{Common Crawl vs.~Wikipedia?}
\label{subsec:homogeneityimpact}

Table~\ref{tab:ablation_data_size} clearly shows that models trained on the 4 GB versions of \oscar 2019 and \ccnet (Common Crawl) perform consistently better than the one trained on the French Wikipedia. This is true both in the fine-tuning and embeddings setting. Unsurprisingly, the gap is larger on tasks involving texts whose genre and style are more divergent from those of Wikipedia, such as tagging and parsing on the Spoken treebank. The performance gap is also very large on the XNLI task, probably as a consequence of the larger diversity of Common-Crawl-based corpora in terms of genres and topics. XNLI is indeed based on multiNLI which covers a range of genres of spoken and written text.

\begin{table}[ht]
    \small\centering
    \resizebox{\textwidth}{!}{
        \tabulinesep =_1pt^1pt
        \begin{tabu}{ l l @{\hspace{0.7cm}}  c  c  @{\hspace{0.7cm}} c  c  @{\hspace{0.7cm}} c  c @{\hspace{0.7cm}} c  c @{\hspace{0.7cm}} c c @{\hspace{0.7cm}} c @{\hspace{0.7cm}} c @{\hspace{0.7cm}}}
            \toprule
                                                    &                                      & \multicolumn{2}{c @{\hspace{0.5cm}}}{\textsc{GSD}} & \multicolumn{2}{c @{\hspace{0.7cm}}}{\textsc{Sequoia}} & \multicolumn{2}{c @{\hspace{0.7cm}}}{\textsc{Spoken}} & \multicolumn{2}{c @{\hspace{0.7cm}}}{\textsc{ParTUT}} & \multicolumn{2}{c @{\hspace{0.7cm}}}{\textsc{\textbf{Average}}} & NER               & NLI                                                                                                                            \\
            \cmidrule(l{2pt}r{0.4cm}){3-4}\cmidrule(l{-0.2cm}r{0.4cm}){5-6}\cmidrule(l{-0.2cm}r{0.4cm}){7-8}\cmidrule(l{-0.2cm}r{0.4cm}){9-10}\cmidrule(l{-0.2cm}r{0.4cm}){11-12}\cmidrule(l{-0.2cm}r{0.4cm}){13-13}\cmidrule(l{-0.2cm}r{0.4cm}){14-14}
            \multirow{-2}{*}[2pt]{\textsc{Dataset}} & \multirow{-2}{*}[2pt]{\textsc{Size}} & \textsc{UPOS}                                      & \textsc{LAS}                                           & \textsc{UPOS}                                         & \textsc{LAS}                                          & \textsc{UPOS}                                                   & \textsc{LAS}      & \textsc{UPOS}              & \textsc{LAS}      & \textsc{UPOS}     & \textsc{LAS}      & \textsc{F1}       & \textsc{Acc.}     \\
            \midrule

            \multicolumn{10}{l}{\hspace*{6mm}\em Fine-tuning}                                                                                                                                                                                                                                                                                                                                                                                                                                                                                   \\[0.5mm]
            %\oscar                                  & 0.1GB                                  & 98.12 & 92.28 & 98.52 &  90.09 & 96.45 &  76.01 & 95.08 & 87.49 & - & - & 83.25 & 72.98\\ 
            %100MB & 98.12 & 92.28 & 98.52 &  90.09 & 96.45 &  76.01 & 95.08 & 87.49  & 83.25 & 72.98\\
            Wiki                                    & 4GB                                  & 98.28                                              & 93.04                                                  & 98.74                                                 & 92.71                                                 & 96.61                                                           & 79.61             & 96.20                      & 89.67             & 97.45             & 88.75             & 89.86             & 78.32             \\ %  10137841 10137842 parsing, ppos 10138173  tagging
            \ccnet                                  & 4GB                                  & 98.34                                              & 93.43                                                  & 98.95                                                 & 93.67                                                 & 96.92                                                           & \textbf{82.09}    & 96.50                      & \textbf{90.98}    & 97.67             & \textbf{90.04}    & 90.46             & \textbf{82.06}    \\
            \oscar                                  & 4GB                                  & \underline{98.35}                                  & \underline{93.55}                                      & \underline{98.97}                                     & \underline{93.70}                                     & \underline{96.94}                                               & \underline{81.97} & \underline{96.58}          & 90.28             & \underline{97.71} & 89.87             & \underline{90.65} & \underline{81.88} \\
            \tabucline[\hbox{$\scriptstyle \cdot$}]{-}
            %\ccnet & 135GB & \underline{98.36} &  90.57 & 98.97 & 94.04 & 96.98 &  82.07 & 96.39  & \textbf{91.18} & 90.13 & \textbf{82.22} \\%  -  &  -  &
            \oscar                                  & 138GB                                & \textbf{98.39}                                     & \textbf{93.80}                                         & \textbf{98.99}                                        & \textbf{94.00}                                        & \textbf{97.17}                                                  & 81.18             & \textbf{96.63}             & \underline{90.56} & \textbf{97.79}    & \underline{89.88} & \textbf{91.55}    & 81.55             \\
            \midrule
            \multicolumn{11}{l}{\hspace*{6mm}\em Embeddings (with UDPipe Future (tagging, parsing) or LSTM+CRF (NER))}                                                                                                                                                                                                                                                                                                                                                                                                                          \\[0.5mm]
            %\oscar                                  & 0.1GB                                  & 98.04 & 91.95 & 98.73 & 92.60 & 96.96 & 81.02 & - & - & - & - & 89.78 & - \\ 
            Wiki                                    & 4GB                                  & 98.09                                              & 92.31                                                  & 98.74                                                 & 93.55                                                 & 96.24                                                           & 78.91             & 95.78                      & 89.79             & 97.21             & 88.64             & 91.23             & -                 \\
            \ccnet                                  & 4GB                                  & \textbf{98.22}                                     & \textbf{92.93}                                         & \underline{99.12}                                     & \underline{94.65}                                     & 97.17                                                           & \textbf{82.61}    & \underline{\textbf{96.74}} & \underline{89.95} & \underline{97.81} & \underline{90.04} & \textbf{92.30}    & -                 \\
            \oscar                                  & 4GB                                  & \underline{98.21}                                  & \underline{92.77}                                      & \underline{99.12}                                     & \textbf{94.92}                                        & \underline{97.20}                                               & \underline{82.47} & \underline{\textbf{96.74}} & \textbf{90.05}    & \textbf{97.82}    & \textbf{90.05}    & \underline{91.90} & -                 \\
            \tabucline[\hbox{$\scriptstyle \cdot$}]{-}
            %\ccnet & 135GB & 98.27   & 92.94  &   99.05  &  94.51   & 97.04  & 82.09  & \underline{96.68}  & 89.89 & 91.88  & - \\  -  &  -  &
            \oscar                                  & 138GB                                & 98.18                                              & \underline{92.77}                                      & \textbf{99.14}                                        & 94.24                                                 & \textbf{97.26}                                                  & 82.44             & 96.52                      & 89.89             & 97.77             & 89.84             & 91.83             & -                 \\

            \bottomrule
        \end{tabu}
    }
    \caption{Results on the four tasks using language models pre-trained on data sets of varying homogeneity and size, reported on validation sets (average of 4 runs for POS tagging, parsing and NER, average of 10 runs for NLI).}
    \label{tab:ablation_data_size}
\end{table}

The downstream task performances of the models trained on the 4 GB version of \ccnet and \oscar are much more similar.\footnote{We provide the results of a model trained on the whole \ccnet corpus in the Appendix. The conclusions are similar when comparing models trained on the full corpora: downstream results are similar when using \oscar or \ccnet.}


\subsection{How much data do you need?}
\label{subsec:sizeimpact}

An unexpected outcome of our experiments is that the model trained ``only'' on the 4 GB sample of \oscar 2019 performs remarkably similarly to the standard \camembert trained on the whole 138 GB \oscar 2019. The only task with a large performance gap is NER, where  ``138 GB'' models are better by 0.9 F1 points. This could be due to the higher number of named entities present in the larger corpora, which is beneficial for this task. On the contrary, other tasks don't seem to gain from the additional data.

In other words, when trained on corpora such as \oscar and \ccnet, which are heterogeneous in terms of genre and style, 4 GB of uncompressed text is large enough as pre-training corpus to reach state-of-the-art results with the BASE architecture, better than those obtained with \mbert (pre-trained on 60 GB of text).\footnote{The OSCAR-4 GB model gets slightly better XNLI accuracy than the full OSCAR-138 GB model (81.88 vs. 81.55). This might be due to the random seed used for pre-training, as each model is pre-trained only once.} This calls into question the need to use a very large corpus such as \oscar or \ccnet when training a monolingual Transformer-based language model such as BERT or \roberta. Not only does this mean that the computational (and therefore environmental) cost of training a state-of-the-art language model can be reduced, but it also means that \camembert-like models can be trained for all languages for which a Common-Crawl-based corpus of 4 GB or more can be created. \oscar is available in more than 150 languages, and provides such a corpus for around 38 languages. Moreover, it is possible that slightly smaller corpora (e.g.~down to 1 GB) could also prove sufficient to train high-performing language models. We obtained our results with BASE architectures. Further research is needed to confirm the validity of our findings on larger architectures and other more complex natural language understanding tasks. However, even with a BASE architecture and 4 GB of training data, the validation loss is still decreasing beyond 100k steps (and 400 epochs). This suggests that we are still under-fitting the 4 GB pre-training dataset, training longer might increase downstream performance.

\section{Discussion}

Between the pre-publication of this work\footnote{\url{https://arxiv.org/abs/1911.03894v1} (First ArXiv version).} and the publication of its peer-reviewed version \citep{martin-etal-2020-camembert}, many monolingual language models appeared, e.g. \citep{le-etal-2020-flaubert-unsupervised,virtanen-etal-2019-multilingual,delobelle-etal-2020-robbert}, and for as much as 30 languages \citep{nozza-etal-2020-what}. In almost all tested configurations they displayed better results than multilingual language models such as \mbert \citep{pires-etal-2019-multilingual}. Interestingly, \citet{le-etal-2020-flaubert-unsupervised} showed that using FlauBERT, another RoBERTa-based language model for Contemporary French, which was trained on less but more edited data, in conjunction to \camembert in an ensemble system could improve the performance of a parsing model and establish a new state-of-the-art in constituency parsing for Contemporary French, highlighting thus the complementarity of both models.\footnote{We refer the reader to \citep{le-etal-2020-flaubert-unsupervised} for a comprehensive benchmark and details therein.}

As it was the case for English when \bert was first released, the availability of similar scale language models for Contemporary French enabled interesting applications, such as large scale anonymization of legal texts, where \camembert-based models established a new state-of-the-art on this task \citep{benesty-2019-ner}, or the first large question answering experiments on a French Squad data set that was released after the publication of \camembert \citep{dhoffschmidt-etal-2020-fquad} where the authors matched human performance using \camembertlarge. Being the first pre-trained Trasnformer-based language model that used the OSCAR corpus and given its impact on the community, \camembert paved the way for many works on monolingual language models that followed. Furthermore, the availability of all its training data favors reproducibility and is a step towards better understanding such models and the impact that the pre-training data has on them. In that spirit, we make the models used in our experiments available via our website\footnote{\url{https://camembert-model.fr}} and via the \texttt{huggingface} and \texttt{fairseq} APIs, in addition to the base \camembert model.

\section{Conclusion}
In this chapter we investigated the feasibility of training a Transformer-based language model for languages other than Contemporary English. Using Contemporary French as an example, we trained \camembert, a language model based on \roberta. We evaluated \camembert on four downstream tasks (part-of-speech tagging, dependency parsing, named entity recognition
and natural language inference) in which our best model reached or improved the state of the art in all tasks considered, even when compared to strong multilingual models such as \mbert, XLM and XLM-R, while also having fewer parameters.

Our experiments confirm the previous findings presented in chapter \ref{chap:monolingual} that using web crawled data with high variability is preferable to using Wikipedia-based data. In addition, we showed that our models could reach surprisingly high performances with as low as 4 GB of pre-training data, questioning thus the need for large scale pre-training corpora. This shows that state-of-the-art Transformer-based language models can be trained on languages with far fewer resources than previously believed, and whenever a few gigabytes of data are available. This paves the way for the rise of monolingual contextual pre-trained language models for mid- and low-resourced languages. The question of knowing whether pre-training on small domain specific content will be a better option than transfer learning techniques such as fine-tuning remains open, and we will partially study it in the context of historical data in upcoming chapters.

%%%%%%%%%%%%%%%%%%%%%%%%%%%%%%%%%%%%%%%%%%%%%%%%%%%%%%%%%%%%%%%%%%%%%%%%
\chapter{FrELMo}\label{chap:frelmo}
%%%%%%%%%%%%%%%%%%%%%%%%%%%%%%%%%%%%%%%%%%%%%%%%%%%%%%%%%%%%%%%%%%%%%%%%

\begin{center}
    \begin{minipage}{0.66\textwidth}
        \begin{small}
            In which we present a part of the work of \citet{ortiz-suarez-etal-2020-establishing} who pre-train an ELMo model for Contemporary French and then evaluate its performance in the NER annotated FTB against all the available versions of CamemBERT. From these experiments \citet{ortiz-suarez-etal-2020-establishing} set a new state of the art for this corpus. We also present part of the work of \citet{popa-fabre-etal-2020-french} who further train pre-train ELMo models with the previously presented \Cabernet and CBT-fr and then evaluate them in multiple downstream tasks in order to assess the importance of representative and balanced corpora as pre-training datasets.\footnotemark
        \end{small}
    \end{minipage}
    \vspace{0.5cm}
\end{center}

\footnotetext{Contributions: for the part of \citep{ortiz-suarez-etal-2020-establishing} presented here I pre-trained FrELMo  and conducted all the experiments involving FrELMo and \camembert. For the part of {popa-fabre-etal-2020-french} presented here, I pre-trained all the ELMo models and conducted all the evaluations in downstream tasks.}

Having trained the \roberta \citep{liu-etal-2019-roberta} based \camembert \citep{martin-etal-2020-camembert} models in the previous chapter, we wanted to fairly compare the Transformer-based architecture with ELMo \citep{peters-etal-2018-deep}, the BiLSTM-based contextualized word representations that predated the BERT model \citep{devlin-etal-2019-bert}. Such a comparison had already been done to an extent in English by \citet{peters-etal-2019-tune}, but in that case, ELMo and BERT where pre-trained with different datasets, which as we saw in previous chapters, can have an enormous impact on the performance of these types of models. We thus decided to train an ELMo model with the French subcorpus of OSCAR 2019 to fairly compare with CamemBERT. We first compare these two models in a benchmarking experiment in named entity recognition that we do in order to find the best possible combination of embeddings and architectures for NER or at least for the NER annotated version of the FTB that we presented in subsection \ref{subsec:alignment}. We then expand our experiments by actually repeating most of the CamemBERT experiments but this for comparing the OSCAR pre-trained ELMo with \Cabernet and CBT-fr ELMos.

\section{FrELMo}

We train an ELMo model for contemporary French using the French subcorpus of OSCAR 2019. Furthermore, we train each model for 10 epochs, as was done for the original English ELMo \citep{peters-etal-2018-deep}. We also use the same hyper-parameters and the same pre-processing as the originals ELMo authors, i.e., we shuffle the French subcorpus of OSCAR 2019 at a line level. In this case we do not bother to save checkpoints as we previously saw that training for longer produced better models (see \ref{chap:monolingual}), so we train for the full 10 epochs as the original authors suggested \citep{peters-etal-2018-deep}.

\subsection{Benchmarking NER Models}\label{subsec:benchmarking-ner-models}

\subsubsection{Experiments}
For our benchmark of NER models for French, we used SEM \citep{dupont-2017-exploration} as our strong baseline because, to the best of our knowledge, it was the previous state-of-the-art for named entity recognition on the FTB-NE corpus. Other French NER systems are available, such as the one given by SpaCy. However, it was trained on another corpus called WikiNER, making the results non-comparable. We can also cite the system of \citep{stern-etal-2012-joint}. This system was trained on another newswire (AFP) using the same annotation guidelines, so the results given in this article are not directly comparable. This model was trained on FTB-NE in \citet{stern-2013-identification} (table C.7, page 303), but the article is written in French. The model yielded an F1-score of 0.7564, which makes it a weaker baseline than SEM. We can cite yet another NER system, namely grobid-ner.\footnote{\url{https://github.com/kermitt2/grobid-ner\#corpus-lemonde-ftb-french}} It was trained on the FTB-NE and yields an F1-score of 0.8739, but two things are to be taken into consideration in grobid-ner's score: the tagset was slightly modified and scores were averaged over a 10-fold cross validation. To see why this is important for FTB-NE, see section \ref{subsubsec:shuffling}.

In this section, we will compare our strong baseline with a series of neural models. We will use the two current state-of-the-art neural architectures for NER, namely seq2seq and LSTM-CRFs models. We will use various pre-trained embeddings in said architectures: fastText, \camembert and FrELMo embeddings.


\subsubsection{SEM}
SEM \citep{dupont-2017-exploration} is a tool that relies on linear-chain CRFs \citep{lafferty-etal-2001-conditional} to perform tagging. SEM uses Wapiti \citep{lavergne-etal-2010-practical} v1.5.0 as linear-chain CRFs implementation. SEM uses the following features for NER:
\begin{itemize}
    \item token, prefix/suffix from 1 to 5 and a Boolean isDigit features in a [-2, 2] window;
    \item previous/next common noun in sentence;
    \item 10 gazetteers (including NE lists and trigger words for NEs) applied with some priority rules in a [-2, 2] window;
    \item a "fill-in-the-gaps" gazetteers feature where tokens not found in any gazetteer are replaced by their POS, as described in \citep{raymond-fayolle-2010-reconnaissance}. These features used token unigrams and token bigrams in a [-2, 2] a window.
    \item tag unigrams and bigrams.
\end{itemize}

We trained our own SEM model by using SEM features on gold tokenization and optimized L1 and L2 penalties on the development set. The metric used to estimate convergence of the model is the error on the development set ($1 - accuracy$). Our best result on the development set was obtained using the rprop algorithm, a 0.1 L1 penalty and a 0.1 L2 penalty.

SEM also uses an NE mention broadcasting post-processing (mentions found at least once are used as a gazetteer to tag unlabeled mentions), but we did not observe any improvement using this post-processing on the best hyperparameters on the development set.


\subsubsection{Neural models}

In order to study the relative impact of different word vector representations and different architectures, we trained a number of NER neural models that differ in multiple ways. They use zero to three of the following vector representations: FastText non-contextual embeddings \citep{bojanowski-etal-2017-enriching}, the FrELMo contextual language model, and one of multiple \camembert language models \citep{martin-etal-2020-camembert} (see Appendix \ref{appendix:camembert}). The \camembert models we use in our experiments differ in multiple ways:
\begin{itemize}
    \item Training corpus: OSCAR 2019 or CCNet \citep{wenzek-etal-2020-ccnet}. For comparison purposes, we also display the results of an experiment using the mBERT multilingual BERT model trained on the Wikpiedias for over 100 languages.
    \item Model size: following \citet{devlin-etal-2019-bert}, we use both ``BASE'' and ``LARGE'' models; these models differ by their number of layers (12 vs.~24), hidden dimensions (768 vs.~1024), attention heads (12 vs.~16) and, as a result, their number of parameters (110M vs.~340M).
    \item Masking strategy: the objective function used to train a \camembert model is a masked language model objective. However, BERT-like architectures like \camembert rely on a fixed vocabulary of explicitly predefined size obtained by an algorithm that splits rarer words into subwords, which are part of the vocabulary together with more frequent words. As a result, it is possible to use a whole-word masked language objective (the model is trained to guess missing words, which might be made of more than one subword) or a subword masked language objective (the model is trained to guess missing subwords). Our models use the acronyms WWM and SWM respectively to indicate the type of masking they used.
\end{itemize}

We use these word vector representations in three types of architectures:
\begin{itemize}
    \item Fine-tuning architectures: in this case, we add a dedicated linear layer to the first subword token of each word, and the whole architecture is then fine-tuned to the NER task on the training data.
    \item Embedding architectures: word vectors produced by language models are used as word embeddings. We use such embeddings in two types of LSTM-based architectures: an LSTM fed to a seq2seq layer and an LSTM fed to a CRF layer. In such configurations, the use of several word representations at the same time is possible, using concatenation as a combination operator. For instance, in Table~\ref{tab:results_ordered}, the model FastText + CamemBERT\textsubscript{OSCAR-BASE-WWM} under the header ``\emph{LSTM-CRF + embeddings} corresponds to a model using the LSTM-CRF architecture and, as embeddings, the concatenation of FastText embeddings, the output of the \camembert ``BASE'' model trained on OSCAR with a whole-word masking objective, and the output of the FrELMo language model. For all LSTM-based architectures we use the implementation of \citet{strakova-etal-2019-neural}.
\end{itemize}

For our neural models, we optimized hyperparameters using F1-score on development set as our convergence metric.

We train each model three times with three different seeds, select the best seed on the development set, and report the results of this seed on the test set in Table~\ref{tab:results_ordered}.

\subsection{Results}

\begin{table}[htp!]
    \centering\small
    \begin{tabular}{lrrr}
        \toprule
        \textsc{Model}                                                 & \textsc{Precision} & \textsc{Recall}   & \textsc{F1-Score} \\
        \midrule
        \multicolumn{4}{c}\emph{baseline}                                                                                           \\
        %         LNSAI &  84.64 & 68.51 & 75.73\\
        SEM (CRF)                                                      & 87.18              & 80.48             & 83.70             \\
        \midrule
        LSTM-seq2seq                                                   & 85.10              & 81.87             & 83.45             \\
        + FastText                                                     & 86.98              & 83.07             & 84.98             \\
        + FastText + FrELMo                                            & 89.49              & 87.48             & 88.47             \\
        + FastText + CamemBERT\textsubscript{OSCAR-BASE-WWM}           & 89.79              & 88.86             & 89.32             \\
        + FastText + CamemBERT\textsubscript{OSCAR-BASE-WWM} + FrELMo  & 90.00              & 88.60             & 89.30             \\
        + FastText + CamemBERT\textsubscript{CCNET-BASE-WWM}           & 90.31              & 89.29             & 89.80             \\
        + FastText + CamemBERT\textsubscript{CCNET-BASE-WWM} + FrELMo  & 90.11              & 88.86             & 89.48             \\
        + FastText + CamemBERT\textsubscript{OSCAR-BASE-SWM}           & 90.09              & 89.46             & 89.77             \\
        + FastText + CamemBERT\textsubscript{OSCAR-BASE-SWM} + FrELMo  & 90.11              & 88.95             & 89.53             \\
        + FastText + CamemBERT\textsubscript{CCNET-BASE-SWM}           & 90.31              & 89.38             & 89.84             \\
        + FastText + CamemBERT\textsubscript{CCNET-BASE-SWM} + FrELMo  & 90.64              & 89.46             & \underline{90.05} \\
        + FastText + CamemBERT\textsubscript{CCNET-500K-WWM}           & \underline{90.68}  & 89.03             & 89.85             \\
        + FastText + CamemBERT\textsubscript{CCNET-500K-WWM} + FrELMo  & 90.13              & 88.34             & 89.23             \\
        + FastText + CamemBERT\textsubscript{CCNET-LARGE-WWM}          & 90.39              & 88.51             & 89.44             \\
        + FastText + CamemBERT\textsubscript{CCNET-LARGE-WWM} + FrELMo & 89.72              & 88.17             & 88.94             \\
        \midrule
        \multicolumn{4}{c}\emph{LSTM-CRF + embeddings}                                                                              \\
        LSTM-CRF                                                       & 85.87              & 81.35             & 83.55             \\
        + FastText                                                     & 88.53              & 84.63             & 86.53             \\
        + FastText + FrELMo                                            & 88.89              & 88.43             & 88.66             \\
        + FastText + CamemBERT\textsubscript{OSCAR-BASE-WWM}           & 90.47              & 88.51             & 89.48             \\
        + FastText + CamemBERT\textsubscript{OSCAR-BASE-WWM} + FrELMo  & 89.70              & 88.77             & 89.24             \\
        + FastText + CamemBERT\textsubscript{CCNET-BASE-WWM}           & 90.24              & 89.46             & 89.85             \\
        + FastText + CamemBERT\textsubscript{CCNET-BASE-WWM} + FrELMo  & 89.38              & 88.69             & 89.03             \\
        + FastText + CamemBERT\textsubscript{OSCAR-BASE-SWM}           & \textbf{90.96}     & \underline{89.55} & \textbf{90.25}    \\
        + FastText + CamemBERT\textsubscript{OSCAR-BASE-SWM} + FrELMo  & 89.44              & 88.51             & 88.98             \\
        + FastText + CamemBERT\textsubscript{CCNET-BASE-SWM}           & 90.09              & 88.69             & 89.38             \\
        + FastText + CamemBERT\textsubscript{CCNET-BASE-SWM} + FrELMo  & 88.18              & 87.65             & 87.92             \\
        + FastText + CamemBERT\textsubscript{CCNET-500K-WWM}           & 89.46              & 88.69             & 89.07             \\
        + FastText + CamemBERT\textsubscript{CCNET-500K-WWM} + FrELMo  & 90.11              & 88.86             & 89.48             \\
        + FastText + CamemBERT\textsubscript{CCNET-LARGE-WWM}          & 89.19              & 88.34             & 88.76             \\
        + FastText + CamemBERT\textsubscript{CCNET-LARGE-WWM} + FrELMo & 89.03              & 88.34             & 88.69             \\
        \midrule
        \multicolumn{4}{c}\emph{fine-tuning}                                                                                        \\
        mBERT                                                          & 80.35              & 84.02             & 82.14             \\ %% Qu'est-ce que c'est ?

        CamemBERT\textsubscript{OSCAR-BASE-WWM}                        & 89.36              & 89.18             & 89.27             \\
        CamemBERT\textsubscript{CCNET-500K-WWM}                        & 89.35              & 88.81             & 89.08             \\
        CamemBERT\textsubscript{CCNET-LARGE-WWM}                       & 88.76              & \textbf{89.58}    & 89.39             \\
        \bottomrule
    \end{tabular}
    \caption{Results on the test set for the best development set scores.}
    \label{tab:results_ordered}
\end{table}

\paragraph{Word Embeddings:} Results obtained by SEM and by our neural models are shown in table \ref{tab:results_ordered}. First important result that should be noted is that LSTM+CRF and LSTM+seq2seq models have similar performances to that of the SEM (CRF) baseline when they are not augmented with any kind of embeddings. Just adding classical fastText word embeddings dramatically increases the performance of the model.

\paragraph{ELMo Embeddings:} Adding contextualized ELMo embeddings increases again the performance for both architectures. However, we note that the difference is not as big as in the case of the pair with/without fastText word embeddings for the LSTM-CRF. For the seq2seq model, it is the contrary: adding ELMo gives a good improvement while fastText does not improve the results as much.

\paragraph{\camembert Embeddings:} Adding the \camembert embeddings always increases the performance of the model LSTM based models. However, as opposed to adding ELMo, the difference with/without \camembert is equally considerable for both the LSTM-seq2seq and LSTM-CRF. In fact adding \camembert embeddings increases the original scores far more than ELMo embeddings does, so much so that the state-of-the-art model is the LSTM + CRF + FastText + CamemBERT\textsubscript{OSCAR-BASE-SWM}.

\paragraph{\camembert + FrELMo:} Contrary to the results given in \citet{strakova-etal-2019-neural}, adding ELMo to \camembert did not have a positive impact on the performances of the models. Our hypothesis for these results is that, contrary to \citet{strakova-etal-2019-neural}, we trained ELMo and \camembert on the same corpus. We think that, in our case, ELMo either does not bring any new information or even interfere with \camembert.

\paragraph{Base vs large:} an interesting observation is that using large model negatively impacts the performances of the models. One possible reason could be that, because the models are larger, the information is more sparsely distributed and that training on the FTB-NE, a relatively small corpus, is harder.

\subsection{Impact of shuffling the data}
\label{subsubsec:shuffling}

One important thing about the FTB is that the underlying text is made of articles from the newspaper Le Monde that are chronologically ordered. Moreover, the standard development and test sets are at the end of the corpus, which means that they are made of articles that are more recent than those found in the training set. This means that a lot of entities in the development and test sets may be new and therefore unseen in the training set. To estimate the impact of this distribution, we shuffled the data, created a new training/development/test split of the same lengths as in the standard split, and retrained and reevaluated our models. We repeated this process 3 times to avoid unexpected biases. The raw results of this experiment are given in table \ref{tab:results_shuffled}. We can see that the shuffled splits result in improvements on all metrics, the improvement in F1-score on the test set ranging from 4.04 to 5.75 (or 25\% to 35\% error reduction) for our SEM baseline, and from 1.73 to 3.21 (or 18\% to 30\% error reduction) for our LSTM-CRF architectures, reaching scores comparable to the English state-of-the-art. This highlights a specific difficulty of the FTB-NE corpus where the development and test sets seem to contain non-negligible amounts of unknown entities. This specificity, however, allows to have a quality estimation which is more in line with real use cases, where unknown NEs are frequent. This is especially the case when processing newly produced texts with models trained on FTB-NE, as the text annotated in the FTB is made of articles around 20 years old.

\begin{table}
    \centering\small
    \begin{tabular}{lrrr}
        \toprule
        \textsc{Model}                                         & \textsc{Precision} & \textsc{Recall}   & \textsc{F1-Score} \\
        \midrule
        \multicolumn{4}{c}\emph{shuf 1}                                                                                     \\
        SEM(dev)                                               & 92.96              & 87.84             & 90.33             \\
        LSTM-CRF+CamemBERT\textsubscript{OSCAR-BASE-SWM}(dev)  & \underline{93.77}  & \underline{94.00} & \underline{93.89} \\
        SEM(test)                                              & 91.88              & 87.14             & 89.45             \\
        LSTM-CRF+CamemBERT\textsubscript{OSCAR-BASE-SWM}(test) & \textbf{92.59}     & \textbf{93.96}    & \textbf{93.27}    \\
        \midrule
        \multicolumn{4}{c}\emph{shuf 2}                                                                                     \\
        SEM(dev)                                               & 91.67              & 85.96             & 88.73             \\
        LSTM-CRF+CamemBERT\textsubscript{OSCAR-BASE-SWM}(dev)  & \underline{93.15}  & \underline{94.21} & \underline{93.68} \\
        SEM(test)                                              & 90.57              & 87.76             & 89.14             \\
        LSTM-CRF+CamemBERT\textsubscript{OSCAR-BASE-SWM}(test) & \textbf{92.63}     & \textbf{94.31}    & \textbf{93.46}    \\
        \midrule
        \multicolumn{4}{c}\emph{shuf 3}                                                                                     \\
        SEM(dev)                                               & 92.53              & 88.75             & 90.60             \\
        LSTM-CRF+CamemBERT\textsubscript{OSCAR-BASE-SWM}(dev)  & \underline{94.85}  & \underline{95.82} & \underline{95.34} \\
        SEM(test)                                              & 90.68              & 85.00             & 87.74             \\
        LSTM-CRF+CamemBERT\textsubscript{OSCAR-BASE-SWM}(test) & \textbf{91.30}     & \textbf{92.67}    & \textbf{91.98}    \\
        \bottomrule
    \end{tabular}
    \caption{Results on the test set for the best development set scores.}
    \label{tab:results_shuffled}
\end{table}

\subsection{Conclusions of the Benchmark}
\label{sec:conclusion}

We establish a new state-of-the-art for French NER using state-of-the-art neural techniques and recently produced neural language models for French. Our best neural model reaches an F1-score which is 6.55 points higher (a 40\% error reduction) than the strong baseline provided by the SEM system.

We also highlight how the FTB-NE is a good approximation of a real use case. Its chronological partition increases the number of unseen entities allows to have a better estimation of the generalization capacities of machine learning models than if it were randomized.

One interesting point to investigate is that using Large embeddings overall has a negative impact on the models performances. It could be because larger models store information relevant to NER more sparingly, making it harder for trained models to capitalize them. We would like to investigate this hypothesis in future research.

\section{Pre-training Corpora Evaluation for ELMo models} \label{sect:EvalMethod}

Having completed this Benchmark in NER, we also wanted to better understand the computational impact of the quality, size and linguistic balance in ELMo's \citep{peters-etal-2018-deep} pre-training. We conducted this experiments with ELMo instead of BERT or RoBERTa, as ELMo is a far less demanding model in terms of computing power when it comes to pre-training, and at the moment when we conducted these experiments we didn't have access to the infrastructure required to pre-train Transformer-based models.

\subsection{ELMo Pre-traing \& Fine-tuning Method}\label{MethodTRAIN}

Two protocols were carried out to evaluate the impact of corpora characteristics on the tasks under analysis. \textit{Method 1} implies a full pre-training ELMo-based language models for each of the corpora mentioned in Table \ref{Table_nb_Words}. While \textit{Method 2} is based on pre-training OSCAR + fine-tuning with our French Balanced Reference Corpus \Cabernet, yielding \ELMocoscar. Hence, the pure pre-traing (i.e. Method 1) yields the following four language models which were pre-trained on the four corpora under comparison :  \ELMooscar (FrELMo in the previous section), \ELMowiki, \ELMococa and \ELMocbt.

We conduct the same experiments that we did for CamemBERT in dependency parsing, POS tagging and NER. We also coupled our ELMo models with the same tasks specific architectures as before, namely textbf{UDPipe Future} \citep{straka-2018-udpipe} for POS tagging and dependency parsing and \citep{strakova-etal-2019-neural} for NER. Experiments were run using the Universal Dependencies (UD) paradigm and its corresponding UD POS-tag set \citep{petrov-etal-2012-universal} and UD treebank collection version 2.2 \citep{nivre-etal-2018-universal}, which was used for the CoNLL 2018 shared task.

\subsection{Results \& Discussion} \label{sect:ResultsCorpora}

\subsubsection{Dependency Parsing and POS-tagging}\label{sect:ResultsParsePOS}

\begin{table}[htp!]
    \small\centering
    \resizebox{\linewidth}{!}{
        \begin{tabular}{ l  c  c  c @{\hspace{0.35cm}}  @{\hspace{0.35cm}} c  c  c @{\hspace{0.35cm}}  @{\hspace{0.35cm}} c  c  c  @{\hspace{0.35cm}}  @{\hspace{0.35cm}} c  c  c }
            \toprule
                                                        & \multicolumn{3}{c @{\hspace{0.5cm}}}{\textsc{GSD}} & \multicolumn{3}{c @{\hspace{0.7cm}}}{\textsc{Sequoia}} & \multicolumn{3}{c @{\hspace{0.7cm}}}{\textsc{Spoken}} & \multicolumn{3}{c @{\hspace{0.35cm}}}{\textsc{ParTUT}}                                                                                                                                                                                                                                                                                                                        \\
            \cmidrule(l{2pt}r{0.4cm}){2-4}\cmidrule(l{-0.2cm}r{0.4cm}){5-7}\cmidrule(l{-0.2cm}r{0.4cm}){8-10}\cmidrule(l{-0.2cm}r{2pt}){11-13}
            \multirow{-2}{*}[1pt]{\textsc{Model}}       & \textsc{UPOS}                                      & \textsc{UAS}                                           & \textsc{LAS}                                          & \textsc{UPOS}                                          & \textsc{UAS}                           & \textsc{LAS}                           & \textsc{UPOS}                              & \textsc{UAS}                           & \textsc{LAS}                           & \textsc{UPOS}     & \textsc{UAS}                           & \textsc{LAS}                           \\
            \midrule
            %\multicolumn{1}{c}{UDPipe Future + ELMo} & \multicolumn{12}{c}{}\\
            %\cmidrule(lr){1-1}

            %\multicolumn{13}{l}{\textit{Baseline}} \\
            \underline{\textit{Baseline} UDPipe Future} & 97.63                                              & 90.65                                                  & 88.06                                                 & 98.79                                                  & 92.37                                  & 90.73                                  & 95.91                                      & 82.90                                  & 77.53                                  & 96.93             & 92.17                                  & 89.63                                  \\

            \:+\ELMocbt                                 & 97.49                                              & 90.21                                                  & 87.37                                                 & 98.40                                                  & 92.18                                  & 90.56                                  & 96.60                                      & 85.05                                  & 79.82                                  & 97.27             & 92.55                                  & 90.44                                  \\

            \:+\ELMowiki                                & \underline{97.92}                                  & 92.13                                                  & 89.77                                                 & 99.22                                                  & 94.28                                  & 92.97                                  & \underline{97.28}                          & 85.61                                  & 80.79                                  & \textbf{97.62}    & 94.01                                  & 91.78                                  \\

            %-FrWak  & \underline{97.89} & 92.04 & 89.70 & 99.25 & 94.53 & 93.36 & 97.20 & \textbf{86.04} & \textbf{81.14} & 97.47 & \textbf{94.78} & 92.40\\ 

            %\midrule 
            %\:+\ELMococa  & 97.76 & 91.91 & 89.49 & \underline{99.27} & \underline{94.65} & \underline{93.40} & \cellcolor[gray]{0.7}\emph{\textbf{97.32}} & 85.63 & 80.61 & \underline{97.58} & 94.24 & 91.90\\ 

            %%%%%%% new results on clean cabernet %%%%%%%%%%%%%%%%
            \:+\ELMocaber                               & 97.87                                              & 92.02                                                  & 89.62                                                 & \underline{99.33}                                      & 94.42                                  & 93.14                                  & \cellcolor[gray]{0.7}\emph{\textbf{97.30}} & 85.39                                  & 80.63                                  & 97.43             & 94.02                                  & 91.86                                  \\
            %\midrule 
            %%%%%%%%%%%%%%%%%%%%%%%%%%%%%%%%%%%%%%%%%%%%%%%%%%%%%%

            \:+\ELMooscar                               & 97.85                                              & \cellcolor[gray]{0.9}\underline{92.41}                 & \cellcolor[gray]{0.9}\underline{90.05}                & 99.30                                                  & \cellcolor[gray]{0.9}\underline{94.43} & \cellcolor[gray]{0.9}\underline{93.25} & 97.10                                      & \cellcolor[gray]{0.9}\underline{85.83} & \cellcolor[gray]{0.9}\textbf{80.94}    & 97.47             & \cellcolor[gray]{0.9}\textbf{94.74}    & \cellcolor[gray]{0.9}\textbf{92.55}    \\

            \midrule
            %\:+\ELMocoscar & \underline{97.88} & \cellcolor[gray]{0.9}\textbf{92.67} & \cellcolor[gray]{0.9} \textbf{90.34} & 99.26 & \cellcolor[gray]{0.9}\textbf{94.75} & \cellcolor[gray]{0.9}\textbf{93.54} & 97.22 & \cellcolor[gray]{0.9}\underline{85.77} & \cellcolor[gray]{0.9}\underline{80.80} & 97.50 & \cellcolor[gray]{0.9}\underline{94.66} & \cellcolor[gray]{0.9}\underline{92.43} \\ 

            %%%%%%%new results on clean cabernet oscar %%%%%%%%%%%%%%%%

            \:+\ELMocabercar                            & \textbf{97.98}                                     & \cellcolor[gray]{0.9}\textbf{92.57}                    & \cellcolor[gray]{0.9} \textbf{90.22}                  & \textbf{99.34}                                         & \cellcolor[gray]{0.9}\textbf{94.51}    & \cellcolor[gray]{0.9}\textbf{93.38}    & 97.24                                      & \cellcolor[gray]{0.9}\textbf{85.91}    & \cellcolor[gray]{0.9}\underline{80.93} & \underline{97.58} & \cellcolor[gray]{0.9}\underline{94.47} & \cellcolor[gray]{0.9}\underline{92.05} \\

            \midrule %%%%%%%%%%%%%%%%%%%%%%%%%%%%%%%%%
            \multicolumn{13}{l}{\textit{State-of-the-art}}                                                                                                                                                                                                                                                                                                                                                                                                                                                                                                                                                    \\

            \underline{UDify}                           & 97.83                                              & 93.60                                                  & 91.45                                                 & 97.89                                                  & 92.53                                  & 90.05                                  & 96.23                                      & 85.24                                  & 80.01                                  & 96.12             & 90.55                                  & 88.06                                  \\

            UDPipe Future + mBERT                       & 97.98                                              & 92.55                                                  & 90.31                                                 & \emph{99.32}                                           & 94.88                                  & 93.81                                  & 97.23                                      & \emph{86.27}                           & \emph{81.40}                           & \emph{97.64}      & 94.51                                  & 92.47                                  \\

            \camembert                                  & \emph{98.19}                                       & \emph{94.82}                                           & \emph{92.47}                                          & 99.21                                                  & \emph{95.56}                           & \emph{94.39}                           & 96.68                                      & 86.05                                  & 80.07                                  & 97.63             & 95.21                                  & \emph{92.90}                           \\

            \bottomrule
        \end{tabular}
    }
    \caption{Final POS and dependency parsing scores on 4 French treebanks (French GSD, Spoken, Sequoia and ParTUT), reported on test sets (4 averaged runs) assuming gold tokenisation. Best scores in bold, second to best underlined, state-of-the-art results in italics.}
    \label{tab:fine-tuning_results}
\end{table}

\paragraph{\ELMococa : A Test for Balance}
The representations offered by \ELMococa are not only competitive but sometimes better than Wikipedia ones. One should keep in mind that almost all the four treebanks we use in this section include Wikipedia data. \ELMococa is reaching state-of-the-are results in POS-tagging on Spoken. Notably, it performs better than \camembert, the previous state of the art on this oral specialized tree-bank (cf. dark gray highlight on Table \ref{tab:fine-tuning_results}). We understand this results as a clear effect of balance when testing upon a purely spoken test-set. Importantly, this effect is difficultly explainable by the size of oral-style data in \Cabernet. The oral sub-part is only one fifth of the total, and in this one fifth, only an even smaller amount of data comes from purely oral transcripts comparable the ones in the Spoken tree-bank, namely 67,444 words from Rhapsodie corpus, and 575,894 words form \textsc{ORFEO}. Hence, \Cabernet's  balanced oral language use shows to pay off in POS-tagging. These results are surprising, especially given the fact that our evaluation method was aiming at comparing the quality of word-embedding representations and not beating the state-of-the-art.

\paragraph{\ELMococa : A Test for Coverage}
From Table \ref{tab:fine-tuning_results}, we discover that not only balance, but also the broad and diverse genre converge of \Cabernet may play a role in its POS-tagging success is we compare its results with \ELMocbt that also features oral dialogues in youth literature. The fact that \ELMocbt does not show a comparable performance in POS-tagging, can be interpreted as linked to its size, but possibly also to its lack of variety in genres, thus, suggesting the advantage of a comprehensive coverage of language use. This suggests that a balanced sample may enhance the convergence of generalization about oral-style from distinct genre that still implies oral-like dialogues like in fiction. In sum, broad coverage may contribute to enhancing representations about oral language.

\paragraph{The effect of balance on Fine-tuning}
For POS-tagging in GSD the results of \ELMooscar are in second place position compared to \ELMocoscar that is extremely close to \ELMowiki. While in POS-tagging in ParTUT, \ELMowiki exhibits better results than \ELMooscar, and \ELMocoscar is in second position.

Comparing GSD and Sequoia scores from \ELMooscar and \ELMocoscar, we observe that fine-tuning with \Cabernet the embeddings that were pre-trained on OSCAR, yields better representations for the three tasks compared to both the original \ELMooscar and \ELMococa. However, fine-tuning does not always yield better findings than \ELMooscar on Spoken and ParTUT, where \ELMocoscar places in second after \ELMooscar for parsing scores UAS/LAS (cf. Table \ref{tab:fine-tuning_results}).

A closer look on Parsing results reveals an interesting pattern of results across treebanks (see light gray highlights on Table \ref{tab:fine-tuning_results}). We see that for GSD and Sequoia the \Cabernet fine-tuned version \ELMocoscar compared to the pure OSCAR pre-trained \ELMooscar is achieving higher scores. While a reverse and less clear-cut pattern is observable for the other two treebanks, namely Spoken and ParTUT. This configuration can be explained if we understand this pattern as due to the reinforcement and unlearning of \ELMooscar representations during the process of fine-tuning. Specifically, we can observe that parsing scores are better on treebanks that share the kind of language use represented in \Cabernet, while they are worse on corpora that are closer in language sample to OSCAR corpus, like Spoken and ParTuT. This calls for further developments of \Cabernet (§\ref{sec:Concl}).

\paragraph{\ELMocbt: small but relevant}
\ELMocbt shows an intriguing pattern of results. Even if its scores are under the baseline on GSD and Sequoia, it yields over the baseline results for Spoken and ParTUT. Given its reduced size, one would expect it to overfit, this would explain the under baseline performance. However, this was not the case on Spoken and ParTUT treebanks, thus showing \ELMocbt contribution in generating representations that are useful to UDPipe model to achieve better results in POS-tagging and parsing tasks on the ParTUT and Spoken tree-banks. The presence of oral dialogues is certainly playing a role in this results' pattern. This unexpected result calls for further investigation on the impact of pre-training with reduced-size, noiseless, domain-specific corpora.


%%%%%%%%%%%%%%%%%%%%%%%%%%%%%%%%%%%%%%%%%%%%%
%%%%%%%%%%%%%%%%%%%%%%%%%%%%%%%%%%%%%%%%%%%%%
\subsubsection{NER} \label{sect:ResultsNER}

\begin{table}[htp!]
    \centering\small
    \begin{tabular}{lccc}
        \toprule
        %\multicolumn{4}{c}{\textsc{NER - Results}}  \\\midrule
        \textsc{NER - Results} on FTB                 & Precision                            & Recall                              & F1                                  \\
        \midrule
        \multicolumn{4}{l}{\textit{Baselines Models}}                                                                                                                    \\
        SEM (CRF) \citep{dupont-2017-exploration}     & 87.89                                & 82.34                               & 85.02                               \\ %baseline 
        LSTM-CRF \citep{dupont-2017-exploration}      & 87.23                                & 83.96                               & 85.57                               \\ \midrule %baseline 2
        LSTM-CRF  test models                         & 85.87                                & 81.35                               & 83.55                               \\
        \:+FastText                                   & 88.53                                & 84.63                               & 86.53                               \\
        \:+FastText+\ELMocbt                          & 79.77                                & 77.63                               & 78.69                               \\
        \:+FastText+\ELMowiki                         & 88.87                                & 87.56                               & 88.21                               \\
        % \:+FastText+\ELMococa                  & 88.82                 & 87.82                 & 88.32                 \\
        \:+FastText+\ELMocaber                        & 88.91                                & 87.22                               & 88.06                               \\
        \:+FastText+\ELMooscar                        & 88.89                                & 88.43                               & 88.66                               \\\midrule %
        %\:+FastText+\ELMocoscar                & \cellcolor[gray]{0.8} \emph{\textbf{88.93}} & \underline{88.08}     & \underline{88.50}     \\
        \:+FastText+\ELMocabercar                     & \cellcolor[gray]{0.8} \textbf{90.70} & \cellcolor[gray]{0.8}\textbf{89.12} & \cellcolor[gray]{0.8}\textbf{89.93} \\
        \midrule

        \multicolumn{4}{l}{\textit{State-of-the-art Models}}                                                                                                             \\
        \camembert \citep{martin-etal-2020-camembert} & \underline{89.35}                    & \underline{88.81}                   & \underline{89.08}                   \\ %baseline state of the art 
        \bottomrule
    \end{tabular}
    \caption{NER Results on French Treebank (FTB): \textbf{best scores}, \underline{second to best}.}
\end{table}

For NER, LSTM-CRF+FastText+\ELMocabercar achieves a better precision, recall and F1 than the traditional CRF-based SEM architectures and even \camembert. Importantly, LSTM-CRF+FastText+\ELMocaber reaches better results in finding entity mentions, than Wikipedia which is a highly specialized corpus in terms of vocabulary variety and size, as can be seen in the overwhelming total number of unique forms it contains (see Table \ref{Table_MorphoRich}). We can conclude that both pre-training and fine-tuning with \Cabernet on \ELMooscar generates better word-embedding representations than Wikipedia in this downstream task.

CBT-fr NER results are under the LSTM-CRF baseline. This can possibly be explained by the distance in terms of topics and domain from FTB treebank (i.e. newspaper articles), or by the reduced-size of the corpus to yield good-enough representation to perform named entity recognition.

All in all, our evaluations confirm the effectiveness of large ELMo-based language models fine-tuned or pre-trained with a balanced and linguistically representative corpus, like \Cabernet as opposed to domain-specific ones.

\subsection{Conclusion} \label{sec:Concl}

We investigated the relevance of different types of corpora on ELMo's pre-training and fine-tuning. It confirms the effectiveness and quality of word-embeddings obtained through balanced and linguistically representative corpora.

The proposed evaluation methods are showing that \Cabernet and CBT-fr are not only relevant for neural NLP and language modeling in French, but that corpus balance shows to be a significant predictor of ELMo's accuracy on Spoken test data-set and for NER tasks.

The results obtained for the parsing tasks on ParTUT open a new perspective for the development of the French Balanced Reference Corpus, involving the enhancement of the terminological coverage of \Cabernet. A sixth sub-part could be included to cover technical domains like legal and medical ones, and thereby enlarge the specialized lexical coverage of \Cabernet.

Further developments of this resource would involve an extension to cover user-generated content, ranging from well written blogs, tweets to more variable written productions like newspaper's comment or forums, as present in the CoMeRe corpus \citep{chanier-etal-2014-the}. The computational experiments conducted here also show that pre-training language models like ELMo on a very small sample like the French Children Book Test corpus or \Cabernet yields unexpected results. This opens a perspective for languages that have smaller training corpora. ELMo could be a better suited language model for those languages than it is for others having larger size resources.

To conclude, our current evaluations show that linguistic quality in terms of \emph{representativeness} and balance yields better performing contextualized word-embeddings.

\chapter{SinNer CLEF-HIPE2020}

\section{Related Work on Named Entity Recognition}
\label{sec:sota}

Named Entity Recognition came into light as a prerequisite for designing robust Information Extraction (IE) systems in the MUC conferences \cite{grishman-sundheim-1995-design}. This task soon began to be treated independently from IE since it can serve multiple purposes, like Information retrieval or Media Monitoring for instance \cite{yangarber-etal-2002-unsupervised}. As such, shared task specifically dedicated to NER started to rise like the CoNLL 2003 shared task \cite{tjong-kim-sang-de-meulder-2003-introduction}. Two main paths were followed by the community: (i) since NER was at first used for general purposes, domain extension start to gain interest \cite{evans-2003-a}; (ii) since the majority of NER systems were designed for English, the extension to novel languages (including low resource languages) became of importance \cite{rossler-2004-adapting}.

One can say that NER followed the different trends in NLP. The first approaches were based on gazeeters and handcrafted rules. Initially NER was considered to be solved by a patient process involving careful syntactic analysis \cite{hobbs-1993-generic}. Supervised learning approaches came to fashion with the increase of available data and the rise of shared tasks on NER. Decision trees and Markov models were soon outperformed by Condition Random Fields (CRF).
%By taking advantage of the sequentiality of textual data, CRF helped to set new state-of-the-art results in the domain \cite{finkel-etal-2005-incorporating}.
Thanks to its ability to model dependencies and to take advantage of the sequentiality of textual data, CRF helped to set new state-of-the-art results in the domain \cite{finkel-etal-2005-incorporating}.
Since supervised learning results were bound by the size of training data, lighter approaches were tested in the beginning of the 2000's, among them we can cite weakly supervision \cite{yangarber-2003-counter} and active learning \cite{shen-etal-2004-multi}.

During a time, most of promising approaches involved an addition to improve CRFs : word embeddings \cite{passos-etal-2014-lexicon}, (bi-)LSTMs \cite{lample-etal-2016-neural} % \cite{Ma-2016}
or contextual embeddings \cite{peters-etal-2018-deep}.
More recently, the improvements in contextual word embeddings made the CRFs disappear as standalone models for systems reaching state-of-the-art results, see \cite{stanislawek-etal-2019-named} for a review on the subject and a very interesting discussion on the limits attained by state-of-the-art systems, the \textit{Glass Ceiling}.

\section{Dataset for the CLEF-HIPE shared task}
\label{sec:dataset}

The dataset of the CLEF-HIPE shared task contains newspaper articles of 17th-20th century. The text is an output of an OCR software, then tokenised and annotated with labels corresponding to each sub-task. This pecularity of historical documents will be detailed later in this section.
The corpus provided for French and German both contained training data (train) and development data (dev) whereas, for English only development data was provided for the shared task. For this reason, we chose to work on French and German only.
%In the development stage, we tried to train our model with the training data and then to evaluate our model with labeled dev data. The submitted results are based on models only trained with the training data.
Table \ref{stats} shows some statistics of this dataset. The size of the train dataset was twice higher for French than for German whereas the development sets have roughly the same size. As usual in NER, persons (Pers) and locations (Loc) are the most frequent entity types.

\begin{table}[!h]
    \centering
    \begin{tabular}{ @{\hspace{0.15cm}} l @{\hspace{0.15cm}}  @{\hspace{0.15cm}}  r @{\hspace{0.15cm}}  @{\hspace{0.15cm}} r @{\hspace{0.15cm}}  @{\hspace{0.15cm}} r  @{\hspace{0.2cm}}  @{\hspace{0.2cm}}  r @{\hspace{0.15cm}}  @{\hspace{0.15cm}} r @{\hspace{0.15cm}}  @{\hspace{0.15cm}} r @{\hspace{0.15cm}}  @{\hspace{0.15cm}} r @{\hspace{0.15cm}}  @{\hspace{0.15cm}} r @{\hspace{0.15cm}} }
        \toprule
                 & \multirow{2}{*}{Tokens} & \multirow{2}{*}{Documents} & \multirow{2}{*}{Segments} & \multicolumn{5}{c}{Labeled named entities}                            \\
        \cmidrule{5-9}
                 &                         &                            &                           & Pers                                       & Loc  & Org & Time & Prod \\
        \midrule
        Train Fr & 166217                  & 158                        & 19183                     & 3067                                       & 2513 & 833 & 273  & 198  \\
        Dev Fr   & 37592                   & 43                         & 4423                      & 771                                        & 677  & 158 & 69   & 48   \\
        Train De & 86960                   & 104                        & 10353                     & 1747                                       & 1170 & 358 & 118  & 112  \\
        Dev De   & 36175                   & 40                         & 4186                      & 664                                        & 428  & 172 & 73   & 53   \\
        \bottomrule
    \end{tabular}
    \caption{Statistics on the training and development data in French and German}
    \label{stats}
\end{table}

Table \ref{extraitCorpus} shows an excerpt of the train dataset (CoNLL format).
For each document, general information were provided. Among them, newspaper and date may have been features useful for recognising entities but we did not take advantage of it.
Each document was composed of segments, starting with "\# segment \dots" corresponding to lines in the original documents. Each segment is tokenized in order to correspond to the CoNLL format with one token per line.
These two notions, segments and tokens, are very important since they do not always match the type of unit usually processed in NLP pipelines.
Segments seldom correspond to sentences so that there is a need to concatenate the segments to get the raw text and then segment it into sentences. This is very interesting since it gets us close to real-world conditions rather than laboratory conditions, and we show in Section \ref{sec:sequence_seg} that this segment vs. sentence question has an important influence on the results.
Regarding tokens, the tokenization is obviously not perfect.
We can see that there are non-standard words and bad tokenization due to the OCR output (in red in Table \ref{extraitCorpus}).
If we concatenate the tokens we get the sequence "Su. \_sss allemands" instead of "Suisse allemande". These non-standard words make the Named Entity Recognition task more complicated and, again, more realistic.

% \begin{figure}[h!]
% \centering
%\includegraphics[width=0.5\textwidth]{./ExtraitFrPartiel.png}
% \caption{Example extracted from French training dataset}
%\label{extraitCorpus}
% \end{figure}
\begin{table}%[!htbp]
    \centering
    \scriptsize
    \scalebox{0.91}{
        \begin{tabular}{l|ll|lll|l|ll|l}
            TOKEN                               & \multicolumn{2}{c|}{NE-COARSE} & \multicolumn{3}{c|}{NE-FINE} & NE-NESTED     & \multicolumn{2}{c|}{NEL} & MISC                                                 \\
                                                & LIT                            & METO                         & LIT           & METO                     & COMP &               & LIT     & METO &              \\

            \multicolumn{10}{l}{\textcolor{blue}{\# language = fr}}                                                                                                                                               \\
            \multicolumn{10}{l}{\textcolor{blue}{\# newspaper = EXP}}                                                                                                                                             \\
            \multicolumn{10}{l}{\textcolor{blue}{\# date = 1918-04-22}}                                                                                                                                           \\
            \multicolumn{10}{l}{\textcolor{blue}{\# document\_id = EXP-1918-04-22-a-i0077}}                                                                                                                       \\
            \multicolumn{10}{l}{\textcolor{blue}{\# segment\_iiif\_link = \url{https://iiif.dhlab.epfl.ch/iiif_impresso}}\dots}                                                                                   \\%.../default.jpg}}} \\
            Lettre                              & O                              & O                            & O             & O                        & O    & O             & \_      & \_   & \_           \\
            de                                  & O                              & O                            & O             & O                        & O    & O             & \_      & \_   & \_           \\
            la                                  & O                              & O                            & O             & O                        & O    & O             & \_      & \_   & \_           \\
            \textbf{\textcolor{red}{Su}}        & B-loc                          & O                            & B-loc.adm.reg & O                        & O    & B-loc.adm.nat & Q689055 & \_   & NoSpaceAfter \\
            \textbf{\textcolor{red}{.}}         & I-loc                          & O                            & I-loc.adm.reg & O                        & O    & I-loc.adm.nat & Q689055 & \_   & \_           \\
            \textbf{\textcolor{red}{\_}}        & I-loc                          & O                            & I-loc.adm.reg & O                        & O    & I-loc.adm.nat & Q689055 & \_   & NoSpaceAfter \\
            \textbf{\textcolor{red}{sss}}       & I-loc                          & O                            & I-loc.adm.reg & O                        & O    & I-loc.adm.nat & Q689055 & \_   & \_           \\
            \textbf{\textcolor{red}{allemands}} & I-loc                          & O                            & I-loc.adm.reg & O                        & O    & O             & Q689055 & \_   & EndOfLine    \\

            \multicolumn{10}{l}{\textcolor{blue}{\# segment\_iiif\_link = \url{https://iiif.dhlab.epfl.ch/iiif_impresso}}\dots}                                                                                   \\% .../default.jpg}}} \\

            (                                   & O                              & O                            & O             & O                        & O    & O             & \_      & \_   & NoSpaceAfter \\
            Nous                                & O                              & O                            & O             & O                        & O    & O             & \_      & \_   & \_           \\
            serons                              & O                              & O                            & O             & O                        & O    & O             & \_      & \_   & \_           \\
            heureux                             & O                              & O                            & O             & O                        & O    & O             & \_      & \_   & \_           \\
            de                                  & O                              & O                            & O             & O                        & O    & O             & \_      & \_   & \_           \\
            publier                             & O                              & O                            & O             & O                        & O    & O             & \_      & \_   & \_           \\
            \dots                                                                                                                                                                                                 \\
            %%%de &	O &	O &	O &	O &	O &	O &	\_ &	\_ &	\_ \\
            %%%temps &	O &	O &	O &	O &	O &	O &	\_ &	\_ &	\_ \\
            %%%à &	O &	O &	O &	O &	O &	O &	\_ &	\_ &	EndOfLine \\
            %%%
            %%%\multicolumn{10}{l}{\textcolor{blue}{\# segment\_iiif\_link = \url{https://iiif.dhlab.epfl.ch/iiif\_impresso/\dots}}}\\%_impresso/.../default.jpg}}} \\
            %%%
            %%%autre &	O &	O &	O &	O &	O &	O & 	\_ &	\_ &	NoSpaceAfter \\ 
            %%%, &	O &	O &	O &	O &	O &	O &	\_ & 	\_ &	\_ \\
            %%%sous &	O &	O &	O &	O &	O &	O &	\_ &	\_ &	\_ \\
            %%%cette &	O &	O &	O &	O &	O &	O &	\_ &	\_ &	\_ \\ 
            %%%rubrique &	O &	O &	O &	O &	O & 	O &	\_ & 	\_ &	NoSpaceAfter \\
            %%%, &	O &	O &	O &	O &	O &	O & 	\_ &	\_ &	\_ \\
        \end{tabular}
    }
    \caption{Example extracted from the French training dataset}
    \label{extraitCorpus}
\end{table}
\vspace{-1cm}
%\input{./parts/table2.tex}

% \begin{figure}[h!]
% \centering
%\includegraphics[width=0.5\textwidth]{./exempleSuisseDetail.png}
% \caption{Extracted from French training dataset}
%\label{egNorm}
% \end{figure}


\section{CRFs and Contextualized Word Embeddings for NER}
\label{sec:method}


\subsection{CRF model (run3)}

SEM (Segmenteur-Étiqueteur Markovien)\footnote{available at: \url{https://github.com/YoannDupont/SEM}}\footnote{translates to: Markovian Tokenizer-Tagger (MTT).} \cite{dupont-2017-exploration} is a free NLP tool that relies on linear-chain CRFs \cite{lafferty-etal-2001-conditional} to perform tagging. SEM uses \textsc{Wapiti} \cite{lavergne-etal-2010-practical} v1.5.0\footnote{available at: \url{https://github.com/Jekub/Wapiti}} as linear-chain CRFs implementation. For this particular NER task, SEM uses the following features:
\begin{itemize}
    \item token, prefix/suffix from 1 to 5 and a Boolean isDigit features in a [-2, 2] window; %TODO: prefix, character ?
    \item previous/next common noun in sentence;
    \item 10 gazetteers (including NE lists and trigger words for NEs) applied with some priority rules in a [-2, 2] window;
    \item a ``fill-in-the-gaps'' gazetteers feature where tokens not found in any gazetteer are replaced by their POS, as described in \cite{raymond-fayolle-2010-reconnaissance}. This feature used token unigrams and token bigrams in a [-2, 2] a window.
    \item tag unigrams and bigrams.
\end{itemize}

We trained a CLEF HIPE specific model by optimizing L1 and L2 penalties on the development set. The metric used to estimate convergence of the model is the error on the development set ($1 - accuracy$).
% Our best result on the development set was obtained using the rprop algorithm, a 0.1 L1 penalty and a 0.1 L2 penalty.
For French, our optimal L1 and L2 penalties were 0.5 and 0.0001 respectively (default Wapiti parameters).
For German, our optimal L1 and L2 penalties were 1.0 and 0.0001 respectively.

One interest of SEM is that it has a built-in sentence tokenizer for French using a rule-based approach. By default, CLEF-HIPE provides a newline segmentation that is the output of the OCR. As a result, some NE mentions span across multiple segments, making it very hard to identify them correctly. It is to be expected that models trained (and labelling on) sentences would yield better performances than those trained (and labelling on) segments. SEM makes it simple to switch between different sequence segmentations, which allowed us to label sentences and output segments.
SEM's sentence segmentation engine works using mainly local rules to determine whether a token is the last of a sequence (eg: is a dot preceded by a known title abbreviation?). It also uses non-local rules to remember whether a token is between parentheses or French quotes to not segment automatically within them. Since we work at token level, we had to adapt some rules to fit CLEF-HIPE tokenization. For example, SEM decides at tokenization stage whether a dot is a strong punctuation or part of a larger token, as for abbreviations. This has the advantage of making sentence segmentation easier. CLEF-HIPE tokenization systematically separates dots, so we adapted some sentence segmentation rules, for example: we decided not to consider a dot as a sentence terminator if the previous token was in a lexica of titles or functions. No specific handling of OCR errors were done.
Another interest is that SEM has an NE mention broadcasting process. Mentions found at least once in a document are used as a gazetteer to tag unlabeled mentions within said document. When a new mention overlaps and is strictly longer than an already found mention, the new mention will replace the previous one in the document.


\subsection{Contextualized word embeddings}

\emph{Embeddings from Language Models} (ELMo) \cite{peters-etal-2018-deep} is a Language Model, i.e, a model that given a sequence of $N$ tokens, $(t_1, t_2, ..., t_N)$, computes the probability of the sequence
by modeling the probability of token $t_k$ given the history $(t_1, ..., t_{k-1})$:
\[
    p(t_1, t_2, \ldots, t_N) = \prod_{k=1}^N p({t_k} \mid t_1, t_2, \ldots, t_{k-1}).
\]
However, ELMo in particular uses a bidirectional language model (biLM) consisting of $L$ LSTM layers, that is, it combines both a forward and a backward language model jointly maximizing the log likelihood of the forward and backward directions:
\begin{align*}
     & \sum_{k=1}^N \left( \right. \log p({t_k} \mid t_1, \ldots, t_{k-1}; \Theta_x, \overrightarrow{\Theta}_{LSTM}, \Theta_s) \\
     & + \log p({t_k} \mid t_{k+1}, \ldots, t_{N}; \Theta_x, \overleftarrow{\Theta}_{LSTM}, \Theta_s)
    \left. \right).
\end{align*}
where at each position $k$, each LSTM layer $l$ outputs a context-dependent representation $\overrightarrow{\mathbf{h}}^{LM}_{k,l}$ with $l=1, \ldots, L$ for a forward LSTM, and $\overleftarrow{\mathbf{h}}^{LM}_{k,l}$ of $t_k$ given $(t_{k+1}, \ldots, t_N)$ for a backward LSTM.

ELMo also computes a context-independent token representation $\mathbf{x}^{LM}_{k}$ via token embeddings or via a CNN over characters. ELMo then ties the parameters for the token representation ($\Theta_x$) and Softmax layer ($\Theta_s$) in the forward and backward direction while maintaining separate parameters for the LSTMs in each direction.

ELMo is a task specific combination of the intermediate layer representations in the biLM, that is,
for each token $t_k$, a $L$-layer biLM computes a set of $2L + 1$ representations
\begin{align*}
    R_k & =  \{\mathbf{x}^{LM}_{k}, \overrightarrow{\mathbf{h}}^{LM}_{k,l}, \overleftarrow{\mathbf{h}}^{LM}_{k,l} \ |\  l =1, \ldots, L \} \\
        & =  \{\mathbf{h}^{LM}_{k,l}\ | \ l=0, \ldots, L\},
\end{align*}
where $\mathbf{h}^{LM}_{k,0}$ is the token layer and
\[
    \mathbf{h}^{LM}_{k,l} = [\overrightarrow{\mathbf{h}}^{LM}_{k,l}; \overleftarrow{\mathbf{h}}^{LM}_{k,l}],
\]
for each biLSTM layer.


When included in a downstream model, as it is the case in this paper, ELMo collapses all $L$ layers in $R$ into a single vector $\mathbf{ELMo}_k = E(R_k; \mathbf{\Theta}_e)$, generally computing a task specific weighting of all biLM layers:
\begin{align*}
    \mathbf{ELMo}^{task}_k & = E(R_k; \Theta^{task})                                       \\
                           & =\gamma^{task} \sum_{l=0}^L s^{task}_l \mathbf{h}^{LM}_{k,l}.
\end{align*}
applying layer normalization to each biLM layer before weighting.

Following \cite{peters-etal-2018-deep}, we use in this paper ELMo models where $L=2$, i.e., the ELMo architecture involves a character-level CNN layer followed by a 2-layer biLSTM.

\subsection{ELMo-LSTM-CRF (run1 and run2)}

The LSTM-CRF is a model originally proposed by Lample et al. \cite{lample-etal-2016-neural} it consists of a Bi-LSTM encoder pre-appended by both character level word embeddings and pre-trained word embeddings, and a CRF decoder layer. For our experiments, we follow the same approach as Ortiz Suárez et al. \cite{ortiz-suarez-etal-2020-establishing} by using the Bi-LSTM-CRF implementation of Straková et al. \cite{strakova-etal-2019-neural} which is open source and readily available\footnote{Available at: \url{https://github.com/ufal/acl2019_nested_ner}.}, and pre-appending contextualized word-embeddings to the model. For French we pre-append the FrELMo model \cite{ortiz-suarez-etal-2020-establishing}, which is the standard ELMo \cite{peters-etal-2018-deep} implementation\footnote{Available at: \url{https://github.com/allenai/bilm-tf}} trained on the French OSCAR\footnote{Available at: \url{https://oscar-corpus.com}} corpus \cite{ortiz-suarez-etal-2020-monolingual} \cite{ortiz-suarez-etal-2019-asynchronous}. For German we pre-append the German ELMo \cite{may-2019-german}, which is again the standard ELMo implementation but trained on the German Wikipedia.

Contrary to the approach of Ortiz Suárez et al. \cite{ortiz-suarez-etal-2020-establishing}, we do not use the CamemBERT model \cite{martin-etal-2020-camembert} for French or the German BERT \cite{chan-etal-2019-german}. Both of these models are BERT-based and as such they are limited to a 512-token contextualized window. Moreover, they both use SentencePiece \cite{kudo-richardson-2018-sentencepiece} meaning that tokens are actually subwords, which considerably increases the number of tokens per sentence, specially for the longer ones, thus decreasing the contextual windows of both CamemBERT and the German BERT. SentencePiece also introduces the problem of a fixed-size vocabulary, which in the case of this shared task might negatively impact the performance of said models, as they could struggle handling OCR problems or just non-standard vocabulary. Since our main goal was to reconstruct the sentences and use long contextualized sequences we opted to use ELMo which can easily handle longer sequences with it's standard implementation and actually has a dynamic vocabulary thanks to the CNN character embedding layer, thus it might be better equipped to handle non-standard orthography and OCR problems.

For the fixed word embeddings we used the Common Crawl-based FastText embeddings \cite{grave-etal-2018-learning} originally trained by Facebook as opposed to the embeddings provided by the HIPE shared task, as we obtained better dev scores using the original FastText embeddings for both French and German.

We used the standard hyperparameters originally\footnote{\url{https://github.com/ufal/acl2019_nested_ner/blob/master/tagger.py\#L484}.} used by Straková et al. \cite{strakova-etal-2019-neural}. Namely a batch size of 8, a dropout of 0.5, a learning rate of 0.001 and 10 epochs. The difference between run 1 and 2, is that run 1 uses the data as is, while run 2 uses the reconstructed sentences.


\section{Results and Discussion}
\label{sec:results}

\subsection{Official shared task results}

The results of our 3 runs compared to the best run on the NERC-coarse shared-task for French and German are given in Table \ref{tab:results-raw} (strict scenario).
For both tasks, we are the third best ranking team.
% We are happy about those results as we
We only did very minimal adaptation of existing systems. We did not modify tokenization for any language. The most notable change was to use custom sentence segmentation instead of given segments for French and using some additional lexica as features for our CRF model in German (for French, we only used existing SEM lexica). Other than that, we only optimized hyper-parameters on the dev set. This clearly illustrates the power of contextual embeddings and today's neural network architectures. This is encouraging in terms of usability of SotA models on real-world data.

\begin{table}
    \centering
    \begin{tabular}{p{0.11\linewidth}x{0.11\linewidth}x{0.11\linewidth}x{0.11\linewidth}x{0.11\linewidth}x{0.11\linewidth}x{0.11\linewidth}}
        \toprule
        \multirow{2}{*}{\textsc{run}} & \multicolumn{3}{c}{\textsc{French}} & \multicolumn{3}{c}{\textsc{German}}                                                                             \\
        \cmidrule(l{0.4cm}r{0.4cm}){2-4}\cmidrule(l{0.4cm}r{0.4cm}){5-7}
                                      & P                                   & R                                   & F1               & P                & R                & F1               \\
        \midrule
        winner                        & 83.1                                & 84.9                                & 84.0             & 79.0             & 80.5             & 79.7             \\
        run 1                         & \underline{77.8}                    & \underline{79.4}                    & \underline{78.6} & \underline{63.1} & \textbf{66.6}    & \underline{64.8} \\
        run 2                         & \textbf{78.8}                       & \textbf{80.2}                       & \textbf{79.5}    & \textbf{65.8}    & \underline{65.8} & \textbf{65.8}    \\
        run 3                         & 70.2                                & 57.9                                & 63.5             & 64.4             & 43.8             & 52.1             \\
        \midrule
        average                       & 70.2                                & 66.7                                & 67.6             & 63.8             & 58.1             & 60.0             \\
        median                        & 71.5                                & 68.6                                & 68.6             & 66.8             & 57.7             & 64.5             \\
        \bottomrule
    \end{tabular}
    \caption{Strict results for our systems compared to the winning system (micro measures)}
    \label{tab:results-raw}
\end{table}

\subsection{Study of sequence segmentation}
\label{sec:sequence_seg}

%~ segments
%~ processed 29854 tokens with 1532 phrases; found: 1476 phrases; correct: 1196.
%~ accuracy: 96.58%; precision: 81.03%; recall: 78.07%; FB1: 79.52
%~  loc: precision: 85.21%; recall: 87.52%; FB1: 86.35 568
%~  org: precision: 70.62%; recall: 62.78%; FB1: 66.47 160
%~  pers: precision: 80.24%; recall: 76.88%; FB1: 78.52 663
%~  prod: precision: 62.96%; recall: 39.53%; FB1: 48.57 27
%~  time: precision: 86.21%; recall: 78.12%; FB1: 81.97 58

%~ sentences
%~ processed 29854 tokens with 1365 phrases; found: 1319 phrases; correct: 1114.
%~ accuracy: 97.61%; precision: 84.46%; recall: 81.61%; FB1: 83.01
%~  loc: precision: 87.73%; recall: 87.08%; FB1: 87.41 538
%~  org: precision: 71.33%; recall: 65.64%; FB1: 68.37 150
%~  pers: precision: 84.64%; recall: 82.09%; FB1: 83.35 547
%~  prod: precision: 75.86%; recall: 56.41%; FB1: 64.71 29
%~  time: precision: 90.91%; recall: 87.72%; FB1: 89.29 55

% \begin{table}
% \centering
% \begin{tabular}{|l|ccc|}
% \hline
% segmentation & P     & R     & F1    \\
% \hline
% segments     & 81.03 & 81.61 & 79.52 \\
% sentences    & 84.46 & 84.46 & 83.01 \\
% \hline
% \end{tabular}
% \caption{result difference between segments and SEM sentence segmentation on dev}
% \label{tab:segment-vs-sentences}
% \end{table}

In this section, we evaluate the influence of sequence segmentation on system performances. This evaluation is done for French only, as we used SEM to provide sentence segmentation and SEM could only provide a proper sentence segmentation for that language. As can be seen in table \ref{tab:segment-vs-sentences}, sentence segmentation allows to improve results by 3.5 F1 points. This is due to the fact that some entities were split across multiple segments in the original data. Using a custom sentence segmentation allows to have entities in a single sequence. This segmentation is applied both with training data and evaluation data, so that our systems can access a more proper context for named entities. The cost of using another segmentation is relatively cheap, as SEM can process nearly 1GB of raw text per hour.

A per entity comparison is also available in Table \ref{tab:segment-vs-sentences}.
One can see that the improvement of sentence segmentation is not very significant for locations (Loc). It is due to two facts : (i) locations are usually small in number of tokens and therefore less prone to be separated in two segments and (ii) there was less room from improvement since they were the easiest entity type to detect (86.35\% F1-score).
To the contrary, entities of type ``product'' (Prod), usually longer in tokens, were very hard to predict with only 48.57\% F1-measure and benefited the most from segmentation in sentences (+16 percentage points in F1-measure).


%~ 2.52 -0.44  1.06
%~ 0.71  2.86  1.90
%~ 4.40  5.21  4.83
%~ 12.90 16.88 16.14
%~ 4.70  9.60  7.32

\begin{table}
    \scalebox{0.92}{
        \begin{tabular}{@{\hspace{0.15cm}} l @{\hspace{0.2cm}}  @{\hspace{0.2cm}} r @{\hspace{0.15cm}}  @{\hspace{0.15cm}} r @{\hspace{0.2cm}}  @{\hspace{0.2cm}} r @{\hspace{0.15cm}}  @{\hspace{0.15cm}} r @{\hspace{0.2cm}}  @{\hspace{0.2cm}} r @{\hspace{0.15cm}}  @{\hspace{0.15cm}} r@ {\hspace{0.15cm}} }% YD: I like that one better
            %\begin{tabular}{lclclcl}%Better aligned this way ?
            \toprule
            \multirow{2}{*}{\textsc{Type}} & \multicolumn{2}{c}{\textsc{P}} & \multicolumn{2}{c}{\textsc{R}} & \multicolumn{2}{c}{\textsc{F1}}                                                                                                \\
            \cmidrule(l{-0.15cm}r{0.3cm}){2-3}\cmidrule(l{-0.15cm}r{0.3cm}){4-5}\cmidrule(l{-0.15cm}r{0.15cm}){6-7}
                                           & \multicolumn{1}{c}{Segments}   & \multicolumn{1}{c}{Sentences}  & \multicolumn{1}{c}{Segments}    & \multicolumn{1}{c}{Sentences} & \multicolumn{1}{c}{Segments} & \multicolumn{1}{c}{Sentences} \\
            \midrule
            Loc                            & 85.21                          & 87.73 (+2.52)                  & 87.52                           & 87.08 (-0.44)                 & 86.35                        & 87.41 (+1.06)                 \\
            Org                            & 70.62                          & 71.33 (+0.71)                  & 62.78                           & 65.64 (+2.86)                 & 66.47                        & 68.37 (+1.90)                 \\
            Pers                           & 80.24                          & 84.64 (+4.40)                  & 76.88                           & 82.09 (+5.21)                 & 78.52                        & 83.35 (+4.83)                 \\
            Prod                           & 62.96                          & 75.86 (+12.90)                 & 39.53                           & 56.41 (+16.88)                & 48.57                        & 64.71 (+16.14)                \\
            Time                           & 86.21                          & 90.91 (+4.70)                  & 78.12                           & 87.72 (+9.60)                 & 81.97                        & 89.29 (+7.32)                 \\
            \midrule
            Global                         & 81.03                          & 84.46 (+3.43)                  & 81.61                           & 84.46 (+2.85)                 & 79.52                        & 83.01 (+3.49)                 \\
            \bottomrule
        \end{tabular}
    }
    \caption{Comparison between segments and sentences on French dev dataset (run 1), strict scenario}
    \label{tab:segment-vs-sentences}
\end{table}


\subsection{To dev or not to dev?}
\label{sec:todev-ornot}

In Table \ref{tab:to-dev} we show the results that could have been obtained by training the Bi-LSTM model %TODO: checker si je dis pas de conneries
on both train and dev dataset. We used the same hyperparameters as we did for our official run. Despite the fact that it does not ensure the robustness of the system, the added-value seem to be quite disappointing\footnote{In particular, if we consider that it would not have given us a better ranking on any language.}. In German the gain may be a bit more significant, probably due to the smaller size of the training dataset.

\begin{table}
    \centering
    \begin{tabular}{lx{0.15\linewidth}x{0.15\linewidth}x{0.15\linewidth}x{0.15\linewidth}}
        \toprule
        \multirow{2}{*}{\textsc{metric}} & \multicolumn{2}{c}{\textsc{french}} & \multicolumn{2}{c}{\textsc{german}}                                     \\
        \cmidrule(l{0.15cm}r{0.15cm}){2-3}\cmidrule(l{0.15cm}r{0.15cm}){4-5}
                                         & not to dev                          & to dev                              & not to dev & to dev               \\
        \midrule
        P                                & 78.8                                & \textbf{79.5} (+0.7)                & 65.8       & \textbf{68.2} (+2.4) \\
        R                                & 80.2                                & \textbf{80.7} (+0.5)                & 65.8       & \textbf{66.1} (+0.3) \\
        F1                               & 79.5                                & \textbf{80.1} (+0.6)                & 65.8       & \textbf{67.1} (+1.3) \\
        \bottomrule
    \end{tabular}
    \caption{Results obtained on the test set (strict metric) with only the train set (not to dev) and with train+dev sets (to dev) with our best system (run 2)\label{tab:to-dev}}

\end{table}

\section{Conclusion}
\label{sec:concl}

In this article we presented three methods developed for the Named Entity Recognition task in French and German historical newspapers.
The first method relied on linear-chain CRFs while the other two methods use a Bidirectional LSTM and a bidirectional Language Model (ELMo).
The later outperformed the CRF model and achieved rank 3 on the NER task in both French and German.
We also showed that the type of sequences used has a significant influence on the results. When we segment in sentences rather than using the segments of the dataset as it is the results are systematically much better, with an exception for locations where the gain is marginal. This proves that sentence segmentation remains a key component of efficient NLP architectures, in particular for models taking advantage of the context.

As a future work it would be interesting to assess the importance of noise in the data. For instance, by comparing the results of NER on texts obtained via different OCR tools.
The influence of the qualitative jumps in the data, which is common in Digital Humanities, is an important aspect to evaluate the robustness of the system in real-world conditions rather than laboratory conditions.
We also plan to provide an in-depth analysis of the impact of word embeddings and neural architecture, as we only provided our best results in this paper.

\chapter{BERTrade}

\section{Data}\label{sec-data}

\begin{figure}[thb]
    \centering
    \includegraphics[scale=0.29]{static/media/mod_eval/bertrade/map-dialects2.png}
    \caption{Oïl languages}
    \label{fig:map-dialects}
\end{figure}

This section describes the raw corpus of Medieval French we gathered in order to train unsupervised language models for Old French.
To our knowledge, it is one of the largest such dataset gathered for Medieval French, although it remains quite small (\SI{55}{\mebi\byte} in total) relatively to the corpora usually used for pre-training contextual embeddings models.

Medieval French covers both Old French (9th-13th c.) and Middle French (14th-15th c.). These stages are linguistically close and both precede the adoption of spelling norms. Middle French is more regular than Old French in some respects such as word order \citep{marchello-Nizia-etal-2020-grande} and less in others such as NP structure and pronouns system \citep{marchello-nizia-etal-1979-histoire}. Medieval French covers a set of \textit{Oïl} Romance languages spoken in the kingdom of France between the 9th and the 15th century (\cref{fig:map-dialects}).
There are around twenty such languages.

Older texts are close to Late Latin, and verse is prevalent until the end of the 13th century. Old French has a relatively free word order.
Until the mid-11th century, the prevalent order is \textit{Subject-Object-Verb} (SOV), which is then gradually supplanted by SVO, which is the default order in contemporary French.
%Unlike most languages with free word order, word functions are not usually given by morphological clues, such as the rich case system of Classical Latin and there are many cases of syntactic ambiguity.
Unlike most languages with free word order, the functions of verbal arguments are not always given away by morphological clues, the already simplistic %\footnote{As compared to that of Latin, for instance.} 
case system of Old French disappears progressively through the covered period.

There are also many cases of syntactic ambiguity. For example, in the following quote from \emph{Lancelot},\footnote{In the edition from Pierre Kunstmann, from the online \textit{Base de français médiéval}: \url{http://catalog.bfm-corpus.org/CharretteKu}.} (verse ~5436),
both \enquote{la dame} and \enquote{Lancelot} could be the subject or the object of \enquote{Vit} and only the context enables the reader to understand that \enquote{la dame} is the subject.

\digloss{Dolant et pansif Lancelot Vit la dame}
{Mournful and meditative Lancelot saw the lady}
{The lady saw that Lancelot was mournful and meditative.}

Word order is also relatively free within constituents. For example, a noun modifier can be on the left or on the right of its governor, and it is not necessarily preceded by a preposition. In contemporary French, it can only appear on the right, and it is found without a preposition only in some cases like named entities. Because of the general free word order and the absence of punctuation in our treebank, this adds up to the ambiguity of the analysis.

In each of the following examples from the SRCMF corpus, the noun following \emph{roi} (\enquote{king}) has a different analysis: head of \emph{roi}, modifier, argument of the same verb or a different one, with no explicit marking:

\begin{center}
    % beroul. modifieur à gauche.
    \begin{dependency}[theme=simple]
        \begin{deptext}[row 2/.style={font=\small}]
            \textit{Fus} \& \textit{tu} \& \textit{donc} \& \textit{pus} \& \textit{a} \& \textit{la} \& \textbf{\textit{roi}} \& \textit{cort} \\
            %VERB \& PRON \& ADV \& ADV \& ADP \& DET \& NOUN \& NOUN \\
            Were \& you \& then \& no more \& at \& the \& king \& court \\
        \end{deptext}
        \depedge{8}{7}{nmod}
        \depedge[edge start x offset=0.5em]{8}{6}{det}
        \depedge[edge start x offset=1em]{8}{5}{case}
    \end{dependency}

    \raggedright
    \enquote{Then were you not at the king's court anymore?} (\emph{Beroul Tristan})
\end{center}

\begin{center}
    % Graal. modifieur à droite.
    \begin{dependency}[theme=simple]
        \begin{deptext}[row 2/.style={font=\small}]
            \textit{la} \& \textit{fille} \& \textit{au} \& \textit{riche} \& \textbf{\textit{roi}} \& \textit{pescheor} \\
            the \& daughter \& of the \& rich \& king \& fisher \\
        \end{deptext}
        \depedge{5}{6}{flat}
    \end{dependency}

    \raggedright
    \enquote{the daughter of the rich Fisher King} (\emph{Queste del Saint Graal})
\end{center}

\begin{center}
    % Roland. arguments du même verbe.
    \begin{dependency}[theme=simple]
        \begin{deptext}[row 2/.style={font=\small}]
            \textit{De} \& \textit{Guenelun} \& \textit{atent} \& \textit{li} \& \textbf{\textit{reis}} \& \textit{nuveles} \\
            From \& Ganelon \& waits \& the \& king \& news \\
        \end{deptext}
        \depedge{3}{5}{nsubj}
        \depedge[edge start x offset=-0.5em]{3}{6}{obj}
    \end{dependency}

    \raggedright
    \enquote{The king waits for news from Ganelon.} (\emph{Chanson de Roland})
\end{center}

\begin{center}
    % Graal. arguments de verbes différents, et absence de ponctuation.
    \begin{dependency}[theme=simple]
        \begin{deptext}[row 2/.style={font=\small}]
            \textit{Biax} \& \textit{sire} \& \textit{fet} \& \textit{li} \& \textbf{\textit{rois}} \& \textit{escu} \& \textit{vos} \& \textit{envoiera} \& \textit{Diex} \\
            Dear \& Sir \& says \& the \& king \& shield \& you \& send-FUT \& God \\
        \end{deptext}
        \depedge{3}{5}{nsubj}
        \depedge{8}{6}{obj}
    \end{dependency}

    \raggedright
    \enquote{Dear Sir, says the king, God will send you a shield.} (\emph{Queste del Saint Graal})
\end{center}

Furthermore, overt subjects are not mandatory, and are often dropped in texts written in verse until the 12th century, after which the presence of subjects increases through time.
These phenomena are particularly prevalent in verse, where metric and rhyming constraints often lead to more contrived syntactic forms than in prose.

Another source of ambiguity is the variety of spellings, due to the lack of spelling standard. For example, the word \textit{moult} (transl. \textit{a lot (of), very}), emblematic of this period, is initially an adjective, and it is progressively grammaticalized, becoming an adverb. Several forms appear at the same time, some with a declension, some without, and the radical does not have a fixed spelling: \textit{molt(e)(s), molz, mult(e)(s), mul(t)z, mou(l)t}…

We chose to include a few texts from the early Middle French period (14th-15th c.) in this raw corpus, which brings a valuable complement of the prose documents that are lacking for Old French, while staying close enough to late Old French, the boundary between the two epochs being somewhat fuzzy.
These texts precede the adoption of norms established by editors after the invention of Gutenberg's printing press. Middle French is more regular than Old French in some respects such as word order \citep{marchello-Nizia-etal-2020-grande} and less in others such as NP structure and pronouns system \citep{marchello-nizia-etal-1979-histoire}, but they share most of their lexicon and for these relatively early texts, the syntax is not too different from that of late Old French texts.

%Corpora
\begin{table}[thb]
    \centering
    \tablefontsize
    \begin{tabular}{l S[table-format=2.1]}
        \toprule
        {\textbf{Corpus}}                              & {\textbf{Size / \si{\mebi\byte}}} \\ %& {\textbf{\# texts}}
        \midrule
        BFM \citep{guillot-etal-2018-base}             & 20.7                              \\ %139
        AND \citep{rothwell-etal-2005-anglo}           & 17.2                              \\ %73
        NCA \citep{kunstmann-stein-2007-le}            & 9.7                               \\ %271
        Chartes Douai \citep{glessen-2003-elaboration} & 3.1                               \\ %1
        OpenMedFr \citep{wrisley-2018-the}             & 1.7                               \\ %19
        Geste \citep{camps-etal-2019-geste}            & 1.5                               \\ %32
        MCVF \citep{martineau-2008-un}                 & 1.4                               \\ %17
        Chartes Aube \citep{reenen-etal-2007-chartes}  & 0.2                               \\ %75
        \midrule
        Total                                          & 55.3                              \\ %627
        \bottomrule
    \end{tabular}
    \caption{Data collection}
    \label{tab:texts_train}
\end{table}

\begin{figure}[thb]
    \centering
    \begin{tikzpicture}
        \begin{axis}[
                xbar,
                colormap name=viridis,
                cycle list={
                        {color of colormap=200, draw=., preaction={fill=.}, pattern=grid},
                        {color of colormap=800, draw=., preaction={fill=.}, pattern=dots},
                    },
                font=\footnotesize,
                width=\linewidth,
                yticklabel style={
                        xshift=0.5cm,
                        align=right,
                    },
                y axis line style = { opacity = 0 },
                xlabel = Datasize (\unit{\mebi\byte}),
                %axis x line       = none,
                tickwidth         = 0pt,
                ytick             = data,
                % enlarge y limits  = 0.15,
                % enlarge x limits  = 0.2,
                symbolic y coords = {Legal, Historical, Didactic, Religious, Literature},
                nodes near coords,
                nodes near coords style={black},
                legend style={at={(0.95,0.6)}},
                reverse legend,
            ]
            \addplot
            coordinates {(13.3321,Literature) (4.35204,Religious)
                    (4.35204,Didactic) (0.687576,Historical) (0,Legal)};
            \addlegendentry{Verse}
            \addplot
            coordinates {(2.352817,Literature) (4.433186,Religious)
                    (3.041925,Didactic) (8.362256,Historical) (15.705591,Legal)};
            \addlegendentry{Prose}
            % 			\legend{Verse,Prose}
        \end{axis}
    \end{tikzpicture}
    \caption{Distribution of form and domain, gathered from documents metadata and manual annotation.}
    \label{fig:metadata}
\end{figure}

Medieval French has many factors of variation: language evolution, dialects, domains, forms of text (verse or prose) and lack of standard. Our dataset gives us a representation of Medieval French that is as accurate and diversified as possible, given the limited amount of material that survived to these days.
The detailed instructions to replicate this dataset are described in the Appendix. No particular processing is done on the original documents.

In order to get a sound evaluation of the contextual embeddings trained with this dataset, we filter out the documents that are also present in the SRCMF treebank used for evaluation purposes in section \ref{sec-experiments}\footnote{As noted by \citet{gururangan-etal-2020-dont}, pre-training on task specific data provides an additional boost, that would muddle our results, since our objective here is not so much task optimization as embeddings benchmarking.}.
The resulting corpus is quite heterogeneous: legal texts and verse literature are in the majority, whereas other domains, such as historical and didactic texts, are under-represented, as can be seen in \cref{fig:metadata}.

\section{Experiments}
\label{sec-experiments}
We evaluate a set of alternative word representations on Old French, using their usefulness for POS-tagging and dependency parsing as a downstream evaluation.
To that end, we use the annotated treebank of Old French (SRCMF,  \citet{prevost-stein-2013-syntactic}) as provided by the 2.7 version of the UD dataset \citep{zeman-etal-2020-universal} as a reference treebank.

Our parser/tagger probe uses \citet{dozat-manning-2018-simpler}'s neural graph parser made as reimplemented by \citet{le-etal-2020-flaubert} and \citet{grobol-crabbe-2021-analyse}, using the same hyperparameters.
Word representations are obtained by concatenating subword embeddings, averaged over transformer layers together with character embeddings and non contextualized word embeddings.  %without weighting or scaling. 
%returned by the various models with character embeddings and word embeddings learned together with the parsing model on the treebank train set.
This representation is similar to those used by \citet{straka-strakova-2019-evaluating,ling-etal-2015-finding}.
In all of our experiments, the contextual embeddings are fine-tuned while training the parser.
%\footnote{For a more comprehensive review of the hyperparameters, see the appendix.}.
Unlike the recent CoNLL challenges settings, we assume gold tokenization, since the syntactic annotations we target provide a reference word-based segmentation. Using a predicted one could only add noise to our experiments.
Furthermore, for most European languages using a Latin script---including Old and Middle French---, word segmentation is acceptably approximated by simple typographic tokenization.

The remaining of this section presents our experimental results, sorted by nature of required data.
We report UPOS POS-tagging scores as well as unlabeled and labeled attachment scores for dependency parsing (respectively UAS and LAS), as given by the CoNLL-2018 scorer, computed on the development set of SRCMF to avoid overfitting the architecture and transfer learning procedure to the test set.
Results on the test set are provided only for the dev-best models to allow us to compare our results to the state of the art.

Due to the number of costly experiments,\footnote{See the Appendix for elements on the carbon footprint of our experiments.} the results are reported on single runs.
The results should therefore be interpreted  only with respects to the broad trends: small score differences between competing settings should be taken with care.

\subsection{Baselines}\label{sec|baselines}
\begin{table}[thb]
    \centering
    \tablefontsize
    \begin{tabular}{l*{3}{S[table-format=2.2]}}
        \toprule
        {\textbf{Embeddings}} & {\textbf{UPOS}} & {\textbf{UAS}} & {\textbf{LAS}} \\
        \midrule
        Vanilla               & 93.51           & 87.60          & 81.54          \\
        Random-base           & 93.17           & 86.97          & 80.71          \\
        finBERT               & 94.44           & 88.44          & 82.47          \\
        \bottomrule
    \end{tabular}
    \caption{Results on SRCMF dev — no additional data.}\label{tab|nodata}
\end{table}

We first compare a baseline where contextual embeddings are not used at all (Vanilla) with two settings using models with no preexisting knowledge of Old French: Random-base, a randomly initialized model using the same architecture and model size as RoBERTa-base \citep{liu-etal-2019-roberta} %and CamemBERT-base \citep{martin-etal-2020-camembert}, 
and finBERT \citep{virtanen-etal-2019-multilingual}, a contextual embedding model from Finnish, a Uralic language that is unrelated to Old French.
These baselines are meant to check that the gain in performances observed when using models with some (possibly indirect) knowledge of Old French are linked to this knowledge and not simply due to an increase in the number of trainable parameters (for the random baseline) or to a weight distribution induced by training on a language modeling task that would be universally good for all languages (for the finBERT baseline, which can thus be seen as a different kind of weight initialization).

\Cref{tab|nodata} shows the results obtained in these configurations, which show that using a model with random weights, even fine-tuned for these tasks, does not bring any improvement, and is in fact even worse than using no contextual embeddings at all.
In contrast, using a model that has been pretrained for language modeling---even for an unrelated language---brings some modest improvements.
This suggests that pretraining gives a structure to this kind of model that makes it suitable for fine-tuning on the downstream task, but the impact of this gain is clearly---and predictably---very limited compared to what can be expected for representations that have been trained on relevant linguistic data.

\subsection{With related contextual embeddings}\label{sec|related}

\begin{table}[thb]
    \centering
    \tablefontsize
    \begin{tabular}{l*{3}{S[table-format=2.2]}}
        \toprule
        {\textbf{Base model}} & {\textbf{UPOS}} & {\textbf{UAS}} & {\textbf{LAS}} \\
        \midrule
        FlauBERT              & 95.70           & 90.43          & 85.45          \\
        CamemBERT             & 95.86           & 91.15          & 86.31          \\
        mBERT                 & 96.06           & 91.52          & 86.83          \\
        \bottomrule
    \end{tabular}
    \caption{Results on SRCMF dev — monolingual models.}\label{tab|pre-trained}
\end{table}

When a low-resource language is close to a well-resourced one, it is possible to leverage models designed for the latter.
For Old French, contemporary French is an obvious candidate and two contextual embeddings models are available: FlauBERT \citep{le-etal-2020-flaubert} and CamemBERT \citep{martin-etal-2020-camembert}.
Furthermore, mBERT \citep{devlin-etal-2019-bert}, a model trained on a multilingual corpus which does not include Old French (possibly apart from some fragments in its contemporary French training data), has been shown to be suitable for many languages, and in particular for Indo-European and Romance languages \citep{straka-strakova-2019-evaluating,muller-etal-2021-unseen}.
We report in \cref{tab|pre-trained} the results obtained when using these language models directly, without additional fine-tuning involving Old French data.

As expected, these results show significant improvements over the baselines, confirming that using contextual embeddings for a related language works better than both randomly initialized embeddings and embeddings pretrained for an unrelated language---even after fine-tuning.
More surprisingly, the best results here are obtained with mBERT.
This could mean that mBERT benefits from having been pretrained for a wider range of languages, including in particular other Romance languages that share with Old French some features,% lost in contemporary French---
for instance null subjects.

\subsection{With raw linguistic data}\label{sec|withraw}

\begin{table*}[ht]
    \centering
    \tablefontsize
    \begin{tabular}{
        l@{\hskip 2ex}
        S[table-format=2.0]
        S[table-format=2.0]
        S[table-format=3.0]@{\hskip 2ex}
        *{3}{S[table-format=2.2]}
        }
        \toprule
        {\textbf{Name}} & {\textbf{Layers}} & {\textbf{Embeddings}} & {\textbf{Heads}} & {\textbf{UPOS}} & {\textbf{UAS}} & {\textbf{LAS}} \\
        \midrule
        BERTrade-tiny   & 2                 & 128                   & 2                & 94.03           & 88.66          & 82.79          \\
        % mini    & 4  & 256 & 4  & 92.60 & 86.48 & 80.12\\
        BERTrade-small  & 4                 & 512                   & 8                & 96.53           & 86.30          & 87.49          \\
        BERTrade-petit  & 12                & 256                   & 4                & 97.14           & 91.90          & 89.18          \\
        BERTrade-medium & 8                 & 512                   & 8                & 96.62           & 91.92          & 87.60          \\
        BERTrade-base   & 12                & 768                   & 12               & 96.74           & 92.37          & 88.42          \\
        % \midrule
        % FastText & {-} & {-} & 00.00 & 00.00 & 00.00\\
        \bottomrule
    \end{tabular}
    \caption{Results on SRCMF dev — Performances of different model sizes when training from scratch}\label{tab|fromscratch}
\end{table*}

\begin{table}[tbh]
    \centering
    \tablefontsize
    \begin{tabular}{l*{3}{S[table-format=2.2]}}
        \toprule
        {\textbf{Base model}} & {\textbf{UPOS}} & {\textbf{UAS}} & {\textbf{LAS}} \\
        \midrule
        BERTrade-petit        & 97.14           & 92.95          & 89.18          \\
        \midrule
        BERTrade-finBERT      & 96.28           & 92.12          & 87.92          \\
        BERTrade-mBERT        & 96.95           & 93.33          & 89.60          \\
        BERTrade-CamemBERT    & 97.16           & 93.75          & 90.06          \\
        BERTrade-FlauBERT     & 96.94           & 93.75          & 90.07          \\
        \bottomrule
    \end{tabular}
    \caption{Results on SRCMF dev — using raw data.}\label{tab|post-train}
\end{table}

We now try to take advantage of the raw Medieval French data described in section \ref{sec-data}.
To that end, we explore two strategies: training a model from scratch and refining existing models by \enquote{post-training} them---running a few more training epochs on the Medieval French raw data.

In the \enquote{from scratch} strategy we first train a BBPE sub-word tokenizer \citep{wang-cho-etal-2020-neural}
on our raw corpus, then train a RoBERTa \citep{liu-etal-2019-roberta} masked language model.
Taking inspiration from \citet{micheli-etal-2020-importance}, who worked in a setting close to ours: a small and noisy pre-training corpus used to create a model from scratch, we used a RoBERTa architecture.
%Since there are no clear silver bullets when it comes to Transformer-based contextual embedding models, the RoBERTa architecture is chosen mostly for comparison with 
As reported in \cref{tab|fromscratch}, we tested several parametrizations of the architecture
also inspired by \citet{turc-etal-2019-well}.
Out of these alternatives, the \enquote{BERTrade-petit} configuration was the
most successful and this is the one we keep for the following experiments.

For the \enquote{post-training} strategy, we continue the training of the pre-trained models used in \cref{sec|baselines,sec|related}, for \num{12} epochs on our raw corpus. We used the same RoBERTa masked language modeling task, using the same parameters as \citet{wang-etal-2020-extending} (but without vocabulary modifications), resulting in the BERTrade-X models, where X is the name of the base model.

The results of these experiments are reported in \Cref{tab|post-train}.
Comparing these to our results of \cref{sec|related} shows that training a model from scratch, even on such limited amounts of data, yields a better model than a simple task-specific fine-tuning of mBERT.
However, post-training mBERT yields even better results, and the best ones are obtained by post-training the models for contemporary French.

\begin{table}[thb]
    \centering
    \tablefontsize
    \begin{tabular}{l*{3}{S[table-format=2.2]}}
        \toprule
        {\textbf{Model}}                        & {\textbf{UPOS}} & {\textbf{UAS}} & {\textbf{LAS}} \\
        \midrule
        \citet{straka-strakova-2019-evaluating} & 96.26           & 91.83          & 86.75          \\
        \midrule
        mBERT                                   & 96.19           & 92.03          & 87.52          \\
        BERTrade-petit                          & 96.60           & 92.20          & 87.95          \\
        BERTrade-mBERT                          & 97.11           & 93.86          & 90.37          \\
        BERTrade-FlauBERT                       & 97.15           & 93.96          & 90.57          \\
        BERTrade-CamemBERT                      & 97.29           & 94.36          & 90.90          \\
        \bottomrule
    \end{tabular}
    \caption{Results on SRCMF test}\label{tab|sota}
\end{table}

\subsection{Putting it all together}
Finally, in \cref{tab|sota}, we compare the performances of our models on the test set of SRCMF with those obtained by  \citet{straka-strakova-2019-evaluating}, with similar methods. The difference between the models is that we fine-tune the word embeddings, while
\citet{straka-strakova-2019-evaluating} keep them frozen.

Our mBERT baseline, which is the closest to their configuration, shows that even without any additional data, task-specific fine-tuning already brings significant improvements, while our models refined using our raw corpus of Medieval French bring further improvements, leading to state-of-the-art results that are consistent with their results on the development set.

\section{Conclusion}

In this work, we have shown that building a monolingual contextual word embeddings model for Medieval French is possible even with limited and heterogeneous linguistic data and that it can bring significant performance gains in parsing and POS-tagging.
To that end, the best strategy seems to be post-training a contextual word embedding model for contemporary French on raw Medieval French documents.
We have not directly addressed the internal heterogeneity issue in both our pretraining and fine-tuning data, relying instead on the versatility of the representation models we considered to bypass it, but it seems a promising perspective for future work---for instance by using finer-grained post-training, concentrating on specific linguistic sub-periods or genres.

For historical languages in general, this suggests that language-specific fine-tuning is more efficient when applied to a model pre-trained for their contemporary counterpart than when applied to a multilingual model.
While this study is not currently easy to replicate for other languages due to the lack of annotated data for a suitable downstream task, it suggests that the considerable amount of work required to gather even a small amount of raw texts in the target language is a sound investment, given the significant improvements it can bring to contextual word representations.
Beyond historical languages, these findings could also help for processing minority dialectal variants and contact languages of well-resourced languages, and we leave for future work the exploration of these generalizations.

% \section*{Acknowledgments}
% 
% The acknowledgments should go immediately before the references. Do not number the acknowledgments section.
% Do not include this section when submitting your paper for review.

\section{Collecting the Data}
\label{subsec:collectdata}
%\todo{Complete AND, MCVF, and Chartes Douai}
The following data can be downloaded directly from their website:
\begin{itemize}
    \item Chartes de l'Aube: \\ \url{https://sites.google.com/site/achimstein/research/resources} \\
          Extract raw text from XML files: <body>, then <s>, then <word>.
    \item Geste: \\ \url{https://github.com/Jean-Baptiste-Camps/Geste} \\
          Raw text is available under /txt/norm/.
    \item OpenMedFr: \\ \url{https://github.com/OpenMedFr/texts} \\
          Remove the header of each file (until \textit{*** START}), its last line (\textit{*** END}), paragraph breaks (\textit{\#|}) and folios or pages numbers.
\end{itemize}

Special permissions are required to access and use these sources:
\begin{itemize}
    \item AND: \\ \url{https://anglo-norman.net/project-members}
    \item BFM: \\ \url{http://bfm.ens-lyon.fr/spip.php?article19} \\
          Raw text is available.
    \item Chartes Douai: \\
          \url{https://www.rose.uzh.ch/docling}
    \item MCVF: \url{http://www.voies.uottawa.ca}
    \item NCA: \\ \url{https://sites.google.com/site/achimstein/research/resources} \\
          Extract raw text from the XML files: <body> then <txm:form>.
\end{itemize}

% \subsection{Oïl languages}
% \begin{figure}[h]
%     \centering
%     \includegraphics[scale=0.3]{map-fr.png}
%     \caption{Map of Medieval Languages}
%     \label{fig:my_label}
% \end{figure}

\section{Details on the Models}

\subsection{Models Trained From Scratch}

These are trained for \num{32} epochs in a masked language modeling task using the same parameters as RoBERTa \citep{liu-etal-2019-roberta} but a smaller batch size of \num{256} samples\footnote{Preliminary experiments with larger batch sizes showed no significant improvement to compensate for the heavier computational load.}, which amounts to a magnitude of \num{e5} steps.
We also use a smaller vocabulary size (\num{8192}) than other works, in line with the observations of \citet{ding-etal-2019-call} that learning large vocabularies on small corpora defeats the purpose of sub-word tokenization.
Using a larger vocabulary size of \num{5e4} (like FlauBERT) also did not seem to bring any improvements in our preliminary experiments and made pre-training more expensive.

\subsection{Post-training}

The pretrained models we used in the post-training settings are those available in the 4.2.0 version of Huggingface Transformers \citep{wolf-etal-2020-transformers} and the exact handles are:

\begin{description}
    \item[mBERT] \href{https://huggingface.co/bert-base-multilingual-cased}{bert-base-multilingual-cased}
    \item[flauBERT] \href{https://huggingface.co/flaubert/flaubert_base_cased}{flaubert/flaubert\_base\_cased}
    \item[camemBERT] \href{https://huggingface.co/camembert-base}{camembert-base}
    \item[finBERT] \href{https://huggingface.co/TurkuNLP/bert-base-finnish-cased-v1}{TurkuNLP/bert-base-finnish-cased-v1}
\end{description}

The post-trained models are those with MLM heads, which we did not reset before post-training, so the post-training phase can be seen as a language transfer task for masked language modeling out of which we extract a contextual word embeddings model.

\section{Carbon Footprint}\label{carbon-footprint}

\begin{table}[t]
    \centering\small
    \scalebox{0.89}{
        \begin{tabular}{@{}lrrrrr@{}}
            \toprule
            \textbf{Model}  & {\textbf{Power (\unit{\watt})}} & {\textbf{\# Models}} & {\textbf{Duration (\unit{\hour})}} & {\textbf{Consumption (\unit{\kWh})}} & {\textbf{CO\textsubscript{2}e (\unit{\kilo\gram})}} \\
            \midrule
            Pre-train       & 10756                           & 11                   & 6                                  & 11216.36                             & 358.92                                              \\
            Post-train      & 1520                            & 4                    & 20                                 & 192.13                               & 6.15                                                \\
            \midrule
            Total emissions &                                 &                      &                                    &                                      & 365.07                                              \\
            \bottomrule
        \end{tabular}
    }
    \caption{Average power draw, number of models trained, training times in hours, mean power consumption including power usage effectiveness (PUE), and CO\textsubscript{2} emissions; for each setting.}
    \label{tab:carbon-bertrade}
\end{table}

In light of recent concerns about the power consumption and carbon footprint of deep learning models \citep{schwartz-etal-2020-green, bender-etal-2021-on} we report the power consumption and carbon footprint of our main experiments following the approach of \citet{strubell-etal-2019-energy}. Two different configurations were used in our experiments, one for pre-training models from scratch (Pre-train) and another one for continuing the training of existing models (Post-train).

\paragraph{Pre-train:} We use a cluster of 4 machines each one having \num{8} GPU Nvidia Tesla V100 SXM2 \qty{32}{\gibi\byte}, \qty{384}{\gibi\byte} of RAM, and two Intel Xeon Gold 6226 processors. One Nvidia Tesla V100 card is rated at around \qty{300}{\watt},\footnote{\href{https://www.nvidia.com/en-us/data-center/v100/}{ Nvidia Tesla V100 specification}} while the Xeon Gold 6226 processor is rated at \qty{125}{\watt},\footnote{\href{https://ark.intel.com/content/www/us/en/ark/products/193957/intel-xeon-gold-6226-processor-19-25m-cache-2-70-ghz.html}{Intel Xeon Gold 6226 specification}}. For the DRAM we can use the work of \citet{desrochers-etal-2016-a} to estimate the total power draw of \qty{384}{\gibi\byte} of RAM at around \qty{39}{\watt}. The total power draw of this setting adds up to around \qty{10756}{\watt}. We train \num{11} different models in this configuration.

\paragraph{Post-train:} We use a single machine having \num{4} GPU Nvidia Tesla V100 SXM2 \qty{32}{\gibi\byte}, \qty{192}{\gibi\byte} of RAM and two Intel Xeon Gold 6248 processors. The Xeon Gold 6248 processor is rated at 150 W,\footnote{\href{https://ark.intel.com/content/www/us/en/ark/products/192446/intel-xeon-gold-6248-processor-27-5m-cache-2-50-ghz.html}{Intel Xeon Gold 6248 specification}}, and the DRAM total power draw can be estimated at around \qty{20}{\watt}. The total power draw of this setting adds up to around \qty{1520}{\watt}. We train \num{4} different models in this configuration.

Having this information, we can now use the formula proposed by \citet{strubell-etal-2019-energy} in order to compute the total power required for each setting:

\begin{equation*}
    p_t = \frac{1.58t(cp_{c} + p_r + gp_g)}{1000}
\end{equation*}

Where $c$ and $g$ are the number of CPUs and GPUs respectively, $p_c$ is the average power draw (in \unit{\watt}) from all CPU sockets, $p_r$ the average power draw from all DRAM sockets, and $p_g$ the average power draw of a single GPU. We estimate the total power consumption by adding GPU, CPU and DRAM consumption, and then multiplying by the \emph{Power Usage Effectiveness} (PUE), which accounts for the additional energy required to support the compute infrastructure. We use a PUE coefficient of \num{1.58}, the 2018 global average for data centers \citep{strubell-etal-2019-energy}. In table \ref{tab:carbon-bertrade} we report the training times in hours, as well as the total power draw (in Watts) of the system used to train the models. We use this information to compute the total power consumption of each setting, also reported in table \ref{tab:carbon-bertrade}.

We can further estimate the CO\textsubscript{2} emissions in kilograms of each single model by multiplying the total power consumption by the average CO\textsubscript{2} emissions per \unit{\kWh} in our region which were around \qty{32}{\gram\per\kWh} in January 2021,\footnote{\href{https://www.rte-france.com/eco2mix/les-emissions-de-co2-par-kwh-produit-en-france}{Rte - éCO\textsubscript{2}mix}.} when the models were trained. Thus the total CO\textsubscript{2} emissions in kg for one single model can be computed as:
\begin{equation*}
    \text{CO}_{2}\text{e} = 0.032 p_t
\end{equation*}

All emissions are also reported in table \ref{tab:carbon-bertrade}.

%%%%%%%%%%%%%%%%%%%%%%%%%%%%%%%%%%%%%%%%%%%%%%%%%%%%%%%%%%%%%%%%%%%%%%%%
\chapter{D'AlemBERT}\label{chap:dalembert}
%%%%%%%%%%%%%%%%%%%%%%%%%%%%%%%%%%%%%%%%%%%%%%%%%%%%%%%%%%%%%%%%%%%%%%%%

\begin{center}
    \begin{minipage}{0.66\textwidth}
        \begin{small}
            In which we present part of the work of \citet{gabay-etal-2022-from} who pre-train and develop RoBERTa-based \citep{liu-etal-2019-roberta} models for Early Modern French from scratch with the \freemmax corpus, and subsequently evaluate it on POS tagging and named entity recognition on the \freemlpm and the \freemner corpora respectively.
        \end{small}
    \end{minipage}
    \vspace{0.5cm}
\end{center}

After having successfully pre-trained and evaluated a Transformer-based language model for Medieval French, we wanted to develop such a model for Early Modern French, the state of language corresponding to that of the \emph{Dictionnaire Universel} in its 1701 edition \citep{furetiere-1701-dictionnaire}, the main text of study of the ANR BASNUM (ANR-18-CE38-0003) that funded this Ph.D. thesis. Thus, in this chapter we develop \dalembert a neural language model for Early Modern French, and we evaluate it in POS tagging and NER on the \freemlpm and the \freemner corpora respectively. Contrary to the approach used in the previous chapter, we only pre-train \dalembert from scratch, and we do not post-train any of the Contemporary French models. We decided to do this mainly because that our \freemmax, our pre-training dataset is around 1.2 GB in size which is around 20 times the size of the corpus used in the BERTrade experiments for Medieval French, we also thought that 1.2 GB would be enough to properly train a RoBERTa-based architecture in light of the results obtained for \camembert in subsection \ref{subsec:sizeimpact}.

\section{D'AlemBERT: a neural language model for Early Modern French}\label{sec:dAlemBERT}
In this section, we describe the pre-training data, architecture, training objective and optimization setup we use for \dalembert, our new neural language model for Early Modern French.

\subsection{Pre-processing}
Similar to \roberta \citep{liu-etal-2019-roberta} we segment the input text data into subword units using Byte-Pair encoding (BPE) \citep{sennrich-etal-2016-neural} in the implementation proposed by \citep{radford-etal-2019-language} that uses bytes instead of unicode characters as the base subword units. The BPE encoding does not require pre-tokenization (at the word or token level), thus removing the need to develop a specific tokenizer for Early Modern French. We use a vocabulary size of 32,768 subword tokens. These subwords are learned on the entire \freemmax dataset.

\subsection{Language Modelling}

\paragraph{Transformer}
\dalembert uses the exact same architecture as \roberta, which is a multi-layer bidirectional Transformer \citep{vaswani-etal-2017-attention}. \dalembert uses the original \emph{base} architecture of \roberta (12 layers, 768 hidden dimensions, 12 attention heads, 110M parameters).

\paragraph{Pre-training Objective}
We train our model on the Masked Language Modelling (MLM) task as proposed by RoBERTa's authors \citep{liu-etal-2019-roberta}: given an input text sequence composed of $N$ tokens $x_1, ..., x_N$, we select 15\% of tokens for possible replacement. Among those selected tokens, 80\% are replaced with the special \texttt{<MASK>} token, 10\% are left unchanged and 10\% are replaced by a random token. The model is then trained to predict the masked tokens using cross-entropy loss.

Again, following the \roberta approach, we dynamically mask tokens instead of fixing them statically for the whole dataset during preprocessing. We also choose not to use the next sentence prediction (NSP) task originally used in \bert \citep{devlin-etal-2019-bert}, as it has been shown that it does not improve downstream task performance \citep{conneau-lample-2019-cross,liu-etal-2019-roberta}.

\paragraph{Optimization}
Optimization for our model in the exact same way as \citep{liu-etal-2019-roberta} using Adam \citep{kingma-ba-2015-adam} ($\beta_1 = 0.9$, $\beta_2 = 0.98$) for 31k steps with large batch sizes of 8,192 sequences, each sequence containing at most 512 tokens.

\paragraph{Pre-training}
We use the \roberta implementation in the Zelda Rose library,\footnote{\url{https://github.com/LoicGrobol/zeldarose}} and again, in the same way as \citet{liu-etal-2019-roberta} our learning rate is warmed up for 10k steps up to a peak value of $0.0003$ instead of the original $0.0001$ used by the original implementation of \roberta \citep{liu-etal-2019-roberta}, as our model diverged with the $0.0001$ value. Furthermore, we hypothesize that this is either due to the smaller size of \freemmax (compared to the corpora used for \roberta or \camembert) or to our large batch size. We train our model for 31k steps, which amounts to 41 epochs. The total pre-training times, the details of the infrastructure we used and even the carbon emissions of our model are reported in Appendix~\ref{carbon-footprint-dalembert}.

\section{Evaluation and Discussion}

\subsection{Part-Of-Speech Tagging}

In order to evaluate our \dalembert model, we first fine-tune it for POS tagging on the \freemlpm corpus. We use the \texttt{flair} framework\footnote{\url{https://github.com/flairNLP/flair}} for sequence tagging \citep{akbik-etal-2019-flair}. To fine-tune \dalembert for POS we follow the same approach as \citet{schweter-akbik-2020-flert} with some modifications: we append a linear layer of size 256 that takes as input the last hidden representation of the \texttt{<s>} special token and the mean of the last hidden representation of the subword units of each token (token as defined for \freemlpm), that is, we use a \emph{``mean''} subword pooling strategy. We fine-tune \dalembert with a learning rate of 0.000005 for a total of 10 epochs. We also fine-tune \camembert using the exact same hyperparameters as that we use for \dalembert.

\freemlpm provides a standard split (train, dev, test), however it also proposes an evaluation on a \emph{out-of-domain} subcorpus that is not contained in the standard split and that is separated by century (from the 16\textsuperscript{th} to the 20\textsuperscript{th} century) and that also contains both the \emph{Normalized} and \emph{Original} versions of the texts for the 16\textsuperscript{th}, 17\textsuperscript{th} and 18\textsuperscript{th} centuries. The idea of this out-of-domain evaluation corpus is to have a fine-grained evaluation of the models to better assess their performance in all the different types of text that one might encounter when working with Early Modern French data.

\begin{table}[ht]
    \centering\small
    \resizebox{\linewidth}{!}{
        \begin{tabular}{lrrrrrr}
            \toprule
            \multicolumn{7}{c}{\textsc{Original}}                                                      \\
            \midrule
            Model        & 16             & 17             & 18             & 19 & 20 & Avg            \\
            \midrule
            \multicolumn{7}{l}{\hspace*{6mm}\emph{Drama}}                                              \\
            \pieextended & \emph{90.34}   & \emph{94.47}   & \emph{94.64}   & -  & -  & \emph{93.15}   \\
            \camembert   & 87.06          & 89.01          & 90.92          & -  & -  & 89.00          \\
            \dalembert   & \textbf{94.17} & \textbf{96.59} & \textbf{96.28} & -  & -  & \textbf{95.68} \\
            \multicolumn{7}{l}{\hspace*{6mm}\emph{Varia}}                                              \\
            \pieextended & \emph{89.85}   & \emph{93.44}   & \emph{95.98}   & -  & -  & \emph{93.09}   \\
            \camembert   & 86.90          & 88.85          & 92.85          & -  & -  & 89.53          \\
            \dalembert   & \textbf{93.86} & \textbf{95.73} & \textbf{96.95} & -  & -  & \textbf{95.51} \\
            \multicolumn{7}{l}{\hspace*{6mm}\emph{Both}}                                               \\
            \pieextended & \emph{90.08}   & \emph{93.95}   & \emph{95.33}   & -  & -  & \emph{ 93.12}  \\
            \camembert   & 86.98          & 88.93          & 91.89          & -  & -  & 89.27          \\
            \dalembert   & \textbf{94.02} & \textbf{96.16} & \textbf{96.62} & -  & -  & \textbf{95.60} \\
            \bottomrule
        \end{tabular}
        \begin{tabular}{lrrrrrr}
            \toprule
            \multicolumn{7}{c}{\textsc{Normalized or Contemporary}}                                                            \\
            \midrule
            Model        & 16             & 17             & 18             & 19             & 20             & Avg            \\
            \midrule
            \multicolumn{7}{l}{\hspace*{6mm}\emph{Drama}}                                                                      \\
            \pieextended & \emph{93.69}   & \emph{95.75}   & \emph{95.61}   & \emph{95.03}   & \emph{93.71}   & \emph{94.76}   \\
            \camembert   & 90.18          & 91.51          & 91.37          & 91.13          & 91.42          & 91.12          \\
            \dalembert   & \textbf{96.25} & \textbf{96.97} & \textbf{96.80} & \textbf{96.25} & \textbf{95.00} & \textbf{96.25} \\
            \multicolumn{7}{l}{\hspace*{6mm}\emph{Varia}}                                                                      \\
            \pieextended & \emph{92.52}   & \emph{94.81}   & \emph{95.98}   & \emph{92.24}   & \emph{94.03}   & \emph{93.94}   \\
            \camembert   & 89.79          & 90.69          & 93.06          & 90.54          & 89.78          & 93.94          \\
            \dalembert   & \textbf{94.52} & \textbf{96.64} & \textbf{96.88} & \textbf{94.90} & \textbf{95.30} & \textbf{95.65} \\
            \multicolumn{7}{l}{\hspace*{6mm}\emph{Both}}                                                                       \\
            \pieextended & \emph{93.08}   & \emph{95.28}   & \emph{95.80}   & \emph{93.65}   & \emph{93.87}   & \emph{94.35}   \\
            \camembert   & 89.99          & 91.10          & 92.22          & 90.84          & 90.60          & 92.53          \\
            \dalembert   & \textbf{95.39} & \textbf{96.81} & \textbf{96.84} & \textbf{95.58} & \textbf{95.15} & \textbf{95.95} \\
            \bottomrule
        \end{tabular}
    }
    \caption{Comparison between \dalembert, \camembert and \pieextended performance on the test set, out-of-domian data of \freemlpm.}
    \label{tab:POS}
\end{table}

Following the approach of \citet{clerice-2020-pie}, we report the scores obtained on the out-of-domain testing dataset of \freemlpm in Table~\ref{tab:POS}. We use the scores previously reported by \citet{clerice-2020-pie} using \emph{Pie Extended}, a stacked BiLSTM-CRF model, as our first baseline as well as the fine-tuned \camembert that serves as a second baseline as well as a rough estimation of how much knowledge can \dalembert transfer from the \freemmax corpus into this task.

We can see that \dalembert consistently outperforms \pieextended and \camembert in both the normalized and original versions of our out-of-domain testing data and for all different periods by a considerable margin. Furthermore, we can also see that on average the difference in score between \dalembert and \pieextended is greater for the original split than the normalized one. This suggests that \dalembert can generalize more effectively to non-normalized data than the more traditional architecture used by \pieextended. Moreover, we can also see that the difference in scores is also greater for the 16\textsuperscript{th}\,c. and 17\textsuperscript{th}\,c. data. This is interesting, especially for the 16\textsuperscript{th}\,c, because, as we can see in Figure~\ref{fig:FreEMmax_desc}, this is the least represented period in the \freemmax corpus. This result actually suggests that \dalembert might be able to do effective transfer learning from the 18\textsuperscript{th}\,c., 19\textsuperscript{th}\,c. and 20\textsuperscript{th}\,c. data to the 16\textsuperscript{th}\,c. and 17\textsuperscript{th}\,c. data.

As for \camembert, we can see that it consistently scores lower than both \dalembert and \pieextended. Moreover, we can see that it struggles particularly with the non-normalized data of the 16\textsuperscript{th}\,c., 17\textsuperscript{th}\,c. and 18\textsuperscript{th}\,c.. This results clearly shows that \camembert cannot easily generalize to these earlier states of languages, or at least not with the quantity of data found in the training set of \freemlpm. These results also show the impressive capacity of \dalembert of quickly generalizing to diverse set of states of language, as well as its capacity to transfer knowledge from the \freemmax corpus into this task. The obtained results are also a testament to the importance of the pre-training data, specially taking in account that the pre-training set of \camembert is more than 100 times bigger than that of \dalembert.

\subsection{Named Entity Recognition}

\begin{table}[ht]
    \centering\small
    \begin{tabular}{lrrr}
        \toprule
        Model & Precision & Recall & F1-Score \\
        \midrule
        LSTM-CRF  &   0.8640  &  0.8533  &  0.8586\\
        \camembert & \emph{0.9303}  &  \emph{0.9309}  &  \emph{0.9306} \\
        \dalembert & \textbf{0.9329}  &  \textbf{0.9323}  &  \textbf{0.9326}\\
        \bottomrule
    \end{tabular}
    \caption{Comparison between \dalembert, \camembert and an LSTM-CRF-based model performance on the test set of \freemner.}
    \label{tab:dalembert-ner}
\end{table}

Now we fine-tune \dalembert on NER with the \freemner corpus. We use once again the \texttt{flair} framework\footnote{\url{https://github.com/flairNLP/flair}} for sequence tagging \citep{akbik-etal-2019-flair} and we follow the same approach as \citet{schweter-akbik-2020-flert} with the exact same modifications as in the previous subsection. We also fine-tune \camembert using the exact same hyperparameters as that we use for \dalembert. For a baseline we use the BiLSTM-CRF implementation provided by the \texttt{flair} library, and we couple it with character embeddings as well as the Common Crawl-based FastText embeddings \citep{grave-etal-2018-learning} originally trained by Facebook.

In contrast to our POS tagging experiments, here we see \dalembert getting marginally better scores than \dalembert, we believe that this is due to the striking size of \freemner which has more than 5 million annotated tokens, that is, we believe that in this case \camembert has enough training data in order to properly fine-tune to this task in Early Modern French and in particular to potentially overcome the poor representations given by the SentencePiece \citep{kudo-richardson-2018-sentencepiece} trained on Contemporary French for the out-of-vocabulary words found in the Early Modern French data. \footnote{We observe that SentencePiece tends to split OOV words by characters which might not be ideal for sequence-tagging tasks, specially for NER.} We believe that to a certain extent, given the size of \freemner, \camembert might be \enquote{\emph{forgetting}} its pre-training contemporary data and \enquote{\emph{re-learning}} the Early Modern French data in \freemner.

In any case, the results obtained here by \dalembert are on par with the state-of-the-art NER models for Contemporary English \citep{wang-etal-2021-automated} while using a much simpler architecture. The results obtained by both Transformer-based models largely outperform those obtained by our LSTM-CRF based baseline, which shows how well the Transformer-based models respond to large quantities of annotated data. We report the results by entity type on the appendix section \ref{dalembert-entity-results}.

\section{Conclusion}

In this chapter we showed that it is possible to successfully train a Transformer-based language model for Early Modern French from scratch with even less data than originally shown in previous works \citep{martin-etal-2020-camembert}. Moreover, with our POS tagging evaluation we were able to observe some form of transfer learning and generalization across multiple states of the language corresponding to different periods of time, while in our experiments in named entity recognition we observed the type of performance one can get when given a big enough annotated corpus, even when the models are not particularly fine-tuned to the specific period of time of the annotated data.

We believe that \dalembert will be of use not only to the BASNUM project, but also to all digital humanists and linguists interested in Early Modern French. For our future work, we hope that we will be able to study the application of our \dalembert model to other NLP tasks such as text normalization or even document structuring, where we hope to more extensively study the transfer learning capabilities of our approach.


\part{Conclusions and Perspectives}
%%%%%%%%%%%%%%%%%%%%%%%%%%%%%%%%%%%%%%%%%%%%%%%%%%%%%%%%%%%%%%%%%%%%%%%%
\chapter{Conclusions and Perspectives}\label{chap:conclusions}
%%%%%%%%%%%%%%%%%%%%%%%%%%%%%%%%%%%%%%%%%%%%%%%%%%%%%%%%%%%%%%%%%%%%%%%%

During this thesis, we have developed multiple resources for several states of language in French, and even for a wide variety of languages in the case of OSCAR. We have chosen to focus on the development of data for the pre-training of language models rather than on the architectures themselves. This approach proved to be extremely effective as we were able to establish a new state of the art for a wide range of tasks in natural language processing for several states of language in French: Medieval French, Modern French and Contemporary French.

We consider that we have reached and even exceeded all the initial objectives set by the BASNUM project for this thesis, by producing models and automatic annotation systems that will make it possible to enrich not only the \emph{Dictionnaire Universel} of Basnage but also any other type of document in French from any historical era. We hope that the resources we have produced will be of great use to researchers in natural language processing and digital humanities.


\appendix
\part{Appendices}
%%%%%%%%%%%%%%%%%%%%%%%%%%%%%%%%%%%%%%%%%%%%%%%%%%%%%%%%%%%%%%%%%%%%%%%%
\chapter{Goclassy: an Asynchronous Language Classification Pipeline for Common Crawl}
%%%%%%%%%%%%%%%%%%%%%%%%%%%%%%%%%%%%%%%%%%%%%%%%%%%%%%%%%%%%%%%%%%%%%%%%


\begin{table}[ht]
    \centering\tiny
    \resizebox{\linewidth}{!}{
        \begin{tabular}{@{}lrrrrclrrrr@{}}\toprule
            \multirow{2}{*}{Language} & \multicolumn{2}{c}{Size} & \multicolumn{2}{c}{Words} & \phantom{a}              & \multirow{2}{*}{Language} & \multicolumn{2}{c}{Size} & \multicolumn{2}{c}{Words}                                                                                                               \\
            \cmidrule(l{2pt}r{2pt}){2-3} \cmidrule(l{2pt}r{2pt}){4-5} \cmidrule(l{2pt}r{2pt}){8-9} \cmidrule(l{2pt}r{2pt}){10-11}
                                      & \multicolumn{1}{c}{Orig} & \multicolumn{1}{c}{Dedup} & \multicolumn{1}{c}{Orig} & \multicolumn{1}{c}{Dedup} & \phantom{a}              &                           & \multicolumn{1}{c}{Orig} & \multicolumn{1}{c}{Dedup} & \multicolumn{1}{c}{Orig} & \multicolumn{1}{c}{Dedup} \\\midrule
            Afrikaans                 & 241M                     & 163M                      & 43,482,801               & 29,533,437                &                          & Lower Sorbian             & 13K                      & 7.1K                      & 1,787                    & 966                       \\
            Albanian                  & 2.3G                     & 1.2G                      & 374,196,110              & 186,856,699               &                          & Luxembourgish             & 29M                      & 21M                       & 4,403,577                & 3,087,650                 \\
            Amharic                   & 360M                     & 206M                      & 28,301,601               & 16,086,628                &                          & Macedonian                & 2.1G                     & 1.2G                      & 189,289,873              & 102,849,595               \\
            Arabic                    & 82G                      & 32G                       & 8,117,162,828            & 3,171,221,354             &                          & Maithili                  & 317K                     & 11K                       & 69,161                   & 874                       \\
            Aragonese                 & 1.3M                     & 801K                      & 52,896                   & 45,669                    &                          & Malagasy                  & 21M                      & 13M                       & 3,068,360                & 1,872,044                 \\
            Armenian                  & 3.7G                     & 1.5G                      & 273,919,388              & 110,196,043               &                          & Malay                     & 111M                     & 42M                       & 16,696,882               & 6,045,753                 \\
            Assamese                  & 113M                     & 71M                       & 6,956,663                & 4,366,570                 &                          & Malayalam                 & 4.9G                     & 2.5G                      & 189,534,472              & 95,892,551                \\
            Asturian                  & 2.4M                     & 2.0M                      & 381,005                  & 325,237                   &                          & Maltese                   & 24M                      & 17M                       & 2,995,654                & 2,163,358                 \\
            Avaric                    & 409K                     & 324K                      & 24,720                   & 19,478                    &                          & Marathi                   & 2.7G                     & 1.4G                      & 162,609,404              & 82,130,803                \\
            Azerbaijani               & 2.8G                     & 1.5G                      & 322,641,710              & 167,742,296               &                          & Mazanderani               & 691K                     & 602K                      & 73,870                   & 64,481                    \\
            Bashkir                   & 128M                     & 90M                       & 9,796,764                & 6,922,589                 &                          & Minangkabau               & 608K                     & 310K                      & 5,682                    & 4,825                     \\
            Basque                    & 848M                     & 342M                      & 120,456,652              & 45,359,710                &                          & Mingrelian                & 5.8M                     & 4.4M                      & 299,098                  & 228,629                   \\
            Bavarian                  & 503                      & 503                       & 399                      & 399                       &                          & Mirandese                 & 1.2K                     & 1.1K                      & 171                      & 152                       \\
            Belarusian                & 1.8G                     & 1.1G                      & 144,579,630              & 83,499,037                &                          & Modern Greek              & 62G                      & 27G                       & 5,479,180,137            & 2,412,419,435             \\
            Bengali                   & 11G                      & 5.8G                      & 623,575,733              & 363,766,143               &                          & Mongolian                 & 2.2G                     & 838M                      & 181,307,167              & 68,362,013                \\
            Bihari                    & 110K                     & 34K                       & 8,848                    & 2,875                     &                          & Nahuatl languages         & 12K                      & 11K                       & 1,234                    & 1,193                     \\
            Bishnupriya               & 4.1M                     & 1.7M                      & 198,286                  & 96,940                    &                          & Neapolitan                & 17K                      & 13K                       & 5,282                    & 4,147                     \\
            Bosnian                   & 447K                     & 116K                      & 106,448                  & 20,485                    &                          & Nepali                    & 1.8G                     & 1.2G                      & 107,448,208              & 71,628,317                \\
            Breton                    & 29M                      & 16M                       & 5,013,241                & 2,890,384                 &                          & Newari                    & 5.5M                     & 4.1M                      & 564,697                  & 288,995                   \\
            Bulgarian                 & 32G                      & 14G                       & 2,947,648,106            & 1,268,114,977             &                          & Northern Frisian          & 4.4K                     & 4.4K                      & 1,516                    & 1,516                     \\
            Burmese                   & 1.9G                     & 1.1G                      & 56,111,184               & 30,102,173                &                          & Northern Luri             & 76K                      & 63K                       & 8,022                    & 6,740                     \\
            Catalan                   & 8.0G                     & 4.3G                      & 1,360,212,450            & 729,333,440               &                          & Norwegian                 & 8.0G                     & 4.7G                      & 1,344,326,388            & 804,894,377               \\
            Cebuano                   & 39M                      & 24M                       & 6,603,567                & 3,675,024                 &                          & Norwegian Nynorsk         & 85M                      & 54M                       & 14,764,980               & 9,435,139                 \\
            Central Bikol             & 885                      & 885                       & 312                      & 312                       &                          & Occitan                   & 5.8M                     & 3.7M                      & 750,301                  & 512,678                   \\
            Central Khmer             & 1.1G                     & 581M                      & 20,690,610               & 10,082,245                &                          & Oriya                     & 248M                     & 188M                      & 14,938,567               & 11,321,740                \\
            Central Kurdish           & 487M                     & 226M                      & 48,478,334               & 18,726,721                &                          & Ossetian                  & 13M                      & 11M                       & 1,031,268                & 878,765                   \\
            Chavacano                 & 520                      & 520                       & 130                      & 130                       &                          & Pampanga                  & 760                      & 304                       & 130                      & 52                        \\
            Chechen                   & 8.3M                     & 6.7M                      & 711,051                  & 568,146                   &                          & Panjabi                   & 763M                     & 460M                      & 61,847,806               & 37,555,835                \\
            Chinese                   & 508G                     & 249G                      & 14,986,424,850           & 6,350,215,113             &                          & Persian                   & 79G                      & 38G                       & 9,096,554,121            & 4,363,505,319             \\
            Chuvash                   & 39M                      & 26M                       & 3,041,614                & 2,054,810                 &                          & Piemontese                & 2.1M                     & 1.9M                      & 362,013                  & 337,246                   \\
            Cornish                   & 44K                      & 14K                       & 8,329                    & 2,704                     &                          & Polish                    & 109G                     & 47G                       & 15,277,255,137           & 6,708,709,674             \\
            Croatian                  & 226M                     & 110M                      & 34,232,765               & 16,727,640                &                          & Portuguese                & 124G                     & 64G                       & 20,641,903,898           & 10,751,156,918            \\
            Czech                     & 53G                      & 24G                       & 7,715,977,441            & 3,540,997,509             &                          & Pushto                    & 361M                     & 242M                      & 46,559,441               & 31,347,348                \\
            Danish                    & 16G                      & 9.5G                      & 2,637,463,889            & 1,620,091,317             &                          & Quechua                   & 78K                      & 67K                       & 10,186                   & 8,691                     \\
            Dhivehi                   & 126M                     & 79M                       & 7,559,472                & 4,726,660                 &                          & Romanian                  & 25G                      & 11G                       & 3,984,317,058            & 1,741,794,069             \\
            Dimli                     & 146                      & 146                       & 19                       & 19                        &                          & Romansh                   & 7.4K                     & 6.5K                      & 1,093                    & 960                       \\
            Dutch                     & 78G                      & 39G                       & 13,020,136,373           & 6,598,786,137             &                          & Russia Buriat             & 13K                      & 11K                       & 963                      & 809                       \\
            Eastern Mari              & 7.2M                     & 6.0M                      & 565,992                  & 469,297                   &                          & Russian                   & 1.2T                     & 568G                      & 92,522,407,837           & 46,692,691,520            \\
            Egyptian Arabic           & 66M                      & 33M                       & 7,305,151                & 3,659,419                 &                          & Sanskrit                  & 93M                      & 37M                       & 4,331,569                & 1,713,930                 \\
            Emilian-Romagnol          & 25K                      & 24K                       & 6,376                    & 6,121                     &                          & Scottish Gaelic           & 1.9M                     & 1.3M                      & 310,689                  & 207,110                   \\
            English                   & 2.3T                     & 1.2T                      & 418,187,793,408          & 215,841,256,971           &                          & Serbian                   & 3.9G                     & 2.2G                      & 364,395,411              & 207,561,168               \\
            Erzya                     & 1.4K                     & 1.2K                      & 90                       & 78                        &                          & Serbo-Croatian            & 25M                      & 5.8M                      & 5,292,184                & 1,040,573                 \\
            Esperanto                 & 299M                     & 228M                      & 48,486,161               & 37,324,446                &                          & Sicilian                  & 3.3K                     & 2.8K                      & 554                      & 468                       \\
            Estonian                  & 4.8G                     & 2.3G                      & 643,163,730              & 309,931,463               &                          & Sindhi                    & 347M                     & 263M                      & 43,530,158               & 33,028,015                \\
            Finnish                   & 27G                      & 13G                       & 3,196,666,419            & 1,597,855,468             &                          & Sinhala                   & 1.4G                     & 802M                      & 93,053,465               & 50,864,857                \\
            French                    & 282G                     & 138G                      & 46,896,036,417           & 23,206,776,649            &                          & Slovak                    & 9.1G                     & 4.5G                      & 1,322,247,763            & 656,346,179               \\
            Galician                  & 620M                     & 384M                      & 102,011,291              & 63,600,602                &                          & Slovenian                 & 2.5G                     & 1.3G                      & 387,399,700              & 193,926,684               \\
            Georgian                  & 3.6G                     & 1.9G                      & 171,950,621              & 91,569,739                &                          & Somali                    & 61K                      & 16K                       & 1,202                    & 472                       \\
            German                    & 308G                     & 145G                      & 44,878,908,446           & 21,529,164,172            &                          & South Azerbaijani         & 27M                      & 19M                       & 2,175,054                & 1,528,709                 \\
            Goan Konkani              & 2.2M                     & 1.8M                      & 124,277                  & 102,306                   &                          & Spanish                   & 278G                     & 149G                      & 47,545,122,279           & 25,928,290,729            \\
            Guarani                   & 36K                      & 24K                       & 7,382                    & 4,680                     &                          & Sundanese                 & 211K                     & 141K                      & 30,321                   & 20,278                    \\
            Gujarati                  & 1.1G                     & 722M                      & 72,045,701               & 50,023,432                &                          & Swahili                   & 13M                      & 8.1M                      & 2,211,927                & 1,376,963                 \\
            Haitian                   & 3.9K                     & 3.3K                      & 1,014                    & 832                       &                          & Swedish                   & 44G                      & 25G                       & 7,155,994,312            & 4,106,120,608             \\
            Hebrew                    & 20G                      & 9.8G                      & 2,067,753,528            & 1,032,018,056             &                          & Tagalog                   & 573M                     & 407M                      & 98,949,299               & 70,121,601                \\
            Hindi                     & 17G                      & 8.9G                      & 1,372,234,782            & 745,774,934               &                          & Tajik                     & 379M                     & 249M                      & 31,758,142               & 21,029,893                \\
            Hungarian                 & 40G                      & 18G                       & 5,163,936,345            & 2,339,127,555             &                          & Tamil                     & 9.3G                     & 5.1G                      & 420,537,132              & 226,013,330               \\
            Icelandic                 & 1.5G                     & 846M                      & 219,900,094              & 129,818,331               &                          & Tatar                     & 670M                     & 305M                      & 51,034,893               & 23,825,695                \\
            Ido                       & 147K                     & 130K                      & 25,702                   & 22,773                    &                          & Telugu                    & 2.5G                     & 1.6G                      & 123,711,517              & 79,094,167                \\
            Iloko                     & 874K                     & 636K                      & 142,942                  & 105,564                   &                          & Thai                      & 36G                      & 16G                       & 951,743,087              & 368,965,202               \\
            Indonesian                & 30G                      & 16G                       & 4,574,692,265            & 2,394,957,629             &                          & Tibetan                   & 187M                     & 138M                      & 1,483,589                & 936,556                   \\
            Interlingua               & 662K                     & 360K                      & 180,231                  & 100,019                   &                          & Tosk Albanian             & 5.0M                     & 2.8M                      & 841,750                  & 459,001                   \\
            Interlingue               & 24K                      & 1.6K                      & 5,352                    & 602                       &                          & Turkish                   & 60G                      & 27G                       & 7,577,388,700            & 3,365,734,289             \\
            Irish                     & 88M                      & 60M                       & 14,483,593               & 10,017,303                &                          & Turkmen                   & 11M                      & 6.8M                      & 1,113,869                & 752,326                   \\
            Italian                   & 137G                     & 69G                       & 22,248,707,341           & 11,250,012,896            &                          & Tuvinian                  & 12K                      & 7.9K                      & 759                      & 540                       \\
            Japanese                  & 216G                     & 106G                      & 4,962,979,182            & 1,123,067,063             &                          & Uighur                    & 122M                     & 83M                       & 8,657,141                & 5,852,225                 \\
            Javanese                  & 659K                     & 583K                      & 104,896                  & 86,654                    &                          & Ukrainian                 & 53G                      & 28G                       & 4,204,381,276            & 2,252,380,351             \\
            Kalmyk                    & 113K                     & 112K                      & 10,277                   & 10,155                    &                          & Upper Sorbian             & 4.2M                     & 1.8M                      & 545,351                  & 236,867                   \\
            Kannada                   & 1.7G                     & 1.1G                      & 81,186,863               & 49,343,462                &                          & Urdu                      & 2.7G                     & 1.7G                      & 331,817,982              & 218,030,228               \\
            Karachay-Balkar           & 2.6M                     & 2.3M                      & 185,436                  & 166,496                   &                          & Uzbek                     & 21M                      & 12M                       & 2,450,256                & 1,381,644                 \\
            Kazakh                    & 2.7G                     & 1.5G                      & 191,126,469              & 108,388,743               &                          & Venetian                  & 18K                      & 17K                       & 3,492                    & 3,199                     \\
            \bottomrule
        \end{tabular}
    }
\end{table}

\begin{table}[t!]
    \centering\tiny
    \resizebox{\linewidth}{!}{
        \begin{tabular}{@{}lrrrrclrrrr@{}}\toprule
            \multirow{2}{*}{Language} & \multicolumn{2}{c}{Size} & \multicolumn{2}{c}{Words} & \phantom{a}              & \multirow{2}{*}{Language} & \multicolumn{2}{c}{Size} & \multicolumn{2}{c}{Words}                                                                                                               \\
            \cmidrule(l{2pt}r{2pt}){2-3} \cmidrule(l{2pt}r{2pt}){4-5} \cmidrule(l{2pt}r{2pt}){8-9} \cmidrule(l{2pt}r{2pt}){10-11}
                                      & \multicolumn{1}{c}{Orig} & \multicolumn{1}{c}{Dedup} & \multicolumn{1}{c}{Orig} & \multicolumn{1}{c}{Dedup} & \phantom{a}              &                           & \multicolumn{1}{c}{Orig} & \multicolumn{1}{c}{Dedup} & \multicolumn{1}{c}{Orig} & \multicolumn{1}{c}{Dedup} \\\midrule
            Kirghiz                   & 600M                     & 388M                      & 44,194,823               & 28,982,620                &                          & Vietnamese                & 68G                      & 32G                       & 12,036,845,359           & 5,577,159,843             \\
            Komi                      & 2.3M                     & 1.2M                      & 201,404                  & 95,243                    &                          & Volapük                   & 2.0M                     & 2.0M                      & 321,121                  & 318,568                   \\
            Korean                    & 24G                      & 12G                       & 2,368,765,142            & 1,120,375,149             &                          & Walloon                   & 273K                     & 203K                      & 50,720                   & 37,543                    \\
            Kurdish                   & 94M                      & 60M                       & 15,561,003               & 9,946,440                 &                          & Waray                     & 2.5M                     & 2.2M                      & 397,315                  & 336,311                   \\
            Lao                       & 174M                     & 114M                      & 4,133,311                & 2,583,342                 &                          & Welsh                     & 213M                     & 133M                      & 37,422,441               & 23,574,673                \\
            Latin                     & 26M                      & 8.3M                      & 4,122,201                & 1,328,038                 &                          & Western Frisian           & 35M                      & 26M                       & 5,691,077                & 4,223,816                 \\
            Latvian                   & 4.0G                     & 1.8G                      & 520,761,977              & 236,428,905               &                          & Western Mari              & 1.2M                     & 1.1M                      & 93,338                   & 87,780                    \\
            Lezghian                  & 3.3M                     & 3.0M                      & 247,646                  & 224,871                   &                          & Western Panjabi           & 12M                      & 9.0M                      & 1,426,986                & 1,111,112                 \\
            Limburgan                 & 29K                      & 27K                       & 4,730                    & 4,283                     &                          & Wu Chinese                & 109K                     & 32K                       & 11,189                   & 4,333                     \\
            Lithuanian                & 8.8G                     & 3.9G                      & 1,159,661,742            & 516,183,525               &                          & Yakut                     & 42M                      & 26M                       & 2,547,623                & 1,789,174                 \\
            Lojban                    & 736K                     & 678K                      & 154,330                  & 141,973                   &                          & Yiddish                   & 141M                     & 84M                       & 13,834,320               & 8,212,970                 \\
            Lombard                   & 443K                     & 433K                      & 75,229                   & 73,665                    &                          & Yoruba                    & 55K                      & 27K                       & 8,906                    & 3,518                     \\
            Low German                & 18M                      & 13M                       & 2,906,347                & 2,146,417                 &                          & Yue Chinese               & 3.7K                     & 2.2K                      & 186                      & 128                       \\
            \midrule
            \textbf{Total}            & 6.3T                     & 3.2T                      & 844,315,434,723          & 425,651,344,234           &                          &                           &                          &                           &                          &                           \\
            \bottomrule
        \end{tabular}
    }
    \caption{Size of the OSCAR corpus by language measured in bytes and number of words. Standard UNIX human-readable notation is used for the size in byte. We define ``words'' as spaced separated tokens, which gives a good estimate of the size of each corpus for languages using Latin or Cyrillic alphabets, but might give a misleading size for other languages such as Chinese or Japanese.}
    \label{tab:langs-goclassy}
\end{table}
%%%%%%%%%%%%%%%%%%%%%%%%%%%%%%%%%%%%%%%%%%%%%%%%%%%%%%%%%%%%%%%%%%%%%%%%
\chapter{A First Evaluation of the OSCAR Corpus}
%%%%%%%%%%%%%%%%%%%%%%%%%%%%%%%%%%%%%%%%%%%%%%%%%%%%%%%%%%%%%%%%%%%%%%%%
\label{sec:appendix}

\section{Computational cost and carbon footprint}\label{cost}

Considering the discussion above, we believe an interesting follow-up to our experiments would be training the ELMo models for more of the languages included in the OSCAR corpus. However training ELMo is computationally costly, and one way to estimate this cost, as pointed out by \citet{strubell-etal-2019-energy}, is by using the training times of each model to compute both power consumption and CO\textsubscript{2} emissions.

\begin{table}[t]
    \centering\small
    \scalebox{0.93}{
        \begin{tabular}{@{}lrrrrr@{}}\toprule
            Language                                        & Power & Hours  & Days  & KWh$\cdotp$PUE & CO\textsubscript{2}e \\
            \midrule
            \multicolumn{6}{l}{\hspace*{6mm}\em OSCAR-Based ELMos}                                                           \\[0.5mm]
            Bulgarian                                       & 1183  & 515.00 & 21.45 & 962.61         & 49.09                \\
            Catalan                                         & 1118  & 199.98 & 8.33  & 353.25         & 18.02                \\
            Danish                                          & 1183  & 200.89 & 8.58  & 375.49         & 19.15                \\
            Finnish                                         & 1118  & 591.25 & 24.63 & 1044.40        & 53.26                \\
            Indonesian                                      & 1183  & 694.26 & 28.93 & 1297.67        & 66.18                \\
            \midrule\multicolumn{6}{l}{\hspace*{6mm}\em Wikipedia-Based ELMos}                                               \\[0.5mm]
            Bulgarian                                       & 1118  & 15.45  & 0.64  & 27.29          & 1.39                 \\
            Catalan                                         & 1118  & 51.08  & 2.13  & 90.22          & 4.60                 \\
            Danish                                          & 1118  & 14.56  & 0.61  & 25,72          & 1.31                 \\
            Finnish                                         & 1118  & 21.79  & 0.91  & 38.49          & 1.96                 \\
            Indonesian                                      & 1118  & 20.28  & 0.84  & 35.82          & 1.82                 \\
            \midrule
            \multicolumn{2}{@{}l}{\textsc{Total emissions}} &       &        &       & 216.78                                \\
            \bottomrule
        \end{tabular}
    }
    \caption{Average power draw (Watts), training times (in both hours and days), mean power consumption (KWh) and CO\textsubscript{2} emissions (kg) for each ELMo model trained.}
    \label{tab:carbon}
\end{table}

In our set-up we used two different machines, each one having 4 NVIDIA GeForce GTX 1080 Ti graphic cards and 128GB of RAM, the difference between the machines being that one uses a single Intel Xeon Gold 5118 processor, while the other uses two Intel Xeon E5-2630 v4 processors. One GeForce GTX 1080 Ti card is rated at around 250 W,\footnote{\url{https://www.geforce.com/hardware/desktop-gpus/geforce-gtx-1080-ti/specifications}} the Xeon Gold 5118 processor is rated at 105 W,\footnote{\url{https://ark.intel.com/content/www/us/en/ark/products/120473/intel-xeon-gold-5118-processor-16-5m-cache-2-30-ghz.html}} while one Xeon E5-2630 v4 is rated at 85 W.\footnote{\url{https://ark.intel.com/content/www/us/en/ark/products/92981/intel-xeon-processor-e5-2630-v4-25m-cache-2-20-ghz.html}} For the DRAM we can use the work of \citet{desrochers-etal-2016-a} to estimate the total power draw of 128GB of RAM at around 13W. Having this information, we can now use the formula proposed by \citet{strubell-etal-2019-energy} in order to compute the total power required to train one ELMo model:
\[
    p_t = \frac{1.58t(cp_{c} + p_r + gp_g)}{1000}
\]
Where $c$ and $g$ are the number of CPUs and GPUs respectively, $p_c$ is the average power draw (in Watts) from all CPU sockets, $p_r$ the average power draw from all DRAM sockets, and $p_g$ the average power draw of a single GPU. We estimate the total power consumption by adding GPU, CPU and DRAM consumptions, and then multiplying by the \emph{Power Usage Effectiveness} (PUE), which accounts for the additional energy required to support the compute infrastructure. We use a PUE coefficient of 1.58, the 2018 global average for data centers \citep{strubell-etal-2019-energy}. In table \ref{tab:carbon} we report the training times in both hours and days, as well as the total power draw (in Watts) of the system used to train each individual ELMo model. We use this information to compute the total power consumption of each ELMo, also reported in table \ref{tab:carbon}.

We can further estimate the CO\textsubscript{2} emissions in kilograms of each single model by multiplying the total power consumption by the average CO\textsubscript{2} emissions per kWh in France (where the models were trained). According to the RTE (Réseau de transport d'électricité / Electricity Transmission Network) the average emission per kWh were around 51g/kWh in November 2019,\footnote{\url{https://www.rte-france.com/fr/eco2mix/eco2mix-co2}} when the models were trained. Thus the total CO\textsubscript{2} emissions in kg for one single model can be computed as:
\[
    \text{CO}_{2}\text{e} = 0.051 p_t
\]
All emissions for the ELMo models are also reported in table \ref{tab:carbon}.

We do not report the power consumption or the carbon footprint of training the UDPipe 2.0 architecture, as each model took less than 4 hours to train on a machine using a single NVIDIA Tesla V100 card. Also, this machine was shared during training time, so it would be extremely difficult to accurately estimate the power consumption of these models.

Even though it would have been interesting to replicate all our experiments and computational cost estimations with state-of-the-art fine-tuning models such as BERT, XLNet, RoBERTa or ALBERT, we recall that these transformer-based architectures are extremely costly to train, as noted by the BERT authors on the official BERT GitHub repository,\footnote{\url{https://github.com/google-research/bert}} and are currently beyond the scope of our computational infrastructure. However we believe that ELMo contextualized word embeddings remain a useful model that still provide an extremely good trade-off between performance to training cost, even setting new state-of-the-art scores in parsing and POS tagging for our five chosen languages, performing even better than the multilingual mBERT model.

\section{Number of training steps for each checkpoint and each corpus}

\begin{table}[ht!]
    \centering\small
    \scalebox{0.96}{
        \begin{tabular}{@{}lrrrr@{}}\toprule
            Language   & 1 Epoch & 3 Epochs & 5 Epochs  & 10 Epochs        \\
            \midrule
            \multicolumn{5}{l}{\hspace*{6mm}\em Wikipedia-Based ELMos}     \\[0.5mm]
            Bulgarian  & 6,268   & 18,804   & 31,340    & 62,680           \\
            Catalan    & 20,666  & 61,998   & 103,330   & 206,660          \\
            Danish     & 5,922   & 17,766   & 29,610    & 59,220           \\
            Finnish    & 8,763   & 26,289   & 43,815    & 87,630           \\
            Indonesian & 7,891   & 23,673   & 39,455    & 78,910           \\
            \midrule\multicolumn{5}{l}{\hspace*{6mm}\em OSCAR-Based ELMos} \\[0.5mm]
            Bulgarian  & 143,169 & 429,507  & 715,845   & 1,431,690        \\
            Catalan    & 81,156  & 243,468  & 405,780   & 811,560          \\
            Danish     & 81,156  & 243,468  & 405,780   & 811,560          \\
            Finnish    & 181,230 & 543,690  & 906,150   & 1,812,300        \\
            Indonesian & 263,830 & 791,490  & 1,319,150 & 2,638,300        \\
            \bottomrule
        \end{tabular}
    }
    \caption{Number of training steps for each checkpoint, for the \elmowiki and \elmooscar of each language.}
    \label{tab:steps}
\end{table}
%%%%%%%%%%%%%%%%%%%%%%%%%%%%%%%%%%%%%%%%%%%%%%%%%%%%%%%%%%%%%%%%%%%%%%%%
\chapter{Quality at a Glance: An Audit of OSCAR 2019 and other Web-Crawled Datasets}
%%%%%%%%%%%%%%%%%%%%%%%%%%%%%%%%%%%%%%%%%%%%%%%%%%%%%%%%%%%%%%%%%%%%%%%%

\begin{table}[th!]
    \centering
    \begin{tabular}{lll}
        \toprule
        \textbf{Dataset } & \textbf{Supercode} & \textbf{Subcode(s)}                       \\
        \midrule
        JW300             & \texttt{kg}        & \texttt{kwy}                              \\
        JW300             & \texttt{mg}        & \texttt{tdx}                              \\
        JW300             & \texttt{qu}        & \texttt{que}, \texttt{qug}, \texttt{qus}, \\
                          &                    & \texttt{quw}, \texttt{quy}, \texttt{quz}, \\
                          &                    & \texttt{qvi}, \texttt{qvz}                \\
        JW300             & \texttt{sw}        & \texttt{swc}                              \\
        \midrule
        OSCAR             & \texttt{ar}        & \texttt{arz}                              \\
        OSCAR             & \texttt{az}        & \texttt{azb}                              \\
        OSCAR             & \texttt{sh}        & \texttt{bs}, \texttt{hr}, \texttt{sr}     \\
        OSCAR             & \texttt{ku}        & \texttt{ckb}                              \\
        OSCAR             & \texttt{ms}        & \texttt{id}, \texttt{min}                 \\
        OSCAR             & \texttt{no}        & \texttt{nn}                               \\
        OSCAR             & \texttt{sq}        & \texttt{als}$^{*}$                        \\
        OSCAR             & \texttt{zh}        & \texttt{yue}, \texttt{wuu}                \\
        %  \midrule
        % Tatoeba & ar & acm,afb,ajp,apc,arq,ary,arz,ayl \\
        % Tatoeba & ber & kab \\
        % Tatoeba & et & vro \\
        % Tatoeba & ku & ckb,kmr,sdh \\
        % Tatoeba & lv & ltg \\
        % Tatoeba & sq & aln \\
        \midrule
        WikiMatrix        & \texttt{ar}        & \texttt{arz}                              \\
        WikiMatrix        & \texttt{sh}        & \texttt{bs}, \texttt{hr}, \texttt{sr}     \\
        WikiMatrix        & \texttt{zh}        & \texttt{wuu}                              \\
        \bottomrule
    \end{tabular}
    \caption{Situations where two language codes are represented, but one is a superset of another by the ISO standard, leading to unclarity about the data in the supercode dataset. $^{*}$The \texttt{als} dataset is actually in \texttt{gsw}.}
    \label{tab:supersets}
\end{table}


\begin{table}[th!]
    \centering
    \small
    \begin{tabular}{ll}
        \toprule
        \textbf{ Actual language} & \textbf{Code in JW300}                                                \\
        \midrule
        \texttt{cs}               & \texttt{cse}                                                          \\
        \texttt{de}               & \texttt{gsg}                                                          \\
        \texttt{el}               & \texttt{gss}                                                          \\
        \texttt{en}               & \texttt{ase}, \texttt{asf}, \texttt{bfi}, \texttt{ins}, \texttt{psp}, \\
                                  & \texttt{sfs}, \texttt{zib}, \texttt{zsl}                              \\
        \texttt{es}               & \texttt{aed}, \texttt{bvl}, \texttt{csf}, \texttt{csg}, \texttt{csn}, \\
                                  & \texttt{csr}, \texttt{ecs}, \texttt{esn}, \texttt{gsm}, \texttt{hds}, \\
                                  & \texttt{lsp}, \texttt{mfs}, \texttt{ncs}, \texttt{prl}, \texttt{pys}, \\
                                  & \texttt{ssp}, \texttt{vsl}                                            \\
        \texttt{fi}               & \texttt{fse}                                                          \\
        \texttt{fr}               & \texttt{fcs},\texttt{fsl}                                             \\
        \texttt{hu}               & \texttt{hsh}                                                          \\
        \texttt{id}               & \texttt{inl}                                                          \\
        \texttt{it}               & \texttt{ise}                                                          \\
        \texttt{ja}               & \texttt{jsl}                                                          \\
        \texttt{ko}               & \texttt{kvk}                                                          \\
        \texttt{pl}               & \texttt{pso}                                                          \\
        \texttt{pt}               & \texttt{bzs}, \texttt{mzy}, \texttt{psr}, \texttt{sgn\_AO}            \\
        \texttt{ro}               & \texttt{rms}                                                          \\
        \texttt{ru}               & \texttt{rsl}                                                          \\
        \texttt{sk}               & \texttt{svk}                                                          \\
        \texttt{sq}               & \texttt{sql}                                                          \\
        \texttt{st}               & \texttt{jw\_ssa}                                                      \\
        \texttt{zh}               & \texttt{csl}, \texttt{tss}                                            \\
        \bottomrule
    \end{tabular}
    \caption{There are 48 languages in the JW300 corpus with language codes that correspond to sign languages, but in reality are unrelated high-resource languages (usually the most spoken language in the country of origin of the sign language). This table shows the actual language of the data corresponding to each sign language code.} %  For instance, the \texttt{ase-en} parallel data is actually \texttt{en-en} parallel data (copied source and target).
    \label{tab:signlanguages}
\end{table}

\section{Details on Language Code Issues}
\label{app:jw300}

Table \ref{tab:supersets} provides a complete lists of the corpora where one code is defined as a superset of the other by the ISO standard, and in Table \ref{tab:signlanguages} we provide a complete list of the language codes in JW300 which purport to be sign language but are actually unrelated high-resource languages.



\begin{table}[th!]
    \centering\small
        \begin{tabular}{lll}


            \toprule
            \textbf{Code in JW300} & \textbf{BCP-47 code} & \textbf{Actual Language Name} \\
            \multicolumn{3}{c}{}                                                          \\
            \multicolumn{3}{c}{\textbf{Incorrect private-use extensions}}                 \\
            \midrule
            hy\_arevmda            & hyw                  & Western Armenian              \\
            jw\_dgr                & os\_x\_dgr           & Digor Ossetian                \\
            jw\_dmr                & naq\_x\_dmr          & Damara Khoekhoe               \\
            jw\_ibi                & yom\_x\_ibi          & Ibinda Kongo                  \\
            jw\_paa                & pap\_x\_paa          & Papiamento (Aruba)            \\
            jw\_qcs                & qxl                  & Salasaca Highland Kichwa      \\
            jw\_rmg                & rmn\_x\_rmg          & Greek Romani (South)          \\
            jw\_rmv                & rmy\_x\_rmv          & Vlax Romani, Russia           \\
            jw\_spl                & nso\_x\_spl          & Sepulana                      \\
            jw\_ssa                & st\_ZA               & Sesotho (South Africa)        \\
            jw\_tpo                & pt\_PT               & Portuguese (Portugal)         \\
            jw\_vlc                & ca\_x\_vlc           & Catalan (Valencia)            \\
            jw\_vz                 & skg\_x\_vz           & Vezo Malagasy                 \\
            rmy\_AR                & rmy\_x\_?            & Kalderash                     \\

            \multicolumn{3}{c}{}                                                          \\
            \multicolumn{3}{c}{\textbf{Equivalent codes used in place of extensions}}     \\
            \midrule
            kmr\_latn              & kmr\_x\_rdu          & Kurmanji (Caucasus)           \\
            nya                    & ny\_x\_?             & Chinyanja (Zambia)            \\
            que                    & qu\_x\_?             & Quechua (Ancash)              \\

            \multicolumn{3}{c}{}                                                          \\
            \multicolumn{3}{c}{\textbf{Deprecated codes}}                                 \\
            \midrule
            daf                    & dnj/lda              & Dan                           \\
            % sgn\_AO & 	pt & 	Portuguese \\ 

            \multicolumn{3}{c}{}                                                          \\
            \multicolumn{3}{c}{\textbf{ISO-693-3 used in place of ISO-693-2}}             \\
            \midrule
            cat                    & ca                   & Catalan                       \\
            gug                    & gn                   & Guarani                       \\
            run                    & rn                   & Kirundi                       \\
            tso\_MZ                & ts\_MZ               & Changana (Mozambique)         \\
            \bottomrule
        \end{tabular}%
    \caption{Language code issues in the JW300 datasets for 22 language varieties not covered by Tables \ref{tab:supersets} and \ref{tab:signlanguages}.
        %Twelve languages have codes starting in \texttt{jw\_}, suggesting they are varieties of Javanese, but are instead mis-parsed private-use extensions. Three codes appear in addition to equivalent ISO codes, making it unclear which languages they are. One language uses a deprecated ISO code. Four languages use the ISO639-3 code instead of the ISO639-2 code, and therefore are not BCP-47. (Note: in this table, 
        Private use extensions are given as they appear in \url{jw.org}, and specified as `?' if they are absent from \url{jw.org}.}
    \label{tab:jw300nonbcp}
\end{table}

Special attention needs to be given to the JW300 dataset, which, in addition to the sign languages and superset code issues, has a variety of other peculiarities. These problems seem to originate in the codes used by \url{jw.org},\footnote{The \url{jw.org} website seems to use correct BCP-47 extensions now, however, and entering a code such as ``jw\_dmr" redirects to ``naq\_x\_dmr".} which were apparently not checked in the creation of the JW300 dataset. An overview is provided in Table \ref{tab:jw300nonbcp}, and the following paragraphs give specifics.

Twelve languages in JW300 have codes starting in \texttt{jw\_}, suggesting they are varieties of Javanese (ISO639-1 \texttt{jw}), but are instead attempts to represent language dialects for which there are no BCP-47 codes. These codes seem to have been updated in \url{jw.org} to appropriate BCP-47 private-use extensions in the form \texttt{<supercode>\_x\_<tag>}, which are provided in Table \ref{tab:jw300nonbcp}.
Twelve languages have codes starting in \texttt{jw\_}, suggesting they are varieties of Javanese, but are instead mis-parsed private-use extensions. Three codes appear in addition to equivalent ISO codes, making it unclear which languages they are. One language uses a deprecated ISO code. Four languages use the ISO639-3 code instead of the ISO639-2 code, and therefore are not BCP-47.

In addition to the \texttt{jw\_} tags, there are two other mis-used private subtags: \texttt{hy\_arevmda}, which in addition to lacking the mandatory \texttt{\_x\_} appears to represent standard Western Armenian (\texttt{hyw}); and \texttt{rmy\_AR}, which, rather than being Romany from Argentina, is Kalderash Romany.

There are also a few anomalies where private use extensions should have been used but other methods were found to convey the distinctions. Three codes appear in addition to equivalent ISO codes, making it unclear which languages they are. Two of these are equivalencies between  ISO639-2 and  ISO639-3 (\texttt{nya} and \texttt{ny} are both Chichewa, \texttt{qu} and \texttt{que} are both Quechua), and one is a script equivalency (\texttt{kmr} and \texttt{kmr\_latn} are both in Latin script). In these three cases the two codes do represent different languages---so a private use extension would have been appropriate.

Finally, there is the more minor issue that three languages use the ISO639-3 code instead of the ISO639-2 code, and therefore are not BCP-47.


In addition to the JW300-specific errors, Table \ref{tab:misc_codes} summarizes miscellaneous errors in CCAligned and OSCAR 2019 that were detailed in Section \ref{sec:codes}.

\begin{table}[!th]
    \small
    \centering
    \begin{tabular}{lll}
        \toprule
        \textbf{Dataset} & \textbf{Code in Corpus} & \textbf{Correct Code} \\
        \midrule
        CCAligned        & \texttt{zz}             & \texttt{zza}          \\
        CCAligned        & \texttt{sz}             & \texttt{szl}          \\
        CCAligned        & \texttt{ns}             & \texttt{nso}          \\
        CCAligned        & \texttt{cb}             & \texttt{ckb}          \\
        CCAligned        & \texttt{tz}             & \texttt{ber}          \\
        CCAligned        & \texttt{qa}             & \texttt{shn}          \\
        CCAligned        & \texttt{qd}             & \texttt{kac}          \\
        CCAligned        & \texttt{cx}             & \texttt{ceb}          \\
        \midrule
        mC4              & \texttt{iw}             & \texttt{he}           \\
        \midrule
        OSCAR            & \texttt{eml}            & \texttt{egl}          \\
        OSCAR            & \texttt{als}            & \texttt{gsw}          \\
        OSCAR            & \texttt{sh}             & \texttt{hbs}          \\
        \midrule
        WikiMatrix       & \texttt{sh}             & \texttt{hbs}          \\
        \bottomrule
    \end{tabular}
    \caption{Miscellaneous errors in language codes.}
    \label{tab:misc_codes}
\end{table}







\section{Complete Error Taxonomy and Instructions}~\label{app:taxonomy}
In addition to the examples given in Table \ref{tab:examples}, raters were provided with the following verbal notes on the error codes:
\begin{itemize}
    \item \textbf{\texttt{CC}: Correct translation, natural sentence:} It's OK if it's a sentence fragment instead of a whole sentence, as long as it is not too short (about 5 words or greater). The translation does not have to be perfect.
    \item \textbf{\texttt{\texttt{CS}}: Correct Translation, but single word or short phrase:} Also includes highly repeated short phrases, like ``the cat the cat the cat the cat the cat ..."
    \item \textbf{\texttt{CB}: Correct translation, but boilerplate: } This can be auto-generated or formulaic content, or content that one deems ``technically correct but generally not very useful to NLP models". Unfortunately, it's often not clear what should be counted as boilerplate...do your best.
    \item \textbf{\texttt{X}: Incorrect translation} [for parallel sentences] both source and target are in the correct language, but they are not adequate translations.
    \item \textbf{\texttt{WL}: Wrong language} For short sentences, especially with proper nouns, there is often a fine line between ``Wrong language" and ``Not language". Do your best.
    \item \textbf{\texttt{NL}: Not language} At least one of source and target are not linguistic content. Any sentence consisting only of a proper noun (e.g. ``Tyrone Ping") should be marked as \texttt{NL}.
    \item \textbf{\texttt{U}: Unknown} for sentences that need verification by a native speaker. This is an auxiliary label that is resolved in most cases.
\end{itemize}







\section{Methodological Notes}\label{app:strategies}

A surprising amount of work can be done without being an expert in the languages involved. The easiest approach is simply to search the internet for the sentence, which usually results in finding the exact page the sentence came from, which in turn frequently contains clues like language codes in the URL, or a headline like \textit{News in X language}, sometimes with references to a translated version of the same page. However, for the cases where this is insufficient, here are a few tips, tricks, and observations.

\paragraph{No Skills Required:}
Things that do not require knowledge of the language(s) in question.
\begin{enumerate}
    \item ``Not language'' can usually be identified by anyone who can read the script, though there are tricky cases with proper nouns.
    \item Frequently, ``parallel" sentences contain different numbers in the source and target (especially autogenerated content), and are easy to disqualify.
    \item Errors tend to repeat. If a word is mistranslated once, it will often be mistranslated many more times throughout a corpus, making it easy to spot.
\end{enumerate}

\paragraph{Basic Research Required:}
Things that do not require knowledge of the language(s) in question but can be done with basic research.
\begin{enumerate}
    \item If it's written in the wrong script it's considered wrong language. (Sometimes the writing system is indicated in the published corpus, e.g. \texttt{bg-Latn}, but usually the language has a ``default" script defined by ISO.)
    \item Some types of texts come with inherent labels or markers, such as enumerators or verse numbers.
          %For example, much of CCAligned's Odia text is Christian Bible verses, which are preceded by an identifier like ``Matt 12:37". 
    \item When all else fails, search the internet for the whole sentence or n-grams thereof! If the whole sentence can be found, frequently the language is betrayed by the web page (the language's autonym is useful in this case).
\end{enumerate}


\section{Complete Audit Results}\label{app:stats}
Table for \ref{tab:oscar-full} give the complete annotation percentages for OSCAR 2019. For each annotation label, we report the ratio of the annotated sentences (of max 100 sentences) that were assigned that label by the primary annotator. Repeated annotations done for agreement measurement are not included. The \texttt{C} column aggregates all correct sub-codes (\texttt{CC}, \texttt{CS}, \texttt{CB}). We also report the total number of sentences that each dataset contains for each language and the average sentence length for the audited sentences to illustrate differences across languages. The original language codes as they are published with the datasets are maintained for the sake of consistency (but should be handled with care in future work, see Section~\ref{sec:codes}), and those with less than 20\% correct sentences are highlighted. For the complete audit results for the other 4 datasets, please refer to the original \citet{kreutzer-etal-2021-quality} work.

%%% CCALIGNED %%%

\begin{table*}[hbt!]
    \centering
    \resizebox*{0.9\textwidth}{!}{ %\textheight}{%
        \begin{tabular}{l|rrrr|rrrr|rr}
            \toprule
            {}                      & C       & CC      & CS      & CB      & X       & WL      & NL      & porn    & \#sentences & avg target length \\
            \midrule
            \textbf{en-sz\_PL}      & 0.00\%  & 0.00\%  & 0.00\%  & 0.00\%  & 0.00\%  & 8.33\%  & 91.67\% & 0.00\%  & 12          & 71.42             \\
            \textbf{en-mt\_MT}      & 3.85\%  & 0.00\%  & 3.85\%  & 0.00\%  & 50.00\% & 26.92\% & 19.23\% & 0.00\%  & 26          & 12.58             \\
            \textbf{en-tz\_MA}      & 12.12\% & 6.06\%  & 6.06\%  & 0.00\%  & 45.45\% & 36.36\% & 6.06\%  & 0.00\%  & 33          & 57.33             \\
            \textbf{en-zz\_TR}      & 0.00\%  & 0.00\%  & 0.00\%  & 0.00\%  & 8.82\%  & 61.76\% & 29.41\% & 0.00\%  & 34          & 46.53             \\
            \textbf{en-kg\_AO}      & 1.35\%  & 0.00\%  & 1.35\%  & 0.00\%  & 14.86\% & 2.70\%  & 81.08\% & 0.00\%  & 74          & 29.20             \\
            \textbf{en-qa\_MM}      & 11.03\% & 5.88\%  & 3.68\%  & 1.47\%  & 72.06\% & 3.68\%  & 13.24\% & 0.00\%  & 136         & 55.28             \\
            \textbf{en-bm\_ML}      & 6.04\%  & 4.03\%  & 2.01\%  & 0.00\%  & 26.85\% & 6.71\%  & 60.40\% & 0.00\%  & 149         & 32.19             \\
            \textbf{en-az\_IR}      & 6.93\%  & 6.93\%  & 0.00\%  & 0.00\%  & 20.79\% & 13.86\% & 58.42\% & 0.00\%  & 158         & 115.85            \\
            \textbf{en-qd\_MM}      & 7.92\%  & 4.95\%  & 1.98\%  & 0.99\%  & 81.19\% & 3.96\%  & 6.93\%  & 0.00\%  & 179         & 60.34             \\
            en-ay\_BO               & 51.00\% & 33.00\% & 18.00\% & 0.00\%  & 29.00\% & 3.00\%  & 17.00\% & 0.00\%  & 475         & 92.19             \\
            \textbf{en-ak\_GH}      & 14.23\% & 13.60\% & 0.63\%  & 0.00\%  & 46.86\% & 19.25\% & 19.67\% & 0.00\%  & 478         & 45.85             \\
            en-st\_ZA               & 48.57\% & 42.14\% & 0.00\%  & 6.43\%  & 40.71\% & 1.43\%  & 9.29\%  & 0.00\%  & 904         & 111.83            \\
            en-ve\_ZA               & 60.40\% & 29.70\% & 21.78\% & 8.91\%  & 28.71\% & 3.96\%  & 6.93\%  & 0.00\%  & 1555        & 82.99             \\
            en-ts\_ZA               & 51.49\% & 34.65\% & 11.88\% & 4.95\%  & 40.59\% & 2.97\%  & 4.95\%  & 0.00\%  & 1967        & 73.93             \\
            en-or\_IN               & 42.61\% & 6.09\%  & 24.35\% & 12.17\% & 38.26\% & 9.57\%  & 9.57\%  & 0.00\%  & 5526        & 71.39             \\
            \textbf{en-ns\_ZA }     & 4.00\%  & 2.00\%  & 0.00\%  & 2.00\%  & 23.00\% & 15.00\% & 58.00\% & 4.00\%  & 14138       & 33.52             \\
            \textbf{en-lg\_UG}      & 6.00\%  & 0.00\%  & 6.00\%  & 0.00\%  & 68.00\% & 17.00\% & 9.00\%  & 2.00\%  & 14701       & 15.83             \\
            \textbf{en-ln\_CD}      & 8.00\%  & 4.00\%  & 3.00\%  & 1.00\%  & 14.00\% & 4.00\%  & 74.00\% & 4.00\%  & 21562       & 28.80             \\
            \textbf{en-om\_KE}      & 2.00\%  & 2.00\%  & 0.00\%  & 0.00\%  & 31.00\% & 38.00\% & 29.00\% & 24.00\% & 22206       & 23.83             \\
            \textbf{en-ss\_SZ}      & 12.65\% & 9.04\%  & 3.61\%  & 0.00\%  & 13.25\% & 24.10\% & 50.00\% & 13.86\% & 22960       & 25.30             \\
            \textbf{en-te\_IN\_rom} & 0.00\%  & 0.00\%  & 0.00\%  & 0.00\%  & 25.00\% & 8.00\%  & 67.00\% & 5.00\%  & 25272       & 24.21             \\
            \textbf{en-cb\_IQ}      & 4.00\%  & 1.00\%  & 3.00\%  & 0.00\%  & 30.00\% & 18.00\% & 48.00\% & 11.00\% & 52297       & 30.04             \\
            \textbf{en-tn\_BW}      & 0.00\%  & 0.00\%  & 0.00\%  & 0.00\%  & 6.90\%  & 8.97\%  & 63.45\% & 10.34\% & 71253       & 16.80             \\
            \textbf{en-ff\_NG}      & 0.00\%  & 0.00\%  & 0.00\%  & 0.00\%  & 0.00\%  & 8.00\%  & 92.00\% & 2.00\%  & 73022       & 33.59             \\
            \textbf{en-sn\_ZW}      & 5.00\%  & 1.00\%  & 3.00\%  & 1.00\%  & 81.00\% & 14.00\% & 0.00\%  & 0.00\%  & 86868       & 102.59            \\
            \textbf{en-wo\_SN}      & 0.00\%  & 0.00\%  & 0.00\%  & 0.00\%  & 1.71\%  & 3.31\%  & 94.98\% & 18.46\% & 88441       & 27.25             \\
            \textbf{en-br\_FR}      & 17.00\% & 3.00\%  & 1.00\%  & 13.00\% & 37.00\% & 14.00\% & 32.00\% & 1.00\%  & 115128      & 41.68             \\
            en-zu\_ZA               & 55.00\% & 39.00\% & 3.00\%  & 13.00\% & 30.00\% & 7.00\%  & 8.00\%  & 3.00\%  & 126101      & 79.32             \\
            en-ku\_TR               & 36.52\% & 12.17\% & 13.04\% & 11.30\% & 33.04\% & 28.70\% & 1.74\%  & 1.74\%  & 137874      & 90.51             \\
            en-ig\_NG               & 58.00\% & 49.00\% & 3.00\%  & 6.00\%  & 29.00\% & 12.00\% & 1.00\%  & 0.00\%  & 148146      & 83.42             \\
            en-kn\_IN               & 46.00\% & 9.00\%  & 6.00\%  & 31.00\% & 46.00\% & 2.00\%  & 5.00\%  & 4.00\%  & 163921      & 70.20             \\
            en-yo\_NG               & 34.93\% & 6.16\%  & 10.96\% & 17.81\% & 34.93\% & 12.33\% & 17.81\% & 0.00\%  & 175192      & 75.01             \\
            en-ky\_KG               & 44.12\% & 24.51\% & 17.65\% & 1.96\%  & 33.33\% & 22.55\% & 0.00\%  & 0.98\%  & 240657      & 69.56             \\
            en-tg\_TJ               & 46.08\% & 18.63\% & 24.51\% & 2.94\%  & 32.35\% & 20.59\% & 0.98\%  & 4.90\%  & 251865      & 75.31             \\
            en-ha\_NG               & 30.00\% & 25.00\% & 3.00\%  & 2.00\%  & 49.00\% & 9.00\%  & 12.00\% & 1.00\%  & 339176      & 60.78             \\
            en-am\_ET               & 59.11\% & 35.47\% & 2.46\%  & 21.18\% & 37.44\% & 2.96\%  & 0.49\%  & 0.00\%  & 346517      & 58.29             \\
            en-km\_KH               & 56.12\% & 12.24\% & 33.67\% & 10.20\% & 42.86\% & 1.02\%  & 0.00\%  & 0.00\%  & 412381      & 71.35             \\
            en-ne\_NP               & 47.00\% & 10.00\% & 13.00\% & 24.00\% & 15.00\% & 8.00\%  & 30.00\% & 14.00\% & 487155      & 79.14             \\
            en-su\_ID               & 35.00\% & 15.00\% & 15.00\% & 5.00\%  & 13.00\% & 13.00\% & 39.00\% & 0.00\%  & 494142      & 57.08             \\
            \textbf{en-ur\_PK\_rom} & 0.50\%  & 0.00\%  & 0.50\%  & 0.00\%  & 18.91\% & 27.36\% & 53.23\% & 5.47\%  & 513123      & 18.41             \\
            en-ht\_HT               & 55.67\% & 8.25\%  & 10.31\% & 37.11\% & 35.05\% & 6.19\%  & 3.09\%  & 1.03\%  & 558167      & 101.95            \\
            en-mn\_MN               & 33.00\% & 8.00\%  & 14.00\% & 11.00\% & 42.00\% & 7.00\%  & 18.00\% & 12.00\% & 566885      & 44.43             \\
            en-te\_IN               & 69.00\% & 42.00\% & 11.00\% & 16.00\% & 27.00\% & 1.00\%  & 3.00\%  & 1.00\%  & 581651      & 97.95             \\
            en-kk\_KZ               & 68.32\% & 40.59\% & 18.81\% & 8.91\%  & 18.81\% & 8.91\%  & 3.96\%  & 1.98\%  & 689651      & 72.36             \\
            en-be\_BY               & 90.00\% & 57.00\% & 13.00\% & 20.00\% & 10.00\% & 0.00\%  & 0.00\%  & 2.00\%  & 1125772     & 118.45            \\
            en-af\_ZA               & 63.00\% & 40.00\% & 23.00\% & 0.00\%  & 31.00\% & 2.00\%  & 4.00\%  & 12.00\% & 1504061     & 105.45            \\
            \textbf{en-jv\_ID}      & 5.05\%  & 1.01\%  & 1.01\%  & 3.03\%  & 25.25\% & 10.10\% & 59.60\% & 8.08\%  & 1513974     & 18.34             \\
            en-nl\_NL               & 46.00\% & 27.00\% & 19.00\% & 0.00\%  & 49.00\% & 2.00\%  & 3.00\%  & 0.00\%  & 36324231    & 85.95             \\
            \textbf{en-hi\_IN\_rom} & 1.00\%  & 0.00\%  & 0.00\%  & 1.00\%  & 39.00\% & 21.00\% & 39.00\% & 8.00\%  & 3789571     & 18.13             \\
            en-lv\_LV               & 59.00\% & 37.00\% & 9.00\%  & 13.00\% & 31.00\% & 7.00\%  & 3.00\%  & 14.00\% & 4850957     & 83.67             \\
            \textbf{en-ar\_AR\_rom} & 0.00\%  & 0.00\%  & 0.00\%  & 0.00\%  & 0.00\%  & 4.00\%  & 96.00\% & 4.00\%  & 5584724     & 16.69             \\
            \textbf{en-tl\_XX}      & 13.00\% & 6.00\%  & 3.00\%  & 4.00\%  & 24.00\% & 26.00\% & 37.00\% & 5.00\%  & 6593250     & 37.03             \\
            en-uk\_UA               & 63.00\% & 42.00\% & 8.00\%  & 13.00\% & 35.00\% & 1.00\%  & 1.00\%  & 5.00\%  & 8547348     & 67.88             \\
            en-zh\_TW               & 46.00\% & 11.00\% & 31.00\% & 4.00\%  & 47.00\% & 6.00\%  & 1.00\%  & 1.00\%  & 8778971     & 24.89             \\
            en-el\_GR               & 49.00\% & 15.00\% & 5.00\%  & 29.00\% & 38.00\% & 3.00\%  & 10.00\% & 8.00\%  & 8878492     & 54.90             \\
            en-da\_DK               & 54.00\% & 31.00\% & 18.00\% & 5.00\%  & 29.00\% & 5.00\%  & 12.00\% & 7.00\%  & 10738582    & 73.99             \\
            en-vi\_VN               & 31.00\% & 18.00\% & 0.00\%  & 13.00\% & 54.00\% & 1.00\%  & 14.00\% & 6.00\%  & 12394379    & 74.19             \\
            en-sv\_SE               & 97.00\% & 91.00\% & 3.00\%  & 3.00\%  & 0.00\%  & 3.00\%  & 0.00\%  & 0.00\%  & 12544075    & 103.91            \\
            en-zh\_CN               & 57.29\% & 22.92\% & 12.50\% & 21.88\% & 31.25\% & 1.04\%  & 10.42\% & 1.04\%  & 15181410    & 33.55             \\
            en-tr\_TR               & 45.00\% & 14.50\% & 14.00\% & 16.50\% & 44.50\% & 5.00\%  & 5.50\%  & 4.00\%  & 20282339    & 83.80             \\
            en-ja\_XX               & 57.00\% & 35.00\% & 21.00\% & 1.00\%  & 34.00\% & 6.00\%  & 0.00\%  & 0.00\%  & 26201214    & 34.44             \\
            en-pt\_XX               & 66.34\% & 36.63\% & 10.89\% & 18.81\% & 20.79\% & 3.96\%  & 8.91\%  & 0.00\%  & 46525410    & 87.20             \\
            en-it\_IT               & 36.00\% & 14.00\% & 18.00\% & 4.00\%  & 60.00\% & 1.00\%  & 3.00\%  & 0.00\%  & 58022366    & 97.44             \\
            en-de\_DE               & 62.00\% & 29.00\% & 14.00\% & 19.00\% & 28.00\% & 2.00\%  & 8.00\%  & 2.00\%  & 92597196    & 78.08             \\
            en-es\_XX               & 58.42\% & 16.83\% & 25.74\% & 15.84\% & 22.77\% & 2.97\%  & 15.84\% & 4.95\%  & 98351611    & 72.18             \\
            \midrule
            \textit{mean}           & 27.01\% & 29.35\% & 8.62\%  & 28.97\% & 14.48\% & 6.49\%  & 5.89\%  & 0.00\%  & 5.26\%      &                   \\
            %nl  36324231 after id_ID                                                                                                                  \\
            \bottomrule
        \end{tabular}%
    }
    \caption{Audit results for a sample of 100 sentences from \textbf{CCAligned} for each language pair, compared to the number of sentences available in the dataset. If fewer than 100 sentences were available, all sentences were audited. Language codes are as originally published.  The length is measured in number of characters and averaged across the audited portion of each corpus. Languages with less than 20\% correct sentences are boldfaced.}

    \label{tab:ccaligned-full}
\end{table*}

% template frame:
%\begin{table*}
%\centering
%\resizebox*{0.8\textwidth}{\textheight}{%
% [INSERT TABLE]
%}
%\caption{Audit results for a sample of 100 sentences from CCAligned for each language.}
%\end{table*}

\clearpage

%%% WIKIMATRIX %%%
\begin{table*}[hbt!]
    \centering
    \resizebox{0.9\textwidth}{!}{%
        \begin{tabular}{l|rrrr|rrrr|rr}
            \toprule
            {}               & C       & CC      & CS     & CB      & X       & WL      & NL     & porn   & \# sentences & avg target length \\
            \midrule
            \textbf{en-ug}   & 12.87\% & 8.91\%  & 1.98\% & 1.98\%  & 72.28\% & 9.90\%  & 1.98\% & 0.00\% & 22012        & 95.55             \\
            en-mwl           & 27.00\% & 26.00\% & 0.00\% & 1.00\%  & 73.00\% & 0.00\%  & 0.00\% & 0.00\% & 33899        & 135.26            \\
            \textbf{en-tg}   & 0.00\%  & 0.00\%  & 0.00\% & 0.00\%  & 95.10\% & 3.92\%  & 0.98\% & 0.00\% & 37975        & 88.87             \\
            \textbf{en-ne}   & 13.00\% & 7.00\%  & 6.00\% & 0.00\%  & 60.00\% & 23.00\% & 4.00\% & 0.00\% & 40549        & 69.26             \\
            \textbf{en-ka}   & 11.88\% & 2.97\%  & 2.97\% & 5.94\%  & 73.27\% & 10.89\% & 2.97\% & 0.00\% & 41638        & 144.74            \\
            \textbf{en-lmo } & 12.75\% & 11.76\% & 0.00\% & 0.98\%  & 81.37\% & 4.90\%  & 0.98\% & 0.00\% & 43790        & 89.38             \\
            en-io            & 28.00\% & 27.00\% & 0.00\% & 1.00\%  & 69.00\% & 2.00\%  & 1.00\% & 0.00\% & 45999        & 83.26             \\
            \textbf{en-jv}   & 13.73\% & 9.80\%  & 0.00\% & 3.92\%  & 70.59\% & 12.75\% & 2.94\% & 0.00\% & 48301        & 91.87             \\
            en-wuu           & 23.23\% & 14.14\% & 7.07\% & 2.02\%  & 65.66\% & 7.07\%  & 4.04\% & 0.00\% & 51024        & 34.77             \\
            \textbf{br-en}   & 8.70\%  & 7.61\%  & 1.09\% & 0.00\%  & 82.61\% & 4.35\%  & 0.00\% & 0.00\% & 58400        & 90.68             \\
            \textbf{bar-en}  & 6.00\%  & 6.00\%  & 0.00\% & 0.00\%  & 75.00\% & 16.00\% & 3.00\% & 0.00\% & 67394        & 103.51            \\
            \textbf{en-kk}   & 5.00\%  & 2.00\%  & 2.00\% & 1.00\%  & 81.00\% & 14.00\% & 0.00\% & 0.00\% & 109074       & 56.03             \\
            en-sw            & 33.33\% & 27.27\% & 4.04\% & 2.02\%  & 64.65\% & 2.02\%  & 0.00\% & 0.00\% & 138590       & 111.61            \\
            \textbf{en-nds}  & 1.96\%  & 1.96\%  & 0.00\% & 0.00\%  & 95.10\% & 1.96\%  & 0.98\% & 0.00\% & 178533       & 91.95             \\
            be-en            & 26.00\% & 24.00\% & 2.00\% & 0.00\%  & 73.00\% & 1.00\%  & 0.00\% & 0.00\% & 257946       & 121.22            \\
            en-hi            & 36.27\% & 32.35\% & 0.98\% & 2.94\%  & 59.80\% & 0.98\%  & 2.94\% & 0.00\% & 696125       & 96.77             \\
            en-ko            & 48.04\% & 33.33\% & 2.94\% & 11.76\% & 48.04\% & 2.94\%  & 0.98\% & 0.00\% & 1345630      & 55.18             \\
            en-uk            & 87.00\% & 84.00\% & 2.00\% & 1.00\%  & 10.00\% & 1.00\%  & 2.00\% & 0.00\% & 2576425      & 104.39            \\
            en-it            & 42.00\% & 42.00\% & 0.00\% & 0.00\%  & 58.00\% & 0.00\%  & 0.00\% & 0.00\% & 4626048      & 140.27            \\
            en-simple        & 37.62\% & 24.75\% & 0.00\% & 12.87\% & 56.44\% & 2.97\%  & 2.97\% & 0.00\% & nan          & 77.53             \\
            \bottomrule
        \end{tabular}%
    }
    \caption{Audit results for a sample of 100 sentences from \textbf{WikiMatrix} for each language pair, compared to the number of sentences available in the dataset. Language codes are as originally published. The length is measured in number of characters and averaged across the audited portion of each corpus. Languages with less than 20\% correct sentences are boldfaced.}

    \label{tab:wikimatrix-full}
\end{table*}


%%% PARACRAWL %%%
\begin{table*}[hbt!]
    \centering
    \resizebox{0.9\textwidth}{!}{%
        \begin{tabular}{l|rrrr|rrrr|rr}
            \toprule
            {}    & C       & CC      & CS      & CB      & X       & WL      & NL      & porn   & \# sentences & avg target length \\
            \midrule
            en-so & 80.81\% & 61.62\% & 1.01\%  & 18.18\% & 14.14\% & 5.05\%  & 0.00\%  & 0.00\% & 14879        & 189.83            \\
            en-ps & 72.00\% & 53.00\% & 9.00\%  & 10.00\% & 17.00\% & 10.00\% & 0.00\%  & 0.00\% & 26321        & 141.01            \\
            en-my & 45.00\% & 9.00\%  & 16.00\% & 20.00\% & 32.00\% & 9.00\%  & 14.00\% & 0.00\% & 31374        & 147.07            \\
            en-km & 76.00\% & 51.00\% & 13.00\% & 12.00\% & 18.00\% & 6.00\%  & 0.00\%  & 0.00\% & 65113        & 121.20            \\
            en-ne & 73.00\% & 48.00\% & 1.00\%  & 24.00\% & 23.00\% & 2.00\%  & 0.00\%  & 0.00\% & 92084        & 153.42            \\
            en-sw & 85.00\% & 60.00\% & 15.00\% & 10.00\% & 11.00\% & 2.00\%  & 2.00\%  & 0.00\% & 132517       & 167.34            \\
            en-si & 37.00\% & 31.00\% & 6.00\%  & 0.00\%  & 62.00\% & 0.00\%  & 1.00\%  & 0.00\% & 217407       & 123.06            \\
            en-nn & 35.92\% & 24.27\% & 8.74\%  & 2.91\%  & 49.51\% & 13.59\% & 0.97\%  & 0.00\% & 323519       & 56.24             \\
            es-eu & 88.00\% & 66.00\% & 15.00\% & 7.00\%  & 10.00\% & 1.00\%  & 1.00\%  & 0.00\% & 514610       & 121.31            \\
            es-gl & 89.00\% & 46.00\% & 6.00\%  & 37.00\% & 4.00\%  & 7.00\%  & 0.00\%  & 0.00\% & 1222837      & 107.88            \\
            en-ru & 81.00\% & 73.00\% & 6.00\%  & 2.00\%  & 19.00\% & 0.00\%  & 0.00\%  & 6.00\% & 5377911      & 101.28            \\
            en-bg & 95.15\% & 85.44\% & 0.97\%  & 8.74\%  & 4.85\%  & 0.00\%  & 0.00\%  & 0.97\% & 6470710      & 112.29            \\
            es-ca & 80.00\% & 54.00\% & 19.00\% & 7.00\%  & 11.00\% & 9.00\%  & 0.00\%  & 5.00\% & 6870183      & 107.21            \\
            en-el & 91.59\% & 68.22\% & 0.93\%  & 22.43\% & 7.48\%  & 0.93\%  & 0.00\%  & 0.00\% & 9402646      & 135.66            \\
            en-pl & 94.12\% & 76.47\% & 0.98\%  & 16.67\% & 3.92\%  & 1.96\%  & 0.00\%  & 0.98\% & 13744860     & 95.95             \\
            en-nl & 49.00\% & 32.00\% & 17.00\% & 0.00\%  & 46.00\% & 3.00\%  & 2.00\%  & 0.00\% & 31295016     & 95.05             \\
            en-pt & 93.07\% & 92.08\% & 0.00\%  & 0.99\%  & 4.95\%  & 1.98\%  & 0.00\%  & 0.00\% & 31486963     & 108.68            \\
            en-it & 60.82\% & 36.08\% & 16.49\% & 8.25\%  & 38.14\% & 0.00\%  & 1.03\%  & 0.00\% & 40798278     & 127.55            \\
            en-es & 87.00\% & 54.00\% & 20.00\% & 13.00\% & 12.00\% & 0.00\%  & 1.00\%  & 0.50\% & 78662122     & 119.72            \\
            en-de & 82.83\% & 64.65\% & 13.13\% & 5.05\%  & 13.13\% & 3.03\%  & 1.01\%  & 0.00\% & 82638202     & 111.43            \\
            en-fr & 89.62\% & 82.08\% & 4.72\%  & 2.83\%  & 10.38\% & 0.00\%  & 0.00\%  & 0.00\% & 104351522    & 144.20            \\
            \bottomrule
        \end{tabular} %
    }
    \caption{Audit results for a sample of 100 sentences from \textbf{ParaCrawl} for each language pair, compared to the number of sentences available in the dataset. Language codes are as originally published.  The length is measured in number of characters and averaged across the audited portion of each corpus.}
    \label{tab:paracrawl-full}
\end{table*}

\clearpage


%%% mC4 %%%

\begin{table*}[hbt!]
    \centering
    \resizebox*{0.9\textwidth}{!}{%
        \begin{tabular}{l|rrrr|rrr|rr}
            \toprule
            {}                & C       & CC      & CS      & CB      & WL      & NL      & porn   & \# sentences & avg length \\
            \midrule
            yo                & 84.69\% & 71.43\% & 2.04\%  & 11.22\% & 14.29\% & 1.02\%  & 0.00\% & 46214        & 117.71     \\
            st                & 56.70\% & 42.27\% & 14.43\% & 0.00\%  & 35.05\% & 8.25\%  & 0.00\% & 66837        & 132.13     \\
            haw               & 44.90\% & 34.69\% & 1.02\%  & 9.18\%  & 33.67\% & 21.43\% & 1.02\% & 84312        & 129.99     \\
            ig                & 55.91\% & 41.73\% & 10.24\% & 3.94\%  & 0.00\%  & 44.09\% & 0.79\% & 92909        & 98.03      \\
            sm                & 60.20\% & 58.16\% & 2.04\%  & 0.00\%  & 27.55\% & 12.24\% & 0.00\% & 98467        & 126.42     \\
            ha                & 80.81\% & 79.80\% & 1.01\%  & 0.00\%  & 14.14\% & 5.05\%  & 2.02\% & 247479       & 155.76     \\
            su                & 59.60\% & 58.59\% & 1.01\%  & 0.00\%  & 25.25\% & 15.15\% & 2.02\% & 280719       & 107.10     \\
            sn                & 36.63\% & 32.67\% & 2.97\%  & 0.99\%  & 58.42\% & 4.95\%  & 0.00\% & 326392       & 145.59     \\
            mg                & 57.00\% & 57.00\% & 0.00\%  & 0.00\%  & 18.00\% & 25.00\% & 0.00\% & 345040       & 116.23     \\
            pa                & 78.30\% & 68.87\% & 3.77\%  & 5.66\%  & 4.72\%  & 10.38\% & 0.00\% & 363399       & 134.43     \\
            ga                & 76.77\% & 58.59\% & 6.06\%  & 12.12\% & 10.10\% & 13.13\% & 0.00\% & 465670       & 147.35     \\
            co                & 33.00\% & 29.00\% & 2.00\%  & 2.00\%  & 48.00\% & 19.00\% & 0.00\% & 494913       & 195.30     \\
            zu                & 51.00\% & 48.00\% & 2.00\%  & 1.00\%  & 30.00\% & 19.00\% & 0.00\% & 555458       & 137.81     \\
            jv                & 52.73\% & 19.09\% & 19.09\% & 14.55\% & 40.00\% & 7.27\%  & 1.82\% & 581528       & 97.96      \\
            km                & 92.86\% & 92.86\% & 0.00\%  & 0.00\%  & 7.14\%  & 0.00\%  & 0.00\% & 756612       & 162.57     \\
            kn                & 85.15\% & 73.27\% & 3.96\%  & 7.92\%  & 2.97\%  & 9.90\%  & 0.00\% & 1056849      & 105.39     \\
            fy                & 56.73\% & 50.00\% & 3.85\%  & 2.88\%  & 39.42\% & 3.85\%  & 0.00\% & 1104359      & 234.25     \\
            te                & 89.00\% & 76.00\% & 9.00\%  & 4.00\%  & 3.00\%  & 8.00\%  & 0.00\% & 1188243      & 108.49     \\
            la                & 82.31\% & 65.38\% & 6.15\%  & 10.77\% & 10.00\% & 7.69\%  & 0.00\% & 1674463      & 67.25      \\
            be                & 92.04\% & 86.73\% & 2.65\%  & 2.65\%  & 4.42\%  & 3.54\%  & 0.00\% & 1742030      & 110.86     \\
            af                & 76.00\% & 76.00\% & 0.00\%  & 0.00\%  & 15.00\% & 9.00\%  & 0.00\% & 2152243      & 99.52      \\
            \textbf{lb}       & 17.48\% & 17.48\% & 0.00\%  & 0.00\%  & 7.77\%  & 74.76\% & 0.00\% & 2740336      & 481.68     \\
            ne                & 78.35\% & 77.32\% & 1.03\%  & 0.00\%  & 21.65\% & 0.00\%  & 0.00\% & 2942785      & 102.88     \\
            sr                & 93.69\% & 85.59\% & 7.21\%  & 0.90\%  & 5.41\%  & 0.00\%  & 0.00\% & 3398483      & 131.72     \\
            gl                & 67.62\% & 57.14\% & 10.48\% & 0.00\%  & 13.33\% & 17.14\% & 0.00\% & 4549465      & 151.45     \\
            bn                & 93.00\% & 86.00\% & 1.00\%  & 6.00\%  & 3.00\%  & 4.00\%  & 0.00\% & 7444098      & 92.60      \\
            mr                & 40.00\% & 35.24\% & 2.86\%  & 1.90\%  & 49.52\% & 10.48\% & 0.00\% & 7774331      & 281.94     \\
            sl                & 92.08\% & 82.18\% & 4.95\%  & 4.95\%  & 2.97\%  & 4.95\%  & 0.00\% & 8499456      & 149.45     \\
            hi                & 80.30\% & 76.77\% & 1.01\%  & 2.53\%  & 19.70\% & 0.00\%  & 2.53\% & 18507273     & 105.54     \\
            bg                & 80.90\% & 75.88\% & 2.51\%  & 2.51\%  & 2.01\%  & 17.09\% & 0.00\% & 23409799     & 93.86      \\
            uk                & 95.48\% & 81.41\% & 7.54\%  & 6.53\%  & 2.01\%  & 2.51\%  & 0.00\% & 38556465     & 116.79     \\
            ro                & 94.95\% & 78.79\% & 12.12\% & 4.04\%  & 3.03\%  & 2.02\%  & 0.00\% & 45738857     & 130.08     \\
            sv                & 91.18\% & 84.31\% & 2.94\%  & 3.92\%  & 4.90\%  & 3.92\%  & 1.96\% & 48570979     & 114.45     \\
            zh                & 92.00\% & 87.00\% & 1.00\%  & 4.00\%  & 1.00\%  & 7.00\%  & 0.00\% & 54542308     & 94.77      \\
            ja                & 99.00\% & 89.00\% & 6.00\%  & 4.00\%  & 0.00\%  & 1.00\%  & 1.00\% & 87337884     & 59.94      \\
            tr                & 95.96\% & 88.89\% & 0.00\%  & 7.07\%  & 3.54\%  & 0.51\%  & 0.00\% & 87595290     & 152.75     \\
            nl                & 92.08\% & 85.15\% & 6.93\%  & 0.00\%  & 1.98\%  & 5.94\%  & 0.00\% & 96210458     & 103.67     \\
            pl                & 96.00\% & 82.00\% & 7.00\%  & 7.00\%  & 2.00\%  & 2.00\%  & 0.00\% & 126164277    & 170.70     \\
            pt                & 86.00\% & 79.00\% & 4.00\%  & 3.00\%  & 2.00\%  & 12.00\% & 1.00\% & 169239084    & 133.51     \\
            it                & 92.00\% & 79.00\% & 9.00\%  & 4.00\%  & 1.00\%  & 7.00\%  & 0.00\% & 186404508    & 180.26     \\
            fr                & 92.00\% & 82.00\% & 7.00\%  & 3.00\%  & 1.00\%  & 7.00\%  & 0.00\% & 332674575    & 143.69     \\
            de                & 91.18\% & 77.45\% & 7.84\%  & 5.88\%  & 6.86\%  & 1.96\%  & 0.00\% & 397006993    & 107.71     \\
            ru                & 91.06\% & 69.11\% & 11.38\% & 10.57\% & 4.07\%  & 4.88\%  & 0.00\% & 755585265    & 109.28     \\
            en                & 93.94\% & 83.84\% & 8.08\%  & 2.02\%  & 1.01\%  & 5.05\%  & 0.00\% & 3079081989   & 130.97     \\
            \textbf{bg\_latn} & 9.09\%  & 9.09\%  & 0.00\%  & 0.00\%  & 51.52\% & 39.39\% & 1.01\% & N/A          & 139.92     \\
            \textbf{ja\_latn} & 13.00\% & 7.00\%  & 4.00\%  & 2.00\%  & 60.00\% & 27.00\% & 0.00\% & N/A          & 218.92     \\
            ru\_latn          & 36.45\% & 25.23\% & 10.28\% & 0.93\%  & 34.58\% & 28.97\% & 0.93\% & N/A          & 123.14     \\
            \textbf{zh\_latn} & 5.00\%  & 4.00\%  & 1.00\%  & 0.00\%  & 64.00\% & 31.00\% & 0.00\% & N/A          & 186.84     \\
            \bottomrule
        \end{tabular}%
    }
    \caption{Audit results for a sample of 100 sentences from \textbf{mC4} for each language, compared to the number of sentences available in the dataset. Language codes are as originally published. The length is measured in number of characters and averaged across the audited portion of each corpus. Languages with less than 20\% correct sentences are boldfaced.}
    \label{tab:mc4-full}
\end{table*}
\clearpage


%%% OSCAR %%%
\begin{table*}[hbt!]
    \centering
    \resizebox*{0.9\textwidth}{!}{%
        \begin{tabular}{l|rrrr|rrr|rr}
            \toprule
            {}           & C        & CC       & CS     & CB      & WL       & NL       & porn   & \# sentences & avg length \\
            \midrule
            diq          & 100.00\% & 100.00\% & 0.00\% & 0.00\%  & 0.00\%   & 0.00\%   & 0.00\% & 1            & 131.00     \\
            \textbf{bcl} & 0.00\%   & 0.00\%   & 0.00\% & 0.00\%  & 0.00\%   & 100.00\% & 0.00\% & 1            & 623.00     \\
            \textbf{cbk} & 0.00\%   & 0.00\%   & 0.00\% & 0.00\%  & 100.00\% & 0.00\%   & 0.00\% & 1            & 519.00     \\
            pam          & 100.00\% & 100.00\% & 0.00\% & 0.00\%  & 0.00\%   & 0.00\%   & 0.00\% & 2            & 139.00     \\
            bar          & 25.00\%  & 25.00\%  & 0.00\% & 0.00\%  & 0.00\%   & 75.00\%  & 0.00\% & 4            & 53.50      \\
            myv          & 100.00\% & 100.00\% & 0.00\% & 0.00\%  & 0.00\%   & 0.00\%   & 0.00\% & 5            & 127.00     \\
            \textbf{yue} & 0.00\%   & 0.00\%   & 0.00\% & 0.00\%  & 57.14\%  & 42.86\%  & 0.00\% & 7            & 177.00     \\
            mwl          & 57.14\%  & 57.14\%  & 0.00\% & 0.00\%  & 42.86\%  & 0.00\%   & 0.00\% & 7            & 141.00     \\
            \textbf{frr} & 0.00\%   & 0.00\%   & 0.00\% & 0.00\%  & 0.00\%   & 100.00\% & 0.00\% & 9            & 231.56     \\
            ht           & 30.00\%  & 30.00\%  & 0.00\% & 0.00\%  & 0.00\%   & 70.00\%  & 0.00\% & 10           & 329.10     \\
            ie           & 30.00\%  & 30.00\%  & 0.00\% & 0.00\%  & 30.00\%  & 40.00\%  & 0.00\% & 11           & 121.70     \\
            scn          & 100.00\% & 100.00\% & 0.00\% & 0.00\%  & 0.00\%   & 0.00\%   & 0.00\% & 17           & 155.59     \\
            tyv          & 96.15\%  & 96.15\%  & 0.00\% & 0.00\%  & 0.00\%   & 3.85\%   & 0.00\% & 26           & 167.96     \\
            mai          & 79.31\%  & 75.86\%  & 0.00\% & 3.45\%  & 20.69\%  & 0.00\%   & 0.00\% & 29           & 141.17     \\
            bxr          & 100.00\% & 100.00\% & 0.00\% & 0.00\%  & 0.00\%   & 0.00\%   & 0.00\% & 37           & 160.76     \\
            dsb          & 100.00\% & 97.56\%  & 0.00\% & 2.44\%  & 0.00\%   & 0.00\%   & 0.00\% & 41           & 155.15     \\
            \textbf{so}  & 0.00\%   & 0.00\%   & 0.00\% & 0.00\%  & 28.57\%  & 71.43\%  & 0.00\% & 42           & 208.24     \\
            rm           & 100.00\% & 100.00\% & 0.00\% & 0.00\%  & 0.00\%   & 0.00\%   & 0.00\% & 47           & 137.66     \\
            nah          & 100.00\% & 96.67\%  & 0.00\% & 3.33\%  & 0.00\%   & 0.00\%   & 0.00\% & 60           & 164.53     \\
            \textbf{nap} & 0.00\%   & 0.00\%   & 0.00\% & 0.00\%  & 0.00\%   & 100.00\% & 0.00\% & 61           & 152.11     \\
            yo           & 98.46\%  & 96.92\%  & 0.00\% & 1.54\%  & 1.54\%   & 0.00\%   & 0.00\% & 64           & 281.57     \\
            gn           & 81.48\%  & 81.48\%  & 0.00\% & 0.00\%  & 2.47\%   & 16.05\%  & 0.00\% & 81           & 234.95     \\
            vec          & 91.36\%  & 91.36\%  & 0.00\% & 0.00\%  & 0.00\%   & 8.64\%   & 0.00\% & 81           & 184.90     \\
            kw           & 91.57\%  & 90.36\%  & 0.00\% & 1.20\%  & 3.61\%   & 4.82\%   & 0.00\% & 83           & 162.75     \\
            \textbf{wuu} & 0.00\%   & 0.00\%   & 0.00\% & 0.00\%  & 98.84\%  & 1.16\%   & 0.00\% & 86           & 157.15     \\
            eml          & 42.57\%  & 42.57\%  & 0.00\% & 0.00\%  & 0.00\%   & 57.43\%  & 0.00\% & 104          & 177.88     \\
            bh           & 89.42\%  & 21.15\%  & 0.00\% & 68.27\% & 1.92\%   & 8.65\%   & 0.00\% & 104          & 137.17     \\
            min          & 64.00\%  & 6.00\%   & 0.00\% & 58.00\% & 27.00\%  & 9.00\%   & 0.00\% & 180          & 649.85     \\
            qu           & 100.00\% & 98.97\%  & 0.00\% & 1.03\%  & 0.00\%   & 0.00\%   & 0.00\% & 425          & 167.27     \\
            su           & 99.00\%  & 99.00\%  & 0.00\% & 0.00\%  & 0.00\%   & 1.00\%   & 0.00\% & 676          & 221.00     \\
            jv           & 97.00\%  & 86.00\%  & 0.00\% & 11.00\% & 1.00\%   & 2.00\%   & 0.00\% & 2350         & 203.08     \\
            als          & 93.00\%  & 93.00\%  & 0.00\% & 0.00\%  & 6.00\%   & 1.00\%   & 0.00\% & 7997         & 375.44     \\
            la           & 98.00\%  & 98.00\%  & 0.00\% & 0.00\%  & 2.00\%   & 0.00\%   & 0.00\% & 33838        & 224.11     \\
            uz           & 98.00\%  & 98.00\%  & 0.00\% & 0.00\%  & 2.00\%   & 0.00\%   & 0.00\% & 34244        & 369.99     \\
            nds          & 97.03\%  & 95.05\%  & 0.00\% & 1.98\%  & 2.97\%   & 0.00\%   & 0.00\% & 35032        & 344.74     \\
            sw           & 98.00\%  & 98.00\%  & 0.00\% & 0.00\%  & 0.00\%   & 2.00\%   & 0.00\% & 40066        & 196.70     \\
            br           & 100.00\% & 96.00\%  & 0.00\% & 4.00\%  & 0.00\%   & 0.00\%   & 0.00\% & 61941        & 239.56     \\
            fy           & 97.00\%  & 97.00\%  & 0.00\% & 0.00\%  & 2.00\%   & 1.00\%   & 0.00\% & 67762        & 340.23     \\
            am           & 81.09\%  & 79.10\%  & 0.00\% & 1.99\%  & 18.91\%  & 0.00\%   & 0.00\% & 287142       & 267.43     \\
            af           & 100.00\% & 100.00\% & 0.00\% & 0.00\%  & 0.00\%   & 0.00\%   & 0.00\% & 517353       & 339.18     \\
            eu           & 100.00\% & 98.00\%  & 0.00\% & 2.00\%  & 0.00\%   & 0.00\%   & 0.00\% & 1099498      & 330.93     \\
            mn           & 98.00\%  & 94.00\%  & 0.00\% & 4.00\%  & 2.00\%   & 0.00\%   & 0.00\% & 1430527      & 309.94     \\
            te           & 98.99\%  & 93.94\%  & 1.01\% & 4.04\%  & 0.00\%   & 1.01\%   & 1.01\% & 1685185      & 412.31     \\
            kk           & 100.00\% & 100.00\% & 0.00\% & 0.00\%  & 0.00\%   & 0.00\%   & 0.00\% & 2719851      & 318.93     \\
            ca           & 99.00\%  & 91.00\%  & 0.00\% & 8.00\%  & 1.00\%   & 0.00\%   & 0.00\% & 13292843     & 333.38     \\
            nl           & 98.00\%  & 94.00\%  & 2.00\% & 2.00\%  & 2.00\%   & 0.00\%   & 4.00\% & 126067610    & 305.01     \\
            it           & 87.13\%  & 71.29\%  & 1.98\% & 13.86\% & 11.88\%  & 0.99\%   & 1.98\% & 210348435    & 393.66     \\
            zh           & 100.00\% & 97.00\%  & 0.00\% & 3.00\%  & 0.00\%   & 0.00\%   & 1.00\% & 232673578    & 195.60     \\
            fr           & 100.00\% & 93.00\%  & 0.00\% & 7.00\%  & 0.00\%   & 0.00\%   & 5.00\% & 461349575    & 306.62     \\
            es           & 100.00\% & 94.00\%  & 0.00\% & 6.00\%  & 0.00\%   & 0.00\%   & 3.00\% & 488616724    & 268.07     \\
            en           & 99.00\%  & 96.00\%  & 0.00\% & 3.00\%  & 0.00\%   & 1.00\%   & 1.00\% & 3809525119   & 364.65     \\
            \bottomrule
        \end{tabular}%
    }
    \caption{Audit results for a sample of 100 sentences from \textbf{OSCAR} for each language, compared to the number of sentences available in the dataset. If fewer than 100 sentences were available, all sentences were audited Language codes are as originally published. Length is measured in number of characters. Languages with less than 20\% correct sentences are boldfaced.}
    \label{tab:oscar-full}
\end{table*}




\begin{table*}[hbt!]
    \centering
    \resizebox*{0.9\textwidth}{!}{ %\textheight}{%
        \begin{tabular}{l|l|rrrr|rrrr|rr}
            \toprule
            corpus     & language                & C        & CC       & CS      & CB      & X       & WL      & NL      & porn    & \#sentences & avg target length \\
            \midrule
            CCAligned  & \textbf{en-tz\_MA}      & 12.12\%  & 6.06\%   & 6.06\%  & 0.00\%  & 45.45\% & 36.36\% & 6.06\%  & 0.00\%  & 33          & 57.33             \\
            CCAligned  & \textbf{en-kg\_AO}      & 1.35\%   & 0.00\%   & 1.35\%  & 0.00\%  & 14.86\% & 2.70\%  & 81.08\% & 0.00\%  & 74          & 29.20             \\
            CCAligned  & \textbf{en-bm\_ML}      & 6.04\%   & 4.03\%   & 2.01\%  & 0.00\%  & 26.85\% & 6.71\%  & 60.40\% & 0.00\%  & 149         & 32.19             \\
            CCAligned  & \textbf{en-ak\_GH}      & 14.23\%  & 13.60\%  & 0.63\%  & 0.00\%  & 46.86\% & 19.25\% & 19.67\% & 0.00\%  & 478         & 45.85             \\
            CCAligned  & en-st\_ZA               & 48.57\%  & 42.14\%  & 0.00\%  & 6.43\%  & 40.71\% & 1.43\%  & 9.29\%  & 0.00\%  & 904         & 111.83            \\
            CCAligned  & en-ve\_ZA               & 60.40\%  & 29.70\%  & 21.78\% & 8.91\%  & 28.71\% & 3.96\%  & 6.93\%  & 0.00\%  & 1555        & 82.99             \\
            CCAligned  & en-ts\_ZA               & 51.49\%  & 34.65\%  & 11.88\% & 4.95\%  & 40.59\% & 2.97\%  & 4.95\%  & 0.00\%  & 1967        & 73.93             \\
            CCAligned  & \textbf{en-ns\_ZA }     & 4.00\%   & 2.00\%   & 0.00\%  & 2.00\%  & 23.00\% & 15.00\% & 58.00\% & 4.00\%  & 14138       & 33.52             \\
            CCAligned  & \textbf{en-lg\_UG}      & 6.00\%   & 0.00\%   & 6.00\%  & 0.00\%  & 68.00\% & 17.00\% & 9.00\%  & 2.00\%  & 14701       & 15.83             \\
            CCAligned  & \textbf{en-ln\_CD}      & 8.00\%   & 4.00\%   & 3.00\%  & 1.00\%  & 14.00\% & 4.00\%  & 74.00\% & 4.00\%  & 21562       & 28.80             \\
            CCAligned  & \textbf{en-om\_KE}      & 2.00\%   & 2.00\%   & 0.00\%  & 0.00\%  & 31.00\% & 38.00\% & 29.00\% & 24.00\% & 22206       & 23.83             \\
            CCAligned  & \textbf{en-ss\_SZ}      & 12.65\%  & 9.04\%   & 3.61\%  & 0.00\%  & 13.25\% & 24.10\% & 50.00\% & 13.86\% & 22960       & 25.30             \\
            CCAligned  & \textbf{en-tn\_BW}      & 0.00\%   & 0.00\%   & 0.00\%  & 0.00\%  & 6.90\%  & 8.97\%  & 63.45\% & 10.34\% & 71253       & 16.80             \\
            CCAligned  & \textbf{en-ff\_NG}      & 0.00\%   & 0.00\%   & 0.00\%  & 0.00\%  & 0.00\%  & 8.00\%  & 92.00\% & 2.00\%  & 73022       & 33.59             \\
            CCAligned  & \textbf{en-sn\_ZW}      & 5.00\%   & 1.00\%   & 3.00\%  & 1.00\%  & 81.00\% & 14.00\% & 0.00\%  & 0.00\%  & 86868       & 102.59            \\
            CCAligned  & \textbf{en-wo\_SN}      & 0.00\%   & 0.00\%   & 0.00\%  & 0.00\%  & 1.71\%  & 3.31\%  & 94.98\% & 18.46\% & 88441       & 27.25             \\
            CCAligned  & en-zu\_ZA               & 55.00\%  & 39.00\%  & 3.00\%  & 13.00\% & 30.00\% & 7.00\%  & 8.00\%  & 3.00\%  & 126101      & 79.32             \\
            CCAligned  & en-ig\_NG               & 58.00\%  & 49.00\%  & 3.00\%  & 6.00\%  & 29.00\% & 12.00\% & 1.00\%  & 0.00\%  & 148146      & 83.42             \\
            CCAligned  & en-yo\_NG               & 34.93\%  & 6.16\%   & 10.96\% & 17.81\% & 34.93\% & 12.33\% & 17.81\% & 0.00\%  & 175192      & 75.01             \\
            CCAligned  & en-ha\_NG               & 30.00\%  & 25.00\%  & 3.00\%  & 2.00\%  & 49.00\% & 9.00\%  & 12.00\% & 1.00\%  & 339176      & 60.78             \\
            CCAligned  & en-am\_ET               & 59.11\%  & 35.47\%  & 2.46\%  & 21.18\% & 37.44\% & 2.96\%  & 0.49\%  & 0.00\%  & 346517      & 58.29             \\
            CCAligned  & en-af\_ZA               & 63.00\%  & 40.00\%  & 23.00\% & 0.00\%  & 31.00\% & 2.00\%  & 4.00\%  & 12.00\% & 1504061     & 105.45            \\
            CCAligned  & \textbf{en-ar\_AR\_rom} & 0.00\%   & 0.00\%   & 0.00\%  & 0.00\%  & 0.00\%  & 4.00\%  & 96.00\% & 4.00\%  & 5584724     & 16.69             \\
            Wikimatrix & en-sw                   & 33.33\%  & 27.27\%  & 4.04\%  & 2.02\%  & 64.65\% & 2.02\%  & 0.00\%  & 0.00\%  & 138590      & 111.61            \\
            ParaCrawl  & en-so                   & 80.81\%  & 61.62\%  & 1.01\%  & 18.18\% & 14.14\% & 5.05\%  & 0.00\%  & 0.00\%  & 14879       & 189.83            \\
            ParaCrawl  & en-sw                   & 85.00\%  & 60.00\%  & 15.00\% & 10.00\% & 11.00\% & 2.00\%  & 2.00\%  & 0.00\%  & 132517      & 167.34            \\
            MC4        & yo                      & 84.69\%  & 71.43\%  & 2.04\%  & 11.22\% & N/A     & 14.29\% & 1.02\%  & 0.00\%  & 46214       & 117.71            \\
            MC4        & st                      & 56.70\%  & 42.27\%  & 14.43\% & 0.00\%  & N/A     & 35.05\% & 8.25\%  & 0.00\%  & 66837       & 132.13            \\
            MC4        & ig                      & 55.91\%  & 41.73\%  & 10.24\% & 3.94\%  & N/A     & 0.00\%  & 44.09\% & 0.79\%  & 92909       & 98.03             \\
            MC4        & ha                      & 80.81\%  & 79.80\%  & 1.01\%  & 0.00\%  & N/A     & 14.14\% & 5.05\%  & 2.02\%  & 247479      & 155.76            \\
            MC4        & sn                      & 36.63\%  & 32.67\%  & 2.97\%  & 0.99\%  & N/A     & 58.42\% & 4.95\%  & 0.00\%  & 326392      & 145.59            \\
            MC4        & mg                      & 57.00\%  & 57.00\%  & 0.00\%  & 0.00\%  & N/A     & 18.00\% & 25.00\% & 0.00\%  & 345040      & 116.23            \\
            MC4        & zu                      & 51.00\%  & 48.00\%  & 2.00\%  & 1.00\%  & N/A     & 30.00\% & 19.00\% & 0.00\%  & 555458      & 137.81            \\
            MC4        & af                      & 76.00\%  & 76.00\%  & 0.00\%  & 0.00\%  & N/A     & 15.00\% & 9.00\%  & 0.00\%  & 2152243     & 99.52             \\
            OSCAR      & \textbf{so}             & 0.00\%   & 0.00\%   & 0.00\%  & 0.00\%  & N/A     & 28.57\% & 71.43\% & 0.00\%  & 42          & 208.24            \\
            OSCAR      & yo                      & 98.46\%  & 96.92\%  & 0.00\%  & 1.54\%  & N/A     & 1.54\%  & 0.00\%  & 0.00\%  & 64          & 281.57            \\
            OSCAR      & sw                      & 98.00\%  & 98.00\%  & 0.00\%  & 0.00\%  & N/A     & 0.00\%  & 2.00\%  & 0.00\%  & 40066       & 196.70            \\
            OSCAR      & am                      & 81.09\%  & 79.10\%  & 0.00\%  & 1.99\%  & N/A     & 18.91\% & 0.00\%  & 0.00\%  & 287142      & 267.43            \\
            OSCAR      & af                      & 100.00\% & 100.00\% & 0.00\%  & 0.00\%  & N/A     & 0.00\%  & 0.00\%  & 0.00\%  & 517353      & 339.18            \\
            \bottomrule
        \end{tabular}%
    }
    \caption{Results on African languages.}

    \label{tab:africa-full}
\end{table*}

%%%%%%%%%%%%%%%%%%%%%%%%%%%%%%%%%%%%%%%%%%%%%%%%%%%%%%%%%%%%%%%%%%%%%%%%
\chapter{Towards a Cleaner Document-Oriented Annotated OSCAR Corpus}
%%%%%%%%%%%%%%%%%%%%%%%%%%%%%%%%%%%%%%%%%%%%%%%%%%%%%%%%%%%%%%%%%%%%%%%%

\section{Carbon Footprint}\label{carbon_footprint_towards}

We use a single machine having 192 GB of RAM and two Intel Xeon Gold 5218 processors, which is rated at 125 W,\footnote{\href{https://ark.intel.com/content/www/us/en/ark/products/192444/intel-xeon-gold-5218-processor-22m-cache-2-30-ghz.html}{Intel Xeon Gold 5218 specification}}. For the DRAM we can use the work of \newcite{desrochers-etal-2016-a} to estimate the total power draw of 192GB of RAM at around 20W. The total power draw of this setting adds up to around 270 W.

Having this information, we can now use the formula proposed by \newcite{strubell-etal-2019-energy} in order to compute the total power required to pre-train one model from scratch:
\[
    p_t = \frac{1.58t(cp_{c} + p_r)}{1000}
\]
Where $c$ is the number of CPUs, $p_c$ is the average power draw (in Watts) from all CPU sockets and $p_r$ the average power draw from all DRAM sockets. We estimate the total power consumption by adding CPU and DRAM consumption, and then multiplying by the \emph{Power Usage Effectiveness} (PUE), which accounts for the additional energy required to support the compute infrastructure. We use a PUE coefficient of 1.58, the 2018 global average for data centers \cite{strubell-etal-2019-energy}. The total time to generate OSCAR 22.01 in this infrastructure was of 42.6 hours. We use this information to compute the total power consumption of the OSCAR generation, which amounts to 0.4266 \unit{\kWh}.

We can further estimate the CO\textsubscript{2} emissions in kilograms of the OSCAR generation by multiplying the total power consumption by the average CO\textsubscript{2} emissions per \unit{\kWh} in our region which were 38.64\unit{\gram/\kWh} in average between the 3rd and the 5th of January 2022\footnote{\href{https://www.rte-france.com/eco2mix/les-emissions-de-co2-par-kwh-produit-en-france}{Rte - éCO\textsubscript{2}mix}.}, the exact time at which the generation was run. Thus the total CO\textsubscript{2} emissions in kg for one single model can be computed as:
\[
    \text{CO}_{2}\text{e} = 0.03864 p_t
\]
Thus total CO\textsubscript{2} emissions amount to 0.01648\unit{\kilo\gram} or 16.48\unit{\gram}.

\section{Language Table}

\begin{table}[ht!]
    \centering\tiny
    \resizebox{\linewidth}{!}{
        \begin{tabular}{lrrrclrrr}
            \toprule
            Language                    & Size      & Documents   & Words           & ~ & Language          & Size      & Documents  & Words          \\
            \midrule
            Afrikaans                   & 47.0 MB   & 12,393      & 6,227,310       & ~ & Luxembourgish     & 15.8 MB   & 5,108      & 1,545,946      \\
            Tosk Albanian               & 363.6 kB  & 139         & 37,381          & ~ & Lezghian          & 375.5 kB  & 124        & 19,250         \\
            Amharic                     & 461.0 MB  & 37,513      & 30,481,153      & ~ & Limburgish        & 1.4 kB    & 2          & 41             \\
            Aragonese                   & 10.6 kB   & 12          & 51              & ~ & Lombard           & 2.6 kB    & 2          & 225            \\
            Arabic                      & 84.2 GB   & 8,718,929   & 6,103,711,887   & ~ & Lao               & 337.1 MB  & 28,914     & 6,682,982      \\
            Egyptian Arabic             & 2.8 MB    & 1,256       & 176,096         & ~ & Lithuanian        & 20.0 GB   & 2,303,070  & 1,712,802,056  \\
            Assamese                    & 221.2 MB  & 17,084      & 11,109,557      & ~ & Latvian           & 8.2 GB    & 1,032,987  & 707,361,898    \\
            Asturian                    & 73.6 kB   & 77          & 3,919           & ~ & Maithili          & 21.6 kB   & 23         & 483            \\
            Avaric                      & 18.6 kB   & 14          & 582             & ~ & Malagasy          & 57.3 MB   & 3,028      & 7,279,056      \\
            Azerbaijani                 & 3.5 GB    & 491,847     & 291,927,692     & ~ & Eastern Mari      & 11.3 MB   & 1,612      & 641,525        \\
            South Azerbaijani           & 14.1 MB   & 5,381       & 693,746         & ~ & Minangkabau       & 6.0 MB    & 585        & 614,613        \\
            Bashkir                     & 95.5 MB   & 11,198      & 5,418,474       & ~ & Macedonian        & 3.6 GB    & 341,775    & 244,058,579    \\
            Belarusian                  & 1.8 GB    & 180,046     & 107,227,860     & ~ & Malayalam         & 4.1 GB    & 250,972    & 137,831,247    \\
            Bulgarian                   & 35.1 GB   & 2,887,115   & 2,405,981,285   & ~ & Mongolian         & 2.8 GB    & 237,719    & 176,405,432    \\
            Bihari languages            & 24.2 kB   & 27          & 569             & ~ & Marathi           & 3.3 GB    & 250,376    & 160,179,233    \\
            Bangla                      & 15.1 GB   & 1,171,501   & 751,877,226     & ~ & Western Mari      & 743.5 kB  & 155        & 43,916         \\
            Tibetan                     & 234.5 MB  & 18,683      & 2,286,269       & ~ & Malay             & 5.3 MB    & 5,228      & 217,818        \\
            Bishnupriya                 & 2.0 MB    & 271         & 98,419          & ~ & Maltese           & 2.5 MB    & 2,208      & 118,190        \\
            Breton                      & 33.7 MB   & 16,119      & 3,111,619       & ~ & Multilingual      & 12.1 GB   & 1,210,685  & 936,187,711    \\
            Bosnian                     & 10.3 kB   & 10          & 422             & ~ & Burmese           & 1.9 GB    & 158,733    & 44,835,970     \\
            Russia Buriat               & 32.9 kB   & 39          & 785             & ~ & Mazanderani       & 128.2 kB  & 76         & 7,337          \\
            Catalan                     & 13.9 GB   & 2,627,307   & 1,508,919,864   & ~ & Nahuatl languages & 8.7 kB    & 12         & 179            \\
            Chechen                     & 14.0 MB   & 4,086       & 798,766         & ~ & Low German        & 9.0 MB    & 1,938      & 1,012,561      \\
            Cebuano                     & 44.6 MB   & 5,742       & 5,253,785       & ~ & Nepali            & 3.7 GB    & 391,947    & 177,885,116    \\
            Central Kurdish             & 716.4 MB  & 84,950      & 43,913,025      & ~ & Newari            & 5.7 MB    & 1,134      & 273,837        \\
            Czech                       & 58.6 GB   & 10,381,916  & 5,452,724,456   & ~ & Dutch             & 114.0 GB  & 20,206,532 & 12,329,127,151 \\
            Chuvash                     & 41.8 MB   & 4,750       & 2,465,782       & ~ & Norwegian Nynorsk & 6.8 MB    & 5,835      & 459,183        \\
            Welsh                       & 409.3 MB  & 90,378      & 49,488,495      & ~ & Norwegian         & 2.8 GB    & 973,188    & 279,182,902    \\
            Danish                      & 12.6 GB   & 2,265,479   & 1,454,439,292   & ~ & Occitan           & 2.1 MB    & 373        & 31,061         \\
            German                      & 496.7 GB  & 70,075,424  & 46,826,676,844  & ~ & Odia              & 487.9 MB  & 52,942     & 23,755,902     \\
            Dimli (individual language) & 706 Bytes & 1           & 19              & ~ & Ossetic           & 13.9 MB   & 3,560      & 800,430        \\
            Lower Sorbian               & 707 Bytes & 1           & 17              & ~ & Punjabi           & 1.1 GB    & 68,094     & 70,068,604     \\
            Divehi                      & 217.2 MB  & 24,067      & 10,112,205      & ~ & Polish            & 139.0 GB  & 19,301,137 & 12,584,498,906 \\
            Greek                       & 78.3 GB   & 6,738,546   & 5,031,242,803   & ~ & Piedmontese       & 1.7 MB    & 698        & 188,270        \\
            Emiliano-Romagnolo.         & 901 Bytes & 1           & 53              & ~ & Western Panjabi   & 46.7 MB   & 6,790      & 4,060,419      \\
            English                     & 3.2 TB    & 431,992,659 & 377,376,402,775 & ~ & Pashto            & 490.3 MB  & 50,312     & 46,293,249     \\
            Esperanto                   & 558.3 MB  & 111,932     & 58,416,628      & ~ & Portuguese        & 170.3 GB  & 23,735,707 & 18,441,864,893 \\
            Spanish                     & 381.9 GB  & 51,386,247  & 42,829,835,316  & ~ & Quechua           & 744 Bytes & 1          & 14             \\
            Estonian                    & 9.2 GB    & 1,362,524   & 820,975,443     & ~ & Romanian          & 49.2 GB   & 4,624,764  & 5,261,803,995  \\
            Basque                      & 1.1 GB    & 233,658     & 97,092,942      & ~ & Russian           & 1.1 TB    & 76,060,844 & 62,811,122,663 \\
            Persian                     & 77.4 GB   & 7,665,871   & 6,430,164,396   & ~ & Sanskrit          & 136.0 MB  & 4,472      & 5,671,369      \\
            Finnish                     & 37.8 GB   & 4,948,961   & 2,900,615,928   & ~ & Sakha             & 65.6 MB   & 6,284      & 3,473,813      \\
            French                      & 382.2 GB  & 52,037,098  & 41,713,990,658  & ~ & Sicilian          & 1.5 kB    & 2          & 50             \\
            Western Frisian             & 75.3 MB   & 21,946      & 6,357,929       & ~ & Sindhi            & 117.1 MB  & 15,516     & 10,685,611     \\
            Irish                       & 45.6 MB   & 12,233      & 4,877,850       & ~ & Serbian (Latin)   & 931.8 kB  & 738        & 92,875         \\
            Scottish Gaelic             & 137.7 kB  & 136         & 7,769           & ~ & Sinhala           & 2.0 GB    & 108,593    & 113,179,741    \\
            Galician                    & 255.2 MB  & 88,803      & 27,051,212      & ~ & Slovak            & 16.5 GB   & 2,409,555  & 1,619,121,944  \\
            Guarani                     & 9.0 kB    & 10          & 374             & ~ & Slovenian         & 1.2 GB    & 351,894    & 118,400,246    \\
            Goan Konkani                & 787.2 kB  & 46          & 38,831          & ~ & Somali            & 2.1 kB    & 3          & 109            \\
            Gujarati                    & 4.8 GB    & 136,467     & 301,170,777     & ~ & Albanian          & 3.0 GB    & 437,287    & 326,325,149    \\
            Hebrew                      & 30.3 GB   & 3,132,396   & 2,249,377,984   & ~ & Serbian           & 6.9 GB    & 577,472    & 482,932,670    \\
            Hindi                       & 23.3 GB   & 1,529,907   & 1,534,799,198   & ~ & Sundanese         & 5.0 MB    & 263        & 547,145        \\
            Croatian                    & 11.2 MB   & 11,462      & 505,369         & ~ & Swedish           & 48.0 GB   & 7,541,278  & 5,078,331,128  \\
            Upper Sorbian               & 132.8 kB  & 110         & 8,825           & ~ & Swahili           & 1.3 MB    & 462        & 123,050        \\
            Hungarian                   & 53.9 GB   & 6,866,062   & 4,598,787,907   & ~ & Tamil             & 11.4 GB   & 556,772    & 452,343,748    \\
            Armenian                    & 4.7 GB    & 379,267     & 268,031,270     & ~ & Telugu            & 3.4 GB    & 249,756    & 137,752,065    \\
            Interlingua                 & 40.2 kB   & 6           & 10,125          & ~ & Tajik             & 870.9 MB  & 46,366     & 56,627,727     \\
            Indonesian                  & 17.4 GB   & 2,244,622   & 1,984,195,207   & ~ & Thai              & 66.1 GB   & 5,030,254  & 1,626,779,846  \\
            Iloko                       & 97.9 kB   & 75          & 8,592           & ~ & Turkmen           & 4.4 MB    & 2,485      & 276,632        \\
            Ido                         & 77.3 kB   & 105         & 2,690           & ~ & Filipino          & 646.5 MB  & 70,394     & 81,881,278     \\
            Icelandic                   & 2.0 GB    & 396,183     & 210,365,124     & ~ & Turkish           & 75.1 GB   & 10,826,031 & 6,421,221,358  \\
            Italian                     & 229.3 GB  & 28,502,092  & 24,294,684,830  & ~ & Tatar             & 915.3 MB  & 76,398     & 51,875,265     \\
            Japanese                    & 258.7 GB  & 36,328,931  & 5,592,948,356   & ~ & Uyghur            & 201.9 MB  & 18,556     & 11,240,889     \\
            Lojban                      & 1.9 MB    & 570         & 260,542         & ~ & Ukrainian         & 48.8 GB   & 4,558,214  & 2,879,585,992  \\
            Javanese                    & 152.7 kB  & 70          & 10,441          & ~ & Urdu              & 3.4 GB    & 336,994    & 332,816,354    \\
            Georgian                    & 7.1 GB    & 488,588     & 281,430,479     & ~ & Uzbek             & 19.9 MB   & 9,526      & 1,370,842      \\
            Kazakh                      & 2.9 GB    & 261,085     & 157,267,307     & ~ & Vietnamese        & 98.9 GB   & 9,587,233  & 12,283,185,482 \\
            Khmer                       & 1.9 GB    & 121,910     & 30,564,131      & ~ & Volapük           & 825.9 kB  & 661        & 57,039         \\
            Kannada                     & 2.6 GB    & 150,850     & 108,450,571     & ~ & Walloon           & 105.7 kB  & 138        & 4,386          \\
            Korean                      & 51.8 GB   & 5,881,481   & 3,854,968,649   & ~ & Waray             & 7.6 MB    & 933        & 830,872        \\
            Karachay-Balkar             & 119.6 kB  & 91          & 4,089           & ~ & Wu Chinese        & 137.2 kB  & 88         & 3,056          \\
            Kurdish                     & 150.3 MB  & 29,906      & 17,390,759      & ~ & Kalmyk            & 9.3 kB    & 9          & 250            \\
            Komi                        & 119.9 kB  & 127         & 3,335           & ~ & Mingrelian        & 7.6 MB    & 2,550      & 253,333        \\
            Cornish                     & 1.4 kB    & 2           & 55              & ~ & Yiddish           & 232.5 MB  & 23,418     & 15,809,780     \\
            Kyrgyz                      & 518.6 MB  & 62,244      & 28,028,986      & ~ & Yoruba            & 24.7 kB   & 26         & 1,042          \\
            Latin                       & 4.1 MB    & 4,397       & 187,446         & ~ & Chinese           & 900.9 GB  & 56,524,518 & 23,149,203,886 \\
            \bottomrule
        \end{tabular}
    }
    \caption{Size of the OSCAR 22.01 corpus by language measured in bytes and number of words. Standard UNIX human-readable notation is used for the size in byte. We define ``words'' as spaced separated tokens, which gives a good estimate of the size of each corpus for languages using Latin or Cyrillic alphabets, but might give a misleading size for other languages such as Chinese or Japanese.}
    \label{tab:langs_towards}
\end{table}

%%%%%%%%%%%%%%%%%%%%%%%%%%%%%%%%%%%%%%%%%%%%%%%%%%%%%%%%%%%%%%%%%%%%%%%%
\chapter{CamemBERT}\label{appendix:camembert}
%%%%%%%%%%%%%%%%%%%%%%%%%%%%%%%%%%%%%%%%%%%%%%%%%%%%%%%%%%%%%%%%%%%%%%%%

\begin{table}[ht]
    \small\centering
    \scalebox{0.65}{
        \begin{tabular}{ l l c c c  c @{\hspace{0.35cm}}  @{\hspace{0.35cm}} c  c @{\hspace{0.35cm}}  @{\hspace{0.35cm}} c  c  @{\hspace{0.35cm}}  @{\hspace{0.35cm}} c  c @{\hspace{0.35cm}}  @{\hspace{0.35cm}} c @{\hspace{0.35cm}}  @{\hspace{0.35cm}} c }
            \toprule
                                                    &                                         &                                       &                                         & \multicolumn{2}{c @{\hspace{0.5cm}}}{\textsc{GSD}} & \multicolumn{2}{c @{\hspace{0.7cm}}}{\textsc{Sequoia}} & \multicolumn{2}{c @{\hspace{0.7cm}}}{\textsc{Spoken}} & \multicolumn{2}{c @{\hspace{0.7cm}}}{\textsc{ParTUT}} & \textsc{NER}      & \textsc{NLI}                                                                                      \\
            \cmidrule(l{2pt}r{0.4cm}){5-6}\cmidrule(l{-0.2cm}r{0.4cm}){7-8}\cmidrule(l{-0.2cm}r{0.4cm}){9-10}\cmidrule(l{-0.2cm}r{0.4cm}){11-12}\cmidrule(l{-0.2cm}r{0.4cm}){13-13} \cmidrule(l{-0.2cm}r{2pt}){14-14}
            \multirow{-2}{*}[2pt]{\textsc{Dataset}} & \multirow{-2}{*}[2pt]{\textsc{Masking}} & \multirow{-2}{*}[2pt]{\textsc{Arch.}} & \multirow{-2}{*}[2pt]{\textsc{\#Steps}} & \textsc{UPOS}                                      & \textsc{LAS}                                           & \textsc{UPOS}                                         & \textsc{LAS}                                          & \textsc{UPOS}     & \textsc{LAS}      & \textsc{UPOS}     & \textsc{LAS}      & \textsc{F1}       & \textsc{Acc.}     \\
            \midrule

            \multicolumn{11}{l}{\hspace*{6mm}\em Fine-tuning}                                                                                                                                                                                                                                                                                                                                                                                                                                                                         \\[0.5mm]
            \toprule
            OSCAR                                   & Subword                                 & \textsc{Base}                         & 100k                                    & \textbf{98.25}                                     & 92.29                                                  & \underline{99.25}                                     & 93.70                                                 & 96.95             & 79.96             & \underline{97.73} & \textbf{92.68}    & 89.23             & 81.18             \\
            OSCAR                                   & Whole-word                              & \textsc{Base}                         & 100k                                    & \underline{98.21}                                  & 92.30                                                  & 99.21                                                 & \underline{94.33}                                     & 96.97             & 80.16             & \textbf{97.78}    & 92.65             & 89.11             & 81.92             \\
            CCNET                                   & Subword                                 & \textsc{Base}                         & 100k                                    & 98.02                                              & 92.06                                                  & \textbf{99.26}                                        & 94.13                                                 & 96.94             & 80.39             & 97.55             & \underline{92.66} & 89.05             & 81.77             \\
            CCNET                                   & Whole-word                              & \textsc{Base}                         & 100k                                    & 98.03                                              & \underline{\textbf{92.43}}                             & 99.18                                                 & 94.26                                                 & \underline{96.98} & \underline{80.89} & 97.46             & 92.33             & \underline{89.27} & 81.92             \\
            CCNET                                   & Whole-word                              & \textsc{Base}                         & 500k                                    & \underline{98.21}                                  & \underline{\textbf{92.43}}                             & 99.24                                                 & \textbf{94.60}                                        & 96.69             & \textbf{80.97}    & 97.65             & 92.48             & 89.08             & \underline{83.43} \\
            CCNET                                   & Whole-word                              & \textsc{Large}                        & 100k                                    & 98.01                                              & 91.09                                                  & 99.23                                                 & 93.65                                                 & \textbf{97.01}    & \underline{80.89} & 97.41             & 92.59             & \textbf{89.39}    & \textbf{85.29}    \\


            \midrule
            \multicolumn{11}{l}{\hspace*{6mm}\em Embeddings (with UDPipe Future (tagging, parsing) or LSTM+CRF (NER))}                                                                                                                                                                                                                                                                                                                                                                                                                \\[0.5mm]
            OSCAR                                   & Subword                                 & \textsc{Base}                         & 100k                                    & \underline{\textbf{98.01}}                         & 90.64                                                  & \textbf{99.27}                                        & 94.26                                                 & \underline{97.15} & \textbf{82.56}    & \textbf{97.70}    & \underline{92.70} & \textbf{90.25}    & -                 \\
            OSCAR                                   & Whole-word                              & \textsc{Base}                         & 100k                                    & 97.97                                              & 90.44                                                  & \underline{99.23}                                     & 93.93                                                 & 97.08             & 81.74             & 97.50             & 92.28             & 89.48             & -                 \\
            CCNET                                   & Subword                                 & \textsc{Base}                         & 100k                                    & 97.87                                              & \textbf{90.78}                                         & 99.20                                                 & \underline{94.33}                                     & \textbf{97.17}    & \underline{82.39} & \underline{97.54} & 92.51             & 89.38             & -                 \\
            CCNET                                   & Whole-word                              & \textsc{Base}                         & 100k                                    & 97.96                                              & \underline{90.76}                                      & \underline{99.23}                                     & \textbf{94.34}                                        & 97.04             & 82.09             & 97.39             & \textbf{92.82}    & \underline{89.85} & -                 \\
            CCNET                                   & Whole-word                              & \textsc{Base}                         & 500k                                    & 97.84                                              & 90.25                                                  & 99.14                                                 & 93.96                                                 & 97.01             & 82.17             & 97.27             & 92.28             & 89.07             & -                 \\
            CCNET                                   & Whole-word                              & \textsc{Large}                        & 100k                                    & \underline{\textbf{98.01}}                         & 90.70                                                  & \underline{99.23}                                     & 94.01                                                 & 97.04             & 82.18             & 97.31             & 92.28             & 88.76             & -                 \\
            % ablation : 
            % %% ["10125734", "10126431", "10126429", "10126640","10126641"] # 
            % large ls_pos = ["10129280","10129224"]+["10129215", "10129223"]
            % ls_long = ["10129339", "10129416", "10129415", "10129415"]
            %  ls_long gsd :  ["10129611"] ( 2 seeds only) 
            \bottomrule
        \end{tabular}}
    \caption{Performance reported on \textbf{Test sets} for all trained models (\textbf{average} over multiple fine-tuning seeds).}
    \label{tab:all_results}
\end{table}

In this appendix, we analyze different design choices of \camembert (Table~\ref{tab:ablation}), namely with respect to the use of whole-word masking, the training dataset, the model size, and the number of training steps in complement with the analyses of the impact of corpus origin and size (Section~\ref{sec:origin_and_size}). In all the ablations, all scores come from at least 4 averaged runs. For POS tagging and dependency parsing, we average the scores on the 4 treebanks. We also report all averaged test scores of our different models in Table~\ref{tab:all_results}.

\begin{table}[!htbp]
    \centering\small
        \begin{tabular}{lcccc @{\hspace{0.7cm}} cccc}
            \toprule
            \textsc{Dataset}     & \textsc{Masking}          & \textsc{Arch.}               & \#\textsc{Param.}   & \#\textsc{Steps}    & \textsc{UPOS}  & \textsc{LAS}   & \textsc{NER}   & \textsc{XNLI}  \\
            \midrule
            \multicolumn{9}{l}{\hspace*{6mm}\em Masking Strategy}                                                                                                                                           \\
            {\color{gray}\oscar} & Subword                   & {\color{gray}\textsc{Base}}  & {\color{gray}110M}  & {\color{gray}100k}  & 97.78          & 89.80          & \textbf{91.55} & 81.04          \\
            {\color{gray}\oscar} & Whole-word                & {\color{gray}\textsc{Base}}  & {\color{gray}110M}  & {\color{gray}100k}  & \textbf{97.79} & \textbf{89.88} & 91.44          & \textbf{81.55} \\
            \midrule
            \multicolumn{9}{l}{\hspace*{6mm}\em Model Size}                                                                                                                                                 \\
            {\color{gray}\ccnet} & {\color{gray}Whole-word}  & \textsc{Base}                & 110M                & {\color{gray}100k}  & 97.67          & 89.46          & 90.13          & 82.22          \\
            {\color{gray}\ccnet} & {\color{gray} Whole-word} & \textsc{Large}               & 335M                & {\color{gray} 100k} & \textbf{97.74} & \textbf{89.82} & \textbf{92.47} & \textbf{85.73} \\
            \midrule
            \multicolumn{9}{l}{\hspace*{6mm}\em Dataset}                                                                                                                                                    \\
            \ccnet               & {\color{gray} Whole-word} & {\color{gray}\textsc{Base}}  & {\color{gray}110M}  & {\color{gray}100k}  & 97.67          & 89.46          & 90.13          & \textbf{82.22} \\
            \oscar               & {\color{gray} Whole-word} & {\color{gray}\textsc{Base}}  & {\color{gray}110M}  & {\color{gray}100k}  & \textbf{97.79} & \textbf{89.88} & \textbf{91.44} & 81.55          \\
            \midrule
            \multicolumn{9}{l}{\hspace*{6mm}\em Number of Steps}                                                                                                                                            \\
            {\color{gray}\ccnet} & {\color{gray} Whole-word} & {\color{gray} \textsc{Base}} & {\color{gray} 110M} & 100k                & \textbf{98.04} & 89.85          & 90.13          & 82.20          \\
            {\color{gray}\ccnet} & {\color{gray} Whole-word} & {\color{gray} \textsc{Base}} & {\color{gray} 110M} & 500k                & 97.95          & \textbf{90.12} & 91.30          & \textbf{83.04} \\
            \bottomrule
        \end{tabular}
    \caption{Comparing scores on the \textbf{Validation sets} of different design choices. POS tagging and parsing datasets are averaged. (average over multiple fine-tuning seeds).
        \label{tab:ablation}}
\end{table}


\section{Impact of Whole-Word Masking}
In Table~\ref{tab:ablation}, we compare models trained using the traditional subword masking with whole-word masking. Whole-Word Masking positively impacts downstream performances for NLI (although only by 0.5 points of accuracy). To our surprise, this Whole-Word Masking scheme does not benefit much lower level task such as Name Entity Recognition, POS tagging and Dependency Parsing.

\section{Impact of model size}
Table~\ref{tab:ablation} compares models trained with the BASE and LARGE architectures. These models were trained with the \ccnet corpus (135 GB) for practical reasons. We confirm the positive influence of larger models on the NLI and NER tasks. The LARGE architecture leads to respectively 19.7\% error reduction and 23.7\%. To our surprise, on POS tagging and dependency parsing, having three time more parameters doesn't lead to a significant  difference compared to the BASE model. \citet{tenney-etal-2019-bert} and \citet{jawahar-etal-2019-bert} have shown that low-level syntactic capabilities are learned in lower layers of \bert while higher level semantic representations are found in upper layers of \bert. POS tagging and dependency parsing probably do not benefit from adding more layers as the lower layers of the BASE architecture already capture what is necessary to complete these tasks.

\section{Impact of training dataset}

Table~\ref{tab:ablation} compares models trained on \ccnet and on \oscar.
The major difference between the two datasets is the additional filtering step of \ccnet that favors Wikipedia-Like texts.
The model pretrained on \oscar gets slightly better results on POS tagging and dependency parsing, but gets a larger +1.31 improvement on NER.
The \ccnet model gets better performance on NLI (+0.67).

\section{Impact of number of steps}
\label{sec:nbsteps}

\begin{figure}[t]
    \centering
    \includegraphics[width=0.75\linewidth]{static/media/mod_eval/camembert/plot_steps_impact_4.pdf}
    \caption{Impact of number of pretraining steps on downstream performance for \camembert.}.
    \label{fig:n_steps_impact}
\end{figure}


Figure~\ref{fig:n_steps_impact} displays the evolution of downstream task performance with respect to the number of steps. All scores in this section are averages from at least 4 runs with different random seeds. For POS tagging and dependency parsing, we also average the scores on the 4 treebanks.

We evaluate our model at every epoch (1 epoch equals 8360 steps). We report the masked language modelling perplexity along with downstream performances.
Figure~\ref{fig:n_steps_impact}, suggests that the more complex the task the more impactful the number of steps is. We observe an early plateau for dependency parsing and NER at around 22k steps, while for NLI, even if the marginal improvement with regard to pretraining steps becomes smaller, the performance is still slowly increasing at 100k steps.

In Table~\ref{tab:ablation}, we compare two models trained on \ccnet, one for 100k steps and the other for 500k steps to evaluate the influence of the total number of steps. The model trained for 500k steps does not increase the scores much from just training for 100k steps in POS tagging and parsing.
The increase is slightly higher for XNLI (+0.84).

Those results suggest that low level syntactic representation are captured early in the language model training process while it needs more steps to extract complex semantic information as needed for NLI.
%%%%%%%%%%%%%%%%%%%%%%%%%%%%%%%%%%%%%%%%%%%%%%%%%%%%%%%%%%%%%%%%%%%%%%%%
\chapter{BERTrade}\label{app:bertrade}
%%%%%%%%%%%%%%%%%%%%%%%%%%%%%%%%%%%%%%%%%%%%%%%%%%%%%%%%%%%%%%%%%%%%%%%%

\section{Collecting the Data}
\label{subsec:collectdata}
The following data can be downloaded directly from their website:
\begin{itemize}
    \item Chartes de l'Aube: \\ \url{https://sites.google.com/site/achimstein/research/resources} \\
          Extract raw text from XML files: <body>, then <s>, then <word>.
    \item Geste: \\ \url{https://github.com/Jean-Baptiste-Camps/Geste} \\
          Raw text is available under /txt/norm/.
    \item OpenMedFr: \\ \url{https://github.com/OpenMedFr/texts} \\
          Remove the header of each file (until \textit{*** START}), its last line (\textit{*** END}), paragraph breaks (\textit{\#|}) and folios or pages numbers.
\end{itemize}

Special permissions are required to access and use these sources:
\begin{itemize}
    \item AND: \\ \url{https://anglo-norman.net/project-members}
    \item BFM: \\ \url{http://bfm.ens-lyon.fr/spip.php?article19} \\
          Raw text is available.
    \item Chartes Douai: \\
          \url{https://www.rose.uzh.ch/docling}
    \item MCVF: \url{http://www.voies.uottawa.ca}
    \item NCA: \\ \url{https://sites.google.com/site/achimstein/research/resources} \\
          Extract raw text from the XML files: <body> then <txm:form>.
\end{itemize}


\section{Details on the Models}

\subsection{Models Trained From Scratch}

These are trained for \num{32} epochs in a masked language modeling task using the same parameters as RoBERTa \citep{liu-etal-2019-roberta} but a smaller batch size of \num{256} samples\footnote{Preliminary experiments with larger batch sizes showed no significant improvement to compensate for the heavier computational load.}, which amounts to a magnitude of \num{e5} steps.
We also use a smaller vocabulary size (\num{8192}) than other works, in line with the observations of \citet{ding-etal-2019-call} that learning large vocabularies on small corpora defeats the purpose of sub-word tokenization.
Using a larger vocabulary size of \num{5e4} (like FlauBERT) also did not seem to bring any improvements in our preliminary experiments and made pre-training more expensive.

\subsection{Post-training}

The pretrained models we used in the post-training settings are those available in the 4.2.0 version of Huggingface Transformers \citep{wolf-etal-2020-transformers} and the exact handles are:

\begin{description}
    \item[mBERT] \href{https://huggingface.co/bert-base-multilingual-cased}{bert-base-multilingual-cased}
    \item[flauBERT] \href{https://huggingface.co/flaubert/flaubert_base_cased}{flaubert/flaubert\_base\_cased}
    \item[camemBERT] \href{https://huggingface.co/camembert-base}{camembert-base}
    \item[finBERT] \href{https://huggingface.co/TurkuNLP/bert-base-finnish-cased-v1}{TurkuNLP/bert-base-finnish-cased-v1}
\end{description}

The post-trained models are those with MLM heads, which we did not reset before post-training, so the post-training phase can be seen as a language transfer task for masked language modeling out of which we extract a contextual word embeddings model.

\section{Carbon Footprint}\label{carbon-footprint}

\begin{table}[t]
    \centering\small
    \scalebox{0.89}{
        \begin{tabular}{@{}lrrrrr@{}}
            \toprule
            \textbf{Model}  & {\textbf{Power (\unit{\watt})}} & {\textbf{\# Models}} & {\textbf{Duration (\unit{\hour})}} & {\textbf{Consumption (\unit{\kWh})}} & {\textbf{CO\textsubscript{2}e (\unit{\kilo\gram})}} \\
            \midrule
            Pre-train       & 10756                           & 11                   & 6                                  & 11216.36                             & 358.92                                              \\
            Post-train      & 1520                            & 4                    & 20                                 & 192.13                               & 6.15                                                \\
            \midrule
            Total emissions &                                 &                      &                                    &                                      & 365.07                                              \\
            \bottomrule
        \end{tabular}
    }
    \caption{Average power draw, number of models trained, training times in hours, mean power consumption including power usage effectiveness (PUE), and CO\textsubscript{2} emissions; for each setting.}
    \label{tab:carbon-bertrade}
\end{table}

We report the power consumption and carbon footprint of our main experiments following the approach of \citet{strubell-etal-2019-energy}. Two different configurations were used in our experiments, one for pre-training models from scratch (Pre-train) and another one for continuing the training of existing models (Post-train).

\paragraph{Pre-train:} We use a cluster of 4 machines each one having \num{8} GPU Nvidia Tesla V100 SXM2 \qty{32}{\gibi\byte}, \qty{384}{\gibi\byte} of RAM, and two Intel Xeon Gold 6226 processors. One Nvidia Tesla V100 card is rated at around \qty{300}{\watt},\footnote{\href{https://www.nvidia.com/en-us/data-center/v100/}{ Nvidia Tesla V100 specification}} while the Xeon Gold 6226 processor is rated at \qty{125}{\watt},\footnote{\href{https://ark.intel.com/content/www/us/en/ark/products/193957/intel-xeon-gold-6226-processor-19-25m-cache-2-70-ghz.html}{Intel Xeon Gold 6226 specification}}. For the DRAM we can use the work of \citet{desrochers-etal-2016-a} to estimate the total power draw of \qty{384}{\gibi\byte} of RAM at around \qty{39}{\watt}. The total power draw of this setting adds up to around \qty{10756}{\watt}. We train \num{11} different models in this configuration.

\paragraph{Post-train:} We use a single machine having \num{4} GPU Nvidia Tesla V100 SXM2 \qty{32}{\gibi\byte}, \qty{192}{\gibi\byte} of RAM and two Intel Xeon Gold 6248 processors. The Xeon Gold 6248 processor is rated at 150 W,\footnote{\href{https://ark.intel.com/content/www/us/en/ark/products/192446/intel-xeon-gold-6248-processor-27-5m-cache-2-50-ghz.html}{Intel Xeon Gold 6248 specification}}, and the DRAM total power draw can be estimated at around \qty{20}{\watt}. The total power draw of this setting adds up to around \qty{1520}{\watt}. We train \num{4} different models in this configuration.

Having this information, we can now use the formula proposed by \citet{strubell-etal-2019-energy} in order to compute the total power required for each setting:

\begin{equation*}
    p_t = \frac{1.58t(cp_{c} + p_r + gp_g)}{1000}
\end{equation*}

Where $c$ and $g$ are the number of CPUs and GPUs respectively, $p_c$ is the average power draw (in \unit{\watt}) from all CPU sockets, $p_r$ the average power draw from all DRAM sockets, and $p_g$ the average power draw of a single GPU. We estimate the total power consumption by adding GPU, CPU and DRAM consumption, and then multiplying by the \emph{Power Usage Effectiveness} (PUE), which accounts for the additional energy required to support the compute infrastructure. We use a PUE coefficient of \num{1.58}, the 2018 global average for data centers \citep{strubell-etal-2019-energy}. In table \ref{tab:carbon-bertrade} we report the training times in hours, as well as the total power draw (in Watts) of the system used to train the models. We use this information to compute the total power consumption of each setting, also reported in table \ref{tab:carbon-bertrade}.

We can further estimate the CO\textsubscript{2} emissions in kilograms of each single model by multiplying the total power consumption by the average CO\textsubscript{2} emissions per \unit{\kWh} in our region which were around \qty{32}{\gram\per\kWh} in January 2021,\footnote{\href{https://www.rte-france.com/eco2mix/les-emissions-de-co2-par-kwh-produit-en-france}{Rte - éCO\textsubscript{2}mix}.} when the models were trained. Thus the total CO\textsubscript{2} emissions in kg for one single model can be computed as:
\begin{equation*}
    \text{CO}_{2}\text{e} = 0.032 p_t
\end{equation*}

All emissions are also reported in table \ref{tab:carbon-bertrade}.
%%%%%%%%%%%%%%%%%%%%%%%%%%%%%%%%%%%%%%%%%%%%%%%%%%%%%%%%%%%%%%%%%%%%%%%%
\chapter{D'AlemBERT}
%%%%%%%%%%%%%%%%%%%%%%%%%%%%%%%%%%%%%%%%%%%%%%%%%%%%%%%%%%%%%%%%%%%%%%%%

\section{Carbon Footprint}\label{carbon-footprint-dalembert}

\begin{table}[th]
    \centering\small
    \begin{tabular}{lrrrr}
        \toprule
        \textbf{Model}               & {\textbf{Power (W)}} & {\textbf{Time (h)}} & {\textbf{(PUE$\cdotp$kWh)}} & {\textbf{CO\textsuperscript{2}e (kg)}} \\
        \midrule
        Pre-train                    & 48640                & 20                  & 1537.02                     & 46.11                                  \\
        Evaluation                   & 589                  & 1                   & 0.93                        & 0.03                                   \\
        \midrule
        Total CO\textsuperscript{2}e &                      &                     &                             & 46.14                                  \\
        \bottomrule
    \end{tabular}
    \caption{Average power draw, number of models trained, training times in hours, mean power consumption including power usage effectiveness (PUE), and CO\textsuperscript{2} emissions; for each setting.}
    \label{tab:carbon-dalembert}
\end{table}

In light of recent interest concerning the energy consumption and carbon emission of machine learning models and specifically of those of language models \cite{schwartz-etal-2020-green,bender-etal-2021-on}, we have decided to report the power consumption and carbon footprint of all our experiments following the approach of \newcite{strubell-etal-2019-energy}. We report the energy consumption and carbon emissions of both the pre-training of D'AlemBERT and its evaluation.

\paragraph{Pre-training:} We use a cluster of 32 machines, each one having 4 GPU Nvidia Tesla V100 SXM2 32GiB, 192GiB of RAM, and two Intel Xeon Gold 6248 processors. One Nvidia Tesla V100 card is rated at around 300W,\footnote{\href{https://www.nvidia.com/en-us/data-center/v100/}{ Nvidia Tesla V100 specification}} while the Xeon Gold 6248 processor is rated at 150W.\footnote{\href{https://ark.intel.com/content/www/us/en/ark/products/192446/intel-xeon-gold-6248-processor-27-5m-cache-2-50-ghz.html}{Intel Xeon Gold 6248 specification}} For the DRAM we can use the work of \newcite{desrochers-etal-2016-a} to estimate the total power draw of 192GiB of RAM at around 20W. Thus, the total power draw of the pre-training adds up to around 48640W.

\paragraph{Evaluation:} We use a single machine with a single GPU Nvidia Tesla V100 SXM2 32GiB, 384GiB of RAM and two Intel Xeon Gold 6226 processors. The Xeon Gold 6226 processor is rated at 125 W,\footnote{\href{https://ark.intel.com/content/www/us/en/ark/products/193957/intel-xeon-gold-6226-processor-19-25m-cache-2-70-ghz.html}{Intel Xeon Gold 6226 specification}} and the DRAM total power draw can be estimated at around 39W. Therefore, the total power draw of the evaluation adds up to around 589W.

With this information, we use the formula proposed by \newcite{strubell-etal-2019-energy} to compute the total power required for each setting:

\begin{equation*}
    p_t = \frac{1.58t(cp_{c} + p_r + gp_g)}{1000}
\end{equation*}

Where $c$ and $g$ are the number of CPUs and GPUs respectively, $p_c$ is the average power draw (in W) from all CPU sockets, $p_r$ the average power draw from all DRAM sockets and $p_g$ the average power draw of a single GPU. We estimate the total power consumption by adding GPU, CPU and DRAM consumption, and then multiplying by the \emph{Power Usage Effectiveness} (PUE), which accounts for the additional energy required to support the compute infrastructure. We use a PUE coefficient of 1.58, the 2018 global average for data centers \cite{strubell-etal-2019-energy}. In Table~\ref{tab:carbon-dalembert} we report the training times in hours, as well as the total power draw (in Watts) of the system used to train the models. We use this information to compute the total power consumption of each setting, also reported in Table~\ref{tab:carbon-dalembert}.

We can further estimate the CO\textsuperscript{2} emissions in kilograms of each single model by multiplying the total power consumption by the average CO\textsuperscript{2} emissions per kWh in our region, which were around 30g/kWh between the 30\textsuperscript{th} and the 31\textsuperscript{st} of December,\footnote{\href{https://www.rte-france.com/eco2mix/les-emissions-de-co2-par-kwh-produit-en-france}{Rte - éCO\textsuperscript{2}mix}.} when the models were trained. Thus the total CO\textsuperscript{2} emissions in kg for one single model can be computed as:

\begin{equation*}
    \text{CO}_{2}\text{e} = 0.030 p_t
\end{equation*}

All emissions are also reported in Table~\ref{tab:carbon-dalembert}.

\section{Detail Results of Experiments in NER by Entity Type}

Here we show the results of each of the trained NER models by entity type.

\begin{table}[ht!]
    \centering\small
    \begin{tabular}{lrrrr}
        \toprule
        \multicolumn{5}{c}{\textsc{LSTM-CRF}}                  \\
        \midrule
        Entity Type  & Precision & Recall & F1-Score & Support \\
        \midrule
        pers         & 0.8808    & 0.8435 & 0.8617   & 2734    \\
        loc          & 0.8109    & 0.8707 & 0.8397   & 1384    \\
        amount       & 0.9040    & 0.9040 & 0.9040   & 250     \\
        time         & 0.9604    & 0.9237 & 0.9417   & 236     \\
        func         & 0.8872    & 0.8429 & 0.8645   & 140     \\
        org          & 0.8824    & 0.6122 & 0.7229   & 49      \\
        prod         & 0.9231    & 0.4444 & 0.6000   & 27      \\
        event        & 0.7273    & 0.6667 & 0.6957   & 12      \\
        \midrule
        micro avg    & 0.8640    & 0.8533 & 0.8586   & 4832    \\
        macro avg    & 0.8720    & 0.7635 & 0.8038   & 4832    \\
        weighted avg & 0.8659    & 0.8533 & 0.8583   & 4832    \\
        samples avg  & 0.7737    & 0.7737 & 0.7737   & 4832    \\
        \bottomrule
    \end{tabular}
    \caption{Results of the BiLSTM-CRF model on the test set of \freemner by entity type.}
\end{table}

\begin{table}[ht!]
    \centering\small
    \begin{tabular}{lrrrr}
        \toprule
        \multicolumn{5}{c}{\textsc{CamemBERT}}                 \\
        \midrule
        Entity Type  & Precision & Recall & F1-Score & Support \\
        \midrule
        pers         & 0.9373    & 0.9236 & 0.9304   & 2734    \\
        loc          & 0.9140    & 0.9371 & 0.9254   & 1384    \\
        amount       & 0.9840    & 0.9840 & 0.9840   & 250     \\
        time         & 0.9447    & 0.9407 & 0.9427   & 236     \\
        func         & 0.9209    & 0.9143 & 0.9176   & 140     \\
        org          & 0.8364    & 0.9388 & 0.8846   & 49      \\
        prod         & 0.7742    & 0.8889 & 0.8276   & 27      \\
        event        & 0.8333    & 0.8333 & 0.8333   & 12      \\
        \midrule
        micro avg    & 0.9303    & 0.9309 & 0.9306   & 4832    \\
        macro avg    & 0.8931    & 0.9201 & 0.9057   & 4832    \\
        weighted avg & 0.9307    & 0.9309 & 0.9307   & 4832    \\
        samples avg  & 0.8856    & 0.8856 & 0.8856   & 4832    \\
        \bottomrule
    \end{tabular}
    \caption{Results of CamemBERT on the test set of \freemner by entity type.}
\end{table}

\begin{table}[ht!]
    \centering\small
    \begin{tabular}{lrrrr}
        \toprule
        \multicolumn{5}{c}{\textsc{D'AlemBERT}}                \\
        \midrule
        Entity Type  & Precision & Recall & F1-Score & Support \\
        \midrule
        pers         & 0.9355    & 0.9279 & 0.9317   & 2734    \\
        loc          & 0.9242    & 0.9335 & 0.9288   & 1384    \\
        amount       & 0.9800    & 0.9800 & 0.9800   & 250     \\
        time         & 0.9456    & 0.9576 & 0.9516   & 236     \\
        func         & 0.9333    & 0.9000 & 0.9164   & 140     \\
        org          & 0.8148    & 0.8980 & 0.8544   & 49      \\
        prod         & 0.8621    & 0.9259 & 0.8929   & 27      \\
        event        & 0.8333    & 0.8333 & 0.8333   & 12      \\
        \midrule
        micro avg    & 0.9329    & 0.9323 & 0.9326   & 4832    \\
        macro avg    & 0.9036    & 0.9195 & 0.9111   & 4832    \\
        weighted avg & 0.9331    & 0.9323 & 0.9327   & 4832    \\
        samples avg  & 0.8893    & 0.8893 & 0.8893   & 4832    \\
        \bottomrule
    \end{tabular}
    \caption{Results of D'AlemBERT model on the test set of \freemner by entity type.}
\end{table}

\section{Entity Distribution by Text in NER Data}

The following diagrams show the detail of the coarse entity distribution by text in \freemner.

\begin{sidewaysfigure}[!ht]
    \centering
    \includegraphics[width=\textwidth]{static/media/mod_eval/dalembert/freem_ner_entities_by_text.png}
    \caption{Number of entities by text on a logarithmic scale.}
    \label{fig:entities-by-text}
\end{sidewaysfigure}

\begin{sidewaysfigure}[!ht]
    \centering
    \includegraphics[width=\textwidth]{static/media/mod_eval/dalembert/freem_ner_entity_type_by_text.png}
    \caption{Entity types by text on a logarithmic scale.}
    \label{fig:entity-type-per-text}
\end{sidewaysfigure}



% This ensures that the subsequent sections are being included as root
% items in the bookmark structure of your PDF reader.
\bookmarksetup{startatroot}
\backmatter

% \begingroup
% \let\clearpage\relax
% \glsaddall
% \printglossary[type=\acronymtype]
% \newpage
% \printglossary
% \endgroup

% \printindex

% Entries for the entire Anthology, followed by custom entries
\addcontentsline{toc}{chapter}{Bibliography}
\bibliography{anthology, thesis}
\bibliographystyle{acl_natbib}

\listoffigures

\listoftables

\newpage  % chktex 1
\thispagestyle{empty}
\phantomsection  % chktex 1
\addcontentsline{toc}{chapter}{License}
\begin{center}
	{\Huge\bfseries License}

	\vfill
	\Large
	This document is available under the terms of the \doclicenseLongName License (\doclicenseName) (\url{\doclicenseURL})

	\vfill
	Copyright © 2022, Pedro Ortiz Suarez <\href{mailto:pedro@portizsu.eu}{\nolinkurl{pedro@portizsu.eu}}>

	\vfill
	\doclicenseImage
\end{center}

\vfill
\include{sources/backmatter/back}

\end{document}
