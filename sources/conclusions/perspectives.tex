%%%%%%%%%%%%%%%%%%%%%%%%%%%%%%%%%%%%%%%%%%%%%%%%%%%%%%%%%%%%%%%%%%%%%%%%
\chapter{Conclusions and Perspectives}\label{chap:conclusions}
%%%%%%%%%%%%%%%%%%%%%%%%%%%%%%%%%%%%%%%%%%%%%%%%%%%%%%%%%%%%%%%%%%%%%%%%

During this thesis, we have developed multiple resources for several states of language in French, and even for a wide variety of languages in the case of OSCAR. We have chosen to focus on the development of data for the pre-training of language models rather than on the architectures themselves. This approach proved to be extremely effective as we were able to establish a new state of the art for a wide range of tasks in natural language processing for several states of language in French: Medieval French, Modern French and Contemporary French.

We consider that we have reached and even exceeded all the initial objectives set by the BASNUM project for this thesis, by producing models and automatic annotation systems that will make it possible to enrich not only the \emph{Dictionnaire Universel} of Basnage but also any other type of document in French from any historical era. We hope that the resources we have produced will be of great use to researchers in natural language processing and digital humanities.