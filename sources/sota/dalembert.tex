%%%%%%%%%%%%%%%%%%%%%%%%%%%%%%%%%%%%%%%%%%%%%%%%%%%%%%%%%%%%%%%%%%%%%%%%
\chapter{D'AlemBERT Related Work}
%%%%%%%%%%%%%%%%%%%%%%%%%%%%%%%%%%%%%%%%%%%%%%%%%%%%%%%%%%%%%%%%%%%%%%%%


Large datasets for historical states of languages or  extinct languages do exist. The \emph{Corpus Middelnederlands} for Medieval Dutch \citep{reenen-etal-1998-corpus} and the \emph{Base Geste} for Medieval French \citep{camps-etal-2019-geste} are freely available online, encoded in TEI. It is also the case for other corpora for later states of language, such as the \emph{Reference corpus of historical Slovene}, covering approximately three centuries of Slovene (1584--1899)  \citep{erjavec-2015-reference}, and the ``corpus noyau'' of \emph{Presto} \citep{blumenthal-2018-presto}. This last corpus, in its extended version, uses other French corpora such as \emph{Espistemon} for Renaissance French \citep{demonet-1998-epistemon} and the University of Chicago's \emph{American and French Research on the Treasury of the French Language} (ARTFL) \citep{morrissey-olsen-1991-american}; or like \textsc{Frantext} \citep{atilf-1998-frantext}, which is a generalist French corpus, covering the different states of the French language between the 11\textsuperscript{th} and the 21\textsuperscript{st} century. Although most of these text collections are free, the two biggest ones, \textsc{Frantext} and ARTFL, are not freely available or open-sourced.

Concerning language modelling in French, two main models are available for contemporary French, \camembert \cite{martin-etal-2020-camembert} and FlauBERT \cite{le-etal-2020-flaubert}. \camembert was trained on a freely available, automatically web-crawled corpus called OSCAR \cite{ortiz-suarez-etal-2019-asynchronous,ortiz-suarez-etal-2020-monolingual} while FlauBERT was trained on a mix of web-crawled data and manually curated (partly non freely available) contemporary French corpora. Neither of these models was explicitly pre-trained for historical French.\footnote{Note however that texts in Old, Middle and Modern French do exist in the internet, and might have found their way to the training corpus of these two models. This is especially the case for Modern French texts, which automatic language classification tools can easily classify as Contemporary French.} However efficient language models have been trained for less-resourced or extinct Languages such as Latin \cite{bamman-burns-2020-latin}, following the approach of \newcite{martin-etal-2020-camembert} for training language models with less data than was previously thought. There have also been some recent projects that specifically target Early Modern French such as that of \pieextended \cite{clerice-2020-pie} that uses the hierarchical encoding architecture originally proposed by \newcite{manjavacas-etal-2019-improving} which itself is constructed by stacking multiple Bi-LSTM-CRFs. \newcite{clerice-2020-pie} distributes pre-trained models for POS tagging and lemmatisation.