%%%%%%%%%%%%%%%%%%%%%%%%%%%%%%%%%%%%%%%%%%%%%%%%%%%%%%%%%%%%%%%%%%%%%%%%
\chapter{Conclusions and Perspectives}
%%%%%%%%%%%%%%%%%%%%%%%%%%%%%%%%%%%%%%%%%%%%%%%%%%%%%%%%%%%%%%%%%%%%%%%%

Au cours de cette thèse, nous avons développé de multiples ressources pour plusieurs états de langue du français, et même pour une grande variété des langues dans le cas d'OSCAR. Nous avons choisi de nous concentrer sur le développement de données pour le pre-entraînement des modèles de langues plutôt que sur les architectures elles-mêmes. Cette approche s'est avérée extrêmement efficace car nous avons pu établir un nouvel état de l'art pour un large éventail de tâches en traitement automatique des langues pour plusieurs états de langue du français: le français médiéval, le français moderne et le français contemporain.

Nous considérons que nous avons atteint et même dépassé tous les objectifs initiaux fixés par le projet BASNUM pour cette thèse, en produisant des modèles et des systèmes d'annotation automatique qui permettront d'enrichir non seulement le Dictionnaire Universel de Basnage mais également n'importe quel autre type de document en français de n'importe quelle époque historique. Nous espérons que les ressources que nous avons produites seront d'une grande utilité pour les chercheurs en traitement automatique des langues et en humanités numériques.