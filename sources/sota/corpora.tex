%%%%%%%%%%%%%%%%%%%%%%%%%%%%%%%%%%%%%%%%%%%%%%%%%%%%%%%%%%%%%%%%%%%%%%%%
\chapter{On Raw Corpora for Language Modeling}
%%%%%%%%%%%%%%%%%%%%%%%%%%%%%%%%%%%%%%%%%%%%%%%%%%%%%%%%%%%%%%%%%%%%%%%%

\begin{center}
    \begin{minipage}{0.66\textwidth}
        \begin{small}
            In which the corpora available for the pre-training of language models both at the beginning and throughout this Ph.D. thesis are described. We also describe one of the pipelines used to produce one of the corpora, and at the end we discuss both the raw and annotated corpora available for some historical languages.
        \end{small}
    \end{minipage}
    \vspace{0.5cm}
\end{center}

As previously stated, the only freely-available corpora, considered \emph{large enough} for the pre-training of language models in Contemporary French at the beginning of this Ph.D thesis, were Wikipedia and frWac. The frWaC corpus \citep{baroni-etal-2009-the} is a French text corpus collected from the \texttt{.fr} domain with using medium-frequency words from the \emph{Le Monde Diplomatique} corpus and basic French vocabulary lists as seeds. The corpus consists of French websites with total size 1.3 billion words. While frWac was almost 2 times bigger than the French Wikipedia at the time, it was still nowhere near the amount of data that was thought to be needed to properly train a Transformer-based language model at the time \citep{liu-etal-2019-roberta}. However, we liked the idea of \citet{baroni-etal-2009-the} of using web pages and crawling as a mean to obtain large quantities of textual data. This is why the work of \citet{mikolov-etal-2018-advances} and its use of Common Crawl and the FastText linear classifier \citep{joulin-etal-2016-fasttext, joulin-etal-2017-bag}, in order to gather large amounts of multilingual text plays a central role in the development of our own multilingual web based corpora.

In this chapter we first describe the work of \citet{mikolov-etal-2018-advances}, then we present and discuss the large web corpora that became available through this thesis that were not available at the begging of it, and the challenges that came with them and with the ever-growing demand for large textual corpora. Finally, we briefly give an overview of the available corpora for historical languages.

\section{Common Crawl}

Common Crawl is a non-profit foundation which produces and maintains an open repository of web crawled data that is both accessible and analyzable.\footnote{\url{http://commoncrawl.org/about/}} Common Crawl's complete web archive consists of petabytes of data collected over 8 years of web crawling. The repository contains raw web page HTML data (WARC files), metadata extracts (WAT files) and plain text extracts (WET files). The organization's crawlers has always respected \texttt{nofollow}\footnote{\url{http://microformats.org/wiki/rel-nofollow}} and \texttt{robots.txt}\footnote{\url{https://www.robotstxt.org/}} policies.

Each monthly Common Crawl snapshot is in itself a massive multilingual corpus, where every single file contains data coming from multiple web pages written in a large variety of languages and covering all possible types of topics. Thus, in order to effectively use this corpus for Natural Language Processing and Machine Learning applications, one has first to extract, filter, clean and classify the data in the snapshot by language.

Throughout this thesis, we will use the WET files which contain the extracted plain texts from the websites mostly converted to UTF-8, as well as headers containing the metadata of each crawled document. Each WET file comes compressed in gzip format\footnote{\url{https://www.gnu.org/software/gzip/}} and is stored on Amazon Web Services.

Common Crawl has already been successfully used to train language models, even multilingual ones. The most notable example is probably FastText which was first trained for English using Common Crawl \citep{mikolov-etal-2018-advances} and then for other 157 different languages \citep{grave-etal-2018-learning}. In fact \citet{grave-etal-2018-learning} proposed a pipeline to filter, clean and classify Common Crawl, which we shall call the ``FastText pre-processing pipeline.'' They used the FastText linear classifier \citep{joulin-etal-2016-fasttext, joulin-etal-2017-bag} to classify each line of Common Crawl by language, and downloaded the initial corpus and schedule the I/O using some simple Bash scripts. Their solution, however, proved to be a synchronous blocking pipeline that works well on infrastructures having the necessary hardware to assure high I/O speeds even when storing tens of terabytes of data at a time. But that downscales poorly to medium-low resource infrastructures that rely on more traditional cost-effective electromechanical mediums in order to store this amount of data.

\subsection{FastText's Pipeline}

The \emph{``FastText pre-processing pipeline''} used by \citet{grave-etal-2018-learning} launches multiple process, preferably as many as available cores. Each of these processes first downloads one Common Crawl WET file which then proceeds to decompress after the download is over. After decompressing, an instance of the FastText linear classifier \citep{joulin-etal-2016-fasttext, joulin-etal-2017-bag} is launched, the classifier processes each WET file line by line, generating a language tag for each line. The tags are then stored in a tag file which holds a one-to-one correspondence between lines of the WET file and its corresponding language tag. The WET file and the tag files are read sequentially and each on the WET file line holding the condition of being longer than 100 bytes is appended to a language file containing only plain text (tags are discarded). Finally, the tag file and the WET files are deleted.

Only when one of these processes finishes another can be launched. This means that one can at most process and download as many files as cores the machine has. That is, if for example a machine has 24 cores, only 24 WET files can be downloaded and processed simultaneously, moreover, the 25\textsuperscript{th} file won't be downloaded until one of the previous 24 files is completely processed.

When all the WET files are classified, one would normally get around 160 language files, each file holding just plain text written in its corresponding language. These files still need to be filtered in order to get rid of all files containing invalid UTF-8 characters, so again a number of processes are launched, this time depending on the amount of memory of the machine. Each process reads a language file, first filters for invalid UTF-8 characters and then performs deduplication. A simple non-collision resistant hashing algorithm is used to deduplicate the files.

The FastText linear classifier works by representing sentences for classification as Bags of Words (BoW) and training a linear classifier. A weight matrix $A$ is used as a look-up table over the words and the word representations are then averaged into a text representation which is fed to the linear classifier. The architecture is in general similar to the CBoW model of \citet{mikolov-etal-2013-distributed}, but the middle word is replaced by a label. They uses a softmax function $f$ to compute the probability distribution over the classes. For a set of $N$ documents, the model is trained to minimize the negative log-likelihood over the classes:
\[
    -\frac{1}{N}\sum_{n=1}^{N} y_n\log\left(f(BAx_n)\right),
\]
where $x_n$ is the normalized bag of features of the $n$-th document, $y_n$ is the $n$-th label, and $A,B$ are the weight matrices. The pre-trained FastText model for language recognition \citep{grave-etal-2018-learning} is capable of recognizing around 176 different languages and was trained using 400 million tokens from Wikipedia as well as sentences from the Tatoeba website.\footnote{\url{https://tatoeba.org/}}

\subsection{CCNet}

We note that the original Common-Crawl-based corpus created by \citet{grave-etal-2018-learning} to train FastText is not freely available. Shortly arfter having started working with Common Crawl data and developing a pipeline to classify it by language \citep{ortiz-suarez-etal-2019-asynchronous}, a new architecture for creating a Common-Crawl-based corpus named CCNet \citep{wenzek-etal-2020-ccnet} was published, although it included specialized filtering based on the KenLM library \citep{heafield-2011-kenlm} and trained on Wikipedia, which might result in a cleaner corpus; the resulting CCNet corpus itself was never published in its entirety.


%%%%%%%%%%%%%%%%%%%%%%%%%%%%%%%%%%%%%%%%%%%%%%%%%%%%%%%%%%%%%%%%%%%%%%%%
\section{Demand for Large Corpora}
%%%%%%%%%%%%%%%%%%%%%%%%%%%%%%%%%%%%%%%%%%%%%%%%%%%%%%%%%%%%%%%%%%%%%%%%

Using large corpora to train neural language models dates back to way before the beginning of this Ph.D. thesis \citep{schwenk-gauvain-2005-training}. During the course of our studies, we observed the demand for large corpora considerably increasing, specially the last two years with the advent of semi-supervised learning methods in NLP, in particular with \emph{contextualized word representations} \citep{howard-ruder-2018-universal,peters-etal-2018-deep,devlin-etal-2019-bert} and more recently \emph{very large generative language models} like GPT-3, T5, GPT-Neo \citep{raffel-etal-2020-exploring,brown-etal-2020-language,black-etal-2021-gpt}. While there have been some recent efforts to manually curate such corpora\footnote{\url{https://bigscience.huggingface.co}} \citep{gao-etal-2020-pile}, the common approach to collect large amounts of raw textual data still relies primarily on crawled web text \citep{ortiz-suarez-etal-2019-asynchronous,ortiz-suarez-etal-2020-monolingual,xue-etal-2021-mt5,el-kishky-etal-2020-ccaligned,espla-etal-2019-paracrawl,banon-etal-2020-paracrawl,gao-etal-2020-pile}, and although some of the initial concerns of using crawled data \citep{trinh-le-2018-a,radford-etal-2019-language} were addressed during the course of this particular Ph.D. thesis \citep{ortiz-suarez-etal-2020-monolingual,martin-etal-2020-camembert} there a many concerns that still need to be tackled \citep{caswell-etal-2020-language} specially for multilingual data \citep{kreutzer-etal-2021-quality}.

In this demand for large raw textual corpora we can observe a clear back and forth in the type of data used to pre-train these models. On one hand some authors have opted for highly curated or edited data like Wikipedia such as \citet{al-rfou-etal-2013-polyglot} and \citet{bojanowski-etal-2017-enriching} for static word embeddings, the 1B Word Benchmark \citep{chelba-etal-2014-one} for ELMo \citep{peters-etal-2018-deep}, and the BookCorpus \citep{zhu-etal-2015-aligning} and Wikipedia for BERT \citep{devlin-etal-2019-bert}. On the other hand projects like those of \citet{pennington-etal-2014-glove} or \citet{grave-etal-2018-learning} used crawled data for the pre-training of fixed word embeddings, CamemBERT \citep{martin-etal-2020-camembert}, our contextualized model for French, successfully used only Crawled data for pre-training, and even large generative language models like T5 have used mainly crawled data successfully \citep{raffel-etal-2020-exploring}. We can of course also see examples of projects successfully using a mix of both manually curated and automatically crawled data such as RoBERTa \citep{liu-etal-2019-roberta}, FauBERT\citep{le-etal-2020-flaubert-unsupervised}, XLNet \citep{yang-etal-2019-xlnet} and GPT-Neo \citep{black-etal-2021-gpt,gao-etal-2020-pile}. However, no matter the chosen approach to build these large corpora, there are in every case concerns that have been expressed, specially for the datasets used in very large generative language models \citep{bender-etal-2021-on}, even when using manually edited resources like Wikipedia \citep{barera-2020-mind}.

\subsection{Problems with Crawled Corpora}

Corpora collected by web crawlers are known to be noisy~\citep{junczys-dowmunt-2019-microsoft,luccioni-viviano-2021-whats}. In highly multilingual settings, past work found that web-crawls of lower-resource languages have serious issues, especially with segment-level LangID~\citep{caswell-etal-2020-language}. Cleaning and filtering web-crawls can boost general language modeling~\citep{gao-etal-2020-pile,brown-etal-2020-language,raffel-etal-2020-exploring} and downstream task performance~\citep{moore-lewis-2010-intelligent,rarrick-etal-2011-mt,xu-koehn-2017-zipporah,khayrallah-koehn-2018-impact,brown-etal-2020-language}.

As the scale of ML research grows, it becomes increasingly difficult to validate automatically collected and curated datasets \citep{biderman-etal-2020-pitfalls,birhane-etal-2021-large,bender-etal-2021-on}. Several works have focused on advancing methodologies and best practices to address these challenges. \citet{bender-friedman-2018-data} introduced data statements, a documentary framework for NLP datasets that seeks to provide a universal minimum bar for dataset description. Similar work has been done on systematizing documentation in other areas in data science and machine learning, including work focusing on online news \citep{kevin-etal-2018-information}, data ethics \citep{sun-etal-2019-mithralabel}, and data exploration \citep{holland-etal-2018-the}, as well as generalist work such as \citep{gebru-etal-2018-datasheets}. Data quality is also implicitly documented by successes of filtering methods. There is a large literature on filtering data for various NLP tasks, e.g. \citep{axelrod-etal-2011-domain,moore-lewis-2010-intelligent,rarrick-etal-2011-mt,wang-etal-2018-denoising,kamholz-etal-2014-panlex,junczys-dowmunt-2018-dual,caswell-etal-2020-language}.

\subsection{Publically Available Web-based Corpora}\label{sec:crawls}

\begin{table*}[th!]
    \centering
    \resizebox{\textwidth}{!}{%
        \begin{tabular}{lccccc}
            \toprule
                            & \multicolumn{3}{c}{\textbf{Parallel}} & \multicolumn{2}{c}{\textbf{Monolingual}}                                                       \\
            \cmidrule(lr){2-4} \cmidrule(lr){5-6}
                            & \textbf{CCAligned}                    & \textbf{ParaCrawl v7.1}                  & \textbf{WikiMatrix} & \textbf{OSCAR} & \textbf{mC4} \\
            \midrule
            \#languages     & 137                                   & 41                                       & 85                  & 166            & 101          \\
            Source          & CC 2013--2020                         & selected websites                        & Wikipedia           & CC 11/2018     & CC all       \\
            Filtering level & document                              & sentence                                 & sentence            & document       & document     \\
            Langid          & FastText                              & CLD2                                     & FastText            & FastText       & CLD3         \\
            Alignment       & LASER                                 & Vec/Hun/BLEU-Align                       & LASER               & -              & -            \\
            Evaluation      & TED-6                                 & WMT-5                                    & TED-45              & POS/DEP-5      & XTREME       \\
            \bottomrule
        \end{tabular}%
    }
    \caption{Comparison of parallel and monolingual corpora extracted from web documents, including their downstream evaluation tasks. All parallel corpora are evaluated for machine translation (BLEU). TED-6: \texttt{da}, \texttt{cr}, \texttt{sl}, \texttt{sk}, \texttt{lt}, \texttt{et}; TED-45: 45-language subset of ~\citep{qi-etal-2018-pre}; WMT-5: \texttt{cs}, \texttt{de}, \texttt{fi}, \texttt{lv}, \texttt{ro}. POS/DEP-5: part-of-speech labeling and dependency parsing for \texttt{bg}, \texttt{ca}, \texttt{da}, \texttt{fi}, \texttt{id}.}
    \label{tab:corpora}
\end{table*}


Table \ref{tab:corpora} provides an overview of the corpora of interest in this work. We selected the corpora for their multilinguality and the inclusion of understudied languages in NLP. With the exception of WikiMatrix and ParaCrawl, all corpora are derived from Common Crawl (CC).\footnote{\url{http://commoncrawl.org/}}

\paragraph{CCAligned~\citep{el-kishky-etal-2020-ccaligned}}is a parallel dataset built off 68 CC snapshots. Documents are aligned if they are in the same language according to FastText LangID~\citep{joulin-etal-2016-fasttext,joulin-etal-2017-bag}, and have the same URL but for a differing language code. These alignments are refined with cross-lingual LASER embeddings \citep{artetxe-schwenk-2019-massively}. For sentence-level data, they split on newlines and align with LASER, but perform no further filtering. Human annotators evaluated the quality of document alignments for six languages (\texttt{de}, \texttt{zh}, \texttt{ar}, \texttt{ro}, \texttt{et}, \texttt{my})\footnote{For language codes please refer to the IETF BCP 47 language tag \url{https://www.iana.org/assignments/language-subtag-registry/language-subtag-registry}} selected for their different scripts and amount of retrieved documents, reporting precision of over 90\%. The quality of the extracted parallel sentences was evaluated in a machine translation (MT) task on six European (\texttt{da}, \texttt{cr}, \texttt{sl}, \texttt{sk}, \texttt{lt}, \texttt{et}) languages of the TED corpus~\citep{qi-etal-2018-pre}, where it compared favorably to systems built on crawled sentences from WikiMatrix and ParaCrawl v6.

\paragraph{Multilingual C4 (mC4)~\citep{xue-etal-2021-mt5}} is a document-level dataset used for training the mT5 language model. It consists of monolingual text in 101 languages and is generated from 71 CC snapshots. It filters out pages that contain less than three lines of at least 200 characters and pages that contain bad words.\footnote{\url{https://github.com/LDNOOBW/}} Since this is a document-level dataset, we split it by sentence and deduplicate it before rating. For language identification, it uses CLD3~\citep{botha-etal-2017-natural},\footnote{\url{https://github.com/google/cld3/}} a small feed-forward neural network that was trained to detect 107 languages. The mT5 model pre-trained on mC4 is evaluated on 6 tasks of the XTREME benchmark~\citep{hu-etal-2020-xtreme} covering a variety of languages and outperforms other multilingual pre-trained language models such as mBERT~\citep{devlin-etal-2019-bert} and XLM\nobreakdash-R~\citep{conneau-etal-2020-unsupervised}.

\paragraph{OSCAR~\citep{ortiz-suarez-etal-2019-asynchronous, ortiz-suarez-etal-2020-monolingual}} our own corpus, to which we devote the whole Part \ref{part:oscar} of this Ph.D thesis, is a set of monolingual corpora extracted from CC snapshots, specifically from the plain text \emph{WET} format distributed by CC which removes all the HTML tags and converts the text to UTF-8. It is deduplicated and follows the approach by~\citep{grave-etal-2018-learning} of using FastText LangID~\citep{joulin-etal-2016-fasttext, joulin-etal-2017-bag} on a line-level.\footnote{Line-level here refers to the `\textbackslash n' separated lines in the Common Crawl WET files.}\textsuperscript{,}\footnote{\url{https://fasttext.cc/docs/en/language-identification.html}} No other filtering was applied. For five languages (\texttt{bg}, \texttt{ca}, \texttt{da}, \texttt{fi}, \texttt{id}) OSCAR was used by its original authors to train language models which were then evaluated on parsing and POS tagging \citep{ortiz-suarez-etal-2020-monolingual}. OSCAR has also been used in independent studies to train monolingual or multilingual language models (\texttt{ar}, \texttt{as}, \texttt{bn}, \texttt{de}, \texttt{el}, \texttt{fr}, \texttt{gu}, \texttt{he}, \texttt{hi}, \texttt{kn}, \texttt{ml}, \texttt{mr}, \texttt{nl}, \texttt{or}, \texttt{pa}, \texttt{ro}, \texttt{ta}, \texttt{te}) and subsequently evaluate them on various downstream tasks \citep{antoun-etal-2021-araelectra, kakwani-etal-2020-indicnlpsuite, wilie-etal-2020-indonlu, chan-etal-2020-germans, koutsikakis-etal-2020-greek, martin-etal-2020-camembert, chriqui-etal-2021-hebert, seker-etal-2021-alephbert, delobelle-etal-2020-robbert, dumitrescu-etal-2020-birth, masala-etal-2020-robert}.


\paragraph{ParaCrawl v7.1} is a parallel dataset with 41 language pairs primarily aligned with English (39 out of 41) and mined using the parallel-data-crawling tool Bitextor \citep{espla-etal-2019-paracrawl,banon-etal-2020-paracrawl} which includes downloading documents, preprocessing and normalization, aligning documents and segments, and filtering noisy data via Bicleaner.\footnote{\url{https://github.com/bitextor/bicleaner}}
ParaCrawl focuses on European languages, but also includes 9 lower-resource, non-European language pairs in v7.1. Sentence alignment and sentence pair filtering choices were optimized for five languages (\texttt{mt}, \texttt{et}, \texttt{hu}, \texttt{cs}, \texttt{de}) by training and evaluating MT models on the resulting parallel sentences. An earlier version (v5) was shown to improve translation quality on WMT benchmarks for~\texttt{cs}, \texttt{de}, \texttt{fi}, \texttt{lv}, \texttt{ro}.


\paragraph{WikiMatrix~\citep{schwenk-etal-2021-wikimatrix}} is a public dataset containing 135M parallel sentences in 1620 language pairs (85 languages) mined from Wikipedia. Out of the 135M parallel sentences, 34M are aligned with English. The text is extracted from Wikipedia pages, split into sentences, and duplicate sentences are removed. FastText LangID is used before identifying bitext with LASER's distance-based mining approach. The margin threshold is optimized by training and evaluating downstream MT models on four WMT benchmarks (\texttt{de-en}, \texttt{de-fr}, \texttt{cs-de}, \texttt{cs-fr}). The final dataset is used to train translation models that are then evaluated by automatically measuring the quality of their translations against human translations of TED talks in 45 languages, with highest quality for translations between English and e.g. \texttt{pt}, \texttt{es}, \texttt{da}, and lowest for \texttt{sr}, \texttt{ja}, \texttt{mr}, \texttt{zh\_TW}.


%%%%%%%%%%%%%%%%%%%%%%%%%%%%%%%%%%%%%%%%%%%%%%%%%%%%%%%%%%%%%%%%%%%%%%%%
\section{Historical French Corpora}
%%%%%%%%%%%%%%%%%%%%%%%%%%%%%%%%%%%%%%%%%%%%%%%%%%%%%%%%%%%%%%%%%%%%%%%%

Large datasets for historical states of languages or extinct languages do exist. The \emph{Corpus Middelnederlands} for Medieval Dutch \citep{reenen-etal-1998-corpus} and the \emph{Base Geste} for Medieval French \citep{camps-etal-2019-geste} are freely available online, encoded in TEI. It is also the case for other corpora for later states of language, such as the \emph{Reference corpus of historical Slovene}, covering approximately three centuries of Slovene (1584--1899)  \citep{erjavec-2015-reference}, and the ``corpus noyau'' of \emph{Presto} \citep{blumenthal-2018-presto}. This last corpus, in its extended version, uses other French corpora such as \emph{Espistemon} for Renaissance French \citep{demonet-1998-epistemon} and the University of Chicago's \emph{American and French Research on the Treasury of the French Language} (ARTFL) \citep{morrissey-olsen-1991-american}; or like \textsc{Frantext} \citep{atilf-1998-frantext}, which is a generalist French corpus, covering the different states of the French language between the 11\textsuperscript{th} and the 21\textsuperscript{st} century. Although most of these text collections are free, the two biggest ones, \textsc{Frantext} and ARTFL, are not freely available or open-sourced.

Regarding corpora annotated corpora for historical languages, very few of them have manually annotated syntactical resources for their medieval states. English has three such treebanks \citep{oxford-2001-the,kroch-etal-2000-the,traugott-pintzuk-2008-coding} for Old and Middle English. The TOROT treebank for Old Church Slavonic, Old East Slavonic and Middle Russian is another large resource \citep{berdicevskis-eckhoff-2020-diachronic}. There is a treebank for Medieval Latin as well, the \emph{Index Thomisticus Treebank} \citep{passarotti-2019-project}. To our knowledge, the last large treebank containing medieval texts is IcePaHC for Icelandic \citep{rognvaldsson-etal-2012-icelandic}. Some other corpora were annotated automatically in order to reduce the cost of annotation. For example, \citet{rocio-etal-2003-automated} adapted a parsing pipeline for contemporary Portuguese and \citet{lee-kong-2014-a} used a previously annotated treebank \citep{lee-kong-2012-dependency} to parse a larger medieval Chinese corpus. Concerning contemporary regional Romance languages, \citet{miletic-etal-2020-building} also used a smaller treebank to generate new annotations, and concluded that using similar languages to train a model does not improve parsing. Although there are many resources for Latin, and some for Ancient Greek, we do not include them here, because they do not face the same challenges as medieval states of language, in particular the high level of spelling variability. And of course for Medieval French there is the SRCMF treebank that will be extensively used in chapter \ref{chap:bertrade}.
