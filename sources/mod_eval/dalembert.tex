\chapter{D'AlemBERT}


% \abstract{
%     Language models for historical states of language are becoming increasingly important to allow the optimal digitisation and analysis of old textual sources. Because these historical states are at the same time more complex to process and more scarce in the corpora available, specific efforts are necessary to train natural language processing (NLP) tools adapted to the data. In this paper, we present our efforts to develop NLP tools for Early Modern French (historical French from the 16\textsuperscript{th} to the 18\textsuperscript{th} centuries). We present the \freemmax corpus of Early Modern French and D'AlemBERT, a RoBERTa-based language model trained on \freemmax. We evaluate the usefulness of D'AlemBERT by fine-tuning it on a part-of-speech tagging task, outperforming previous work on the test set. Importantly, we find evidence for the transfer learning capacity of the language model, since its performance on lesser-resourced time periods appears to have been boosted by the more resourced ones. We release D'AlemBERT and the open-sourced subpart of the \freemmax corpus.
%     \\ \newline \Keywords{Digital humanities, Early Modern French, Language modelling, Neural language representation models, Less-resourced languages, Corpus creation, POS tagging} }



\section{D'AlemBERT: a neural language model for Early Modern French}\label{sec:dAlemBERT}
In this section, we describe the pretraining data, architecture, training objective and optimisation setup we use for \dalembert, our new neural language model for Early Modern French.


\subsection{Pre-processing}
Similar to \roberta \cite{liu-etal-2019-roberta} we segment the input text data into subword units using Byte-Pair encoding (BPE) \cite{sennrich-etal-2016-neural} in the implementation proposed by \cite{radford-etal-2019-language} that uses bytes instead of unicode characters as the base subword units. The BPE encoding does not require pre-tokenisation (at the word or token level), thus removing the need to develop a specific tokeniser for Early Modern French. We use a vocabulary size of 32,768 subword tokens. These subwords are learned on the entire \freemmax dataset.


\subsection{Language Modelling}

\paragraph{Transformer}
\dalembert uses the exact same architecture as \roberta, which is a multi-layer bidirectional Transformer \cite{vaswani-etal-2017-attention}.
\dalembert uses the original \emph{base} architecture of \roberta (12 layers, 768 hidden dimensions, 12 attention heads, 110M parameters).

\paragraph{Pretraining Objective}
We train our model on the Masked Language Modelling (MLM) task as proposed by RoBERTa's authors \cite{liu-etal-2019-roberta}: given an input text sequence composed of $N$ tokens $x_1, ..., x_N$, we select 15\% of tokens for possible replacement. Among those selected tokens, 80\% are replaced with the special \texttt{<MASK>} token, 10\% are left unchanged and 10\% are replaced by a random token. The model is then trained to predict the masked tokens using cross-entropy loss.

Again, following the \roberta approach, we dynamically mask tokens instead of fixing them statically for the whole dataset during preprocessing. We also choose not to use the next sentence prediction (NSP) task originally used in \bert \cite{devlin-etal-2019-bert}, as it has been shown that it does not improve downstream task performance \cite{conneau-lample-2019-cross,liu-etal-2019-roberta}.

\paragraph{Optimisation}
We optimise our model in the exact same way as \cite{liu-etal-2019-roberta} using Adam \cite{kingma-ba-2015-adam} ($\beta_1 = 0.9$, $\beta_2 = 0.98$) for 100k steps with large batch sizes of 8,192 sequences, each sequence containing at most 512 tokens.

\paragraph{Pre-training}
We use the \roberta implementation in the Zelda Rose library,\footnote{\url{https://github.com/LoicGrobol/zeldarose}} and again, in the same way as \newcite{liu-etal-2019-roberta} our learning rate is warmed up for 10k steps up to a peak value of $0.0003$ instead of the original $0.0001$ used by the original implementation of \roberta \cite{liu-etal-2019-roberta}, as our model diverged with the $0.0001$ value. We hypothesise that this is either due to the smaller size of \freemmax (compared to the corpora used for \roberta or \camembert) or to our large batch size. We train our model for 31k steps, which amounts to 41 epochs. The total pre-training times, the details of the infrastructure we used and even the carbon emissions of our model are reported in Appendix~\ref{carbon-footprint-dalembert}.

\section{Evaluation and Discussion}

\begin{table}[ht]
    \centering\small
    \resizebox{\linewidth}{!}{
        \begin{tabular}{lrrrrrr}
            \toprule
            \multicolumn{7}{c}{\textsc{Original}}                                                      \\
            \midrule
            Model        & 16             & 17             & 18             & 19 & 20 & Avg            \\
            \midrule
            \multicolumn{7}{l}{\hspace*{6mm}\emph{Drama}}                                              \\
            \pieextended & \emph{90.34}   & \emph{94.47}   & \emph{94.64}   & -  & -  & \emph{93.15}   \\
            \camembert   & 87.06          & 89.01          & 90.92          & -  & -  & 89.00          \\
            \dalembert   & \textbf{94.17} & \textbf{96.59} & \textbf{96.28} & -  & -  & \textbf{95.68} \\
            \multicolumn{7}{l}{\hspace*{6mm}\emph{Varia}}                                              \\
            \pieextended & \emph{89.85}   & \emph{93.44}   & \emph{95.98}   & -  & -  & \emph{93.09}   \\
            \camembert   & 86.90          & 88.85          & 92.85          & -  & -  & 89.53          \\
            \dalembert   & \textbf{93.86} & \textbf{95.73} & \textbf{96.95} & -  & -  & \textbf{95.51} \\
            \multicolumn{7}{l}{\hspace*{6mm}\emph{Both}}                                               \\
            \pieextended & \emph{90.08}   & \emph{93.95}   & \emph{95.33}   & -  & -  & \emph{ 93.12}  \\
            \camembert   & 86.98          & 88.93          & 91.89          & -  & -  & 89.27          \\
            \dalembert   & \textbf{94.02} & \textbf{96.16} & \textbf{96.62} & -  & -  & \textbf{95.60} \\
            \bottomrule
        \end{tabular}
        \begin{tabular}{lrrrrrr}
            \toprule
            \multicolumn{7}{c}{\textsc{Normalised or Contemporary}}                                                            \\
            \midrule
            Model        & 16             & 17             & 18             & 19             & 20             & Avg            \\
            \midrule
            \multicolumn{7}{l}{\hspace*{6mm}\emph{Drama}}                                                                      \\
            \pieextended & \emph{93.69}   & \emph{95.75}   & \emph{95.61}   & \emph{95.03}   & \emph{93.71}   & \emph{94.76}   \\
            \camembert   & 90.18          & 91.51          & 91.37          & 91.13          & 91.42          & 91.12          \\
            \dalembert   & \textbf{96.25} & \textbf{96.97} & \textbf{96.80} & \textbf{96.25} & \textbf{95.00} & \textbf{96.25} \\
            \multicolumn{7}{l}{\hspace*{6mm}\emph{Varia}}                                                                      \\
            \pieextended & \emph{92.52}   & \emph{94.81}   & \emph{95.98}   & \emph{92.24}   & \emph{94.03}   & \emph{93.94}   \\
            \camembert   & 89.79          & 90.69          & 93.06          & 90.54          & 89.78          & 93.94          \\
            \dalembert   & \textbf{94.52} & \textbf{96.64} & \textbf{96.88} & \textbf{94.90} & \textbf{95.30} & \textbf{95.65} \\
            \multicolumn{7}{l}{\hspace*{6mm}\emph{Both}}                                                                       \\
            \pieextended & \emph{93.08}   & \emph{95.28}   & \emph{95.80}   & \emph{93.65}   & \emph{93.87}   & \emph{94.35}   \\
            \camembert   & 89.99          & 91.10          & 92.22          & 90.84          & 90.60          & 92.53          \\
            \dalembert   & \textbf{95.39} & \textbf{96.81} & \textbf{96.84} & \textbf{95.58} & \textbf{95.15} & \textbf{95.95} \\
            \bottomrule
        \end{tabular}
    }
    \caption{Comparison between \dalembert, \camembert and \pieextended performance on \freemlpm.}
    \label{tab:POS}
\end{table}

In order to evaluate our \dalembert model, we fine-tune it for POS tagging on the \freemlpm corpus. We use the \texttt{flair} framework\footnote{\url{https://github.com/flairNLP/flair}} for sequence tagging \cite{akbik-etal-2019-flair}. To fine-tune \dalembert for POS we follow the same approach as \newcite{schweter-akbik-2020-flert} with some modifications: we append a linear layer of size 256 that takes as input the last hidden representation of the \texttt{<s>} special token and the mean of the last hidden representation of the subword units of each token (token as defined for \freemlpm), that is, we use a \emph{``mean''} subword pooling strategy. We fine-tune \dalembert with a learning rate of 0.000005 for a total of 10 epochs. We also fine-tune \camembert using the exact same hyperparameters as that we use for \dalembert.

\freemlpm provides a standard split (train, dev, test), however it also proposes an evaluation on a \emph{out-of-domain} subcorpus that is not contained in the standard split and that is separated by century (from the 16\textsuperscript{th} to the 20\textsuperscript{th} century) and that also contains both the \emph{Normalised} and \emph{Original} versions of the texts for the 16\textsuperscript{th}, 17\textsuperscript{th} and 18\textsuperscript{th} centuries. The idea of this out-of-domain evaluation corpus is to have a fine-grained evaluation of the models to better assess their performance in all the different types of text that one might encounter when working with Early Modern French data.

\begin{table}[ht]
    \centering\small
    \begin{tabular}{lrrr}
        \toprule
        Model & Precision & Recall & F1-Score \\
        \midrule
        LSTM-CRF  &   0.8640  &  0.8533  &  0.8586\\
        \camembert & \emph{0.9303}  &  \emph{0.9309}  &  \emph{0.9306} \\
        \dalembert & \textbf{0.9329}  &  \textbf{0.9323}  &  \textbf{0.9326}\\
        \bottomrule
    \end{tabular}
    \caption{Comparison between \dalembert, \camembert and \pieextended performance on \freemlpm.}
    \label{tab:dalembert-ner}
\end{table}

Following the approach of \newcite{clerice-2020-pie}, we report the scores obtained on the out-of-domain testing dataset of \freemlpm in Table~\ref{tab:POS}. We use the scores previously reported by \newcite{clerice-2020-pie} using \emph{Pie Extended} as our baseline as well as the fine-tuned \camembert that serves as a second baseline as well as a rough estimation of how much knowledge can \dalembert transfer from the \freemmax into this task.

We can see that \dalembert consistently outperforms \pieextended and \camembert in both the normalised and original versions of our out-of-domain testing data and for all different periods by a considerable margin. We can also see that on average the difference in score between \dalembert and \pieextended is greater for the original split than the normalised one. This suggests that \dalembert can generalise more effectively to non-normalised data than the more traditional architecture used by \pieextended. Moreover we can also see that the difference in scores is also greater for the 16\textsuperscript{th}\,c. and 17\textsuperscript{th}\,c. data. This is interesting, especially for the 16\textsuperscript{th}\,c, because, as we can see in Figure~\ref{fig:FreEMmax_desc}, this is the least represented period in the \freemmax corpus. This result actually suggests that \dalembert might be able to do effective transfer learning from the 18\textsuperscript{th}\,c., 19\textsuperscript{th}\,c. and 20\textsuperscript{th}\,c. data to the 16\textsuperscript{th}\,c. and 17\textsuperscript{th}\,c. data.

As for \camembert, we can see that it consistently scores lower than both \dalembert and \pieextended. Moreover, we can see that it struggles particularly with the non-normalised data of the 16\textsuperscript{th}\,c., 17\textsuperscript{th}\,c. and 18\textsuperscript{th}\,c.. This results clearly shows that \camembert cannot easily generalised to these earlier states of languages, or at least not with the quantity of data found in the training set of \freemlpm. These results also show the impressive capacity of \dalembert of quickly generalising to diverse set of states of language, as well as its capacity to transfer knowledge from the \freemmax corpus into this task. The obtained results are also a testament to the importance of the pre-training data, specially taking in account that the pre-training set of \camembert is more than 100 times bigger than that of \dalembert.

\section{Conclusion}

In this paper we presented the manually curated \freemmax corpus of Early Modern French as well as \dalembert, a RoBERTa-based language model trained on \freemmax. With \dalembert, we showed that it is possible to successfully train a transformer-based language model for historical French with even less data than originally shown in previous works \cite{martin-etal-2020-camembert}. Moreover with our POS tagging evaluation we were able to observe some form of transfer learning and generalisation across multiple states of the language corresponding to different periods of time. Both our corpus and our model will be of use to digital humanists and linguists interested in Early Modern French. For our future work, we hope that will be able to study the application of our \dalembert model to other NLP tasks such as text normalisation, named entity recognition and even document structuring, where we hope to more extensively study the transfer learning capabilities of our approach.
