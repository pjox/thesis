%%%%%%%%%%%%%%%%%%%%%%%%%%%%%%%%%%%%%%%%%%%%%%%%%%%%%%%%%%%%%%%%%%%%%%%%
\chapter{On Language Models and Downstream Tasks}
%%%%%%%%%%%%%%%%%%%%%%%%%%%%%%%%%%%%%%%%%%%%%%%%%%%%%%%%%%%%%%%%%%%%%%%%

\begin{center}
    \begin{minipage}{0.66\textwidth}
        \begin{small}
            In which we give a brief overview of word representations, we present a few of the non-English representations that were available when we began this Ph.D. thesis. We also give an overview of the downstream tasks that we will use for evaluation as well as brief description of the state-of-the-art models that were available when we started working on these tasks. We finally give an overview of the available neural models for historical languages. 
        \end{small}
    \end{minipage}
    \vspace{0.5cm}
\end{center}

Language models and more specifically word representations take a central part in the approach that we have chosen to tackle the tasks laid by the BASNUM project. In this chapter we give an overview of the available word representations that were available we began this Ph.D. thesis as well as a small description of the architecture of said representations.

We will also discuss the downstream tasks that we will use for evaluation focusing on particular on what was available for the French language throughout the course of this Ph.D. thesis. We also give briefly discuss the state-of-the-art models and studies for these tasks. Finally, we end the chapter by discussing some advancements in neural models for historical languages.

\section{On Word Representations}

One of the key elements that has pushed the state of the art considerably in neural NLP in recent years has been the introduction and spread of transfer learning methods to the field. These methods can normally be classified in two categories according to how they are used:
\begin{itemize}
    \item \emph{Feature-based} methods, which involve pre-training real-valued vectors (``embeddings'') at the word, sentence, or paragraph level; and using them in conjunction with a specific architecture for each individual downstream task.
    \item \emph{Fine-tuning} methods, which introduce a minimal number of task-specific parameters, and instead copy the weights from a pre-trained network and then tune them to a particular downstream task.
\end{itemize}
Embeddings or language models can be divided into \emph{fixed}, meaning that they generate a single representation for each word in the vocabulary; and \emph{contextualized}, meaning that a representation is generated based on both the word and its surrounding context, so that a single word can have multiple representations, each one depending on how it is used.

In practice, most fixed embeddings are used as feature-based models. The most notable examples are \emph{word2vec} \citep{mikolov-etal-2013-distributed}, \emph{GloVe} \citep{pennington-etal-2014-glove} and \emph{FastText} \citep{mikolov-etal-2018-advances}. All of them are extensively used in a variety of applications nowadays. On the other hand, contextualized word representations and language models have been developed using both feature-based architectures, the most notable examples being ELMo and Flair \citep{peters-etal-2018-deep,akbik-etal-2018-contextual}, and transformer based architectures, that are commonly used in a fine-tune setting, as is the case of GPT-1, GPT-2 \citep{radford-etal-2018-improving,radford-etal-2019-language}, BERT and its derivatives \citep{devlin-etal-2019-bert,liu-etal-2019-roberta,lan-etal-2020-albert} and more recently T5 \citep{raffel-etal-2020-exploring}. All of them have repeatedly improved the state-of-the art in many downstream NLP tasks over the last year.

In general, the main advantage of using language models is that they are mostly built in an \emph{unsupervised} manner, and they can be trained with raw, unannotated plain text. However, their main drawback is that enormous quantities of data seem to be required to properly train them especially in the case of contextualized models, for which larger corpora are thought to be needed to properly address polysemy and cover the wide range of uses that commonly exist within languages.

For the first English version of word2vec, \citet{mikolov-etal-2013-distributed} used a one billion word dataset consisting of various news articles. Later \citet{al-rfou-etal-2013-polyglot} and then \citet{bojanowski-etal-2017-enriching} used the plain text from Wikipedia to train distributions of word2vec and FastText respectively, for languages other than English. Now, the problem of obtaining large quantities of data aggravates even more for contextual models, as they normally need multiple instances of a given word in order to capture all its different uses and in order to avoid overfitting due to the large quantity of hyperparameters that these models have. \citet{peters-etal-2018-deep} for example use a 5.5 billion token\footnote{Punctuation marks are counted as tokens.} dataset comprised of crawled news articles plus the English Wikipedia in order to train ELMo, \citet{devlin-etal-2019-bert} use a 3.3 billion word\footnote{Space sparated tokens.} corpus made by merging the English Wikipedia with the BooksCorpus \citep{zhu-etal-2015-aligning}, and \citet{radford-etal-2019-language} use a 40GB English corpus created by scraping outbound links from Reddit.\footnote{\url{https://www.reddit.com/}}

For gathering data in a wide range of languages, Wikipedia is a commonly used option. It has been used to train fixed embeddings \citep{al-rfou-etal-2013-polyglot,bojanowski-etal-2017-enriching} and more recently the multilingual BERT (mBERT) \citep{devlin-etal-2019-bert}. However, for some languages, Wikipedia might not be large enough to train good quality contextualized word embeddings. Moreover, Wikipedia data all belong to the same specific genre and style. To address this problem, one can resort to crawled text from the internet; the largest and most widespread dataset of crawled text being Common Crawl.\footnote{\url{https://commoncrawl.org}} Such an approach generally solves the quantity and genre/style coverage problems but might introduce noise in the data, an issue which has earned the corpus some criticism, most notably by \citet{trinh-le-2018-a} and \citet{radford-etal-2019-language}. Using Common Crawl also leads to data management challenges as the corpus is distributed in the form of a large set of plain text each containing a large quantity of unclassified multilingual documents from different websites.

Concerning contextual models, \citet{baevski-etal-2019-cloze} trained a BERT-like model for English using Common Crawl. They followed the ``fastText pre-processing pipeline'' but they removed all copies of Wikipedia inside Common Crawl. They also trained their model using News Crawl \citep{bojar-etal-2018-findings} and using Wikipedia + BooksCorpus, they compared three models and showed that Common Crawl gives the best performance out of the three corpora in spite of the previously mentioned criticism.

The XLNet model was trained for English by joining the BookCorpus, English Wikipedia, Giga5 \citep{parker-etal-2011-english}, ClueWeb 2012-B \citep{callan-etal-2009-clueweb09} and Common Crawl. Particularly for Common Crawl, \citet{yang-etal-2019-xlnet} say they use ``heuristics to aggressively filter out short or low-quality articles'' from Common Crawl, however they don't give any detail about these ``heuristics'' nor about the pipeline they use to classify and extract the English part of Common Crawl. It is important to note that none of these projects distributed their classified, filtered and cleaned versions of Common Crawl, making it difficult in general to faithfully reproduce their results.

\subsection{Some Language Model Architectures}

\paragraph{ELMo: Contextualized word embeddings}
\emph{Embeddings from Language Models} (ELMo) \citep{peters-etal-2018-deep} is a Language Model, i.e, a model that given a sequence of $N$ tokens, $(t_1, t_2, ..., t_N)$, computes the probability of the sequence by modeling the probability of token $t_k$ given the history $(t_1, ..., t_{k-1})$:
\[
    p(t_1, t_2, \ldots, t_N) = \prod_{k=1}^N p({t_k} \mid t_1, t_2, \ldots, t_{k-1}).
\]
More precisely, ELMo uses a bidirectional language model, which combines a forward and a backward LSTM-based language model. ELMo also computes a context-independent token representation via a CNN over characters.

\paragraph{BERT and RoBERTa} \roberta \citep{liu-etal-2019-roberta} which itself is based on \bert \citep{devlin-etal-2019-bert}. \bert is a multi-layer bidirectional Transformer encoder trained with a masked language modeling (MLM) objective, inspired by the Cloze task \citep{taylor-1953-cloze}. It comes in two sizes: the \bertbase architecture and the \bertlarge architecture. The \bertbase architecture is 3 times smaller and therefore faster and easier to use while \bertlarge achieves increased performance on downstream tasks. \roberta improves the original implementation of \bert by identifying key design choices for better performance, using dynamic masking, removing the next sentence prediction task, training with larger batches, on more data, and for longer.

%%%%%%%%%%%%%%%%%%%%%%%%%%%%%%%%%%%%%%%%%%%%%%%%%%%%%%%%%%%%%%%%%%%%%%%%
\section{Non-English Word Representations}
%%%%%%%%%%%%%%%%%%%%%%%%%%%%%%%%%%%%%%%%%%%%%%%%%%%%%%%%%%%%%%%%%%%%%%%%

Since the introduction of \emph{word2vec} \citep{mikolov-etal-2013-distributed}, many attempts have been made to create multilingual language representations; for fixed word embeddings the most remarkable works are those of \citep{al-rfou-etal-2013-polyglot} and \citep{bojanowski-etal-2017-enriching} who created word embeddings for a large quantity of languages using Wikipedia, and later \citep{grave-etal-2018-learning} who trained the FastText word embeddings for 157 languages using Common Crawl and who in fact showed that using crawled data significantly increased the performance of the embeddings especially for mid- to low-resource languages.

Regarding contextualized models, the most notable non-English contribution has been that of the mBERT \citep{devlin-etal-2019-bert}, which is distributed as (i)~a single multilingual model for 100 different languages trained on Wikipedia data, and as (ii)~a single multilingual model for both Simplified and Traditional Chinese. Four monolingual fully trained ELMo models have been distributed for Japanese, Portuguese, German and Basque\footnote{\url{https://allennlp.org/elmo}}; 44 monolingual ELMo models\footnote{\url{https://github.com/HIT-SCIR/ELMoForManyLangs}} where also released by the \emph{HIT-SCIR} team \citep{che-etal-2018-towards} during the \emph{CoNLL 2018 Shared Task} \citep{zeman-etal-2018-conll}, but their training sets where capped at 20 million words.

Following the success of large pre-trained language models, they were extended to the multilingual setting with multilingual \bert (hereafter \mbert) \citep{devlin-etal-2019-bert}, a single multilingual model for 104 different languages trained on Wikipedia data, and later XLM \citep{conneau-lample-2019-cross}, which significantly improved unsupervized machine translation. More recently XLM-R \citep{conneau-etal-2020-unsupervised}, extended XLM by training on 2.5TB of data and outperformed previous scores on multilingual benchmarks. They show that multilingual models can obtain results competitive with monolingual models by leveraging higher quality data from other languages on specific downstream tasks.

A few non-English monolingual models have been released: ELMo models for Japanese, Portuguese, German and Basque\footnote{\url{https://allennlp.org/elmo}} and BERT for Simplified and Traditional Chinese \citep{devlin-etal-2019-bert} and German \citep{chan-etal-2019-german}.

However, to the best of our knowledge, at the beginning of this Ph.D. thesis, no particular effort had been made toward training models for languages other than English at a scale similar to the latest English models (e.g.~\roberta trained on more than 100GB of data).

\subsection{Recent Developments in Contextualized Representations}
Since the introduction of contextualized word representations \citep{peters-etal-2018-deep,akbik-etal-2018-contextual,devlin-etal-2019-bert} and the many improvements proposed for them in the consumption of computational resources \citep{clark-etal-2020-electra}, in the amount of data required to fine-tune them \citep{raffel-etal-2020-exploring}, and more recently in the length of the contextual window \citep{xiong-etal-2021-nystromformer}; there have also been important advancements from a digital humanities point of view on \emph{unsupervised domain adaptation} \citep{ramponi-plank-2020-neural}. In this case, one specializes a language model to a particular domain with unlabeled data in order to improve performance in downstream tasks. This can be achieved by  pre-training the models from scratch with specialized data \citep{beltagy-etal-2019-scibert} or by continuing the training of a general model with a new corpus \citep{lee-etal-2019-BioBERT, peng-etal-2019-transfer}. This last method has already been successfully implemented in the context of historical languages, in particular \citet{han-eisenstein-2019-unsupervised} showed that one can successfully adapt the original BERT \citep{devlin-etal-2019-bert} to Early Modern English by continuing the pre-training on historical raw texts.

In a multilingual context, transformer-based models such as mBERT have been adapted to low-resource languages and evaluated in dependency parsing and POS-tagging showing promising results \citep{chau-etal-2020-parsing, muller-etal-2021-unseen, gururangan-etal-2020-dont, wang-etal-2020-extending}. However, this multilingual approach has also been criticized for favoring monolingual pre-training even when data is scarce \citep{virtanen-etal-2019-multilingual, ortiz-suarez-etal-2020-monolingual}. Indeed, even when only small pre-training corpora are available, BERT-like models have also been successfully pre-trained, resulting in well-performing models \citep{micheli-etal-2020-importance}. Furthermore, compact BERT-like models have also been studied \citep{turc-etal-2019-well} and might prove useful in data constrained conditions, such as monolingual pre-training of contextualized word representation for low-resource languages. In any case BERT-Like models have been one of the most prolific subjects of study in NLP since the start of this Ph.D. thesis \citep{rogers-etal-2020-primer}.

\section{Downstream Tasks Evaluation}\label{MethodEVAL}

POS-tagging is a low-level syntactic task, which consists in assigning to each word its corresponding grammatical category. Dependency-parsing consists of higher order syntactic task like predicting the labeled syntactic tree capturing the syntactic relations between words. We evaluate the performance of our models throughout this text, by using the standard UPOS accuracy for POS-tagging, and Unlabeled Attachment Score (UAS) and Labeled Attachment Score (LAS) for dependency parsing. We will always assume gold tokenization and gold word segmentation as provided in the UD treebanks.

For both of these tasks we always run our experiments using the Universal Dependencies (UD)\footnote{\url{https://universaldependencies.org}} framework and its corresponding UD POS tag set \citep{petrov-etal-2012-universal} and UD treebank collection \citep{nivre-etal-2018-universal}, which was used for the CoNLL 2018 shared task \citep{seker-etal-2018-universal}. We often perform the evaluations of our Contemporary French models on the four freely available French UD treebanks in UD~v2.2: GSD \citep{mcdonald-etal-2013-universal}, Sequoia\footnote{\url{https://deep-sequoia.inria.fr}} \citep{candito-seddah-2012-le,candito-etal-2014-deep}, Spoken \citep{lacheret-etal-2014-rhapsodie,bawden-etal-2014-correcting}\footnote{Speech transcript uncased that includes annotated disfluencies without punctuation}, and ParTUT \citep{sanguinetti-Bosco-2015-parttut}.

We will further evaluate some of our models on NLI, using the French part of the XNLI dataset \citep{conneau-etal-2018-xnli}. NLI consists in predicting whether a hypothesis sentence is entailed, neutral or contradicts a premise sentence. The XNLI dataset is the extension of the Multi-Genre NLI (MultiNLI) corpus \citep{williams-etal-2018-broad} to 15 languages by translating the validation and test sets manually into each of those languages. The English training set is machine translated for all languages other than English. The dataset is composed of 122k train, 2490 development and 5010 test examples for each language. As usual, NLI performance is evaluated using accuracy.

Finally, we also evaluate our models in NER, which is a sequence labeling task predicting which words refer to real-world objects, such as people, locations, artifacts and organizations. We use the French Treebank\footnote{This dataset has only been stored and used on Inria's servers after signing the research-only agreement.} (FTB) \citep{abeille-etal-2003-building} in its 2008 version introduced by \citet{candito-crabbe-2009-improving} and with NER annotations by \citet{sagot-etal-2012-annotation}. The FTB contains more than 11 thousand entity mentions distributed among 7 different entity types. A brief overview of the FTB can also be found in Table \ref{treebanks-tab}. As NER plays a central role in this Ph.D. thesis, beign the task in which we evaluate our models the most, we give an expanded overview of the task here below.

%%%%%%%%%%%%%%%%%%%%%%%%%%%%%%%%%%%%%%%%%%%%%%%%%%%%%%%%%%%%%%%%%%%%%%%%
\subsubsection{Named Entity Recognition}
%%%%%%%%%%%%%%%%%%%%%%%%%%%%%%%%%%%%%%%%%%%%%%%%%%%%%%%%%%%%%%%%%%%%%%%%

Named entity recognition (NER) is the widely studied task consisting in identifying text spans that denote \emph{named entities} such as person, location and organization names, to name the most important types. Such text spans are called named entity \emph{mentions}. In NER, mentions are generally not only identified, but also classified according to a more or less fine-grained ontology, thereby allowing for instance to distinguish between the telecommunication company \emph{Orange} and the town \emph{Orange} in southern France (among others). Importantly, it has long been recognized that the type of named entities can be defined in two ways, which underlies the notion of metonymy: the intrinsic type (\emph{France} is always a location) and the contextual type (in \emph{la France a signé un traité} `France signed a treaty', \emph{France} denotes an organization).

NER has been an important task in natural language processing for quite some time. It was already the focus of the MUC conferences and associated shared tasks \citep{marsh-perzanowski-1998-muc}, and later that of the CoNLL~2003 and ACE shared tasks \citep{tjong-kim-sang-de-meulder-2003-introduction,doddington-etal-2004-automatic}. Traditionally, as for instance was the case for the MUC shared tasks, only person names, location names, organization names, and sometimes ``other proper names'' are considered. However, the notion of named entity mention is sometimes extended to cover any text span that does not follow the general grammar of the language at hand, but a type- and often culture-specific grammar, thereby including entities ranging from product and brand names to dates and from URLs to monetary amounts and other types of numbers.

As for many other tasks, NER was first addressed using rule-based approaches, followed by statistical and now neural machine learning techniques (see Section~\ref{subsec:sota} for a brief discussion on NER approaches). Of course, evaluating NER systems as well as training machine-learning-based NER systems, statistical or neural, require named-entity-annotated corpora. Unfortunately, most named entity annotated French corpora are oral transcripts, and they are not always freely available. The ESTER and ESTER2 corpora (60 plus 150 hours of NER-annotated broadcast transcripts) \citep{galliano-etal-2005-the,galliano-etal-2009-the}, as well as the Quaero \citep{grouin-etal-2011-proposal} corpus are based on oral transcripts (radio broadcasts). Interestingly, the Quaero corpus relies on an original, very rich and structured  definition of the notion of named entity \citep{rosset-etal-2011-entites}. It contains both the intrinsic and the contextual types of each mention, whereas the ESTER and ESTER2 corpora only provide the contextual type.

%%%%%%%%%%%%%%%%%%%%%%%%%%%%%%%%%%%%%%%%%%%%%%%
%%%%%%%%%%%%%%%%%%%%%%%%%%%%%%%%%%%%%%%%%%%%%%%
\subsection{Evaluation Datasets}

\paragraph{Treebanks}
As previously stated, we perform most our evaluations on the four freely available French UD treebanks in UD~v2.2: GSD, Sequoia, Spoken, and ParTUT, presented in Table \ref{treebanks-tab}.

\textbf{GSD} treebank \citep{mcdonald-etal-2013-universal} is the second-largest tree-bank available for French after the FTB (described in subsection \ref{ner-section}), it contains data from blogs, news, reviews, and Wikipedia.

\textbf{Sequoia} tree-bank \citep{candito-etal-2014-deep} comprises more than 3000 sentences, from the French Europarl, the regional newspaper \emph{L’Est Républicain}, the French Wikipedia and documents from the European Medicines Agency.

\textbf{Spoken} was automatically converted from the Rhapsodie tree-bank \citep{lacheret-etal-2014-rhapsodie} with manual corrections. It consists of 57 sound samples of spoken French with phonetic transcription aligned with sound (word boundaries, syllables, and phonemes), syntactic and prosodic annotations.

\textbf{ParTUT} is a conversion of a multilingual parallel treebank developed at the University of Turin, and consisting of a variety of text genres, including talks, legal texts, and Wikipedia articles, among others; ParTUT data is derived from the already-existing parallel treebank, Par(allel)TUT \citep{sanguinetti-Bosco-2015-parttut}. Table~\ref{treebanks-tab} contains a summary comparing the sizes of the treebanks.


\begin{table}[ht]
    \centering\small
        \begin{tabular}{lccl}
            \toprule
            Treebank                         & \#Tokens                         & \#Sentences                     & \multicolumn{1}{l}{Genres} \\
            \midrule
                                             &                                  &                                 & Blogs, News                \\
            \multirow{-2}{*}[1.5pt]{GSD}     & \multirow{-2}{*}[1.5pt]{389,363} & \multirow{-2}{*}[1.5pt]{16,342} & Reviews, Wiki              \\ \tabucline[\hbox {$\scriptstyle \cdot$}]{-}
                                             &                                  &                                 & Medical, News              \\
            \multirow{-2}{*}[0.7pt]{Sequoia} & \multirow{-2}{*}[0.7pt]{68,615}  & \multirow{-2}{*}[0.7pt]{3,099}  & Non-fiction, Wiki          \\ \tabucline[\hbox {$\scriptstyle \cdot$}]{-}
            Spoken                           & 34,972                           & 2,786                           & Spoken                     \\ \tabucline[\hbox {$\scriptstyle \cdot$}]{-}
            ParTUT                           & 27,658                           & 1,020                           & Legal, News, Wikis         \\ \tabucline[\hbox {$\scriptstyle \cdot$}]{-}
            FTB                              & 350,930                          & 27,658                          & News                       \\
            \bottomrule
        \end{tabular}
    \caption{Statistics on the treebanks used in POS tagging, dependency parsing, and NER (FTB).}\label{treebanks-tab}
\end{table}

Again as we pay special attention to the NER task, and we actually work directly on the FTB by aligning both the Universal Dependencies and the TEI-annotated NER version of it in section \ref{subsec:alignment}, we provide an expanded overview of the corpus here below.

\subsubsection{The original named entity FTB annotation layer}\label{ner-section}
\label{evalner}\label{subsec:originalannotations}

\citet{sagot-etal-2012-annotation} annotated the FTB with the span, absolute type\footnote{Every mention of \emph{France} is annotated as a \texttt{Location} with subtype \texttt{Country}, as given in \aleda database, even if in context the mentioned entity is a political organization, the French people, a sports team, etc.}, sometimes subtype and \aleda unique identifier of each named entity mention.\footnote{Only proper nouns are considered as named entity mentions, thereby excluding other types of referential expressions.} Annotations are restricted to person, location, organization and company names, as well as a few product names.\footnote{More precisely, we used a tagset of 7 base NE types: \texttt{Person}, \texttt{Location}, \texttt{Organization}, \texttt{Company}, \texttt{Product}, \texttt{POI} (Point of Interest) and \texttt{FictionChar}.} There are no nested entities. Non capitalized entity mentions (e.g.~\emph{banque mondiale} `World Bank') are annotated only if they can be disambiguated independently of their context. Entity mentions that require the context to be disambiguated (e.g.~\emph{Banque centrale}) are only annotated if they are capitalized. \footnote{So for instance, in \emph{université de Nantes} `Nantes university', only \emph{Nantes} is annotated, as a city, as \emph{université} is written in lowercase letters. However, \emph{Université de Nantes} `Nantes University' is wholly annotated as an organization. It is non-ambiguous because \emph{Université} is capitalized. \emph{Université de Montpellier} `Montpellier University' being ambiguous when the text of the FTB was written and when the named entity annotations were produced, only \emph{Montpellier} is annotated, as a city.} For person names, grammatical or contextual words around the mention are not included in the mention (e.g.~in \emph{M.~Jacques Chirac} or \emph{le Président Jacques Chirac}, only \emph{Jacques Chirac} is included in the mention).

Tags used for the annotation have the following information:
\begin{itemize}
    \item the identifier of the NE in the \aleda database (\texttt{eid} attribute); when a named entity is not present in the database, the identifier is \texttt{null},\footnote{Specific conventions for entities that have merged, changed name, ceased to exist as such (e.g.~\emph{Tchequoslovaquie}) or evolved in other ways are described in \citet{sagot-etal-2012-annotation}.}
    \item the normalized named of the named entity as given in \aleda; for locations it is their name as given in GeoNames and for the others, it is the title of the corresponding French Wikipedia article,
    \item the type and, when relevant, the subtype of the entity.
\end{itemize}
Here are two annotation examples:\\
\noindent{\small\texttt{<ENAMEX type="Organization" eid="1000000000016778"
        name="Confédération\\
        française démocratique du travail">CFDT</ENAMEX>\\
        <ENAMEX type="Location"
        sub\_type="Country"
        eid="2000000001861060"\\
        name="Japan">Japon</ENAMEX>}}

\citet{sagot-etal-2012-annotation} annotated the 2007 version of the FTB treebank (with the exception of sentences that did not receive any functional annotation), i.e.~12,351 sentences comprising 350,931 tokens. The annotation process consisted in a manual correction and validation of the output of a rule- and heuristics-based named entity recognition and linking tool in an XML editor.
Only a single person annotated the corpus, despite the limitations of such a protocol, as acknowledged by \citet{sagot-etal-2012-annotation}.

In total, 5,890 of the 12,351 sentences contain at least a named entity mention. 11,636 mentions were annotated, which are distributed as follows: 3,761 location names, 3,357 company names, 2,381 organization names, 2,025 person names, 67 product names, 29 fiction character names and 15 points of interest.

\subsection{Brief State of the Art for Dependency Parsing and POS Tagging}

For dependency parsing and POS tagging the most notable non-English specific contribution is that of the \emph{CoNLL 2018 Shared Task} \citep{zeman-etal-2018-conll}, where the 1\textsuperscript{st} place (LAS Ranking) was awarded to the \emph{HIT-SCIR} team \citep{che-etal-2018-towards} who used \citet{dozat-manning-2017-deep}'s \emph{Deep Bi-affine parser} and its extension described in \citep{dozat-etal-2017-stanfords}, coupled with deep contextualized ELMo embeddings \citep{peters-etal-2018-deep} (capping the training set at 20 million words). The 1\textsuperscript{st} place in universal POS tagging was awarded to \citet{smith-etal-2018-82} who used two separate instances of \citet{bohnet-etal-2018-morphosyntactic}'s tagger.

More recent developments in POS tagging and parsing include those of \citet{straka-strakova-2019-evaluating} which couples another CoNLL 2018 shared task participant, UDPipe 2.0 \citep{straka-2018-udpipe}, with mBERT greatly improving the scores of the original model, and UDify \citep{kondratyuk-straka-2019-75}, which adds an extra attention layer on top of mBERT plus a Deep Bi-affine attention layer for dependency parsing and a Softmax layer for POS tagging. UDify is actually trained by concatenating the training sets of 124 different UD treebanks, creating a single POS tagging and dependency parsing model that works across 75 different languages.

\subsection{Brief State of the Art for NER}
\label{subsec:sota}

As mentioned above, NER was first addressed using rule-based approaches, followed by statistical and now neural machine learning techniques. In addition, many systems use a lexicon of named entity mentions, usually called a ``gazetteer'' in this context.

Most of the advances in NER  have been achieved on English, in particular with the CoNLL 2003 \citep{tjong-kim-sang-de-meulder-2003-introduction} and  Ontonotes~v5 \citep{pradhan-etal-2012-conll,pradhan-etal-2013-towards} corpora. In recent years, NER was traditionally tackled using Conditional Random Fields (CRF) \citep{lafferty-etal-2001-conditional} which are quite suited for NER; CRFs were later used as decoding layers for Bi-LSTM architectures \citep{huang-etal-2015-bidirectional,lample-etal-2016-neural} showing considerable improvements over CRFs alone. These Bi-LSTM-CRF architectures were later enhanced with contextualized word embeddings which yet again brought major improvements to the task \citep{peters-etal-2018-deep,akbik-etal-2018-contextual}. Finally, large pre-trained architectures settled the current state of the art showing a small yet important improvement over previous NER-specific architectures \citep{devlin-etal-2019-bert,baevski-etal-2019-cloze}.

For French, rule-based system have been developed until relatively recently, due to the lack of proper training data \citep{sekine-nobata-2004-definition,rosset-etal-2005-interaction,stern-sagot-2010-resources,nouvel-etal-2014-pattern}. The limited availability of a few annotated corpora made it possible to apply statistical machine learning techniques \citep{bechet-charton-2010-unsupervised,dupont-tellier-2014-named,dupont-2017-exploration} as well as hybrid techniques combining handcrafted grammars and machine learning \citep{bechet-etal-2011-cooperation}. To the best of our knowledge, the best results previously published on FTB NER are those obtained by \citet{dupont-2017-exploration}, who trained both CRF and BiLSTM-CRF architectures and improved them using heuristics and pre-trained word embeddings.

Leaving aside French and English, the CoNLL 2002 shared task included NER corpora for Spanish and Dutch corpora \citep{tjong-kim-sang-2002-introduction} while the CoNLL 2003 shared task included a German corpus \citep{tjong-kim-sang-de-meulder-2003-introduction}. The recent efforts by \citet{strakova-etal-2019-neural} settled the state of the art for Spanish and Dutch, while \citet{akbik-etal-2018-contextual} did so for German.

\subsection{Common Baselines}
In dependency parsing and POS-tagging we will often compare our models with:

\begin{itemize}
    \item \emph{\mbert}: The multilingual cased version of \bert. We often fine-tune \mbert on each of the treebanks with an additional layer for POS-tagging and dependency parsing.
    \item \emph{\xlmmlmtlm}: A multilingual pretrained language model from \citet{conneau-lample-2019-cross}, which showed better performance than \mbert on NLI. We use the version available in the Hugging's Face transformer library \citep{wolf-etal-2019-huggingface}.
    \item \emph{UDify} \citep{kondratyuk-straka-2019-75}: A multitask and multilingual model based on \mbert, UDify is trained simultaneously on 124 different UD treebanks, creating a single POS tagging and dependency parsing model that works across 75 different languages. We always report the scores from \citet{kondratyuk-straka-2019-75} paper.
    \item \emph{UDPipe Future} \citep{straka-2018-udpipe}: An LSTM-based model ranked 3\textsuperscript{rd} in dependency parsing and 6\textsuperscript{th} in POS tagging at the CoNLL~2018 shared task \citep{seker-etal-2018-universal}. We always report the scores from \citet{kondratyuk-straka-2019-75} paper.
    \item \emph{UDPipe Future + \mbert + Flair} \citep{straka-strakova-2019-evaluating}: The original UDPipe Future implementation using \mbert and Flair as feature-based contextualized word embeddings. We always report the scores from \citet{straka-strakova-2019-evaluating} paper.
\end{itemize}

In French, as discussed avobe, no extensive work has been done on NER due to the limited availability of annotated corpora. Thus, we compare our models with the only recent available baselines set by \citet{dupont-2017-exploration}, who trained both CRF \citep{lafferty-etal-2001-conditional} and BiLSTM-CRF \citep{lample-etal-2016-neural} architectures on the FTB and enhanced them using heuristics and pretrained word embeddings. Additionally, as for POS and dependency parsing, we often compare our models to a fine-tuned version of \mbert for the NER task.

For XNLI, we provide the scores of \mbert which has been reported for French by \citet{wu-dredze-2019-beto}. We report scores from \xlmmlmtlm (described above), the best model from \citet{conneau-lample-2019-cross}. We will also report the results of \mbox{XLM-R} \citep{conneau-etal-2020-unsupervised}.

%%%%%%%%%%%%%%%%%%%%%%%%%%%%%%%%%%%%%%%%%%%%%%%%%%%%%%%%%%%%%%%%%%%%%%%%
\section{Neural Models for Historical Languages}
%%%%%%%%%%%%%%%%%%%%%%%%%%%%%%%%%%%%%%%%%%%%%%%%%%%%%%%%%%%%%%%%%%%%%%%%

Efficient language models have been trained for less-resourced or extinct Languages such as Latin \citep{bamman-burns-2020-latin}, following the approach of \citet{martin-etal-2020-camembert} for training language models with fewer data than was previously thought. There have also been some recent projects that specifically target Early Modern French such as that of \pieextended \citep{clerice-2020-pie} that uses the hierarchical encoding architecture originally proposed by \citet{manjavacas-etal-2019-improving} which itself is constructed by stacking multiple Bi-LSTM-CRFs. \citet{clerice-2020-pie} distributes pre-trained models for POS tagging and lemmatisation.

Lastly, concerning dependency parsing and POS-tagging of Old French in particular, the works of \citet{guibon-etal-2014-parsing} and \citet{stein-2014-parsing, stein-2016-old} are noteworthy. However, they use very different approaches to the one used throughout this thesis and evaluate on previous versions of SRCMF, with incompatible annotation choices and slightly different texts. For the UD version of SRCMF, the most notable work is that of UDPipe 2.0 \citep{straka-2018-udpipe}, which was later enhanced by including contextualized word embeddings \citep{straka-strakova-2019-evaluating}.
