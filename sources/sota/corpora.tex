%%%%%%%%%%%%%%%%%%%%%%%%%%%%%%%%%%%%%%%%%%%%%%%%%%%%%%%%%%%%%%%%%%%%%%%%
\chapter{On Raw Corpora for Language Modeling}
%%%%%%%%%%%%%%%%%%%%%%%%%%%%%%%%%%%%%%%%%%%%%%%%%%%%%%%%%%%%%%%%%%%%%%%%

The frWaC corpus is a French text corpus collected from the .fr domain with using medium-frequency words from the Le Monde Diplomatique corpus and basic French vocabulary lists as seeds. The corpus consists of French websites with total size 1.3 billion words.



\section{Common Crawl}

Common Crawl is a non-profit foundation which produces and maintains an open repository of web crawled data that is both accessible and analyzable.\footnote{\url{http://commoncrawl.org/about/}} Common Crawl's complete web archive consists of petabytes of data collected over 8 years of web crawling. The repository contains raw web page HTML data (WARC files), metadata extracts (WAT files) and plain text extracts (WET files). The organization's crawlers has always respected \texttt{nofollow}\footnote{\url{http://microformats.org/wiki/rel-nofollow}} and \texttt{robots.txt}\footnote{\url{https://www.robotstxt.org/}} policies.

Each monthly Common Crawl snapshot is in itself a massive multilingual corpus, where every single file contains data coming from multiple web pages written in a large variety of languages and covering all possible types of topics. Thus, in order to effectively use this corpus for Natural Language Processing and Machine Learning applications, one has first to extract, filter, clean and classify the data in the snapshot by language.

Throughout this thesis, we will use the WET files which contain the extracted plain texts from the websites mostly converted to UTF-8, as well as headers containing the metadata of each crawled document. Each WET file comes compressed in gzip format\footnote{\url{https://www.gnu.org/software/gzip/}} and is stored on Amazon Web Services.

Common Crawl has already been successfully used to train language models, even multilingual ones. The most notable example is probably fastText which was first trained for English using Common Crawl \citep{mikolov-etal-2018-advances} and then for other 157 different languages \citep{grave-etal-2018-learning}. In fact \citet{grave-etal-2018-learning} proposed a pipeline to filter, clean and classify Common Crawl, which we shall call the ``fastText pre-processing pipeline.'' They used the fastText linear classifier \citep{joulin-etal-2016-fasttext, joulin-etal-2017-bag} to classify each line of Common Crawl by language, and downloaded the initial corpus and schedule the I/O using some simple Bash scripts. Their solution, however, proved to be a synchronous blocking pipeline that works well on infrastructures having the necessary hardware to assure high I/O speeds even when storing tens of terabytes of data at a time. But that downscales poorly to medium-low resource infrastructures that rely on more traditional cost-effective electromechanical mediums in order to store this amount of data.

\subsection{fastText's Pipeline}

The \emph{``fastText pre-processing pipeline''} used by \citet{grave-etal-2018-learning} launches multiple process, preferably as many as available cores. Each of these processes first downloads one Common Crawl WET file which then proceeds to decompress after the download is over. After decompressing, an instance of the fastText linear classifier \citep{joulin-etal-2016-fasttext, joulin-etal-2017-bag} is launched, the classifier processes each WET file line by line, generating a language tag for each line. The tags are then stored in a tag file which holds a one-to-one correspondence between lines of the WET file and its corresponding language tag. The WET file and the tag files are read sequentially and each on the WET file line holding the condition of being longer than 100 bytes is appended to a language file containing only plain text (tags are discarded). Finally, the tag file and the WET files are deleted.

Only when one of these processes finishes another can be launched. This means that one can at most process and download as many files as cores the machine has. That is, if for example a machine has 24 cores, only 24 WET files can be downloaded and processed simultaneously, moreover, the 25\textsuperscript{th} file won't be downloaded until one of the previous 24 files is completely processed.

When all the WET files are classified, one would normally get around 160 language files, each file holding just plain text written in its corresponding language. These files still need to be filtered in order to get rid of all files containing invalid UTF-8 characters, so again a number of processes are launched, this time depending on the amount of memory of the machine. Each process reads a language file, first filters for invalid UTF-8 characters and then performs deduplication. A simple non-collision resistant hashing algorithm is used to deduplicate the files.

The fastText linear classifier works by representing sentences for classification as Bags of Words (BoW) and training a linear classifier. A weight matrix $A$ is used as a look-up table over the words and the word representations are then averaged into a text representation which is fed to the linear classifier. The architecture is in general similar to the CBoW model of \citet{mikolov-etal-2013-distributed}, but the middle word is replaced by a label. They uses a softmax function $f$ to compute the probability distribution over the classes. For a set of $N$ documents, the model is trained to minimize the negative log-likelihood over the classes:
\[
    -\frac{1}{N}\sum_{n=1}^{N} y_n\log\left(f(BAx_n)\right),
\]
where $x_n$ is the normalized bag of features of the $n$-th document, $y_n$ is the $n$-th label, and $A,B$ are the weight matrices. The pre-trained fastText model for language recognition \citep{grave-etal-2018-learning} is capable of recognising around 176 different languages and was trained using 400 million tokens from Wikipedia as well as sentences from the Tatoeba website\footnote{\url{https://tatoeba.org/}}.

\subsection{CCNet}

We note that the original Common-Crawl-based corpus created by \citet{grave-etal-2018-learning} to train fastText is not freely available. Since running the experiments described in this paper, a new architecture for creating a Common-Crawl-based corpus named CCNet \citep{wenzek-etal-2020-ccnet} has been published, although it includes specialized filtering based on the KenLM library \cite{heafield-2011-kenlm} and trained on Wikipedia, which might result in a cleaner corpus, the resulting CCNet corpus itself was never published in its entirety.


%%%%%%%%%%%%%%%%%%%%%%%%%%%%%%%%%%%%%%%%%%%%%%%%%%%%%%%%%%%%%%%%%%%%%%%%
\section{Quality at Glance Related Work}
%%%%%%%%%%%%%%%%%%%%%%%%%%%%%%%%%%%%%%%%%%%%%%%%%%%%%%%%%%%%%%%%%%%%%%%%

Corpora collected by web crawlers are known to be noisy~\citep{junczys-dowmunt-2019-microsoft,luccioni-viviano-2021-whats}. In highly multilingual settings, past work found that web-crawls of lower-resource languages have serious issues, especially with segment-level LangID~\citep{caswell-etal-2020-language}.

Cleaning and filtering web-crawls can boost general language modeling~\citep{gao-etal-2020-the,brown-etal-2020-language,raffel-etal-2020-exploring} and downstream task performance~\citep{moore-lewis-2010-intelligent,rarrick-etal-2011-mt,xu-koehn-2017-zipporah,khayrallah-koehn-2018-impact,brown-etal-2020-language}.

As the scale of ML research grows, it becomes increasingly difficult to validate automatically collected and curated datasets \citep{biderman-etal-2020-pitfalls,birhane-etal-2021-large,bender-etal-2021-on}.

Several works have focused on advancing methodologies and best practices to address these challenges. \citet{bender-friedman-2018-data} introduced data statements, a documentary framework for NLP datasets that seeks to provide a universal minimum bar for dataset description. Similar work has been done on systematizing documentation in other areas in data science and machine learning, including work focusing on
online news \citep{kevin-etal-2018-information}, data ethics \citep{sun-etal-2019-mithralabel}, and data exploration \citep{holland-etal-2018-the}, as well as generalist work such as \citep{gebru-etal-2018-datasheets}. 
Data quality is also implicitly documented by successes of filtering methods. There is a large literature on filtering data for various NLP tasks, e.g. \citet{axelrod-etal-2011-domain,moore-lewis-2010-intelligent,rarrick-etal-2011-mt,wang-etal-2018-denoising,kamholz-etal-2014-panlex,junczys-dowmunt-2018-dual,caswell-etal-2020-language}.

Closest to our work is the analysis of a highly multilingual (non-publicly available) web-crawl and LangID related quality issues by \citet{caswell-etal-2020-language}. They perform a brief analysis of the quality of OSCAR with the focus only on the presence of in-language content. \citet{dodge-etal-2021-documenting} automatically documented and analyzed the contents and sources of C4~\citep{raffel-etal-2020-exploring}, the English counterpart of mC4, which surfaced the presence of machine-translated contents and NLP benchmark data.

\subsection{Multilingual Corpora}\label{sec:crawls}

\begin{table*}[th!]
    \centering
    \resizebox{\textwidth}{!}{%
        \begin{tabular}{lccccc}
            \toprule
                            & \multicolumn{3}{c}{\textbf{Parallel}} & \multicolumn{2}{c}{\textbf{Monolingual}}                                                       \\
            \cmidrule(lr){2-4} \cmidrule(lr){5-6}
                            & \textbf{CCAligned}                    & \textbf{ParaCrawl v7.1}                  & \textbf{WikiMatrix} & \textbf{OSCAR} & \textbf{mC4} \\
            \midrule
            \#languages     & 137                                   & 41                                       & 85                  & 166            & 101          \\
            Source          & CC 2013--2020                         & selected websites                        & Wikipedia           & CC 11/2018     & CC all       \\
            Filtering level & document                              & sentence                                 & sentence            & document       & document     \\
            Langid          & FastText                              & CLD2                                     & FastText            & FastText       & CLD3         \\
            Alignment       & LASER                                 & Vec/Hun/BLEU-Align                       & LASER               & -              & -            \\
            Evaluation      & TED-6                                 & WMT-5                                    & TED-45              & POS/DEP-5      & XTREME       \\
            \bottomrule
        \end{tabular}%
    }
    \caption{Comparison of parallel and monolingual corpora extracted from web documents, including their downstream evaluation tasks. All parallel corpora are evaluated for machine translation (BLEU). TED-6: \texttt{da}, \texttt{cr}, \texttt{sl}, \texttt{sk}, \texttt{lt}, \texttt{et}; TED-45: 45-language subset of ~\citep{qi-etal-2018-pre}; WMT-5: \texttt{cs}, \texttt{de}, \texttt{fi}, \texttt{lv}, \texttt{ro}. POS/DEP-5: part-of-speech labeling and dependency parsing for \texttt{bg}, \texttt{ca}, \texttt{da}, \texttt{fi}, \texttt{id}.}
    \label{tab:corpora}
\end{table*}


Table \ref{tab:corpora} provides an overview of the corpora of interest in this work. We selected the corpora for their multilinguality and the inclusion of understudied languages in NLP. With the exception of WikiMatrix and ParaCrawl, all corpora are derived from CommonCrawl (CC).\footnote{\url{http://commoncrawl.org/}}

\paragraph{CCAligned~\citep{el-kishky-etal-2020-ccaligned}}is a parallel dataset built off 68 CC snapshots. Documents are aligned if they are in the same language according to FastText LangID~\citep{joulin-etal-2016-fasttext,joulin-etal-2017-bag}, and have the same URL but for a differing language code. These alignments are refined with cross-lingual LASER embeddings \citep{artetxe-schwenk-2019-massively}. For sentence-level data, they split on newlines and align with LASER, but perform no further filtering.
Human annotators evaluated the quality of document alignments for six languages (\texttt{de}, \texttt{zh}, \texttt{ar}, \texttt{ro}, \texttt{et}, \texttt{my}) selected for their different scripts and amount of retrieved documents, reporting precision of over 90\%. The quality of the extracted parallel sentences was evaluated in a machine translation (MT) task on six European (\texttt{da}, \texttt{cr}, \texttt{sl}, \texttt{sk}, \texttt{lt}, \texttt{et}) languages of the TED corpus~\citep{qi-etal-2018-pre}, where it compared favorably to systems built on crawled sentences from WikiMatrix and ParaCrawl v6.

\paragraph{Multilingual C4 (mC4)~\citep{xue-etal-2021-mt5}} is a document-level dataset used for training the mT5 language model. It consists of monolingual text in 101 languages and is generated from 71 CC snapshots. It filters out pages that contain less than three lines of at least 200 characters and pages that contain bad words.\footnote{\url{https://github.com/LDNOOBW/}} Since this is a document-level dataset, we split it by sentence and deduplicate it before rating. For language identification, it uses CLD3~\citep{botha-etal-2017-natural},\footnote{\url{https://github.com/google/cld3/}} a small feed-forward neural network that was trained to detect 107 languages. The mT5 model pre-trained on mC4 is evaluated on 6 tasks of the XTREME benchmark~\citep{hu-etal-2020-xtreme} covering a variety of languages and outperforms other multilingual pre-trained language models such as mBERT~\citep{devlin-etal-2019-bert} and XLM\nobreakdash-R~\citep{conneau-etal-2020-unsupervised}.

\paragraph{OSCAR~\citep{ortiz-suarez-etal-2019-asynchronous, ortiz-suarez-etal-2020-monolingual}}is a set of monolingual corpora extracted from CC snapshots, specifically from the plain text \emph{WET} format distributed by CC which removes all the HTML tags and converts the text to UTF-8. It is deduplicated and follows the approach by~\citep{grave-etal-2018-learning} of using FastText LangID~\citep{joulin-etal-2016-fasttext, joulin-etal-2017-bag} on a line-level.\footnote{\url{https://fasttext.cc/docs/en/language-identification.html} } No other filtering was applied. For five languages (\texttt{bg}, \texttt{ca}, \texttt{da}, \texttt{fi}, \texttt{id}) OSCAR was used by its original authors to train language models which were then evaluated on parsing and POS tagging \citep{ortiz-suarez-etal-2020-monolingual}. OSCAR has also been used in independent studies to train monolingual or multilingual language models (\texttt{ar}, \texttt{as}, \texttt{bn}, \texttt{de}, \texttt{el}, \texttt{fr}, \texttt{gu}, \texttt{he}, \texttt{hi}, \texttt{kn}, \texttt{ml}, \texttt{mr}, \texttt{nl}, \texttt{or}, \texttt{pa}, \texttt{ro}, \texttt{ta}, \texttt{te}) and subsequently evaluate them on various downstream tasks \citep{antoun-etal-2021-araelectra, kakwani-etal-2020-indicnlpsuite, wilie-etal-2020-indonlu, chan-etal-2020-germans, koutsikakis-etal-2020-greek, martin-etal-2020-camembert, chriqui-etal-2021-hebert, seker-etal-2021-alephbert, delobelle-etal-2020-robbert, dumitrescu-etal-2020-birth, masala-etal-2020-robert}.


\paragraph{ParaCrawl v7.1} is a parallel dataset with 41 language pairs primarily aligned with English (39 out of 41) and mined using the parallel-data-crawling tool Bitextor \citep{espla-etal-2019-paracrawl,banon-etal-2020-paracrawl} which includes downloading documents, preprocessing and normalization, aligning documents and segments, and filtering noisy data via Bicleaner.\footnote{\url{https://github.com/bitextor/bicleaner}}
ParaCrawl focuses on European languages, but also includes 9 lower-resource, non-European language pairs in v7.1. Sentence alignment and sentence pair filtering choices were optimized for five languages (\texttt{mt}, \texttt{et}, \texttt{hu}, \texttt{cs}, \texttt{de}) by training and evaluating MT models on the resulting parallel sentences. An earlier version (v5) was shown to improve translation quality on WMT benchmarks for~\texttt{cs}, \texttt{de}, \texttt{fi}, \texttt{lv}, \texttt{ro}.


\paragraph{WikiMatrix~\citep{schwenk-etal-2021-wikimatrix}} is a public dataset containing 135M parallel sentences in 1620 language pairs (85 languages) mined from Wikipedia. Out of the 135M parallel sentences, 34M are aligned with English. The text is extracted from Wikipedia pages, split into sentences, and duplicate sentences are removed. FastText LangID is used before identifying bitext with LASER's distance-based mining approach. The margin threshold is optimized by training and evaluating downstream MT models on four WMT benchmarks (\texttt{de-en}, \texttt{de-fr}, \texttt{cs-de}, \texttt{cs-fr}). The final dataset is used to train translation models that are then evaluated by automatically measuring the quality of their translations against human translations of TED talks in 45 languages, with highest quality for translations between English and e.g. \texttt{pt}, \texttt{es}, \texttt{da}, and lowest for \texttt{sr}, \texttt{ja}, \texttt{mr}, \texttt{zh\_TW}. In the audit we focus on language pairs with English on one side.

%%%%%%%%%%%%%%%%%%%%%%%%%%%%%%%%%%%%%%%%%%%%%%%%%%%%%%%%%%%%%%%%%%%%%%%%
\section{Monolingual Related Work}
%%%%%%%%%%%%%%%%%%%%%%%%%%%%%%%%%%%%%%%%%%%%%%%%%%%%%%%%%%%%%%%%%%%%%%%%

Since the introduction of \emph{word2vec} \citep{mikolov-etal-2013-distributed}, many attempts have been made to create multilingual language representations; for fixed word embeddings the most remarkable works are those of \citep{al-rfou-etal-2013-polyglot} and \citep{bojanowski-etal-2017-enriching} who created word embeddings for a large quantity of languages using Wikipedia, and later \citep{grave-etal-2018-learning} who trained the fastText word embeddings for 157 languages using Common Crawl and who in fact showed that using crawled data significantly increased the performance of the embeddings especially for mid- to low-resource languages.

Regarding contextualized models, the most notable non-English contribution has been that of the mBERT \citep{devlin-etal-2019-bert}, which is distributed as (i)~a single multilingual model for 100 different languages trained on Wikipedia data, and as (ii)~a single multilingual model for both Simplified and Traditional Chinese. Four monolingual fully trained ELMo models have been distributed for Japanese, Portuguese, German and Basque\footnote{\url{https://allennlp.org/elmo}}; 44 monolingual ELMo models\footnote{\url{https://github.com/HIT-SCIR/ELMoForManyLangs}} where also released by the \emph{HIT-SCIR} team \citep{che-etal-2018-towards} during the \emph{CoNLL 2018 Shared Task} \citep{zeman-etal-2018-conll}, but their training sets where capped at 20 million words.

For dependency parsing and POS tagging the most notable non-English specific contribution is that of the \emph{CoNLL 2018 Shared Task} \citep{zeman-etal-2018-conll}, where the 1\textsuperscript{st} place (LAS Ranking) was awarded to the \emph{HIT-SCIR} team \citep{che-etal-2018-towards} who used \citet{dozat-manning-2017-deep}'s \emph{Deep Bi-affine parser} and its extension described in \citep{dozat-etal-2017-stanfords}, coupled with deep contextualized ELMo embeddings \citep{peters-etal-2018-deep} (capping the training set at 20 million words). The 1\textsuperscript{st} place in universal POS tagging was awarded to \citet{smith-etal-2018-82} who used two separate instances of \citet{bohnet-etal-2018-morphosyntactic}'s tagger.

More recent developments in POS tagging and parsing include those of \citet{straka-strakova-2019-evaluating} which couples another CoNLL 2018 shared task participant, UDPipe 2.0 \citep{straka-2018-udpipe}, with mBERT greatly improving the scores of the original model, and UDify \citep{kondratyuk-straka-2019-75}, which adds an extra attention layer on top of mBERT plus a Deep Bi-affine attention layer for dependency parsing and a Softmax layer for POS tagging. UDify is actually trained by concatenating the training sets of 124 different UD treebanks, creating a single POS tagging and dependency parsing model that works across 75 different languages.

Concerning contextual models, \citet{baevski-etal-2019-cloze} trained a BERT-like bi-directional Transformer for English using Common Crawl. They followed the ``fastText pre-processing pipeline'' but they removed all copies of Wikipedia inside Common Crawl. They also trained their model using News Crawl \citep{bojar-etal-2018-findings} and using Wikipedia + BooksCorpus, they compared three models and showed that Common Crawl gives the best performance out of the three corpora.

The XLNet model was trained for English by joining the BookCorpus, English Wikipedia, Giga5 \citep{parker-etal-2011-english}, ClueWeb 2012-B \citep{callan-etal-2009-clueweb09} and Common Crawl. Particularly for Common Crawl, \citet{yang-etal-2019-xlnet} say they use ``heuristics to aggressively filter out short or low-quality articles'' from Common Crawl, however they don't give any detail about these ``heuristics'' nor about the pipeline they use to classify and extract the English part of Common Crawl.

It is important to note that none of these projects distributed their classified, filtered and cleaned versions of Common Crawl, making it difficult in general to faithfully reproduce their results.


